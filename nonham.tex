\chapter[Негамильтоновы диффеоморфизмы]{Геометрия негамильтоновых диффеоморфизмов}\label{chap:14}

В этой главе мы обсудим связь между группами $\Ham(M,\Omega)$ и $\Symp_0(M,\Omega)$ и роль негамильтоновых диффеоморфизмов в хоферовской геометрии.

\section{Гомоморфизм потока}\label{sec:14.1}

Пусть $(M,\Omega)$ --- замкнутое симплектическое многообразие, напомним, что $\Symp_0(M,\Omega)$ обознчает компоненту единицы в группе симплектоморфизмов (см. \ref{1.4.C}).
Пусть $\{f_t\}$  путь симплектоморфизмов с $f_0 = \1$,
рассмотрим векторное поле $\xi_t$ на $M$, порождающее этот путь.
Заметим, что $\L_{\xi_t}\Omega=0$ и значит
$i_{\xi_t}\Omega=\lambda_t$ --- семейство замкнутых 1-форм.
Подчеркнём, что эти формы не обязательно точны.
Напомним базовый пример, который мы начали обсуждать в начале книги (см. \ref{1.4.C}).


\begin{thm}{Пример}\label{14.1.A}
Рассмотрим двумерный тор $(\TT^2=\RR^2/\ZZ^2$, $\d p\z\wedge \d q)$ и следующую систему уравненинй
\[
\begin{cases}
\quad\dot p=0,
\\
\quad\dot q=1.
\end{cases}
\]
соответствующую пути симплектоморфизмов $f_t(p, q) = (p, q \z+ t)$.
Этот путь порождён автономным семейством замкнутых 1-форм $\lambda_t = dp$.
\end{thm}


Наш первый вопрос заключается в следующем.
Пусть дан симплектоморфизм $f\in\Symp_0(M,\Omega)$ (скажем, заданный явной формулой).
\textit{Как бы понять, является ли $f$ гамильтоновым?}
Мощным инструментом, позволияющем ответить на этот вопрос в приведённом выше примере, является понятие \?{гомоморфизма потока}{?}; оно было введенно \rindex{Калаби}Калаби и изучалось далее \rindex{Баньяга}Баньягой \cite{B1}.
Это понятие является предметом настоящей главы.
Обратитим внимание на то, что если $f$ задан неточной формой, то вообще говоря этого недостаточно для того чтобы утверждать, что $f$ негамильтонов.
Например, в приведённом выше примере $f_1 = \1$ является гамильтоновым.
Введём следующее полезное понятие.
Пусть $\{f_t\}$, $t\in[0;1]$ --- петля симплектоморфизмов с $f_0 = f_1 = \1$.
Пусть ${\lambda_t}$ --- семейство замкнутых 1-форм, порождающих эту петлю.

\begin{ex*}{Определение}
\rindex{поток петли}\emph{Поток петли} $\{f_t\}$ выражается как \index[symb]{$\flux$}
\[\flux(\{f_t\}) = \int_0^1 [\lambda_t]\,\d t \in H^1(M, \RR).\]
\end{ex*}

Приведём более геометрическое описание.
Пусть $C$ --- 1-цикл на $M$.
Определим 2-цикл $\partial[C] = \bigcup_t f_t(C)$, являющийся образом $C$ относительно потока $f_t$.
Заметим, что $\partial$ является линейным отображением $H_1(M,\ZZ)\to H_2(M,\ZZ)$.

\begin{ex*}{Упражнение}
Покажите, что $(\flux(\{f_t\}), [C]) = ([\Omega], \partial[C])$ для всех $[C]\in H^1(M,\ZZ)$.\?{}{сказать, что это спаривание?}
\end{ex*}

В частности, это показывает, что $\flux(\{f_t\})$ зависит только от гомотопического класса $\{f_t\}$ в $\pi_1 (\Symp_0(M,\Omega))$, поскольку правая часть сохраняется при гомотопиях.
Таким образом, мы получаем гомоморфизм потока
\[\flux\:\pi_1(\Symp_0(M,\Omega))\to H^1(M,\RR).\]

\begin{ex*}{Definition}
Образ $\Gamma \subset H^1(M, \RR)$ гомоморфизма потока называется \rindex{группа потока}\emph{группой потока}.
\end{ex*}


Если $\gamma$ представлен гамильтоновой петлёй, то $\flux(\gamma) = 0$.
Фундаментальный результат \rindex{Баньяга}Баньяги \cite{B1} говорит, что верно и обратное.
То есть если $\flux(\gamma) = 0$ то $\gamma$ гомотопна с гамильтоновой петле.

Полезно обобщить понятие потока на произвольные (не обязательно замкнутые) гладкие пути симплектоморфизмов.
Для заданного пути $\{f_t\}$, порождённого семейством замкнутых 1-форм $\lambda_t$, положим 
\[\flux(\{f_t\})=\int_0^1[\lambda_t]\,\d t\in H^1(M,\RR).\]
Возьмём симплектоморфизм $f \in \Symp_0(M,\Omega)$ и выберем любой путь $\{f_t\}$, для которого $f_0 = \1$ и $f_1 = f$.
Ясно, что $\flux (\{f_t\})$ зависит от выбора пути, соединяющего $\1$ с $f$, но разность между потоками любых двух таких путей принадлежит $\Gamma$.
Таким образом, мы получаем отображение
\[\Delta\:\Symp_0(M,\Omega)\to H^1(M,\RR)/\Gamma.\]
Проверку того, что $\Delta$ является гомоморфизмом оставляем читателю.
Баньяга показал, что $\ker(\Delta) = \Ham(M,\Omega)$, поэтому 
\[\Symp_0(M,\Omega)/\Ham(M,\Omega) = H^1(M, \RR)/\Gamma.\]

\begin{ex}{Упражнение}\label{14.1.B}
Воспользуйтесь последним равенством вместе с теоремой \ref{1.5.A} дабы докать, что всякая нормальная подгруппа $\Symp_0(M,\Omega)$ имеет вид $\Delta^{-1}(K)$ для некоторой подгруппы $K$ в $H^1(M, \RR)/\Gamma$,
\end{ex}

Вычислим группу потока для $(\TT^2,\d p\wedge \d q)$ (см. \ref{14.1.A} выше).
Отождествим следующие группы: 
\begin{align*}
H_1(\TT^2,\ZZ)&=\ZZ^2,
&
H_2(\TT^2,\ZZ)&=\ZZ,
&
H^1(\TT^2,\ZZ)&=\ZZ^2
\subset
\RR^2=
H^1(\TT^2,\RR).
\end{align*}
На этом языке $\Gamma = \ZZ^2$.
В самом деле, для каждого $\gamma\in\pi_1(\Symp_0(\TT^2))$ имеем 
\[(\flux(\gamma), a) = ([dp \wedge dq], \partial a).\]
Здесь $\partial$ --- функционал, переводящий $\ZZ^2$ в $\ZZ$, значение класса $[\d p \wedge \d q]$ на образующей $H_2 (\TT^2,\ZZ)$ равно $1$, и $a \in H_1 (\TT^2, \ZZ)$.\?{}{не понял?}
Таким образом, $\flux(\gamma)\in H^1(\TT^2,\ZZ)$, откуда следует, что $\Gamma\subset \ZZ^2$.
С другой стороны, потоки полных оборотов тора задаются формулами 
\begin{align*}
\flux( (p, q) \mapsto (p , q+ t))
&=
[dp],
\\
\flux( (p, q) \mapsto (p+ t , q))
&=
-[dq],
\end{align*}
и мы заключаем, что $\ZZ^2 \subset  \Gamma$.
Поэтому $\Gamma = \ZZ^2$.
В частности, мы видим, что $f_T(p, q) = (p, q + T)$ является гамильтоновым ровно когда $T \in \ZZ$.
Действительно, $\Delta(f_T) = T[dp] \mod{\ZZ^2}$.

\section{Гипотеза потока}

В общем случае вычислить группу потоков не просто.
Простой вопрос, является ли $\Gamma$ дискретным, оказывается важным по следующей причине.
Напомним (см. \ref{1.4.F} выше), что гипотеза потока%
\footnote{Точнее гипотеза $C^\infty$-потока.
Обратите внимание, что она эквивалентна гипотезе $C^1$-потока.
Однако, гипотеза $C^0$-потока является гораздо более сильным утверждением (см. обсуждение в \cite{LMP1}).}
утверждает, что подгруппа $\Ham(M,\Omega)$ является $C^\infty$-замкнутой в $\Symp_0(M,\Omega)$.

\begin{thm}{Теорема}\label{14.2.A}
Если подгруппа $\Gamma$ дискретна, то гипотеза потока верна. 
\end{thm}

\parit{Доказательство.}
Пусть $\phi_k$ --- последовательность гамильтоновых диффеоморфизмов $C^\infty$-сходящаяся к $\phi\in\Symp_0(M,\Omega)$.
Выберем путь симплектоморфизмов, соединяющий $\phi$ с единицей.
Обозначим через $\lambda \in H^1 (M, \RR)$ его поток.
Мы утверждаем, что существует последовательность путей, соединяющих $\phi_k$ с $\phi$, 
таких, что $\epsilon_k=\flux \phi_k\to 0$ при $k\to +\infty$.
Предположим, что это утверждение даказано.
Тогда (см. рис. \ref{pic-13}) для каждого $k$ существует петля с потоком $\lambda + \epsilon_k\in \Gamma$.
\begin{figure}[ht!]
\centering
\includegraphics{mppics/pic-13}
\caption{}\label{pic-13}
\vskip0mm
\end{figure}
Поскольку $\Gamma$ дискретно, мы заключаем, что $\epsilon_k=0$ при всех достаточно больших $k$.



Таким образом, диффеоморфизм $\phi$ гамильтонов, так как $\Delta(\phi) \z= 0 \mod{\Gamma}$  --- теорема доказана.

Остаётся доказать наше утверждение.
Для этого достаточно найти последовательность путей $\gamma_k$, от $\1$ до $\phi_k^{-1}\phi$, потоки которых сходятся к $0$.

\begin{figure}[ht!]
\centering
\includegraphics{mppics/pic-14}
\caption{}\label{pic-14}
\vskip0mm
\end{figure}

Заметим, что $\phi_k^{-1}\phi \to \1$ при $k \to +\infty$, поэтому график $\phi_k^{-1}\phi$ близок к диагонали $L$ в $(M \times M, \Omega \oplus -\Omega)$.
Теперь воспользуемся следующим приёмом.
Напомним, что $L$ --- лагранжево подмногообразие в $(M \times M, \Omega \oplus -\Omega)$.

\begin{thm}{Лемма}
Пусть $(P, \omega)$ --- симплектическое многообразие, а $L\subset P$ --- замкнутое лагранжево подмногообразие.
Тогда существуют окрестность $U$ многообразия $L$ в $P$ и вложение $f\: U \to \T^\ast L$ со следующими свойствами:
\begin{itemize}
\item $f^\ast\omega_0 = \omega$, где $\omega_0$ --- стандартная симплектическая структура на $\T^\ast L$;
\item $f(x) = (x,0)$ для всех $x\in L$.
\end{itemize}
\end{thm}
Это вариант \rindex{теорема Дарбу}\emph{теоремы Дарбу} для лагранжевых подмногообразий \cite{MS}.

Применим эту лемму к диагонали $L \subset M \times M$.
Она позволяет отождествить трубчатую окрестность $L$ с трубчатой окрестностью нулевого сечения в $\T^\ast L$.
Поскольку $\T^\ast L \simeq \T^\ast M$, мы видим, что при больших $k$ график $(\phi_k^{-1}\phi)$ соответствует сечению $\T^\ast M$, которое является замкнутой $1$-формой $\alpha_k$ на $M$.\footnote{См. упражнение на странице \pageref{1-form-lagrange}. --- \textit{Прим. ред.}}

Ясно, что $\alpha_k \to 0$ при $k\z\to +\infty$.
Зафиксируем достаточно большое $k$ и рассмотрим деформацию лагранжевых подмногообразий 
\[\graph (s\alpha_k) \subset \T^\ast M\]
для $s \in [0;1]$.
Для каждого $s$ график $s\alpha_k$ отождествляется с графиком симплектоморфизма $M$.
Обозначим этот симплектоморфизм через $\gamma_k(s)$.
Ясно, что поток пути $\gamma_k(s)$ равен $[\alpha_k]$.
Мы получаем требуемый путь и завершаем доказательство.
\qeds

Из приведённого выше доказательства становится ясно где сидит
сложность гипотезы потока. 
Она заключается в том, что для произвольной гамильтоновой изотопии
$\{f_t\}$ график $\{f_t\}$ может выйти из трубчатой окрестности $L$
при некотором $t$. 
Он вернётся в эту окрестность, но останется ли он графиком точной 1-формы.
Отметим также, что верна и обратная теорема (см. \cite{LMP1}).

\begin{thm}{Следствие}
Если $[\Omega] \in H^2(M, \QQ)$, то гипотеза потока верна.
\end{thm}

\parit{Доказательство.}
Мы можем переписать это предположение как $[\Omega] \z\in \tfrac1kH^2(M, \ZZ)$ для некоторого $k\in\ZZ$.
Таким образом, 
\[(\flux(\gamma), [C]) \z= ([\Omega], \partial[C]) \z\in \frac1k\ZZ\] для каждого $\gamma \in \pi_1(\Symp_0(M,\Omega))$.
Поэтому $\Gamma\subset \tfrac1k H^1(M,\ZZ)\subset H^1(M,\RR)$, и мы заключаем, что $\Gamma$ дискретно.
Остаётся применить приведённую выше теорему.
\qeds

\section{Связь с жёсткой симплектической топологией}

Приведём более концептуальное доказательство гипотезы потока для $\TT^2$.
Ясно, что $C^k$-замыкание $\Ham(M,\Omega)$ в $\Symp_0(M,\Omega)$ является нормальной подгруппой в $\Symp_0(M,\Omega)$.
Следовательно, учитывая \ref{14.1.B} выше, достаточно показать, что для любого 
\[\alpha\in H^1(\TT^2, \RR)/\Gamma \backslash\{0\}
\quad\text{существует}\quad
f \in \Symp_0(M,\Omega)
\quad\text{с}\quad
\Delta(f)=\alpha\]
который непредставим как предел гамильтоновых диффеоморфизмов.
Не умоляя общности можно считать, что $f$ является сдвигом $(p, q) \mapsto (p, q+T)$, где $T \notin\ZZ$.
Знаменитая гипотеза \rindex{Арнольд}Арнольда, доказанная в \rindex{Зендер}\rindex{Конли}\cite{CZ}, утверждает, что каждый $\phi \in \Ham(\TT^2)$ имеет неподвижную точку.
Таким образом, если $\phi_k \to f$, это означало бы, что $f$ также имеет неподвижную точку --- противоречие.
Это рассуждение принадлежит \rindex{Герман}М. Герману (1983), и оно работает и для $\TT^{2n}$.
Отметим также, что оно доказывает гипотезу $C^0$-потока.

Можно попытаться обобщить это рассуждение на другие симплектические многообразия.
Идея состоит в том, чтобы вместо предельного поведения неподвижных точек рассматривать предельное поведение гомологий Флоера.
На этом пути, получается следующий результат.


\begin{thm}[(\cite{LMP1})]{Теорема}\label{14.3.A}
Предположим, что первый класс Черна $c_1(\T M)$ обращается в нуль на $\pi_2(M)$.
Тогда гипотеза потока верна.
\end{thm}


И вот ещё одно приложение теории псевдоголоморфных кривых к гипотезе потока.
Оно было найдено в \rindex{Лалонд}\cite{LMP2} для 4-мерных симплектических многообразий и позже доказано в \cite{McD2} в полной общности.

\begin{thm}{Теорема}\label{14.3.B}
Ранг группы потоков $\Gamma$ конечен и удовлетворяет неравенству
\[\rank_\ZZ \Gamma\le b_1(M) = \dim H^1(M,\RR).\]
Как следствие, мы получаем, что для симплектических многообразий, у которых первое число Бетти равно $1$, группа $\Gamma$ дискретна, и значит для них верна гипотеза потока.
\end{thm}

\section{Изометрии в хоферовской геометрии }

Естественная метрика на $\Symp_0(M,\Omega)$ неизвестна.
Однако общие негамильтоновы симплектоморфизмы можно включить в рамки хоферовской геометрии следующим образом (см. \cite{LP}).
Заметим, что группа $\Symp(M,\Omega)$ всех симплектоморфизмов $(M,\Omega)$ действует на $\Ham(M,\Omega)$ изометриями.
Для $\phi \in \Symp(M,\Omega)$ определим 
\[T_\phi\: \Ham(M,\Omega) \to \Ham(M,\Omega),
\quad
f \mapsto \phi f\phi^{-1}.\]
Легко проверить, что $T_\phi$ корректно определено и является изометрией для хоферовской метрики $\rho$.

\begin{ex*}{Определение}
Изометрия $T$ называется \rindex{ограниченная изометрия}\emph{ограниченной}, если $\sup \rho(f, Tf)\z<\infty$, где точная верхняя грань берётся по всем $f\in\Ham(M,\Omega)$.
\end{ex*}


Если $\phi$ гамильтонов, то изометрия $T_\phi$ ограничена.
Действительно, следующее неравенство даёт оценку, не зависящую от $f$:
\[\rho(f,T_\phi f) = \rho(f,\phi f\phi^{-1}) = \rho(\1,\phi f\phi^{-1}f^{-1})\le 2\rho(\1, \phi).\]
Определим множество $BI_0 \subset \Symp_0(M,\Omega)$ как множество всех $\phi \z\in \Symp_0(M,\Omega)$ таких, что $T_\phi$ ограничено.
Как мы видели выше, $\Ham(M,\Omega)\subset BI_0$.

\begin{ex}{Упражнение}\label{14.4.A}
Докажите, что $BI_0$ --- нормальная подгруппа в $\Symp_0(M,\Omega)$.
\end{ex}

Следующая гипотеза даёт возможную характеристику гамильтоновых диффеоморфизмов в метрических терминах.

\begin{ex*}{Гипотеза}
$\Ham(M,\Omega) = BI_0$.
\end{ex*}

\begin{thm}[(\cite{LP})]{Теорема}\label{14.4.B}\rindex{Лалонд}
Гипотеза верна для поверхностей рода $\ge1$ и их произведений.
\end{thm}

Приведём идею доказательства для случая, когда $M=\TT^2$.
Как следует из \ref{14.4.A}, для доказательства теоремы достаточно показать следующее.
Пусть
\[(a,b)\in H^1(\TT^2,\RR)\backslash\Gamma=\RR^2\backslash\ZZ^2,\]
тогда существует симплектоморфизм $\phi\in\Symp_0(\TT^2)$ с $\Delta (\phi) = (a, b) \mod \ZZ^2$ такой, что соответствующая изометрия $T_\phi$ неограничена.
Не умоляя общности можно предположить, что $a= 0$, $b \in (0; 1)$ и $\phi(p, q) = (p, q + b)$.
Заметим, что для кривой $C = \{q = 0\}$ выполняется $C \cap \phi(C) = \emptyset$.
Пусть $F = F(q)$ --- нормированный гамильтониан на $\TT^2$, носитель которого лежит в малой окрестности $C$ и такой, что $F|_C \equiv 1$.
Обозначим через $f_t$ соответствующий гамильтонов поток и рассмотрим поток, образованный коммутаторами $g_t = \phi^{-1}f_t^{-1}\phi f_t$.
Этот поток порождается гамильтонианом $G(q) = F(q) - F(q + b)$.
Поскольку $G|_C \equiv 1$, из \ref{7.4.A} следует, что $\rho(\1, g_t)$ стремится к бесконечности при $t \to \infty$.
Таким образом, изометрия $T_\phi$ неограничена.

Задача описания всех изометрий $(\Ham(M,\Omega), \rho)$ открыта и кажется очень сложной даже для поверхностей.
В связи с этим напомню следующий классический \rindex{теорема Мазура --- Улама}результат Мазура --- Улама \cite{MU} (1932): всякая изометрия линейного нормированного пространства с фиксированной точкой в нуле является линейным отображением.
Было бы интересно доказать или опровергнуть нелинейный вариант этого утверждения: всякая изометрия группы $\Ham(M,\Omega)$ с фиксированной точкой $\1$ является изоморфизмом групп (возможно, с точностью до композиции с инволюцией $f\mapsto f^{-1}$).
Предположим на мгновение, что это действительно так.
Тогда теорема \rindex{Баньяга}Баньяги \ref{1.5.D} влекла бы, что каждая такая изометрия (с точностью до упомянутой выше инволюции) совпадает с $T_\phi$, где $\phi\: M \to M$ --- либо симплектоморфизм, либо антисимплектоморфизм (то есть $\phi^\ast\Omega = -\Omega$).
Это означало бы, что хоферовская геометрия определяет симплектическую топологию.
