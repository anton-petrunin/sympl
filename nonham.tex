\chapter[Негамильтоновы диффеоморфизмы]{Геометрия негамильтоновых диффеоморфизмов}

В этой главе мы обсудим связь между группами $\Ham(M,\Omega)$ и $\Symp_0(M,\Omega)$ и роль негамильтоновых диффеоморфизмов в хоферовской геометрии.

\section{Гомоморфизм потока}

Пусть $(M,\Omega)$ --- замкнутое симплектическое многообразие, а $\Symp_0(M,\Omega)$ --- компонента единицы группы симплектоморфизмов (см. \ref{1.4.C}).
Пусть $\{f_t\}$  путь симплектоморфизмов с $f_0 = \1$,
рассмотрим векторное поле $\xi_t$ на $M$, порождающее этот путь.
Заметим, что $\L_{\xi_t}\Omega=0$ и значит  $i_{\xi_t}\Omega=\lambda_t$ ​​--- семейство замкнутых 1-форм.
Подчеркнём, что эти формы не обязательно точны.
Давайте рассмотрим базовый пример, который мы начали обсуждать в начале книги (см. \ref{1.4.C}).

\begin{thm}{Пример}\label{14.1.A}
Рассмотрим двумерный тор $(\TT^2=\RR^2/\ZZ^2$, $\d p\z\wedge \d q)$ и следующую систему уравненинй
\[
\begin{cases}
\dot p=0,
\\
\dot q=1.
\end{cases}
\]
соответствующую пути симплектоморфизмов $f_t(p, q) = (p, q \z+ t)$.
Этот путь порождён автономным семейством замкнутых 1-форм $\lambda_t = dp$.
\end{thm}


Первый вопрос, который мы хотим исследовать, заключается в следующем.
Возьмём любое $f \in \Symp_0(M,\Omega)$ (скажем, заданное явной формулой).
\textit{Как бы нам понять, является $f$ гамильтоновым или нет?}
Мощным инструментом, позволияющем ответить на этот вопрос в приведенном выше примере, является понятие \?{гомоморфизма потока}{?}; оно было введенно Калаби и изучалось далее Баньягой \cite{B1}.
Это понятие и является предметом настоящей главы.
Обратитим внимание на то, что если путь $f$ индуцирован неточной формой, то этого вообще говоря этого недостаточно чтобы утверждать, что $f$ не гамильтонов.
Например, в приведенном выше примере $f_1 = \1$ является гамильтоновым.
Введем следующее полезное понятие.
Пусть $\{f_t\}$, $t\in[0,1]$ --- петля симплектоморфизмов с $f_0 = f_1 = \1$.
Пусть ${\lambda_t}$ --- семейство замкнутых 1-форм, порождающих эту петлю.

\begin{ex*}{Определение}
\emph{Поток} петли определяется как 
\[\flux(\{f_t\}) = \int_0^1 [\lambda_t]\,\d t \in H^1(M, \RR).\]
\end{ex*}

Приведём более геометрическое описание.
Пусть $C$ --- 1-цикл на $M$.
Определим 2-цикл $\partial[C] = \bigcup_t f_t(C)$, являющийся образом $C$ относительно потока $f_t$.
Заметим, что $\partial$ является линейным отображением из $H_1(M,\ZZ)$ в $H_2(M,\ZZ)$.

\begin{ex*}{Упражнение}
Покажите, что $(\flux(\{f_t\}), [C]) = ([\Omega], \partial[C])$ для всех $[C]\in H^1(M,\ZZ)$.
\end{ex*}

В частности, это показывает, что $\flux(\{f_t\})$ зависит только от гомотопического класса $\{f_t\}$ в $\pi_1 (\Symp_0(M,\Omega))$, поскольку правая часть сохраняется при гомотопиях.
Таким образом, мы получаем гомоморфизм потока
\[\flux\:\pi_1(\Symp_0(M,\Omega))\to H^1(M,\RR).\]

\begin{ex*}{Definition}
Образ гомоморфизма потока $\Gamma \subset H^1(M, \RR)$ называется \emph{группой потокa}.
\end{ex*}


Если $\gamma$ представлен гамильтоновой петлёй, то $\flux(\gamma) = 0$.
Фундаментальный результат Баньяги (\cite{B1}) говорит, что верно и обратное.
То есть если $\flux(\gamma) = 0$ то $\gamma$ гомотопна с гамильтоновой петле.

Полезно обобщить понятие потока на произвольные (не обязательно замкнутые) гладкие пути симплектоморфизмов.
Для заданного пути $\{f_t\}$, порождённого семейством замкнутых 1-форм $\lambda_t$, положим 
\[\flux(\{f_t\})=\int_0^1[\lambda_t]\,\d t\in H^1(M,\RR).\]
Возьмём симплектоморфизм $f \in \Symp_0(M,\Omega)$ и выберем любой путь $\{f_t\}$, для которого $f_0 = \1$ и $f_t = f$.
Ясно, что $\flux (\{f_t\})$ зависит от выбора пути, соединяющего $\1$ с $f$, но разность между потоками любых двух таких путей принадлежит $\Gamma$.
Таким образом, мы получаем отображение
\[\Delta\:\Symp_0(M,\Omega)\to H^1(M,\RR)/\Gamma.\]
Проверку того, что $\Delta$ является гомоморфизмом оставим читателю.
Баньяга показал, что $\ker(\Delta) = \Ham(M,\Omega)$, поэтому 
\[\Symp_0(M,\Omega)/\Ham(M,\Omega) = H^1(M, \RR)/\Gamma.\]

\begin{ex}{Упражнение}\label{14.1.B}
Воспользуйтесь последним равенством вместе с теоремой \ref{1.5.A} дабы докать, что всякая нормальная подгруппа $\Symp_0(M,\Omega)$ имеет вид $\Delta^{-1}(K)$ для некоторой подгруппы $K$ группы $H^1(M, \RR)/\Gamma$,
\end{ex}

Вычислим группу потока для $(\TT^2,\d p\wedge \d q)$ (см. \ref{14.1.A} выше).
Отождествим следующие группы: 
\begin{align*}
H_1(\TT^2,\ZZ)&=\ZZ^2,
&
H_2(\TT^2,\ZZ)&=\ZZ,
&
H^1(\TT^2,\ZZ)&=\ZZ^2
\subset
\RR^2=
H^1(\TT^2,\RR).
\end{align*}
Мы утверждаем, что на этом языке $\Gamma = \ZZ^2$.
В самом деле, для каждого $\gamma\in\pi_1(\Symp_0(\TT^2))$ имеем 
\[(\flux(\gamma), a) = ([dp \wedge dq], \partial a).\]
Здесь $\partial$ --- функционал, переводящий $\ZZ^2$ в $\ZZ$, значение класса $[\d p \wedge \d q]$ на образующей $H_2 (\TT^2,\ZZ)$ равно $1$, и $a \in H_1 (\TT^2, \ZZ)$.\?{}{не понял?}
Таким образом, $\flux(\gamma)\in H^1(\TT^2,\ZZ)$, откуда следует, что $\Gamma\subset \ZZ^2$.
С другой стороны, поскольку потоки полных оборотов тора задаются формулами 
\begin{align*}
\flux( (p, q) \mapsto (p , q+ t))
&=
[dp],
\\
\flux( (p, q) \mapsto (p+ t , q))
&=
-[dq],
\end{align*}
и мы заключаем, что $\ZZ^2 \subset  \Gamma$.
Поэтому $\Gamma = \ZZ^2$.
В частности, мы видим, что $f_T(p, q) = (p, q + T)$ является гамильтоновым ровно когда $T \in \ZZ$.
Действительно, $\Delta(f_T) = T[dp] \mod{\ZZ^2}$.
