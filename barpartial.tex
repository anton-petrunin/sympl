\chapter
[\texorpdfstring{$\bm{\bar\partial}$-уравнение}{∂-уравнение}]
{Лагранжевы граничные условия на $\bm{\bar\partial}$-уравнение}

В этой главе доказывается теорема~\ref{3.2.B}, которая утверждает, что $\gamma (L) \le \pi r^2$ для любого замкнутого рационального лагранжева подмногообразия $L \subset B^2 (r) \times \RR^{2n-2}$.
Доказательство основано на громовской технике псевдоголоморфных дисков.

\section[\texorpdfstring{Знакомство с $\bar\partial$-оператором}{Знакомство с ∂-оператором}]{Знакомство с $\bm{\bar\partial}$-оператором}\label{sec:4.1}

Отождествим $\RR^{2n} (p_1,q_1,\dots, p_n, q_n)$ с комплексным пространством
\[\CC^n (p_1 + iq_1,\dots, p_n + iq_n)
=
\CC^n (w_1,\dots, w_n)\]
и обозначим через $\langle\ ,\  \rangle$ евклидово скалярное произведение.
У нас появились три геометрические структуры: евклидова, симплектическая и комплексная.
Они связаны следующим образом
\[\langle \xi, \eta\rangle = \omega (\xi, i\eta).\]
Ограничимся проверкой этой формулы при $n = 1$.
Если $\xi \z= (p', q')$ и $\eta = (p'', q'')$, то
\[dp\wedge dq(\xi, i\eta)
=
dp\wedge dq\left(\binom{p'}{q'},\binom{-q''}{p''}\right)
=
p' p'' + q' q''
=
\langle\xi, \eta\rangle.\]
Далее мы будем измерять площади и длины с помощью евклидовой метрики.
Рассмотрим единичный круг $D^2 \subset \CC$ с координатой $z \z= x \z+ iy$.
Пусть $f\: D^2 \to \CC^n$ гладкое отображение.
Определим $\bar\partial$-оператор $\bar\partial\: C^\infty (D^2, \CC^n) \to C^\infty  (D^2, \CC)$ как 
\[\bar\partial f=\frac12\left(\frac{\partial f}{\partial x} + i \frac{\partial f}{\partial x}\right).\]

\begin{ex*}{Пример}
Пусть $f\: \CC \to \CC$, $z \mapsto \bar z$.
Тогда $f(x,y)=x-iy$ и $\bar\partial f \z= \tfrac12(1+1)=1$.
Заметим, что $\bar\partial f  = \frac{\partial f}{\partial \bar z}$.
\end{ex*}

Введём пару полезных геометрических характеристик отображения $f\: D^2 \to \CC^n$:
\rindex{симплектическая площадь}\emph{симплектическая площадь} $f$ задаваемую как
\[\omega(f) =\int_{D^2}f^\ast\omega\]
и \rindex{евклидова площадь}\emph{евклидову площадь} $f$ определяемую как
\[\Area(f)
=\int_{D^2}
\sqrt{
\left\langle\frac{\partial f}{\partial x},\frac{\partial f}{\partial x}\right\rangle
\left\langle \frac{\partial f}{\partial y},\frac{\partial f}{\partial y}\right\rangle
-
\left\langle\frac{\partial f}{\partial x},\frac{\partial f}{\partial y}\right\rangle^2
}
dxdy.
\]

\begin{thm}{Предложение}\label{4.1.A}
\begin{enumerate}[i)]
\item\label{4.1.A.i} $\displaystyle{\Area(f)\le 2\int_{D^2}|\bar\partial f|^2dxdy+\omega(f)}$ 
\item\label{4.1.A.ii} $\displaystyle{\Area(f)\ge |\omega(f)|}$
\end{enumerate}
\end{thm}

\parit{Доказательство.}
Для $\xi, \eta \in \CC^n$ справедливо неравенство\?{}{+$\cdot$}
\[
\sqrt{|\xi|^2 | \eta |^2 - \langle\xi, \eta\rangle^2}
\le
|\xi|{\cdot}|\eta| 
\le
\tfrac12(|\xi|^2 +|\eta|^2).
\]
Но
\[\tfrac12|\xi + i\eta|^2 + \omega (\xi, \eta)
=
\tfrac12(|\xi|^2 + |\eta|^2) + \langle\xi, i\eta\rangle + \langle\xi, -i\eta\rangle
=
\tfrac12(| \xi |^2 + | \eta |^2 ),\]
и поэтому 
\[\sqrt{| \xi |^2 | \eta |^2 - \langle\xi, \eta\rangle^2}
\le
\tfrac12| \xi + i\eta |^2 + \omega (\xi, \eta).
\]

Интегрируя полученное поточечное неравенство, получаем 
\ref{4.1.A}.\ref{4.1.A.i}.

Чтобы доказать \ref{4.1.A}.\ref{4.1.A.ii}, нам нужно снова проверить поточечное неравенство.
Предположим, что $\eta \ne 0$, и заметим, что $\langle\eta, i\eta\rangle = 0$.
Проецируя $\xi$ на $\eta$ и $i\eta$, получаем 
\[\left\langle\xi, \frac{\eta}{|\eta|}\right\rangle^2
+
\left\langle\xi, \frac{i\eta}{|i\eta|} \right\rangle^2 \le | \xi |^2.\]
Поскольку $| \eta | = | i\eta |$, это неравенство читается как
\[\langle\xi, \eta\rangle^2 + \omega (\xi, \eta)^2 \le | \xi |^2 | \eta |^2,\]
следовательно, 
\[
|\omega(\xi,\eta)|
\le \sqrt{ | \xi |^2 | \eta |^2 - \langle\xi, \eta\rangle^2}.
\]
\qeds

\section{Краевая задача}\label{sec:4.2}

Пусть $L \subset \CC^n$ — замкнутое лагранжево подмногообразие и $g\: D^2 \z\times \CC^n \to \CC^n$ — гладкое отображение, ограниченное вместе со всеми своими производными.
Выберем класс $\alpha \in H_2 (\CC^n, L)$ и рассмотрим следующую краевую задачу.

\emph{Найти гладкое отображение $f\: (D^2, \partial D^2) \to (\CC^n, L)$ такое, что} 
\[
\begin{cases}
\bar\partial f(z) = g (z, f (z))
\\
[f] = \alpha 
\end{cases}
\eqno{(P \big(\alpha, g)\big)}
\]


\begin{ex*}{Пример}
Если $g = 0$ и $\alpha = 0$, то пространство решений $P (0, 0)$ состоит из постоянных отображений $f (z) \equiv w$ при $w \in L$.
Чтобы в этом убедиться, заметим, что $\omega (f) = 0$.
В самом деле, поскольку $\alpha = 0$ и $L$ лагранжево, кривая $f (\partial D^2)$ ограничивает 2-цепь в $L$ с нулевой симплектической площадью.
Эта цепь вместе с $f (D^2)$ образует замкнутую поверхность в $\CC^n$.
В силу точности $\omega$, симплектическая площадь этой поверхности равна нулю.
Значит $\omega (f) = 0$.
Далее, поскольку $g = 0$, получаем $\bar\partial f=0$.
Итак, первая часть \ref{4.1.A} влечёт, что $\Area (f) = 0$ и, следовательно, $\tfrac{\partial f}{\partial x}$ и $\tfrac{\partial f}{\partial y}$ параллельны.
С другой стороны, $\tfrac{\partial f}{\partial x}=-i\tfrac{\partial f}{\partial y}$,
следовательно $\tfrac{\partial f}{\partial x}\perp\tfrac{\partial f}{\partial y}$.
Отсюда $\tfrac{\partial f}{\partial x}=\tfrac{\partial f}{\partial y}=0$.
Итак, $f$ является постоянным отображением.
Учтя граничные условия, получаем, что образ $f$ лежит в $L$.
\end{ex*}

Предположим теперь, что у нас есть последовательность функций
$\{g_n\}$, которая $C^\infty$-сходится к некоторой функции $g$. 
Пусть $f_n$ — решения соответствующих задач $P(\alpha, g_n)$.
Знаменитая \rindex{Громов}\rindex{теорема о компактности}\emph{теорема Громова о компактности} (см. \cite{G1,AL})
утверждает, что либо $\{f_n\}$ содержит подпоследовательность,
сходящуюся к решению $P (\alpha, g)$, либо происходит выдувание.
Чтобы объяснить, что такое выдувание, мы вводим понятие составного решения. 

\begin{ex*}{Определение}
Рассмотрим следующие данные:
\begin{itemize}
\item Разложение $\alpha = \alpha' + \beta_1 +\dots + \beta_k$, где $\beta_j \ne 0$, $j = 1,\dots,k$.
\item Решение $f$ уравнения $P (\alpha', g)$.
\item Решения $h_j$ уравнения  $P (\beta_j, 0)$, это так называемые
  псевдоголоморфные диски.\?{}{Сказать, что вообще-то, в данном
    случае, они будут голоморфными, т.к. $g=0$ и комплексная структура
    стандартная?} 
\end{itemize}
Этот объект называется \rindex{составное решение}\emph{составным решением} $P(\alpha,g)$ и 
$f(D^2)\cup h_1(D^2) \cup\dots\cup h_k (D^2)$ называется его образом.
\end{ex*}

Мы говорим, что просходит выдувание если существует подпоследовательность $\{f_n\}$ (которую мы снова обозначим через $\{f_n\}$), которая сходится к составному решению $P(\alpha,g)$.
Нам будет важно одно свойство этой сходимости — непрерывность
евклидовой площади
\[\Area (f_n)
\to
\Area(f)
+\sum_{j=1}^k\Area (h_j).\]
Mы не приводим точное определение сходимости, оно довольно сложное (см. \cite{G1}, \cite{AL}).
Иллюстративный пример будет дан в разделе~\ref{sec:4.4}.

Используя теорему компактности, Громов установил следующий важный результат \cite{G1}.

\begin{thm*}{Принцип продолжения}\rindex{принцип продолжения}
Рассмотрим семейство «общего положения» $g_s (z, w)$, $s \in [0;1]$ с
$g_0 = 0$. 
Тогда либо $P (0, g_s)$ имеет решение при всех $s$, либо при некотором $s_\infty \le 1$ происходит выдувание, т. е. существует
подпоследовательность  $s_j \to s_\infty$ такая, что
последовательность решений $P(0,g_{s_j})$ сходится к составному решению $P
(0, g_{s_\infty})$. 
\end{thm*}

Понятие «общего положения» следует толковать следующим образом.
Пространство всех семейств $g_s$ можно наделить подходящей структурой
банахова многообразия. 
Семейства общего положения образуют плотное G-дельта множество (то есть счётное пересечение открытых и плотных подмножеств) в этом пространстве.
В частности, каждое семейство $g_s$ переходит в общее положение после
сколь угодно малого возмущения. 
Мы отсылаем к \cite{G1}, \cite{AL} за дополнительной информацией.



\section{Приложение класса Лиувилля}

Мы приведём доказательство теоремы \ref{3.2.B} данное \rindex{Сикорав}Сикоравым \cite{S1}.
Предположим, что $L \subset B^2 (r) \times \CC^{n-1}$ — замкнутое
лагранжево подмногообразие. 
Возьмём $g (z, w) = (\sigma, 0 ,\dots, 0) \in \CC^n$ при некотором $\sigma \in \CC$.

\begin{thm}{Лемма}\label{4.3.A}
Если $| \sigma | > r$, то $P (0, g)$ не имеет решений.
\end{thm}


\parit{Доказательство.}
Предположим, что $f$ — решение.
Обозначим через $\phi$ его первую (комплексную) координату.
Тогда 
\[\frac{\partial\phi}{\partial x}+i\frac{\partial\phi}{\partial y} = 2\sigma.\]
Поскольку $L \subset B^2 (r) \times \CC^{n-1}$, получаем $\bigl|\phi|_{\partial D^2}\bigr|\le r$.
Далее\?{}{добавил скобки в интеграл} 
\begin{align*}
2\pi\sigma &= \int_{D^2}\left(\frac{\partial\phi}{\partial x}+i\frac{\partial\phi}{\partial y}\right)dxdy =
\\
&=\int_{D^2} d (\phi dy - i\phi dx)  = 
\\
&=\int_{S^1}\phi dy - i\phi dx.
\end{align*}
Далее $x + iy \z= e^{2\pi i t}$ и $dx + idy \z= 2\pi i e^{2\pi it} dt$, поэтому $dy - idx \z= 2\pi e^{2\pi i t} dt$.
Следовательно,
\[2\pi | \sigma | 
= 
2\pi\left|\int_0^1  e^{\pi i t} \phi(e^{2\pi i t})\,dt\right|
\le
2\pi r \]
и, следовательно, $| \sigma | \le r$.
\qeds


Возьмём теперь любое $\sigma$ такое, что $|\sigma|>r$ и применим принцип продолжения к семейству $g_s = (s\sigma, 0 ,\dots, 0)$, $s \in [0;1]$.
Предыдущая лемма говорит нам, что не существует решения при $s = 1$, поэтому для немного возмущённого $g_s$ происходит выдувание.
Для простоты будем считать, что выдувание происходит на семействе $g_s$.
В общем случае рассуждения остаются без изменений (надо только делать оценки с точностью до $\epsilon$), проверка предоставляются читателю.

Итак, у нас есть последовательность $s_n \to s_\infty \le 1$ и разложение $0 = \alpha + \beta_1 +\dots+ \beta_k$, $\beta_j \ne 0$.
Пусть $f_n$ — решения $P (0, g_{s_n})$, а $f_\infty$ — решение $P
(\alpha, g_{s_\infty})$ с голоморфными дисками $h_1,\dots,h_k$ такими, что $[h_j] \z= \beta_j$ и
\[\Area (f_n)
\to 
\Area (f_\infty) + \sum_{j=1}^k\Area (h_j).\]
Применяя обе части \ref{4.1.A} и используя то, что диски $h_j$
голоморфны, получаем
$\Area (h_j) = \omega (h_j) \ge \gamma (L)$.
Это неравенство следует из того, что $[h_j] = \beta_j \ne 0$.
Из \ref{4.1.A}.\ref{4.1.A.ii} мы заключаем, что 
\[\Area (f_\infty)
\ge
|\omega(f_\infty)|
=
\left|\sum\omega (h_j)\right|
\ge
\gamma(L).\]
Таким образом, $\Area (f_\infty) + \sum \Area (h_j) \ge 2\gamma (L)$.
С другой стороны \ref{4.1.A}.\ref{4.1.A.i} влечёт, что 
\[\Area(f_n)
\le
2\pi s^2_n|\sigma|^2
\le
2\pi|\sigma|^2.
\]
Здесь мы пользуемся тем, что $\omega (f_n) = 0$ (поскольку $[f_n] = 0$) и $\bar\partial f_n=g_{s_n}$.
Вместе эти два неравенства дают $2\pi | \sigma | \ge 2\gamma (L)$.
Это верно при всех $\sigma$ с $| \sigma | > r$, поэтому $\pi r^2 \ge \gamma (L)$, что доказывает теорему.
\qeds

\parit{Доказательство \ref{3.2.A}.}
Рассмотрим замкнутое лагранжево подмногообразие $L \subset B^2 (r) \times \CC^{n-1}$.
Согласно \ref{4.3.A}, задача
\[
\begin{cases}
\quad\bar\partial f(z)=(s\sigma,0,\dots,0),&|\sigma|>r
\\
\quad[f]=0
\end{cases}
\]
не имеет решения при $s = 1$.
По принципу продолжения произошло выдувание.
Это означает, что существует ненулевой класс $\beta_1$, который представлен голоморфным диском $h_1$.
Поскольку $h_1 \z\ne \const$, получаем $\omega (h_1)> 0$.
Так мы нашли диск в $\CC^n$, натянутый на $h_1 (\partial D^2)$, который имеет ненулевую симплектическую площадь.
Отсюда заключаем, что $\lambda_L \ne 0$.
\qeds

\section{Пример}\label{sec:4.4}

В этом разделе мы разберём конкретный пример, иллюстрирующий процессы в предыдущем доказательстве.
Пусть $L = \partial D^2 \subset \CC$, и пусть $\sigma = 1$.
Требуется найти все отображения $f\: D^2 \to \CC$ такие, что $f
(\partial D^2) \subset \partial D^2$ и 
\begin{equation}
\begin{cases}
\bar\partial f(z,\bar z)=s,
\\
[f|_{\partial D}]=0
\end{cases}
\label{eq:4.4.A}
\end{equation}
Так как $\tfrac{\partial f}{\partial \bar z} = s$ получаем, что $f (z,
\bar z) = s\bar z + u (z)$ для некоторой голоморфной функции $u$ на
$D^2$. 
Мы утверждаем, что $s+ zu (z)$ голоморфная функция, отображающая
$\partial D^2$ в $\partial D^2$ и что $(s+ zu (z))|_{\partial D^2}$
имеет степень 1. 
Действительно, 
\[zf (z,\bar z) = s | z |^2 + zu (z),\]
так что\?{}{добавил $\cdot$}
\[|z|{\cdot}| f (z, \bar z) | = \left| s \left| z \right|^{2} + zu (z)\right|.\]
При $|z|=1$ это выражение можно переписать как
$|f(z,\bar z)|=|s\z+zu(z)|$.
Поскольку $|f(z,\bar z)|=1$ получаем, что $s+zu(z)$ — голоморфная функция, перводящая $\partial D^2$ в $\partial D^2$.
Заметим, что $\deg f = 0$ и $\deg z = 1$, так что $\deg zf= 1 $ и, следовательно, $\deg (s + zu (z)) = 1$.
Все такие голоморфные функции описывают изометрию гиперболической метрики в круге.
Их можно записать как
\[e^{i\theta}\frac{1 - \bar\alpha z}{z-\alpha}\]
при $\theta \in \RR$ и $| \alpha | > 1$.
Таким образом, $s + zu (z) = e^{i\theta}\frac{1 - \bar\alpha z}{z-\alpha}$ и 
\[zu (z)
=
\frac{e^{i\theta} + \alpha s - z (s + e^{i\theta} \bar\alpha)}{z-\alpha}.\]
Поскольку $u$ голоморфна, у неё нет полюсов, так что $e^{i\theta} + \alpha s = 0$ и, следовательно, $\alpha =-\frac{e^{i\theta}}{s}$.
Теперь $1 <| \alpha | = | \tfrac1s |$, это влечёт, что $s<1$ и что у уравнения \ref{eq:4.4.A} нет решений при $s \ge 1$.
Значит, при $s = 1$ происходит выдувание.
Для простоты положим $\theta = 0$.
Тогда 
\[u(z)
=
-\frac{s-\frac1s}{z+\frac1s}
=
\frac{1-s^2}{sz+1}\]
и $f_s(z,\bar z)=s\bar z+\frac{1-s^2}{sz+1}$.
Для любого $z \ne -1$, $f_s (z, \bar z) \to z$ при $s\to1$, 
более того, эта сходимость равномерна вне произвольной окрестности $-1$.
Рассмотрим графики $f_s$ в \?{$D^2 \times D^2 \subset \CC \times
  \CC$}{Должно быть $D^2 \times \CC \subset \CC \times \CC$}.
Положим $w \z= f_s (z, \bar z)$, так что 
\[(w - s\bar z) (sz + 1) = 1 - s^2.\]
При $s \to 1$ это уравнение переходит в $(w - z) (z + 1) = 0$ его график приближается к объединению двух кривых \?{}{здесь не и, а или — обозначается квадратной скобкой — вроде по-русски так пишется.}
\[
\left[
\begin{aligned}
\quad w&=\bar z,
\\
\quad z&=-1.
\end{aligned}
\right.
\]

Здесь $w = z$ является графиком $f_\infty$ и $\{z = -1\}$ соответствует голоморфному диску с краем на $\{-1\} \times L$.
Проецируя на $w$-координату, \?{получаем выдувание}{не по-русски?}. %???
Действительно, $f_\infty (z) = z$ является решением $P (-a, 1)$, где $a = [S^1]$ и голоморфный диск $h (z) = z$ является решением $P (a, 0)$.

\begin{figure}[ht!]
\vskip-0mm
\centering
\includegraphics{mppics/pic-4}
\caption{}\label{pic-4}
\vskip0mm
\end{figure}

Увидеть выдувание можно посмотрев на вещественную часть уравнений.
Если рассмотреть графики соответствующих функций
$f_s(x)\z=sx+\frac{1-s^2}{sx+1}$ при $x \in [-1;1]$, 
то получим картинку на рисунке \ref{pic-4}. 
Графики $f_s$ сходятся к объединению двух кривых, график вещественной
части $f_\infty$ и отрезок $I = [-1;1]$, который является
вещественной частью голоморфного диска $\{-1\} \times D^2$.  
