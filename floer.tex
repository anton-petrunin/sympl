\chapter[Гомологии Флоера]{В гостях у гомологий Флоера}\label{13}

В этой главе мы дадим набросок доказательства теоремы
\ref{12.6.F}, в которой утверждается, что любая однопараметрическая
подгруппа группы $\Ham(M, \Omega)$, порождённая гамильтонианом в общем
положении, является локально минимальной, если $π_{2}(M) = 0$.
Наш подход основан на теории \rindex{гомологии Флоера}гомологий Флоера.
Изложение не является
ни полным, ни стопроцентно строгим.
Его цель состоит в том, чтобы дать представление об очень сложном и
все ещё развивающемся наборе инструментов, а не
предоставить систематическое введение в теорию.
Мы будем довольно точно следовать двум статьям \rindex{Шварц}М. Шварца \cite{Sch2,
  Sch3}.

\section{У входа}\label{13.1}

Гомологии Флоера — один из мощнейших инструментов современной
симплектической топологии. 
Его создание было мотивировано следующим вопросом:
\textit{Какие бывают инварианты гамильтоновых диффеоморфизмов?}
План для ответа приблизительно следующий.
Мы будем работать на связном замкнутом симплектическом многообразии
$(M,\Omega)$, для простоты предполагая асферичность многообразия:
$π_{2}(M) = 0$.
Роль этого предположения скоро прояснится.
Введём некоторые обозначения.
Обозначим через $\widetilde\Ham(M,\Omega)$ универсальное накрытие
группы гамильтоновых диффеоморфизмов с отмеченной точкой $\1$.
Будем обозначать $\L M$ пространство стягиваемых петель $S^{1}\to
M$.
Через $\L\Ham(M,\Omega)$ обозначим группу стягиваемых петель
$\{h_{t}\}$ гамильтоновых 
диффеоморфизмов, 
начинающихся в $\1$,
\?{и порождённых гамильтонианами из $\H$}{Just to remind that by loop we mean such a loop (as opposed to maps $[0;1] \to \Ham$ sending $0,1$ to $\1$)}.
Чтобы упростить обозначения мы часто будем опускать зависимость от $M$
и $\Omega$ и писать $\Ham$ вместо $\Ham(M, \Omega)$ и т. д. 

Группа $\L\Ham$ канонически действует на $\L M$
следующим образом
\[
T_{h}: \{z(t)\}\mapsto \{h_{t}z(t)\}
\]

  
\parbf{Первое наблюдение:} существует естественное
отображение $\widetilde\Ham\z\to C^{\infty}(\L M)/\L\Ham$.
Опишем его.
Выберем элемент $\phi\in\widetilde\Ham$.
Обозначим
через $\F(\phi)$ множество всех гамильтонианов $F\in\F$,
порождающих $\phi$.

Группа $\L\Ham$ действует транзитивно на $\F(\phi)$. Это
действие определяется следующим образом. Рассмотрим петлю $h\in\L\Ham$ и
гамильтониан $F\in\F(\phi)$. Обозначим через $\{f_{t}\}$
гамильтонов поток $F$.
Тогда $h(F)$ определяется как нормализованный гамильтониан, порождающий поток
$h^{-1}_{t}\circ f_{t}$.
Из формулы \ref{1.4.D} вытекает, что
\[
h(F)(x,t) = -H(h_{t}x,t) + F(h_{t}x,t)
\]
где $H$ это гамильтониан порождающий $\{h_{t}\}$.

Для $F\in\F(\phi)$ определим функцию $A_{F}:\L M\to\RR$,
называющуюся \textit{функционалом симплектического действия}:
\[
A_{F}(z)=\int_{0}^{1}F(z(t),t)\d t - \int_{D}\Omega
\]
где $D$ это диск затягивающий петлю $z$.
Так как $M$ асферично, результат не зависит от выбора $D$.

\begin{ex}{Упражнение}\label{13.1.A}
Докажите, что
\[
(T_{h}^{-1})^{*}A_{F}= A _{h(F)}
\]
для всех $h\in\L\Ham$ и $F\in\F(\phi)$.
\textit{Подсказка:}
Воспользуйтесь тем, что для каждой стягиваемой петли $\{h_{t}\}\in\L\Ham$, порождённой некоторым $H\in\H$, действие тождественно равно нулю на её орбитах, то есть, $A_{H}(\{h_{t}x\}) \z= 0$ для любого $x\in M$.
На самом деле на асферических многообразиях это верно даже для нестягиваемых петель $\{h_{t}\}$.
Этот трудный результат был недавно доказан \rindex{Шварц}Шварцем [Sch3].
\end{ex}

Таким образом мы получили естественное отображение, которое переводит
$\phi\in\widetilde\Ham$ в класс эквивалентности $[A_{F}]\in C^{\infty} (\L M)/\L\Ham$.

Функция с точностью до диффеоморфизма — очень богатый объект.
Например, в конечномерном случае, можно извлечь много информации посмотрев на критические точки и топологию поверхностей уровня.
Мощным инструментом для получения такой информации является теория Морса.
Нам придётся работать с бесконечномерным многообразием — пространством петель $\L M$.
Ключевое наблюдение, сделанное \rindex{Флоер}Флоером, состоит в том, что существует подходящая версия теории Морса, работающая в бесконечномерных пространствах.
Эта версия теории будет описана в следующих разделах.
Теория Морса — Флоера порождает довольно сложную структуру, естественным образом связанную 
с группой $\widetilde\Ham$.
Наша основная задача — выяснить роль
хоферовской нормы гамильтонава сиплектоморфизма $\phi$ в этой структуре.
Как мы увидим, она тесно связана со значениями функционала действия
в так называемых гомологически существенных критических точках. 

Начнём с экскурса в конечномерный случай.%
\footnote{Смотри книжку \rindex{Шварц}\cite{Sch1}.}
Следующее замечание, может помочь читателю развить
правильную интуицию.  Многообразие $M$ естественным образом
отождествляется с подмножеством $\L M$, состоящим из постоянных
петель.
Если функция $F\in\F$ не зависит от времени, то сужение $A_{F}$ на $M$ равно $F$.
Таким образом, обычная теория функций на $M$ «сидит» внутри теории функционалов действия на $\L М$. 



\section[Конечномерный случай]{Гомологии Морса в конечномерном\\ случае}\label{13.2}
\rindex{гомологии Морса}

Пусть $F$ — функция Морса на замкнутом связном $N$-мерном
многообразии $M$. 
Множество критических точек $F$ будем обозначать через $\Crit F$.
Индекс Морса%
\footnote{То есть число отрицательных квадратов в канонической форме $d^{2}_{х}F$} критической точки $x$ будет обозначаться через $i(x)$, а через $\Crit_{m} F$ — множество критических точек с индекса $m$.
Обозначим через $C(F)$ векторное пространство над $\ZZ_{2}$,
порождённое $\Crit F$, а через $C_{m}(F)$ его подпространство,
порождённое критическими точками индекса $m$.

Выберем риманову метрику \textbf{общего положения}%
\footnote{Здесь и даллее понятие «общего положения» следует понимать так же как и в последнем абзаце раздела \ref{sec:4.2}: метрика общего положения это элемент некоторого плотного подмножества пространства всех метрик, которое является счётным пересечением открытых всюду плотных подможеств.}
$r$ на $M$ и рассмотрим отрицательный градиентный поток
\[
\frac{\d u}{\d s} (s) = -\nabla_{r} F(u(s)).
\]
Выберем пару точек $x_{-},x_{+} \in \Crit F$.

\begin{thm}{Факт}\label{13.2.A}
Пространство орбит $u(s)$ градиентного потока, удовлетворяющих условиям $u(s)\z\to x_{-}$ при $s\z\to-\infty$ и $u(s)\z\to x_{+}$ при $s\z\to+\infty$ является гладким многообразием размерности $i(x_{-})-i(x_{+})$.
\end{thm}
  
Заметим, что это пространство допускает естественное свободное
$\RR$-действие. 
В самом деле, если $u(t)$ это решение, то и $u(t+\const)$ — тоже решение.
Таким образом, если $i(x_{-})-i(x_{+}) = 1$, то фактор-пространство
является нуль-мерным многообразием.
На самом деле можно показать, что \textit{оно состоит из конечного числа точек.}
Обозначим через $k_{r}(x_{-},x_{+})\in \ZZ_{2}$ чётность этого числа.
Определим линейный оператор
\[
\partial_{r}: C_{m}(F)\to C_{m-1}(F)
\]
следующим образом. Для каждого $x \in \Crit_{m}(F)$ зададим
\[
\partial_{r}x = \sum_{y\in\Crit_{m-1}(F)}k_{r}(x,y)y
\]

\begin{thm}{Факт}\label{13.2.B}
  Оператор $\partial_{r}$ является дифференциалом: $\partial_{r}^{2}=0$.
  Таким образом, $(C(F),\partial_{r})$ это цепной комплекс. 
\end{thm}

\begin{thm}{Факт}\label{13.2.C}
  Группа $H_{m}(C(F),\partial_{r})$ гомологий размерности $m$ этого
  комплекса изоморфна группе гомологий $H_{m}(M; \ZZ_{2})$ многообразия.
\end{thm}

В частности, хотя эти группы и зависят от дополнительных параметров
$F$ и $r$, все они изоморфны друг-другу.
Замечательным фактом является то, что эти изоморфизмы могут быть
организованы в естественное семейство
следующим образом. 

Рассмотрим пространство пар $(F, r)$, где $F$ — функция, а $r$ —
риманова метрика. 
Выберем две пары $\alpha = (F_{0}, r_{0})$ и
$\beta = (F_{1},r_{1})$ в общем положении. Выберем так же путь общего положения
$(F_{s},r_{s})$, $s\in\RR$ такой, что
$(F_{s}, r_{}) = (F_{0}, r_{0})$ при $s\le0$ и
$(F_{s}, r_{s}) = (F_{1},r_{1})$ при $s\ge1$.
Рассмотрим уравнение
\begin{equation}\label{13.2.D}
  \frac{\d u}{\d s}(s)=-\nabla_{r_{s}}F_{s}(u(s))
\end{equation}

Пусть $x_{-}\in\Crit F_{0}$ и $x_{+}\in\Crit F_{1}$ — две критические точки.
Как и прежде, в общем положении пространство решений $u(s)$,
удовлетворяющих условиям $u(s)\to x_{-}$ при $s\to-\infty$ и $u(s)\to
x_{+}$ при  $s\to+\infty$, это гладкое многообразие размерности
$i(x_{-})-i(x_{+})$.
Далее, если $i(x_{-}) = i(x_{+})$, то существует лишь конечное число
решений.%
\footnote{Существенное различие между уравнением~\ref{13.2.D} и
  градиентным потоком состоит в том, что пространство его решений
  не допускает $\RR$-действия, если семейство $(F_{s},r_{s})$ зависит
  от $s$ нетривиальным образом.}
Обозначим через $b(x_{-},x_{+})\in\ZZ_{2}$ чётность этого числа.
Определим линейный оператор
$I^{\beta,\alpha} : C_{*}(F_{0})\to C_{*}(F_{1})$ формулой
\[
I^{\beta,\alpha}(x) = \sum_{i(y)=i(x)}b(x, y)y.
\]

\begin{thm}{Факт}\label{13.2.E}
  \begin{itemize}
  \item
    Каждый из операторов $I^{\beta,\alpha}$ является цепным
    отображением и индуцирует изоморфизм
    \[
    I^{\beta,\alpha}_{*} :
    H_{*}(C(F_{0}),\partial_{r_{0}}) \to
    H_{*}(C(F_{1}),\partial_{r_{1}})
    \]
  \item
    Oператор $I^{\beta,\alpha}_{*}$ не зависит от выбора пути
    $(F_{s},r_{s})$ общего положения.
  \item
    $I^{\alpha,\alpha}=\1$ и $I^{\gamma,\beta}_{*}\circ
    I^{\beta,\alpha}_{*}=I^{\gamma,\alpha}_{*}$, где $\alpha$, $\beta$ и
    $\gamma$ находятся в общем положении.
  \end{itemize}
\end{thm}
%% \?{}{По-моему всё, что написано на странице 108 оригинала до этого момента,
  %% неверно. Для разных функций Морса есть цепные гомотопические эквивалентности
  %% между комплексами, но не такие.}

Назовём семейство операторов $I^{\beta,\alpha}_{*}$ удовлетворяющее
последнему свойству, естественным семейством. 

\begin{ex}{Определение}\label{13.2.F}
Пусть $(C, \partial)$ — цепной комплекс над полем $\ZZ_{2}$ с заданным
базисом $B = \{e_{1},\dots,e_{k}\}$. 
Элемент $e\in B$ называется \rindex{гомологически
  существенный}\emph{гомологически существенным}, если для 
любого $\partial$-инвариантного подпространства $K\subset
\Span(B\setminus \{e\})$ индуцированное вложением отображение 
\[
H_{*}(K,\partial)\to H_{*}(C,\partial)
\]
\textbf{не является} сюръективным.
\end{ex}

Следующее утверждение играет решающую роль в наших дальнейших рассуждениях.
Пусть $F$ — функция Морса общего положения.
Предположим, что $x_{+}\in M$ является его \?{единственной}{Почему это
  важно?} точкой
абсолютного максимума. 
Для типичной римановой метрики $r$ на $M$ рассмотрим комплекс
$(C(F),r)$ с базисом $\Crit F$. 

\begin{thm}{Предложение}\label{13.2.G}
  Точка $x_{+}\in\Crit F$ гомологически существенна.  
\end{thm}
\?{}{То же самое верно для любой точки локального максимума.}
\parbf{Набросок доказательства:} Поскольку функция $F$
убывает вдоль траекторий потока с отрицательным градиентом,
пространство
$Q \z= \Span(\Crit F \setminus \{x_{+}\})$ является $\partial_{r}$-инвариантным.
Более того, $H_{N}(Q,\partial_{r})$ обращается в нуль, где $N = \dim M$.
Это отражает тот факт, что многообразие $\{F<a\}$ открыто для $a\le
\max F$ и, таким образом, у него нет фундаментального цикла. 
Мы заключаем, что для любого $\partial$-инвариантного подпространства
$Q$ образ его группы гомологий в $H_{N}(C(F),\partial_{r})$ равен
нулю. Так как
\[
H_{N}(C(F),\partial_{r})=H_{N}(M;\ZZ_{2})=\ZZ_{2}
\]
критическая точка $x_{+}$ является гомологически существенной.
\qeds

Это предложение даёт следующее важное свойство коэффициентов $b(x,
y)$, которые считают чётность числа решений
уравнения~\ref{13.2.D}. 
Пусть $F$ — функция Морса общего положения с единственным абсолютным
максимумом $x_{+}$.
Рассмотрим семейство $(F_{s},r_{s})$, $s\in\RR$, как в
уравнении~\ref{13.2.D} такое, что $F_{s}$ равна некоторой функции
Морса $F_{0}$ для всех $s\le0$ и $F_{s} = F$ для всех $s\ge1$.


\begin{thm}{Следствие}\label{13.2.H}
  Существует $x\in\Crit F_{0}$ такой, что $b(x, x_{+})\z\neq0$.  
\end{thm}

\parit{Доказательство.}
Рассмотрим оператор,
\[
I:(C(F_{0}),\partial_{r_{0}})\to (C(F),\partial_{r_{1}})
\]
определённый нашими данными.
Если все $b(x,x_{+})$ равны нулю, то образ оператора $I$ содержится в
$\Span(\Crit F\setminus\{x_{+}\})$. 
Это $\partial_{r_{1}}$-ин\-ва\-ри\-ант\-ный подкомплекс.
Более того, его гомологии совпадают с $H(C(F),\partial_{r_{1}})$,
так как $I_{*}$ это изоморфизм.
Получаем противоречие с тем, что $x_{+}$ гомологически существенен.
\qeds

\section{Гомологии Флоера}\label{13.3}
Существенные аспекты теории, описанной в предыдущем разделе,
обобщаются на бесконечномерный случай.
Заменим многообразие $M$ пространством $\L M$ стягиваемых петель в нём, а функциональное
пространство $C^{\infty}(M)$ пространством симплектических
функционалов действия $A_{F}$, где $F\in\F$.
Начнём с описания критических точек $A_{F}$.
Обозначим через $\Pcal(F)\subset\L M$ множество стягиваемых
$1$-периодических орбит гамильтонова потока, порождённого $F$.

\begin{ex}{Упражнение}\label{13.3.A}
Докажите, что критические точки $A_{F}$ это в точности элементы $\Pcal(F)$.
Более того, если отображение $f_{1}$ в момент времени $t=1$ невырождено в том смысле, что его график трансверсален диагонали в $M\times M$, то каждая критическая точка $A_{F}$ невырождена.
\end{ex}

Обозначим через $\J$ пространство всех почти комплексных структур
на $M$ совместимых с $\Omega$.
Выберем элемент $J\in C^{\infty}(S^{1},\J)$.
Каждый такой $J$ определяет риманову метрику на $\L M$ следующим образом.
Касательное пространство к $\L M$ в петле $z\in\L M$
\?{состоит}{не совсем верно, строго говоря} из векторных полей на $M$ вдоль $z$.
Для двух таких векторных полей, скажем, $\xi(t)$ и $\eta(t)$ определим
их скалярное произведение как 
\[
\int_{S^{1}}\Omega\big(\xi(t),J(t)\eta(t)\big)\d t
\]

Для выбранного $J\in\J$ обозначим через $\nabla_{J}$ градиент
относительно римановой метрики $\Omega(\xi,J\eta)$ на $M$.
\begin{ex}{Упражнение}\label{13.3.B}
  Докажите, что градиент функционала $A_{F}$ по отношению к римановой метрике,
  описанной выше, определяется выражением 
  \[
  \grad A_{F}(u)=J\frac{\d u}{\d t}(t)+\nabla_{J(t)}F(u(t),t).
  \]
\end{ex}
Таким образом, для определения комплекса Морса функционала $A_{F}$
необходимо исследовать решения следующей задачи.
\begin{quote}
  Найти гладкое отображение $u \: \RR(s)\times S^{1}(t)\to M$ такое,
  что
  \begin{equation}\label{13.3.C}
    \frac{\partial u}{\partial s}(s,t) +
    J(t)\frac{\partial u}{\partial t}(s,t) +
    \nabla_{J(t)}F(u(s,t),t) = 0
  \end{equation}
  и удовлетворяющее условиям $u(s,t)\to z_{\pm}$ при $s\to\pm\infty$,
  где $z_{\pm}\z\in\Pcal(F)$.
\end{quote}
Обозначим через $\M_{F,J}(z_{-},z_{+})$ пространство решений
уравнения~\ref{13.3.C}. 
Флоер установил, что это пространство имеет очень красивую структуру,
которую мы сейчас опишем.
Важным моментом является то, что уравнение~\ref{13.3.C} это задача Фредгольма.
И в самом деле, видно, что с точностью до членов нулевого порядка это
хорошо знакомое нам уравнение Коши — Римана. 
Оказывается, что для $F$ и $J$ общего положения пространство
$\M_{F,J}(z_{-},z_{+})$ снабжённое естественной топологией является
гладким многообразием.  
Более того, размерность $d(z_{-}, z_{+})$ этого многообразия не меняется, ни при гомотопиях гамильтонова пути $\{f_{t}\}$, $t\in [0,1]$ с фиксированными конечными точками, ни при возмущении $J$.
Таким образом, это число зависит только от поднятия отображения $f_{1}$ на универсальное накрытие группы $\Ham(M,\Omega)$. 

Мы подошли к одному из самых удивительных моментов всей теории.
Заметим, что конечномерная интуиция подсказывает, что число $d(z_{-},
z_{+})$ это ничто иное как разность индексов Морса критических точек
$z_{-}$ and $z_{+}$ функционала $A_{F}$.
Однако легко увидеть, что эти индексы бесконечны.
Тем не менее их разность имеет смысл!
Общая формула для
$d(z_{-},z_{+})$ довольно сложна — она требует понятия индекса Конли — Цендера, который мы не будем обсуждать в этой книге.
Однако ситуация резко упрощается для случая, когда
$F\in\F(g_{\epsilon})$, где $\{g_{t}\}$ это некоторый гамильтонов
поток, порождённый нормализованной, не зависящей от времени функцией
Морса $G$, а $\epsilon>0$ достаточно мало.
Ниже мы описываем гомологии Флоера для гамильтонианов
$F\in\F(g_{\epsilon})$.

\begin{ex}{Упражнение}\label{13.3.D}
  \begin{itemize}
  \item
    Показать, что при достаточно малом $\epsilon > 0$ каждая
    $\epsilon$-пе\-ри\-оди\-чес\-кая орбита автономного потока $\{g_{t}\}$
    является неподвижной точкой потока. Таким образом, $\Pcal(\epsilon
    G)$ совпадает с множеством $\Crit G$ критических точек функции $G$.
  \item
    Докажите, что для $F\in\F(g_{\epsilon})$ любая 1-периодическая
    орбита потока $\{f_{t}\}$ стягиваема. 
    Более того, отображение
    \[
    j : \Pcal(F)\to\Crit G,
    \quad
    \{z(t)\}\mapsto z(0)
    \]
    устанавливает взаимно однозначное соответсвие между
    1-пе\-ри\-оди\-чес\-кими орбитами потока $\{f_{t}\}$ и критическими
    точками функции $G$.
  \end{itemize}
\end{ex}

\pagebreak%%%убрать

Напомним, что $i(x)$ обозначает индекс Морса точки $x\in\Crit G$.

\begin{thm}[(Формула размерности)]{Факт}\label{13.3.E}
  Для $F\in\F(g_{\epsilon})$ в общем положении и для всех
  $z_{\pm}\in\Pcal(F)$ имеет место следующее равенство:
  \[
  d(z_{-}, z_{+}) = i(z_{-}(0)) - i(z_{+}(0)).
  \]
\end{thm}

Заметим, что пространство решений $\M_{F}(z_{-},z_{+})$ допускает
естественное $\RR$-действие сдвигами $u(s, t)\mapsto u(s+c, t)$, где
$c\in\RR$.
Граничные условия гарантируют, что такое действие будет свободным.
Предположим дополнительно, что $d(z_{-}, z_{+}) = 1$.
Тогда из формулы \?{размерности}{Причём тут формула?} следует, что
фактор-пространство $\M_{F,J}(z_{-},z_{+})$ является нульмерным
многообразием.
Вариант теоремы Громова о компактности решений
уравнения~\ref{13.3.C} утверждает, что это многообразие компактно,
следовательно, состоит из конечного числа точек.
Обозначим через $k_{J}(z_{-},z_{+})\in\ZZ_{2}$ чётность этого числа.

Далее поступим точно так же, как и в конечномерном случае.
Напомним, что критические точки $A_{F}$ отождествляются с $\Crit G$ посредством отображения $j$, см. упражнение~\ref{13.3.D}.
Положим $C_{m}(A_{F}) \z= C_{m}(G)$ и $C(A_{F}) = C(G)$.
Зафиксируем петлю $\{J(t)\}$ и определим линейный оператор
$\partial_{J}:C(A_{F})\to C(A_{F})$ следующим образом.
Для элемента $x\in\Crit_{m} G$ положим
\[
\partial_{J}(x)=\sum_{y\in\Crit_{m-1}G}k_{J}(j^{-1}x,j^{-1}y)y
\]


\begin{thm}{Факт}\label{13.3.F}
  Для $F$ и $J$ в общем положении оператор $\partial_{J}$ является
  дифференциалом: $\partial_{J}^{2}=0$.
  Таким образом $(C_{A},\partial_{J})$ это цепной комплекс.
\end{thm}

\begin{thm}{Пример}\label{13.3.G}
  Рассмотрим уравнение~\ref{13.3.C} для $F = \epsilon G$ и для
  постоянной петли $J(t)\equiv J$.
  Каждое решение $u(s)$ уравнения градиентного потока
  \[
  \frac{\d u}{\d s} (s) = -\epsilon\nabla_{J} G(u(s))
  \]
  так же является решением уравнения~\ref{13.3.C}.
  Довольно сложный аргумент \rindex{Саламон}\cite[Lemma 7.1]{HS} показывает, что при
  достаточно малом $\epsilon>0$ это единственные решения при
  условии, что $G$ и $J$ находятся в общем положении.
  Поэтому $(C(A_{F}),\partial_{J})$ совпадает с обычным комплексом
  Морса функции $F$!
\end{thm}

Следующее рассуждение илюстрирует важный и деликатный
момент в теории Флоера.
Попробуем доказать сформулированное выше утверждение для случая, когда
$d(z_{-}, z_{+})=1$.  
Мы должны показать, что каждое решение уравнения~\ref{13.3.C} не
зависит от времени.
Поскольку $G$ не зависит от времени, пространство решений
$\M_{\epsilon G,J}(z_{-},z_{+})$ допускает естественное $\RR\times
S^{1}$-действие $u(s, t) \mapsto (s\z+c_{1},t+c_{2})$, где
$(c_{1},c_{2})\in\RR\times S^{1}$.
Предположим, что $u(s, t)$ — решение, нетривиально зависящее от $t$.
Тогда действие группы $\RR\times S^{1}$ свободно в окрестности точки $u$,
а значит, размерность $d(z_{-}, z_{+})$ не меньше двух.
Это противоречие доказывает, что $u$ не может зависеть от $t$.
Но к сожалению, в этом рассуждении есть дыра.
А именно, не зависящие от времени функции образуют «очень тонкое»
множество в $\F$, поэтому, никакие рассуждения использующие общее
положение не могут гарантировать, что пространство решений
$\M_{\epsilon G,J}(z_{-}, z_{+})$ является многообразием!
Полное доказательство приводится в~\cite{HS}.

Пусть $F_{0}\in\F(g_{\delta})$ и $F_{1}\in\F(g_{\epsilon})$ две
функции общего положения. 
Рассмотрим семейство $(F_{s}, J_{s})$, $s\in\RR$, где
$F_{s}\in\F$, $J_{s}\in C^{\infty}(S^{1},\J)$, удовлетворяющее
условиям 
$F_{s}\equiv F_{0}$ при $s\le0$ и $F_{s}\equiv F_{1}$ при $s\z\ge1$.
Точно так же, как и в конечномерной случае, мы определим
естественный гомоморфизм
\[
I:(C(A_{F_{0}}),\partial_{J_{0}})\to(C(A_{F_{1}}),\partial_{J_{1}})
\]
Следующая задача аналогична~\ref{13.2.D}.

\begin{quote}
  Найти гладкое отображение $u:\RR(s)\times S^{1}(t)\to M$
  такое что 
  \begin{equation}\label{13.3.H}
    \frac{\partial u}{\partial s}(s,t)+
    J_{s}(t)\frac{\partial u}{\partial t}(s,t) +
    \nabla_{J_{s}(t)}F_{s}(u(s,t),t) = 0
  \end{equation}
  и удовлетворяющее $u(s,t)\to z_{\pm}$ при $s\to\pm\infty$, где
  $z_{-}\in\Pcal(F_{0})$ и $z_{+}\in\Pcal(F_{1})$.
\end{quote}
Анализ этого уравнения аналогичен тому, который мы обсудили выше.
В частности, если $i(z_{-})=i(z_{+})$, то при некоторых предположениях
общего положения пространство решений нульмерно и компактно. 
Таким образом, оно состоит из конечного числа точек.
Обозначим через $b(z_{-},z_{+})\in\ZZ_{2}$ чётность этого числа.
В отличие от уравнения~\ref{13.3.C} решения
уравнения~\ref{13.3.H} не инвариантны относительно сдвигов
$u(s,t)\mapsto u(s+c,t)$, если $(F_{s}, J_{s})$ нетривиально зависит
от переменной $s$.
Определим теперь линейный оператор $I$ следующим образом.
Для $x\in\Crit_{m}G$ положим
\begin{equation}
I(x) = \sum_{y\in\Crit_{m}G}b(j^{-1}x,j^{-1}y)\,y
\label{13.3.I}
\end{equation}
Совершенно аналогично утверждениям в~\ref{13.2.E} можно показать, что
$I$ индуцирует изоморфизм $I_{*}$ в гомологиях, не зависящий от выбора
пути общего положения $(F_{s}, J_{s})$. 
Причём, точно так же как и в разделе~\ref{13.2}, семейство отображений
$I_{*}$, связанных с различным выбором параметров $(F,J)$, можно
организовать в естественное семейство.
В частности, из утверждений~\ref{13.3.G} и~\ref{13.2.C} следует,
что 
\[
H_{*}(C(A_{F}),\partial_{J}) = H_{*}(M;\ZZ_{2}).
\]
для $F$ и $J$ общего положения.

\section{Приложение к геодезическим}\label{sec:13.4}

Теперь у нас есть всё необходимое, чтобы доказать теорему~\ref{12.6.F}.
Пусть $G$ это нормализованная функция Морса на $M$ с
\?{единственной}{В теореме 12.6.Ф никакой единственности нет (хотя
  есть общность), и кажется для доказательства она не нужна.}
точкой абсолютного максимума $x_{+}$ и единственной точкой абсолютного
минимума $x_{-}$. 
Будем писать $\{g_{t}\}$ для её гамильтонова потока.
Возьмём любой гамильтонов путь $\{f_{t}\}$, $t\in[0;1]$ с $f_{0}=\1$,
$f_{1}=g_{\epsilon}$ такой, что $\{f_{t}\}$ гомотопен $\{g_{\epsilon
t}\}$, $t\in[0;1]$ с неподвижными концами.
Пусть $F\in\F$ — его нормализованный гамильтониан.
\begin{thm}{Предложение}\label{13.4.A}
  Имеют место следующие неравенства:
  \begin{align*}
    \int_{0}^{1}\max_{x}F(x,t) &\ge \epsilon\max G\\
    \int_{0}^{1}\min_{x}F(x,t) &\le \epsilon\min G\\
  \end{align*}
\end{thm}

Теорема~\ref{12.6.F} немедленно следует из этого предложения.

Для доказательства~\ref{13.4.A} начнём со следующего вспомогательного утверждения.
Рассмотрим петлю $h_{t} = f_{t} \circ g_{\epsilon t}^{-1}$.
В силу~\ref{13.1.A}, $A_{F} = T_{h}^{*}A_{\epsilon G}$.
Выберем \emph{не зависящую от времени} почти комплексную структуру
$J\in\J$ общего положения на $M$ и рассмотрим петлю $J(t) \z= h_{t*}^{-1}
J_{0}h_{t*}$. 
Обозначим через $r_{0}$ и $r$ римановы метрики на $\L M$, связанные с $J_{0}$ и $J(t)$, соответственно. 
Тогда выполняется $r = T_{h}^{*}r_{0}$.
Таким образом, с помощью $T_{h}$ можно отождествить комплексы
$(C(A_{\epsilon G}), \partial_{J_{0}})$ и $(C(A_{F}), \partial_{J})$.
Кроме того, это отождествление сохраняет базис $\Crit G$ обоих
комплексов поэлементно.

\begin{thm}{Лемма}\label{13.4.B}
  Точка $x_{+}$ гомологически существенна в $(C(A_{F}), \partial_{J})$.  
\end{thm}

\parit{Доказательство}.
Утверждения~\ref{13.3.G} и~\ref{13.2.G} влекут, что
элемент $x_{+}$ гомологически существенен в $(C(A_{\epsilon G}), \partial_{J_{0}})$. 
Результат немедленно следует из приведённого выше отождествления.
\qeds

\parit{Доказательство~\ref{13.4.A}}.
Возьмём $\delta\in(0,\epsilon)$.
Пусть $a(s)$, $s\in\RR$ — неубывающая функция такая, что
$a(s)\equiv0$ при $s\le0$ и $a(s)\equiv1$ при $s\ge1$.
Положим
\[
F_{s}(x, t) = (1 - a(s))\cdot\delta\cdot G(x) + a(s)F(x)
\]

Выберем путь $J_{s}:S^{1}\to\J$ такой, что $J_{s}(t)\equiv J_{0}$
при $s\le0$ и $J_{s}(t)\equiv J(t)$ при $s\ge1$
Рассмотрим гомоморфизм, $I$ определённый уравнением~\ref{13.3.I}.
Положим $z_{+}(t)=f_{t}x_{+}\in\Pcal(F)$.
Пользуясь тем, что точка $x_{+}$ гомологически существенна
(см.~\ref{13.4.B}) и рассуждая в точности как в доказательстве
следтсвия~\ref{13.2.H} получаем, что $b(z_{-}, z+)\z\neq 0$ для
некоторого $z_{-}\in\Pcal(\delta G)$.
Таким образом существует решение $u(s, t)$ задачи~\ref{13.3.H}.
Рассмотрим интеграл энергии
\[
E=
\int_{-\infty}^{+\infty}\d s
\int_{0}^{1}
\Omega\left(\frac{\partial u}{\partial s}(s,t),
J_{s}(t)\frac{\partial u}{\partial s}(s,t)\right)
\,\d t
\]
Используя уравнение~\ref{13.3.H} и тот факт, что
$\sgrad F=J\nabla_{J}F$ для каждого $J\in\J$, мы можем вычислить
\begin{equation}\label{13.4.C}
  \Omega\left(\frac{\partial u}{\partial s},
  J_{s}(t)\frac{\partial u}{\partial s}\right)
  =
  \Omega\left(\frac{\partial u}{\partial s},
         \frac{\partial u}{\partial t}\right)
  -
  \d F_{s}\left(\frac{\partial u}{\partial s}\right)
\end{equation}
Пусть $D_{-}$ и $D_{+}$ — ориентированные диски в $M$,
затягивающие $z_{-}$ и
$z_{+}$, соответствено.  Поскольку $π_{2}(M)=0$, мы получаем, что
\begin{equation}\label{13.4.D}
  \int_{-\infty}^{+\infty}\d s
  \int_{0}^{1}
  \Omega\left(\frac{\partial u}{\partial s},
  \frac{\partial u}{\partial t}\right)
  \,\d t
  =
  \int_{D_{+}}\Omega-\int_{D_{-}}\Omega
\end{equation}
Кроме того,
\begin{equation}\label{13.4.E}
  \d F_{s}\left(\frac{\partial u}{\partial s}\right)
  =
  \frac{\d}{\d s}\big(F_{s}(u(s,t),t)\big) -
  \frac{\partial F_{s}}{\partial s}(u(s,t),t)  
\end{equation}
Интегрируя равенство~\ref{13.4.C} и подставляя~\ref{13.4.D}, \ref{13.4.E}
получаем,
\[
E=A_{\delta G}(z_{-})-A_{F}(z_{+})+Q
\]
где
\[
Q=\int_{-\infty}^{+\infty}\d s
  \int_{0}^{1}\d t\;
  \frac{\d a}{\partial s}(s)\big(F(u(s,t),t)-\delta G(u(s,t))\big),
\]
Ясно что
\begin{align*}
  Q\le
  \int_{0}^{1} \max_{x}\big(F(x,t)-\delta G(x)\big)\,\d t\\
  A_{F}(z_{+})=A_{\epsilon G}(x_{+})=\epsilon\max G
\end{align*}
и $E\ge0$.
Более того, $A_{\delta G}(z_{-})\le \delta\max G$, так как все
замкнутые орбиты потока $\{g_{\delta t}\}$ являются просто
критическими точками функции~$G$.

Следовательно,
\[
\int_{0}^{1}\max_{x}\big(F(x,t)-\delta G(x)\big)\,\d t
\ge
(\epsilon-\delta)\max G
\]
Поскольку это верно для всех  $\delta > 0$, мы получаем, что
\[
\int_{0}^{1}\max_{x} F(x,t)\,\d t
\ge
\epsilon\max G
\]
Это завершает доказательство первого неравенства в~\ref{13.4.A}.
Второе доказывается аналогично.
\qeds

\section{К выходу}\label{13.5}

Для начала просуммируем всё сказанное в этой главе.
Пусть $(M,\Omega)$ — асферическое симплектическое многообразие.

\let\subsectionsave=\subsection
\makeatletter
\renewcommand{\subsection}{%
  \@startsection{subsection}%
  {2}%
  {0pt}%
  {1ex}%
  {0pt}%
  {\it}}
\makeatother
\def\thesubsection{\thesection.\Alph{subsection}}

\begin{ex}{}\label{13.5.A}
Каждой функции $F\in\F$ в общем положении ставится в соответствие
векторное пространство $C(A_{F})$ над полем $\ZZ_{2}$.
Это пространство снабжено выделенным базисом $\Pcal(F)$, состоящим из
всех стягиваемых периодических орбит соответствующего гамильтонова
потока. 
Кроме того, $C(A_{F})$ имеет $\ZZ$-градуировку в терминах индекса
Конли — Цендера. 
Мы явно описали эту градуировку в терминах индекса Морса в простейшем
случае, когда $F$ принадлежит $\F(g_{\epsilon})$.
\end{ex}

\begin{ex}{}\label{13.5.B}
Выбор петли совместимых почти комплексных структур в общем положении
$J\in C^{\infty}(S^{1}, \J)$ определяет дифференциал
$\partial_{J}\:C_{*}(A_{F})\to C_{*-1}(A_{F})$.
Гомологии комплекса $(C(A_{F}),\partial_{J})$ изоморфны
$H_{*}(M;\ZZ_{2})$. 
\end{ex}

\begin{ex}{}\label{13.5.C}
Для заданных $(F_{0},J_{0})$ и $(F_{1},J_{1})$ и соединяющего их пути
$(F_{s},J_{s})$ общего положения, существует естественное (с точностью до цепной гомотопии) отображение
комплексов $I:(C(A_{F_{0}}),\partial_{J_{0}})\to(C(A_{F_{1}}),\partial_{J_{1}})$.
Отображение $I$ индуцирует изоморфизм $I_{*}$ на гомологиях, не
зависящий от выбора пути. 
Изоморфизмы $I_{*}$ образуют естественное
семейство в соответсвии с 
утверждением~\ref{13.2.E}. 
\end{ex}

\begin{ex}{}\label{13.5.D}
Предположим, что $F_{0}$ и $F_{1}$ порождают один и тот же элемент в
$\widetilde\Ham(M,\Omega)$. 
Тогда $F_{0} = h(F_{1})$ для некоторой петли $h\in\L\Ham(M,\Omega)$
(см.~\ref{13.1.A}).
\end{ex}

Для заданного $J_{0}\in C^{\infty}(S^{1},\J)$ положим
$J_{1}(t)=h_{t*}^{-1}J_{0}(t)h_{t*}$. Тогда отображение $T_{h}:\L
M\to\L M$ введёнoe в разделе~\ref{13.1} отождествляет
комплекс $(C(A_{F_{0}}),\Pcal(F_{0}),\partial_{J_{0}})$ с
$(C(A_{F_{1}}),\Pcal(F_{1}),\partial_{J_{1}})$ 
Это отождествление намного сильнее, чем описанное в~\ref{13.5.C}
— оно работает на уровне цепных комплексов, тогда как $I$ индуцируют
изоморфизм только на гомологическом уровне. 

\medskip
Структура описаная в~\ref{13.5.A}--\ref{13.5.D} --- это простейшая
часть так называемой теории гомологий 
Флоера, связанной с симплектическим многообразием. 
Следующее утверждение связывает эту теорию с нормой Хофера.

\begin{ex}{}\label{13.5.E}
Предположим, что для некоторого $J\in C^{\infty}(S^{1},\J)$
стягиваемая периодическая орбита $z\in\Pcal(F)$ является гомологически
существенной в $(C(A_{F}),\Pcal(F),\partial_{J})$. Тогда
\[
A_{F}(z)\le\int_{0}^{1}\max_{x}F(x,t)\,\d t
\]
\end{ex}

Доказательство совершенно аналогично доказательству~\ref{13.4.A}
приведённому в разделе~\ref{sec:13.4}.
Далее, обозначим через $\phi\in\widetilde\Ham(M,\Omega)$ элемент порождённый $F$.
С учётом~\ref{13.5.D} гомологическая
существенность орбиты $z$ для некоторого $J$ является внутренним свойством
неподвижной точки $z(0)$ гамильтонова автоморфизма $\phi$.
Значение $A_{F}(z)$ не зависит от конкретного выбора $F\in\F(\phi)$.
Таким образом, неравенство~\ref{13.5.E} даёт способ оценки хоферовского расстояния от $\1$ до $\phi$ на универсальном накрытии $\widetilde\Ham(M,\Omega)$.
Остаётся разработать механизм, позволяющий решить, какие орбиты
гомологически существенны. 
В~\ref{13.4.B} мы рассмотрели простейшую ситуацию, когда $z$ соответствует абсолютному максимуму малого неавтономного гамильтониана. 
В~\cite{Sch3} \rindex{Шварц}Шварц использовал более изощрённое
рассуждение, позволившее доказать локальную минимальность широкого
класса геодезических для асферических симплектических многообразий.  
На самом деле существует важная дополнительная структура,
канонически связанная с $(C(A_{F}),\Pcal(F))$
, а именно
\textit{каноническая вещественная фильтрация} комплекса $C(A_{F})$.
Для $a\in\RR$ обозначим через $C^{a}$ подпространство $C(A_{F})$,
порождённое теми $z\in\Pcal(F)$, которые удовлетворяют неравенству
$A_{F}(z)\le a$.
Поскольку функционал действия убывает вдоль траекторий его
отрицательного градиентного потока, это подпространство является
$\partial_{J}$-инвариантным. 
Таким образом, можно определить относительные группы гомологий
$H(C^{a}/C^{b},\partial_{J})$.
Оказывается, эти гомологии содержат много интересной информации о $\phi$.
Такая фильтрация впервые систематически рассматривалась \rindex{Витербо}Витербо \cite{V1}.
Дальнейшие результаты были получены Ё.-Г. О~\cite{O4} и Шварцем~\cite{Sch3}.

Замечу, что существуют два естественных направления обобщения
обрисованной выше теории. 
Первый — распространить её на симплектические многообразия с
нетривиальной $π_{2}$. 
Для таких многообразий функционалы действия $A_{F}$ являются
многозначными функциями $\L M\to\RR$. На самом деле их дифференциалы
$\d A_{F}$ являются корректно определёнными замкнутыми 1-формами на
$\L M$.
Гомологии Флоера в этой ситуации устроены как обобщение гомологий
Морса — \rindex{Новиков}Новикова замкнутых 1-форм (см.~\rindex{Хофер}\cite{HS}).
Второе направление состоит в том, чтобы распространить
когомологические операции (такие как произведение в когомологиях) на
гомологии Флоера (см.~\rindex{Пиунихин}\cite{PSS}). 

Наконец, замечу, что гомологии Флоера
представляют собой очень сложную структуру, связанную с
$\widetilde\Ham(M,\Omega)$. 
Интересная задача, которая ещё далека от решения, состоит в том, чтобы
разработать алгебраический язык, пригодный для прозрачного описания
этой структуры. 
Мы ссылаемся на~\rindex{Фукая}\cite{Fu} для знакомства с первыми шагами в
этом направлении.  




