\chapter[Гомологии Флоера]{Знакомство с гомологиями Флоера}\label{sec:13}

В настоящей главе мы дадим набросок доказательства теоремы
\ref{12.6.F}, в которой утверждается, что любая однопараметрическая
подгруппа группы $\Ham(M, \Omega)$, порожденная гамильтонианом в общем
положении, является локально минимальной, если $π_{2}(M) = 0$.
Наш подход основан на теории гомологий Флоера. Изложение не является
ни полным, ни стопроцентно строгим.
Его цель состоит в том, чтобы дать представление об очень сложном и
все еще развивающемся \?{наборе инструментов}{machinery}, а не
предоставить систематическое введение \?{в теорию}.
Мы будем довольно точно следовать двум статьям М.Шварца \cite{Sch2,
  Sch3}.

\section{Начало}\label{sec:13.1}
Гомологии Флоера --- один из мощнейших инструментов современной
симплектической топологии. 
Его создание было мотивировано следующим вопросом: Какие бывают инварианты
гамильтоновых диффеоморфизмов?
Дадим набросок ответа.
Мы будем работать на связном замкнутом симплектическом многообразии
$(M,\Omega)$, для простоты предполагая асферичность многообразия:
$π_{2}(M) = 0$.
Роль этого предположения скоро станет ясной.
Введем некоторые обозначения.
Обозначим через $\widetilde\Ham(М,\Omega)$ универсальное накрытие
группы гамильтоновых диффеоморфизмов с отмеченной точкой $\1$.
Будем обозначать $\Lcal M$ пространство стягиваемых петель $S^{1}\to
M$.
Через $\Lcal\Ham(M,\Omega)$ обозначим группу стягиваемых петель
$\{h_{t}\}$ гамильтонoвых диффеоморфизмов, начинающихся в $\1$,
\?{и порожденных функциями из $\Hcal$}{Зачем это? Они вроде все такие}.
Чтобы упростить обозначения мы часто будем опускать зависимость от $M$
и $\Omega$ и писать просто $\Ham$ для $\Ham(M, \Omega)$ и т. д. 

Группа $\Lcal\Ham$ канонически действует на $\Lcal M$
следующим образом\?{}{Исправил обозначения в математике}
\[
T_{h}: \{z(t)\}\mapsto \{h_{t}z(t)\}
\]

  
\noindent\textbf{первое наблюдение:} существует естественное
отображение $\widetilde\Ham\to C^{\infty}(\Lcal M)/\Lcal\Ham$.
Опишем его. Зафиксируйте элемент $\phi\in\widetilde\Ham$. Обозначим
через $\Fcal(\phi)$ множество всех гамильтонианов $F\in\Fcal$,
порождающих ф.

Группа $\Lcal\Ham$ действует транзитивно на $\Fcal(\phi)$. Это
действие определяется следующим образом. Рассмотрим петлю $h\in\Lcal\Ham$ и
гамильтониан $F\in\Fcal(\phi)$. Обозначим через $\{f_{t}\}$
гамильтонов поток $F$.

Тогда $h(F)$ определяется как нормализованная функция Гамильтона, порождающая поток
$h^{-1}_{t}\circ f_{t}$. Из формулы \ref{1.4.D} вытекает, что
\[
h(F)(x,t) = -H(h_{t}x,t) + F(h_{t}x,t)
\]
\?{где $H$ это гамильтониан порождающий $\{h_{t}\}$.}{Добавил}

Для $F\in\Fcal(\phi)$ определим функцию $A_{F}:\Lcal M\to\Rbb$,
называющуюся {\em \change{}{функционалом} симплектического действия}:
\[
A_{F}(z)=\int_{0}^{1}F(z(t),t)\d t - \int_{D}\Omega
\]
где $D$ это диск затягивающий петлю $z$. Так как $М$ асферично,
определение сформулированное выше не зависит от выбора $D$.

\begin{ex}{Упражнение}\label{13.1.A}
  Докажите, что
  \[
  (T_{h}^{-1})^{*}A_{F}= A _{h(F)}
  \]
  для всех $h\in\Lcal\Ham$ и $F\in\Fcal(\phi)$. {\em Подсказка:}
  Используйте, что для каждой стягиваемой петли
  $\{h_{t}\}\in\Lcal\Ham$, порожденной некоторым $H\in\Hcal$, действие
  тождественно равно нулю на орбитах: $A_{H}(\{h_{t}x\}) = 0$ для всех
  $x\in M$.
  На самом деле на асферических многообразиях это верно даже для
  нестягиваемых петель $\{h_{t}\}$.
  Этот трудный результат был недавно доказан Шварцем [Sch3].
\end{ex}
Это завершает описание естественного отображения, которое отображает
$\phi\in\widetilde\Ham$ в
класс эквивалентности $[A_{F}]\in C^{\infty} (\Lcal M)/\Lcal\Ham$.

Функция с точностью до диффеоморфизма — очень богатый объект. Например, в
конечной размерности, можно извлечь много информации из eö критических
точек  и топологии поверхностей уровня.
Мощным инструментом для получения такой информации является
Теория Морса. В нашей ситуации нам приходится имеем дело с бесконечномерным
многообразием --- пространством петель $\Lcal M$.
Фундаментальное наблюдение, сделанное Флоером, состоит в том
что существует подходящая версия теории Морса, которую можно распространить на
бесконечномерные \?{пространства}{settings}.
Мы опишем эту \change{}{конструкцию} в следующих разделах. Теория Морса-
Флоера порождает довольно сложную структуру, связанную канонически
с группой $\widetilde\Ham$. Наша основная задача — выяснить роль
хоферовской нормы $\phi$ в этой структуре.
Как мы увидим, она тесно связана со значениями функционала действия
в так называемых гомологически существенных критических точках. 

Начнем с экскурса в конечномерную ситуацию.%
\footnote{Смотри книжку \cite{Sch1}.}
Следующее замечание, по-видимому, поможет читателю развить
правильную интуицию.  Многообразие $M$ естественным образом
отождествляется с подмножеством $\Lcal M$, состоящим из постоянных
петель. Когда функция $F\in\Fcal$ не зависит от времени, сужение
$A_{F}$ на $M$ это просто $F$. Таким образом, обычная теория функций на $M$
\?{``сидит''}{Не знаю, как кирилические кавычки печатать} внутри теории функционалов действия на $\Lcal М$.





