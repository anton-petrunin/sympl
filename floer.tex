\chapter[Гомологии Флоера]{Знакомство с гомологиями Флоера}\label{sec:13}

В настоящей главе мы дадим набросок доказательства теоремы
\ref{12.6.F}, в которой утверждается, что любая однопараметрическая
подгруппа группы $\Ham(M, \Omega)$, порожденная гамильтонианом в общем
положении, является локально минимальной, если $π_{2}(M) = 0$.
Наш подход основан на теории гомологий Флоера. Изложение не является
ни полным, ни стопроцентно строгим.
Его цель состоит в том, чтобы дать представление об очень сложном и
все еще развивающемся \?{наборе инструментов}{machinery}, а не
предоставить систематическое введение \?{в теорию}{}.
Мы будем довольно точно следовать двум статьям М.Шварца \cite{Sch2,
  Sch3}.

\section{Начало}\label{sec:13.1}
Гомологии Флоера --- один из мощнейших инструментов современной
симплектической топологии. 
Его создание было мотивировано следующим вопросом: Какие бывают инварианты
гамильтоновых диффеоморфизмов?
Дадим набросок ответа.
Мы будем работать на связном замкнутом симплектическом многообразии
$(M,\Omega)$, для простоты предполагая асферичность многообразия:
$π_{2}(M) = 0$.
Роль этого предположения скоро станет ясной.
Введем некоторые обозначения.
Обозначим через $\widetilde\Ham(M,\Omega)$ универсальное накрытие
группы гамильтоновых диффеоморфизмов с отмеченной точкой $\1$.
Будем обозначать $\Lcal M$ пространство стягиваемых петель $S^{1}\to
M$.
Через $\Lcal\Ham(M,\Omega)$ обозначим группу стягиваемых петель
$\{h_{t}\}$ гамильтонoвых диффеоморфизмов, начинающихся в $\1$,
\?{и порожденных функциями из $\Hcal$}{Зачем это? Они вроде все такие}.
Чтобы упростить обозначения мы часто будем опускать зависимость от $M$
и $\Omega$ и писать просто $\Ham$ для $\Ham(M, \Omega)$ и т. д. 

Группа $\Lcal\Ham$ канонически действует на $\Lcal M$
следующим образом\?{}{Исправил обозначения в математике}
\[
T_{h}: \{z(t)\}\mapsto \{h_{t}z(t)\}
\]

  
\noindent\textbf{первое наблюдение:} существует естественное
отображение $\widetilde\Ham\to C^{\infty}(\Lcal M)/\Lcal\Ham$.
Опишем его. Зафиксируем элемент $\phi\in\widetilde\Ham$. Обозначим
через $\Fcal(\phi)$ множество всех гамильтонианов $F\in\Fcal$,
порождающих $\phi$.

Группа $\Lcal\Ham$ действует транзитивно на $\Fcal(\phi)$. Это
действие определяется следующим образом. Рассмотрим петлю $h\in\Lcal\Ham$ и
гамильтониан $F\in\Fcal(\phi)$. Обозначим через $\{f_{t}\}$
гамильтонов поток $F$.

Тогда $h(F)$ определяется как нормализованная функция Гамильтона, порождающая поток
$h^{-1}_{t}\circ f_{t}$. Из формулы \ref{1.4.D} вытекает, что
\[
h(F)(x,t) = -H(h_{t}x,t) + F(h_{t}x,t)
\]
\?{где $H$ это гамильтониан порождающий $\{h_{t}\}$.}{Добавил}

Для $F\in\Fcal(\phi)$ определим функцию $A_{F}:\Lcal M\to\Rbb$,
называющуюся {\em \change{}{функционалом} симплектического действия}:
\[
A_{F}(z)=\int_{0}^{1}F(z(t),t)\d t - \int_{D}\Omega
\]
где $D$ это диск затягивающий петлю $z$. Так как $M$ асферично,
определение сформулированное выше не зависит от выбора $D$.

\begin{ex}{Упражнение}\label{13.1.A}
  Докажите, что
  \[
  (T_{h}^{-1})^{*}A_{F}= A _{h(F)}
  \]
  для всех $h\in\Lcal\Ham$ и $F\in\Fcal(\phi)$. {\em Подсказка:}
  Используйте, что для каждой стягиваемой петли
  $\{h_{t}\}\in\Lcal\Ham$, порожденной некоторым $H\in\Hcal$, действие
  тождественно равно нулю на орбитах: $A_{H}(\{h_{t}x\}) = 0$ для всех
  $x\in M$.
  На самом деле на асферических многообразиях это верно даже для
  нестягиваемых петель $\{h_{t}\}$.
  Этот трудный результат был недавно доказан Шварцем [Sch3].
\end{ex}
Это завершает описание естественного отображения, которое отображает
$\phi\in\widetilde\Ham$ в
класс эквивалентности $[A_{F}]\in C^{\infty} (\Lcal M)/\Lcal\Ham$.

Функция с точностью до диффеоморфизма — очень богатый объект. Например, в
конечной размерности, можно извлечь много информации из eö критических
точек  и топологии поверхностей уровня.
Мощным инструментом для получения такой информации является
Теория Морса. В нашей ситуации нам приходится иметь дело с бесконечномерным
многообразием --- пространством петель $\Lcal M$.
Фундаментальное наблюдение, сделанное Флоером, состоит в том,
что существует подходящая версия теории Морса, которую можно распространить на
бесконечномерные \?{пространства}{settings}.
Мы опишем эту \change{}{конструкцию} в следующих разделах. Теория
Морса-Флоера порождает довольно сложную структуру, естественным образом связанную 
с группой $\widetilde\Ham$. Наша основная задача --- выяснить роль
хоферовской нормы $\phi$ в этой структуре.
Как мы увидим, она тесно связана со значениями функционала действия
в так называемых гомологически существенных критических точках. 

Начнем с экскурса в конечномерную ситуацию.%
\footnote{Смотри книжку \cite{Sch1}.}
Следующее замечание, по-видимому, поможет читателю развить
правильную интуицию.  Многообразие $M$ естественным образом
отождествляется с подмножеством $\Lcal M$, состоящим из постоянных
петель. Когда функция $F\in\Fcal$ не зависит от времени, сужение
$A_{F}$ на $M$ это просто $F$. Таким образом, обычная теория функций на $M$
\?{``сидит''}{Не знаю, как кирилические кавычки печатать} внутри
теории функционалов действия на $\Lcal М$. 



\section{Гомологии Морса в конечных размерностях}\label{sec:13.2}
Пусть $F$ это функция Морса на замкнутом связном $N$-мерном
многообразии $M$. 
Мы будем писать $\Crit F$ для множества критических точек функции $F$.
Обозначим через $i(x)$ индекс Морса%
\footnote{То есть число отрицательных квадратов в канонической форме
  $d^{2}_{х}F$}
критической точки $x$, а через $\Crit_{m} F$ множество критических
точек с индексом Морса равным $m$.
Обозначим через $C(F)$ векторное пространство над $\Zbb_{2}$,
порожденное $\Crit F$, а через $C_{m}(F)$ его подпространство,
порожденное критическими точками индекса $m$.
Возьмем риманову метрику общего положения%
\footnote{Здесь и даллее понятие ``общего положения'' должно
  трактоваться так же как и в последнем абзаце пункта \label{sec:4.2}:
  метрика общего положения это элемент некоторого плотного подмножества 
  пространства всех метрик, которое является счётным пересечением
  открытых всюду плотных подможеств}\?{}{Исправить кавычки в сноске}
$r$ на $M$ и рассмотрим отрицательный градиентный поток
\[
\frac{\d u}{\d s} (s) = -\nabla_{r} F(u(s)).
\]
Выберем пару точек $x_{-},x_{+} \in \Crit F$.

\begin{thm}{Факт}\label{13.2.A}
  Пространство орбит $u(s)$ градиентного потока, удовлетворяющих
  условиям $u(s)\to x_{-}$ при $s\to-\infty$ и $u(s)\to x_{+}$ при
  $s\to+\infty$ это гладкое многообразие размерности $i(x_{-})-i(x_{+})$.
\end{thm}
  
Заметим, что это пространство допускает естественное свободное
$\Rbb$-действие. 
В самом деле, если $u(t)$ это решение, то и $u(t+const)$ --- тоже решение.
Таким образом, когда $i(x_{-})-i(x_{+}) = 1$, фактор-пространство
является нуль-мерным многообразием.
На самом деле можно показать, что оно состоит из конечного числа точек.
Обозначим через $k_{r}(x_{-},x_{+})\in \Zbb_{2}$ чётность этого числа.
Определим линейный оператор
\[
\partial_{r}: C_{m}(F)\to C_{m-1}(F)
\]
следующим образом. Для каждого $x \in \Crit_{m}(F)$ зададим
\[
\partial_{r}x = \sum_{y\in\Crit_{m-1}(F)}k_{r}(x,y)y
\]

\begin{thm}{Факт}\label{13.2.B}
  Оператор $\partial_{r}$ является дифференциалом: $\partial_{r}^{2}=0$.
  Таким образом, $(C(F),\partial_{r})$ это цепной комплекс. 
\end{thm}

\begin{thm}{Факт}\label{13.2.C}
  Группа $H_{m}(C(F),\partial_{r})$ гомологий степени $m$ этого
  комплекса изоморфна группе гомологий $H_{m}(M, \Zbb_{2})$ многообразия.
\end{thm}

В частности, хотя эти группы и зависят от дополнительных параметров
$F$ и $r$, все они взаимно изоморфны.
Замечательным фактом является то, что изоморфизмы могут быть
организованы в каноническое семейство следующим образом.

Рассмотрим пространство пар $(F, r)$, где $F$ --- функция, а $r$ ---
риманова метрика. 
Выберем две пары в общем положении $\alpha = (F_{0}, r_{0})$ и
$\beta = (F_{1},r_{1})$. Выберем так же путь общего положения
$(F_{s},r_{s})$, $s\in\Rbb$ такой, что
$(F_{s}, r_{}) = (F_{0}, r_{0})$ при $s\leq0$ и
$(F_{s}, r_{s}) = (F_{1},r_{1})$ при $s\geq1$.
Рассмотрим уравнение
\begin{equation}\label{eq:13.2.D}
  \frac{\d u}{\d s}(s)=-\nabla_{r_{s}}F_{s}(u(s))
\end{equation}

Пусть $x_{-}\in\Crit F_{0}$ и $x_{+}\in\Crit F_{1}$ --- две критические точки.
Как и прежде, в общем положении пространство решений $u(s)$,
удовлетворяющих условиям $u(s)\to x_{-}$ при $s\to-\infty$ и $u(s)\to
x_{+}$ при  $s\to+\infty$, это гладкое многообразие размерности
$i(x_{-})-i(x_{+})$.
Далее, если $i(x_{-}) = i(x_{+})$, то существует лишь конечное число
решений.%
\footnote{Существенное различие между уравнением~\ref{eq:13.2.D} и
  градиентным потоком состоит в том, что пространство его решений
  не допускает $\Rbb$-действия, если семейство $(F_{s},r_{s})$ зависит
  от $s$ нетривиальным образом.}
Обозначим через $b(x_{-},x_{+})\in\Zbb_{2}$ чётность этого числа.
Определим линейный оператор
$I^{\beta,\alpha} : C_{*}(F_{0})\to C_{*}(F_{1})$ формулой
\[
I^{\beta,\alpha}(x) = \sum_{i(y)=i(x)}b(x, y)y.
\]

\begin{thm}{Факт}\label{13.2.E}
  \begin{itemize}
  \item
    Каждый из операторов $I^{\beta,\alpha}$ является цепным
    отображением и индуцирует изоморфизм
    \[
    I^{\beta,\alpha}_{*} :
    H_{*}(C(F_{0}),\partial_{r_{0}}) \to
    H_{*}(C(F_{1}),\partial_{r_{1}})
    \]
  \item
    Oператор $I^{\beta,\alpha}$ \?{не зависит}{с точностью до цепной
      гомотопии?}
    от выбора пути $(F_{s},r_{s})$ общего положения.
  \item
    $I^{\alpha,\alpha}=\1$ и \?{$I^{\gamma,\beta}\circ
    I^{\beta,\alpha}=I^{\gamma,\alpha}$}{ditto}, где $\alpha$, $\beta$ и
    $\gamma$ находятся в общем положении.
  \end{itemize}
\end{thm}
\?{}{По-моему всё, что написано на странице 108 оригинала до этого момента,
  неверно. Для разных функций Морса есть цепные гомотопические эквивалентности
  между комплексами, но не такие.}

Назовем семейство операторов $I^{\beta,\alpha}$ удовлетворяющее
последнему свойству, каноническим семейством. 

\begin{ex}{Определение}\label{13.2.F}
  Пусть $(C, \partial)$ --- цепной комплекс над полем $\Zbb_{2}$
  с заданным базисом $B = \{e_{1},...,e_{k}\}$.
  Элемент $e\in B$ называется гомологически существенным, если для
  любого $\partial$-инвариантного подпространства
  $K\subset \Span(B\setminus \{e\})$ индуцированное вложением отображение
  \[
  H_{*}(K,\partial)\to H_{*}(C,\partial)
  \]
  \textbf{не является} сюръективным.
\end{ex}

Следующее утверждение играет решающую роль в наших дальнейших рассуждениях.
Пусть $F$ — функция Морса общего положения.
Предположим, что $x_{+}\in M$ является его единственной точкой
абсолютного максимума. 
Для типичной римановой метрики $r$ на $M$ рассмотрим комплекс
$(C(F),r)$ с базисом $\Crit F$. 

\begin{thm}{Предложение}\label{13.2.G}
  Точка $x_{+}\in\Crit F$ гомологически существенна.  
\end{thm}

\noindent\textbf{Набросок доказательства:} Поскольку функция $F$
убывает вдоль траекторий потока с отрицательным градиентом,
пространство
$Q = \Span( Crit F \setminus \{x_{+}\})$ является $\partial_{r}$-инвариантным.
Более того, $H_{N}(Q,\partial_{r})$ обращается в нуль, где $N = \dim M$.
Это отражает тот факт, что многообразие $\{F<a\}$ открыто для $a\leq
\max F$ и, таким образом, не несет фундаментального класса. 
Мы заключаем, что для любого $\partial$-инвариантного подпространства
$Q$ образ его группы гомологий в $H_{N}(C(F),\partial_{r})$ равен
нулю. Так как
\[
H_{N}(C(F),\partial_{r})=H_{N}(M;\Zbb_{2})=\Zbb_{2}
\]
критическая точка $x_{+}$ является гомологически существенной.
\qeds

Это предложение дает следующее важное свойство коэффициентов $b(x,
y)$, которые считают  \change{$(mod 2)$}{чётность количества} решений
уравнения~\ref{eq:13.2.D}. 
Пусть $F$ --- функция Морса общего положения с единственным абсолютным
максимумом $х_{+}$.
Рассмотрим семейство $(F_{s},r_{s})$, $s\in\Rbb$, как в
уравнении~\ref{eq:13.2.D} такое, что $F_{s}$ равна некоторой функции
Морса $F_{0}$ для всех $s\leq0$ и $F_{s} = F$ для всех $s\geq1$.


\begin{thm}{Следствие}\label{13.2.H}
  Существует $х\in\Crit F_{0}$ такой, что $b(x, x+)\neq0$.  
\end{thm}
\?{}{Tут что-то не так. Я посмотрю дальше, используется ли и, если да,
  то как, это утверждение.}

\noindent\textit{Доказательство.}
Рассмотрим оператор,
\[
I:(C(F_{0}),\partial_{r_{0}})\to (C(F),\partial_{r_{1}})
\]
определенный нашими данными.
Если все $b(x,x+)$ равны нулю, то образ $I$ содержится в $\Span(\Crit
F\setminus\{x_{+}\})$. 
Это $\partial_{r_{1}}$-инвариантный подкомплекс.
Более того, его гомологии совпадают с $H(C(F),\partial_{r_{1}})$,
так как $I_{*}$ это изоморфизм.
Получаем противоречие с тем, что $x_{+}$ гомологически существенен.
\qeds
