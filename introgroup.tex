\chapter{Знакомство с группой}

В этой главе мы пройдёмся по некоторым хорошо известным фактам касающихся группы гамильтоновых диффеоморфизмов.

\section[Гамильтоновы диффеоморфизмы]{Гамильтоновы диффеоморфизмы}

Рассмотрим движение частицы массы 1 в $\RR^n(q)$, где $q$ обозначает \?{координату}{а не координаты?} в $\RR^n$, под действием потенциальной силы $\Phi(q, t)  \z= - \tfrac{\partial U}{\partial q} (q, t)$.
Соглачно второму закону Ньютона, $\ddot q= \Phi (q, t)$.
За исключением некоторых редких случаев, это уравнение явно не решить.
Однако можно понять некоторые качественные свойства его решений.

Проделаем небольшую хитрость.
Введём вспомогательную переменную $p = \dot q$, и рассмотрим функцию $F(p,q,t)= \tfrac {p^2} 2 + U (q, t)$.
Функция $F$ представляет собой полную энергию частицы (сумму её кинетической и потенциальной энергий).
В этих обозначениях приведенное выше уравнение Ньютона можно переписать следующим образом:
\[
\begin{cases}
\dot p &= - \tfrac{\partial F}{\partial q} (p, q, t),\\
\dot q &= \tfrac{\partial F}{\partial p} (p, q, t).
\end{cases}
\]
Эта система дифференциальных уравнений первого порядка называется гамильтоновой системой.
Её следует рассматривать как дифференциальное уравнение в $2n$-мерном пространстве $\RR^{2n}$ с координатами $p$ и $q$.
Первый шаг любого качественного исследования состоит в том, что надо отказаться думать о явной форме интересующего вас объекта.
Приняв это к сведению, давайте оставим в стороне выражение для функции $F$ выше и сосредоточимся на свойствах общих гамильтоновых систем, связанных с более-менее произвольными гладкими функциями энергии $F (p, q, t)$.
«Более-менее» означает, что мы накладываем определенные ограничения на поведение $F$ на бесконечности, гарантирующие то, что решения гамильтоновой системы существуют для всех $t\in \RR$.
Итак, выберите такое $F$ и рассмотрите поток $f_t\: \RR^{2n} \to \RR^{2n}$, который переводит любое начальное условие $(p (0), q (0))$ в соответствующее решение $(p (t), q (t))$ в момент времени $t$.
Возникающие таким образом диффеоморфизмы $f_t$ мы будем неформально называть \emph{механическими движениями}.
Суть нашего трюка состоит в том, что эти диффеоморфизмы $2n$-мерного пространства $\RR^{2n}$ обладают следующими замечательными геометрическими свойствами, которые нельзя увидеть в исходном конфигурационном пространстве $\RR^n$.

\begin{thm}[(Теорема Лиувилля)]{Теорема}\label{1.1.A}
Механические движения сохраняют форму объема 
$\Vol=dp_1\wedge dq_1\wedge\dots\wedge d p_n\wedge dq_n$
\end{thm}

\begin{thm}[(Более тонкий вариант \ref{1.1.A})]{Теорема}\label{1.1.B}
Механические движения сохраняют \?{2-форму}{Здесь она $\omega$, но до и после --- $\Omega$ --- это так надо?} 
$\omega = dp_1 \wedge dq_1 +\dots
+ dp_n \wedge dq_n.$
\end{thm}

Обратите внимание, что $\Vol = \tfrac{\omega^n}{n!}$ и значит, \ref{1.1.A} влечёт \ref{1.1.B}.
Далее, при $n = 1$ теоремы \ref{1.1.A} и \ref{1.1.B} равносильны.
Теорема \ref{1.1.B} является простым следствием того факта, что механические движения происходят из гамильтоновой системы.
Мы оставляем доказательство до следующего раздела.

Оба приведенных выше результата восходят к прошлому.
Сохранение объема механическими движениями привлекало большое внимание уже более века назад.
Это послужило основной движущей силой для создания эргодической теории, ныне хорошо известной математической дисциплины, изучающей различные свойства повторяемости преобразований, сохраняющих меру.
Однако значение роли инвариантной 2-формы $\omega$ было замечено сравнительно недавно.
Насколько я знаю, это был В. Я. Арнольд, он впервые прямо указал на это в 1960-х годах.
Попытки понять разницу между механическими движениями и диффеоморфизмами сохраняющими объём породила \emph{симплектическую топологию} --- область математики которая исследует неожиданные явления жесткости, возникающие в теории симплектических многообразий и их морфизмов.

Вот пример такого явления, которое обнаружил Дж. -C. Сикорав в \cite{S1}.
Пусть $B^2(r) \subset \RR^2$ --- евклидов диск радиуса $r$, ограниченный окружностью $S^1(r)$.
Рассмотрим тор 
\[L_R = 
S^1 (R) \times\dots\times S^1(R) \subset \RR^2 (p_1, q_1) \times\dots\times \RR^2 (p_n, q_n) =\RR^{2n} (p, q)\]
и цилиндр $C_r = B^2 (r) \times \RR^{2n-2}$.

\begin{thm}[(Не сжимающее свойство)]{Теорема}\label{1.1.C}
Не существует механического движения, переводящего $L_R$ в $C_r$, при условии, что $R>r$.
\end{thm}

Мы докажем это утверждение в более общем контексте в разделе \ref{3.2.A} ниже.
Отметим, что при $n = 1$ результат очевиден.
Действительно, площадь, ограниченная $S^1 (R)$, больше, чем площадь $B^2 (r)$.
Таким образом, нельзя перевести $S^1(R)$ в $B^2(r)$ преобразованием, сохраняющим площадь.
Однако если $n \ge 2$, то $L_R$ --- подмногообразие коразмерности $n \ge 2$, а $C_r$ имеет бесконечный объем.
Таким образом, нет видимой причины, по которой это утверждение должно быть верным.
Более того, это определенно неправильно в категории сохранения объёма!

\begin{thm*}{Упражнение}
Найти линейное преобразование $\RR^{2n} \to \RR^{2n}$, сохраняющее объём, переводящее $L_R$ в $C_r$ для произвольных положительных $r$ и $R$.
\end{thm*}

В дальнейшем мы будем рассматривать эволюцию механической системы как кривую в группе всех механических движений и изучать эту кривую геометрическими средствами.
Чтобы всё работало, мы вынуждены ограничить класс механических систем, с которыми мы имеем дело.
Например, неограниченные гамильтонианы, такие как $F(p,q,t) = \tfrac {p^2}2 + U(q,t)$ рассматриваемые выше, слишком сложны для нас.
Мы всегда будем предполагать, что гамильтонианы (и, следовательно, соответствующие им механические движения) имеют компактный носитель.
Другими словами, всё движение происходит в ограниченной части пространства, в котором мы живем.

В этой главе мы вводим гамильтонову механику на симплектических многообразиях, которая является естественным обобщением модели, описанной выше.
Соответственно, гамильтонов диффеоморфизм --- это просто механическое движение, порожденное гамильтонианом с компактным носителем. 
