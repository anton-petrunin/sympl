\chapter{Знакомство с группой}

В этой главе мы пройдёмся по некоторым хорошо известным фактам касающихся группы гамильтоновых диффеоморфизмов.

\section[Гамильтоновы диффеоморфизмы]{Гамильтоновы диффеоморфизмы}

Рассмотрим движение частицы массы 1 в $\RR^n(q)$, где $q$ обозначает \?{координату}{а не координаты?} в $\RR^n$, под действием потенциальной силы $\Phi(q, t)  \z= - \tfrac{\partial U}{\partial q} (q, t)$.
Соглачно второму закону Ньютона, $\ddot q= \Phi (q, t)$.
За исключением некоторых редких случаев, это уравнение явно не решить.
Однако можно понять некоторые качественные свойства его решений.

Проделаем небольшую хитрость.
Введём вспомогательную переменную $p = \dot q$, и рассмотрим функцию $F(p,q,t)= \tfrac {p^2} 2 + U (q, t)$.
Функция $F$ представляет собой полную энергию частицы (сумму её кинетической и потенциальной энергий).
В этих обозначениях приведенное выше уравнение Ньютона можно переписать следующим образом:
\[
\begin{cases}
\dot p &= - \tfrac{\partial F}{\partial q} (p, q, t),\\
\dot q &= \tfrac{\partial F}{\partial p} (p, q, t).
\end{cases}
\]
Эта система дифференциальных уравнений первого порядка называется гамильтоновой системой.
Её следует рассматривать как дифференциальное уравнение в $2n$-мерном пространстве $\RR^{2n}$ с координатами $p$ и $q$.
Первый шаг любого качественного исследования состоит в том, что надо отказаться думать о явной форме интересующего вас объекта.
Приняв это к сведению, давайте оставим в стороне выражение для функции $F$ выше и сосредоточимся на свойствах общих гамильтоновых систем, связанных с более-менее произвольными гладкими функциями энергии $F (p, q, t)$.
«Более-менее» означает, что мы накладываем определенные ограничения на поведение $F$ на бесконечности, гарантирующие то, что решения гамильтоновой системы существуют для всех $t\in \RR$.
Итак, выберем такое $F$ и рассмотрим поток $f_t\: \RR^{2n} \to \RR^{2n}$, переводящий любое начальное условие $(p(0),q(0))$ в соответствующее решение $(p (t), q (t))$ в момент времени $t$.
Возникающие таким образом диффеоморфизмы $f_t$ мы будем неформально называть \emph{механическими движениями}.
Сила трюка состоит в том, что эти диффеоморфизмы $2n$-мерного пространства $\RR^{2n}$ обладают следующими замечательными геометрическими свойствами, которые не увидеть в исходном конфигурационном пространстве $\RR^n$.

\begin{thm}[(Теорема Лиувилля)]{Теорема}\label{1.1.A}
Механические движения сохраняют форму объема 
$\Vol=dp_1\wedge dq_1\wedge\dots\wedge d p_n\wedge dq_n$
\end{thm}

\begin{thm}[(Более тонкий вариант \ref{1.1.A})]{Теорема}\label{1.1.B}
Механические движения сохраняют \?{2-форму}{Здесь она $\omega$, но до и после --- $\Omega$ --- это так надо?} 
$\omega = dp_1 \wedge dq_1 +\dots
+ dp_n \wedge dq_n.$
\end{thm}

Обратите внимание, что $\Vol = \tfrac{\omega^n}{n!}$ и значит, \ref{1.1.A} влечёт \ref{1.1.B}.
Далее, заметим, что при $n = 1$ теоремы \ref{1.1.A} и \ref{1.1.B} равносильны.
Теорема \ref{1.1.B} является простым следствием того факта, что механические движения происходят из гамильтоновой системы.
Мы приведём доказательство в следующем разделе.

Оба приведенных выше результата получены давно.
Сохранение объема механическими движениями привлекало большое внимание уже более века назад.
Это послужило основной движущей силой для создания эргодической теории, ныне хорошо известной математической дисциплины, изучающей различные свойства повторяемости преобразований, сохраняющих меру.
Однако значение роли инвариантной 2-формы $\omega$ было замечено сравнительно недавно.
Насколько я знаю, это был В. Я. Арнольд, он впервые прямо указал на это в 1960-х годах.
Попытки понять разницу между механическими движениями и диффеоморфизмами сохраняющими объём породила \emph{симплектическую топологию} --- область математики которая исследует неожиданные явления жесткости, возникающие в теории симплектических многообразий и их морфизмов.

Вот пример такого явления, которое обнаружил Дж. -C. Сикорав \cite{S1}.
Пусть $B^2(r) \subset \RR^2$ --- евклидов диск радиуса $r$, ограниченный окружностью $S^1(r)$.
Рассмотрим тор 
\[L_R = 
S^1 (R) \times\dots\times S^1(R) \subset \RR^2 (p_1, q_1) \times\dots\times \RR^2 (p_n, q_n) =\RR^{2n} (p, q)\]
и цилиндр $C_r = B^2 (r) \times \RR^{2n-2}$.

\begin{thm}[(Не сжимающее свойство)]{Теорема}\label{1.1.C}
Не существует механического движения, переводящего $L_R$ в $C_r$, при условии, что $R>r$.
\end{thm}

Мы докажем это утверждение в более общем виде в разделе \ref{3.2.A} ниже.
Отметим, что при $n = 1$ результат очевиден.
Действительно, площадь, ограниченная $S^1 (R)$, больше, чем площадь $B^2 (r)$.
Таким образом, нельзя перевести $S^1(R)$ в $B^2(r)$ преобразованием, сохраняющим площадь.
Однако если $n \ge 2$, то $L_R$ --- подмногообразие коразмерности $n \ge 2$, а $C_r$ имеет бесконечный объем.
Таким образом, нет видимой причины, по которой это утверждение должно быть верным.
Более того, это определенно неверно в категории диффеоморфизмов сохраняющих объём!

\begin{thm*}{Упражнение}
Найти линейное преобразование $\RR^{2n} \to \RR^{2n}$, сохраняющее объём и переводящее $L_R$ в $C_r$ для произвольных положительных $r$ и $R$.
\end{thm*}

В дальнейшем мы будем рассматривать эволюцию механической системы как кривую в группе всех механических движений и изучать эту кривую геометрическими средствами.
Чтобы всё работало, мы вынуждены ограничить класс механических систем, с которыми мы имеем дело.
Например, неограниченные гамильтонианы, такие как $F(p,q,t) = \tfrac {p^2}2 + U(q,t)$ рассматриваемые выше, будут слишком сложны для нас.
Мы всегда будем предполагать, что гамильтонианы (и, следовательно, соответствующие им механические движения) имеют компактный носитель.
Другими словами, всё движение происходит в ограниченной части нашего пространства.

В этой главе мы вводим гамильтонову механику на симплектических многообразиях, которая является естественным обобщением модели, описанной выше.
Соответственно, гамильтонов диффеоморфизм --- это просто механическое движение, порожденное гамильтонианом с компактным носителем. 

\section{Потоки и пути диффеоморфизмов}

Для начала мы объясним связь между потоками и дифференциальными уравнениями и дадим геометрическую интерпретацию потоков как путей диффеоморфизмов.
Имея в виду будущие приложениями мы предполагаем, что все рассматриваемые нами объекты имеют компактный носитель.
Однако основные построения, описанные ниже, можно расширить на более общий случай.

Рассмотрим гладкое многообразие $M$ без края.
Для диффеоморфизма $\phi\: M \to M$ определим его носитель $\supp (\phi)$ как замыкание всех  $x \in M$ таких, что $\phi(x) \ne x$.
Обозначим через $\Diff^c (M)$ группу всех диффеоморфизмов с компактным носителем.
Пусть $I \subset \RR$ --- интервал.%
\footnote{Мы определяем интервал как связное подмножество $\RR$ с непустой внутренней частью.}
Путь диффеоморфизмов --- это отображение 
\[f\: I \to \Diff^c (M),\quad t \to f_t\]
со следующими свойствами:
\begin{itemize}
\item отображение $M \times I \to M$, переводящее $(x, t)$ в $f_t x$, гладкое;
\item существует компактное подмножество $K$ в $M$, содержащее $\supp f_t$ для всех $t \in I$.
\end{itemize}
Мы часто обозначаем такой путь через $\{f_t\}$.
Отметим, что на замкнутых многообразиях автоматически выполняется второе условие.

Каждый путь диффеоморфизмов порождает семейство векторных полей $\xi_t$, $t \in I$ на $M$ следующим образом: 
\begin{equation}\tfrac{d}{dt} f_t x = \xi_t (f_t x).
\label{eq:1.2.A}
\end{equation}
Отметим, что это семейство гладкое и имеет компактный носитель: $\xi_t (x) = 0$ для всех $x \in M \backslash K$.
Такое семейство называется \emph{зависящим от времени векторным полем с компактным носителем} на $M$.
Приведенное выше соответствие не является инъективным.
В самом деле, каждый путь вида $\{f_t g\}$, где $g$ --- произвольный элемент $\Diff^c (M)$, порождает то же самое зависящее от времени векторное поле $\xi$.
Однако для каждой точки $s \in I$ существует единственный путь $\{f_t\}$, который 
порождает $\xi$ такой, что $f_s$ равно тождественному отображению $\1$.
Этот путь определяется как единственное решение \ref{eq:1.2.A}, которое теперь рассматривается как обыкновенное дифференциальное уравнение с начальным условием $f_s = \1$.
Предположим, что $0 \in I$, и возьмём $s = 0$.
Построенный выше путь $\{f_t\}$ при $f_0 = \1$ называется потоком зависящего от времени векторного поля $\xi$.
Таким образом, потоки --- это просто пути $\{f_t\}$, такие, что $f_0 = \1$.

\section{Математическая модель классической механики}

Роль фазового пространства в классической механике играет симплектическое многообразие $(M^{2n},\Omega)$.
Здесь $M$ --- связное многообразие без границы чётной размерности $2n$, а $\Omega$ --- замкнутая дифференциальная 2-форма на $M$.
Форма $\Omega$ считается невырожденной.
Это означает, что его максимальная степень $\Omega^n$ не обращается в нуль ни в какой точке.
Форма $\Vol =  \tfrac{\Omega^n}{n!}$ называется канонической формой объема на $(M, \Omega)$.
Полезно иметь в виду два элементарных примера симплектических многообразий:
ориентируемую поверхность, наделенную формой площади, и линейное пространство $\RR^{2n} (p_1,\dots, p_n, q_1,\dots, q_n)$ с формой $\omega = \sum^n_{j = 1} dp_j \wedge dq_j$.
Второй пример очень важен с точки зрения классической теоремы Дарбу \cite{MS}.
Она утверждает, что локально каждое симплектическое многообразие выглядит как $(\RR^{2n}, \omega)$.
Другими словами, для каждой точки $M$ можно выбрать такие локальные координаты $(p, q)$, что в этих координатах $\Omega$ записывается как $\sum^n_{j = 1} dp_j \wedge dq_j$.
Мы называем $(p, q)$ \emph{каноническими локальными координатами}.

Пусть $F$ --- гладкая функция на $M$.
Векторное поле $\xi$ на $M$ называется гамильтоновым векторным полем поля $F$, если оно поточечно удовлетворяет линейному алгебраическому уравнению $i_\xi \Omega \z= −dF$.
Элементарное рассуждение из линейной алгебры (основанное на невырожденности $\Omega$) даёт, что $\xi$ всегда существует и единственно \cite{MS}.
Иногда $\xi$ обозначают $\sgrad F$ (косой градиент $F$).

\begin{thm}{Упражнение}\label{1.3.А}
Докажите, что в канонических локальных координатах $(p, q)$ на
$M$ выполняется равенство  $\sgrad F = (-\tfrac{\partial F}{\partial q},\tfrac{\partial F}{\partial p})$
\end{thm}

\begin{thm}{Упражнение}\label{1.3.B}
Пусть  $\phi\: M \to M$ --- симплектический диффеоморфизм
(т. е. $\phi^\ast \Omega = \Omega$).
Докажите, что $\sgrad (F \circ \phi^{-1}) \z= \phi^\ast\sgrad F$ для любой функции $F$ на $M$.
Это свойство, конечно, отражает тот факт, что операция $\sgrad$ определяется бескординатным образом.
\end{thm}


В классической механике энергия определяет эволюцию.
Энергия --- это семейство функций $F_t$ на $M$, которое зависит от дополнительной временного параметра $t$.
Время $t$ определено на некотором интервале $I$.
Эквивалентно, можно рассматривать энергию как одну единственную функцию $F$ на $M \times I$.
Мы будем использовать оба варианта на протяжении всей книги, сохраняя обозначение $F_t (x) = F (x, t)$.
Традиционно $F$ называется гамильтоновой функцией (зависящей от времени).

Эволюция системы описывается уравнением Гамильтона $\dot x \z= \sgrad F_t (x)$.
В локальных канонических координатах $(p, q)$ на $M$ уравнение Гамильтона имеет знакомый вид (сравните с
\ref{1.3.А})
\[
\begin{cases}
\dot p &= - \tfrac{\partial F}{\partial q} (p, q, t),\\
\dot q &= \tfrac{\partial F}{\partial p} (p, q, t).
\end{cases}
\]

Введём линейное функциональное пространство $A = A (M)$, которое будет играть важную роль ниже.
Если $M$ замкнуто, определим $A(M)$ как пространство всех гладких функций на $M$ с нулевым средним относительно канонической формы объема.
Если $M$ открыто, то $A(M)$ состоит из всех гладких функций с компактным носителем.

\begin{thm}{Определение}
Пусть $I \subset \RR$ --- интервал.
Гамильтонова функция $F$ (зависящая от времени) на $M \times I$ называется нормализованной, если $F_t$ принадлежит $A$ для всех $t$.
В случае, когда $M$ открыто, мы дополнительно потребуем, чтобы существовало компактное подмножество $M$, содержащее носители всех функций $F_t$, $t \in I$ одновременно.
\end{thm}

В дальнейшем мы будем рассматривать только нормализованные гамильтонианы.
Приведём пару соображений в пользу этого соглашения.

Прежде всего, на открытых многообразиях необходимо наложить некоторые ограничения на поведение гамильтоновых функций на бесконечности.
В самом деле, в противном случае решения гамильтонова уравнения могут взорваться за конечное время и, таким образом, гамильтонов поток может быть плохо определен.
Важной особенностью приведенного выше определения является то, что зависящее от времени гамильтоново векторное поле $\sgrad F_t$ нормализованной гамильтоновой функции $F$ имеет компактный носитель.
Таким образом, когда $I$ содержит $0$, такое поле определяет поток с компактным носителем, который вписывается в идеологию предыдущего раздела.

Во-вторых, как на открытых, так и на замкнутых многообразиях отображение, переводящее функцию из $A$ в его гамильтоново векторное поле, инъективно.
Действительно, гамильтоново векторное поле определяет соответствующую гамильтонову функцию однозначно с точностью до аддитивной константы.
Понятно, что наша нормализация запрещает добавлять константы!
Это свойство нормированных гамильтоновых функций будет полезно в дальнейшем.
