\chapter{Знакомство с группой}\label{chap:1}

В этой главе мы пройдёмся по некоторым хорошо известным фактам касающихся группы гамильтоновых диффеоморфизмов.

\section[Гамильтоновы диффеоморфизмы]{Гамильтоновы диффеоморфизмы}

Рассмотрим движение частицы массы 1 в $\RR^n(q)$, где $q$ обозначает \?{координату}{Может стоит сказатать, что $q$ это сокращённая запись для $q_1,\dots,q_n$?} в $\RR^n$, под действием потенциальной силы $\Phi(q, t)  \z= - \tfrac{\partial U}{\partial q} (q, t)$.
Соглачно второму закону Ньютона, $\ddot q= \Phi (q, t)$.
За исключением некоторых редких случаев, это уравнение явно не решить.
Однако можно понять некоторые качественные свойства его решений.

Теперь применим небольшую хитрость.
Введём вспомогательную переменную $p = \dot q$, и рассмотрим функцию $F(p,q,t)= \tfrac {p^2} 2 + U (q, t)$.
Функция $F$ представляет собой полную энергию частицы (сумму её кинетической и потенциальной энергий).
В этих обозначениях приведённое выше уравнение Ньютона можно переписать следующим образом:
\[
\begin{cases}
\dot p &= - \tfrac{\partial F}{\partial q} (p, q, t),\\
\dot q &= \tfrac{\partial F}{\partial p} (p, q, t).
\end{cases}
\]
Эта система дифференциальных уравнений первого порядка называется гамильтоновой системой.
Её следует рассматривать как обыкновенное дифференциальное уравнение в $2n$-мерном пространстве $\RR^{2n}$ с координатами $p$ и $q$.
Первый шаг любого исследования качественных свойств состоит в том, что надо отказаться думать о явной форме интересующего нас объекта.
Приняв это к сведению, давайте оставим в стороне выражение для функции
$F$ приведённое выше
и сосредоточимся на свойствах общих гамильтоновых систем, связанных с более-менее произвольными гладкими функциями энергии $F (p, q, t)$.
«Более-менее» означает, что мы накладываем определенные ограничения на поведение $F$ на бесконечности, гарантирующие то, что решения гамильтоновой системы существуют для всех $t\in \RR$.
Итак, выберем такое $F$ и рассмотрим поток $f_t\: \RR^{2n} \to \RR^{2n}$, переводящий любое начальное условие $\big(p(0),q(0)\big))$ в соответствующее решение $\big(p (t), q (t)\big)$ в момент времени $t$.
Возникающие таким образом диффеоморфизмы $f_t$ будут неформально называться \emph{механическими движениями}.
Преимущество \change{трюка}{такого подхода} состоит в том, что эти диффеоморфизмы $2n$-мерного пространства $\RR^{2n}$ обладают следующими замечательными геометрическими свойствами, которые не увидеть в исходном конфигурационном пространстве $\RR^n$.

\begin{thm}[(Теорема Лиувилля)]{Теорема}\label{1.1.A}
Механические движения сохраняют форму объёма 
$\Vol=dp_1\wedge dq_1\wedge\dots\wedge d p_n\wedge dq_n$
\end{thm}

\begin{thm}[(Более тонкий вариант \ref{1.1.A})]{Теорема}\label{1.1.B}
Механические движения сохраняют 2-форму \?{}{Здесь она $\omega$, но до
  и после --- $\Omega$ --- это так надо? Я бы переправил на $\Omega$.}$\omega = dp_1 \wedge dq_1 +\dots
+ dp_n \wedge dq_n.$
\end{thm}

Обратите внимание, что $\Vol = \tfrac{\omega^n}{n!}$ и значит, \ref{1.1.A} влечёт \ref{1.1.B}.
Далее, заметим, что при $n = 1$ теоремы \ref{1.1.A} и \ref{1.1.B} равносильны.
Теорема \ref{1.1.B} является простым следствием того факта, что механические движения происходят из гамильтоновой системы.
Мы приведём доказательство в следующем разделе.

Оба приведённых выше результата получены давно.
Сохранение объёма механическими движениями привлекало большое внимание уже более века назад.
Это послужило основной движущей силой для создания \emph{эргодической теории}, ныне хорошо известной математической дисциплины, изучающей различные свойства \?{повторяемости}{возвращений(?)} преобразований, сохраняющих меру.
Однако значение роли инвариантной 2-формы $\omega$ было замечено сравнительно недавно.
Насколько я знаю, это был В. И. Арнольд, он впервые прямо указал на это в 1960-х годах.
Попытки понять разницу между механическими движениями и диффеоморфизмами сохраняющими объём породила \emph{симплектическую топологию}, которая исследует неожиданные явления жёсткости, возникающие в теории симплектических многообразий и их морфизмов.

Вот пример такого явления, которое обнаружил Дж. -C. Сикорав \cite{S1}.
Пусть $B^2(r) \subset \RR^2$ --- евклидов диск радиуса $r$, ограниченный окружностью $S^1(r)$.
Рассмотрим тор 
\[L_R = 
S^1 (R) \times\dots\times S^1(R) \subset \RR^2 (p_1, q_1) \times\dots\times \RR^2 (p_n, q_n) =\RR^{2n} (p, q)\]
и цилиндр $C_r = B^2 (r) \times \RR^{2n-2}$.

\begin{thm}[(Не сжимающее свойство)]{Теорема}\label{1.1.C}
Не существует механического движения, переводящего $L_R$ в $C_r$, при условии, что $R>r$.
\end{thm}

Мы докажем это утверждение в более общем виде в разделе \ref{3.2.A} ниже.
Отметим, что при $n = 1$ результат очевиден.
Действительно, площадь, ограниченная $S^1 (R)$, больше, чем площадь $B^2 (r)$.
Таким образом, нельзя перевести $S^1(R)$ в $B^2(r)$ преобразованием, сохраняющим площадь.
Однако если $n \ge 2$, то $L_R$ --- подмногообразие коразмерности $n \ge 2$, а $C_r$ имеет бесконечный объем.
Таким образом, нет видимой причины, по которой это утверждение должно быть верным.
Более того, подобное утверждение совершенно неверно в категории диффеоморфизмов сохраняющих объём!


\begin{thm*}{Упражнение}
Найти линейное преобразование $\RR^{2n} \to \RR^{2n}$, сохраняющее объём и переводящее $L_R$ в $C_r$ для произвольных положительных $r$ и $R$.
\end{thm*}

В дальнейшем эволюция механической системы будет рассматриваться как кривая в группе всех механических движений и эта кривая будет изучаться геометрическими средствами.
Чтобы всё работало, нам придётся ограничить класс механических систем.
Например, неограниченные гамильтонианы, такие как $F(p,q,t) = \tfrac {p^2}2 + U(q,t)$ рассматриваемые выше, будут слишком сложны для нас.
Мы всегда будем предполагать, что гамильтонианы (и, следовательно, соответствующие им механические движения) имеют компактный носитель.
Другими словами, всё движение происходит в ограниченной части нашего пространства.

В этой главе вводится гамильтонова механика на симплектических многообразиях, которая является естественным обобщением модели, описанной выше.
Соответственно, гамильтонов диффеоморфизм --- это просто механическое движение, порождённое гамильтонианом с компактным носителем. 

\section{Потоки и пути диффеоморфизмов}

Для начала объясним связь между потоками и дифференциальными уравнениями и дадим геометрическую интерпретацию потоков как путей диффеоморфизмов.
Имея в виду будущие приложениями мы предполагаем, что все рассматриваемые нами объекты имеют компактный носитель.
Однако основные построения, описанные ниже, можно расширить на более общий случай.

Рассмотрим гладкое многообразие $M$ без края.
Для диффеоморфизма $\phi\: M \to M$ определим его носитель $\supp (\phi)$ как замыкание всех  $x \in M$ таких, что $\phi(x) \ne x$.
Обозначим через $\Diff^c (M)$ группу всех диффеоморфизмов с компактным носителем.
Пусть $I \subset \RR$ --- интервал.%
\footnote{Мы определяем интервал как связное подмножество $\RR$ с непустой внутренней частью.}
\emph{Путь} диффеоморфизмов --- это отображение 
\[f\: I \to \Diff^c (M),\quad t \mapsto f_t\]
со следующими свойствами:
\begin{itemize}
\item отображение $M \times I \to M$, переводящее $(x, t)$ в $f_t x$, гладкое;
\item существует компактное подмножество $K$ в $M$, содержащее $\supp f_t$ для всех $t \in I$.
\end{itemize}
Мы часто обозначаем такой путь через $\{f_t\}$.
Отметим, что на замкнутых многообразиях второе условие выполняется автоматически.

Каждый путь диффеоморфизмов порождает семейство векторных полей $\xi_t$, $t \in I$ на $M$ следующим образом: 
\begin{equation}\tfrac{d}{dt} f_t x = \xi_t (f_t x).
\label{eq:1.2.A}
\end{equation}
Отметим, что это семейство гладкое и имеет компактный носитель: $\xi_t (x) = 0$ для всех $x \in M \backslash K$.
Такое семейство называется \emph{зависящим от времени векторным полем с компактным носителем} на $M$.
Приведённое выше соответствие не является инъективным.
В самом деле, каждый путь вида $\{f_t g\}$, где $g$ --- произвольный элемент $\Diff^c (M)$, порождает то же самое зависящее от времени векторное поле $\xi$.
Однако для каждой точки $s \in I$ существует единственный путь $\{f_t\}$, который 
порождает $\xi$ такой, что $f_s$ равно тождественному отображению $\1$.
Этот путь определяется как единственное решение уравнения~(\ref{eq:1.2.A}), которое теперь рассматривается как обыкновенное дифференциальное уравнение с начальным условием $f_s = \1$.
Предположим, что $0 \in I$, и возьмём $s = 0$.
Построенный выше путь $\{f_t\}$ при $f_0 = \1$ называется \emph{потоком} зависящего от времени векторного поля $\xi$.
Таким образом, потоки --- это просто пути $\{f_t\}$, такие, что $f_0 = \1$.

\section{Математическая модель классической механики}

Роль фазового пространства в классической механике играет симплектическое многообразие $(M^{2n},\Omega)$.
Здесь $M$ --- связное многообразие без границы и имеющее чётную размерность $2n$, а $\Omega$ --- замкнутая дифференциальная 2-форма на $M$.
Форма $\Omega$ предпологается невырожденной.
Это означает, что его максимальная степень $\Omega^n$ не обращается в нуль ни в какой точке.
Форма $\Vol =  \tfrac{\Omega^n}{n!}$ называется канонической формой объёма на $(M, \Omega)$.
Полезно иметь в виду два элементарных примера симплектических многообразий:
ориентируемую поверхность, наделённую формой площади, и линейное пространство $\RR^{2n} (p_1,\dots, p_n, q_1,\dots, q_n)$ с формой $\omega = \sum^n_{j = 1} dp_j \wedge dq_j$.
Второй пример очень важен с точки зрения классической теоремы Дарбу \cite{MS}.
Она утверждает, что локально каждое симплектическое многообразие выглядит как $(\RR^{2n}, \omega)$.
Другими словами, для каждой точки $M$ можно выбрать такие локальные координаты $(p, q)$, что в этих координатах $\Omega$ записывается как $\sum^n_{j = 1} dp_j \wedge dq_j$.
Мы называем $(p, q)$ \emph{каноническими локальными координатами}.

Пусть $F$ --- гладкая функция на $M$.
Векторное поле $\xi$ на $M$ называется \emph{гамильтоновым векторным полем} гамильтониана $F$, если оно поточечно удовлетворяет линейному алгебраическому уравнению $i_\xi \Omega \z= -dF$.
Элементарное рассуждение из линейной алгебры (основанное на невырожденности $\Omega$) даёт, что $\xi$ всегда существует и единственно \cite{MS}.
Иногда $\xi$ обозначают $\sgrad F$ (косой градиент $F$).

\begin{thm}{Упражнение}\label{1.3.A}
Докажите, что в канонических локальных координатах $(p, q)$ на
$M$ выполняется равенство  $\sgrad F = (-\tfrac{\partial F}{\partial q},\tfrac{\partial F}{\partial p})$
\end{thm}

\begin{thm}{Упражнение}\label{1.3.B}
Пусть  $\phi\: M \to M$ --- симплектический диффеоморфизм
(т. е. $\phi^\ast \Omega = \Omega$).
Докажите, что $\sgrad (F \circ \phi^{-1}) \z= \phi_\ast \sgrad F$ для любой функции $F$ на $M$.
Это свойство, конечно, отражает тот факт, что операция $\sgrad$ определяется бескординатным образом.
\end{thm}


В классической механике энергия определяет эволюцию системы.
Энергия -- это семейство функций $F_t$ на $M$, которое зависит от
дополнительного временного параметра $t$.
Время $t$ определено на некотором интервале $I$.
Эквивалентно, можно рассматривать энергию как одну единственную функцию $F$ на $M \times I$.
Мы будем использовать оба варианта на протяжении всей книги, сохраняя обозначение $F_t (x) = F (x, t)$.
Традиционно $F$ называется \emph{гамильтонианом (зависящим от времени)}.

Эволюция системы описывается уравнением Гамильтона $\dot x \z= \sgrad F_t (x)$.
В локальных канонических координатах $(p, q)$ на $M$ уравнение
Гамильтона имеет знакомый вид (сравните с \ref{1.3.A})
\[
\begin{cases}
\dot p &= - \tfrac{\partial F}{\partial q} (p, q, t),\\
\dot q &= \tfrac{\partial F}{\partial p} (p, q, t).
\end{cases}
\]

Введём линейное функциональное пространство $\A = \A (M)$, которое будет играть важную роль ниже.
Если $M$ замкнуто, определим $\A(M)$ как пространство всех гладких функций на $M$ с нулевым средним относительно канонической формы объёма.
Если $M$ открыто, то $\A(M)$ состоит из всех гладких функций с компактным носителем.

\begin{thm}{Определение}
Пусть $I \subset \RR$ --- интервал.
Гамильтониан $F$ (зависящий от времени) на $M \times I$ называется \textbf{нормализованным}, если $F_t$ принадлежит $\A$ для всех $t$.
Если $M$ открыто, то мы дополнительно требуем, чтобы существовало компактное подмножество $M$, содержащее носители всех функций $F_t$, $t \in I$ одновременно.
\end{thm}

В дальнейшем будут рассматриваться только нормализованные гамильтонианы.
Приведём пару соображений в пользу этого соглашения.

Прежде всего, на открытых многообразиях необходимо наложить некоторые ограничения на поведение гамильтонианов на бесконечности.
В самом деле, в противном случае решения гамильтонова уравнения могут
убежать на бесконечность за конечное время и, таким образом, гамильтонов поток может быть плохо определен.
Важной особенностью приведенного выше определения является то, что зависящее от времени гамильтоново векторное поле $\sgrad F_t$ нормализованного гамильтониана $F$ имеет компактный носитель.
Таким образом, когда $I$ содержит $0$, такое поле определяет поток с компактным носителем, который вписывается в идеологию предыдущего раздела.

Во-вторых, как на открытых, так и на замкнутых многообразиях отображение, переводящее функцию из $\A$ в его гамильтоново векторное поле, инъективно.
Действительно, гамильтоново векторное поле определяет соответствующий гамильтониан однозначно с точностью до аддитивной константы.
Понятно, что наша нормализация запрещает добавлять константы!
Это свойство нормированных гамильтонианов будет полезно в дальнейшем.

\section{Группа гамильтоновых диф\-фе\-о\-мор\-физ\-мов}\label{1.4}

Пусть $F: M \times I \to R$ --- нормализованный гамильтониан, зависящая от времени.
Предположим, что $I$ содержит ноль.
Рассмотрим поток $\{f_t\}$ зависящего от времени векторного поля $\sgrad F_t$.
Мы будем говорить, что $\{f_t\}$ --- гамильтонов поток, порождённый $F$.
Каждый диффеоморфизм $f_a$, $a \in I$ этого потока называется \emph{гамильтоновым диффеоморфизмом}.
Из определения конечно следует, что гамильтоновы диффеоморфизмы имеют компактный носитель.

\begin{thm}[(репараметризация потоков)]{Упражнение}\label{1.4.A}
Пусть $\{f_t\}, t \z\in [0; a]$ --- гамильтонов поток, порождённый нормированным гамильтонианом $ F (x, t)$.
Докажите, что $\{f_{at}\}$, $t \in [0; 1]$ тоже является гамильтоновым потоком, порождённым $aF (x, at)$.
Следовательно, любой гамильтонов диффеоморфизм на самом деле является отображением некоторого гамильтонова потока с единичным временем.
Более общо, покажите, что для любой гладкой функции $b (t)$ с $b (0) =
0$ поток $\{f_{b (t)}\}$ является гамильтоновым потоком, нормированный
гамильтониан которого равен $\tfrac{\d b}{\d t} (t) F (x, b (t))$.
\end{thm}

Ключевым свойством гамильтоновых диффеоморфизмов является то, что они сохраняют симплектическую форму $\Omega$.
В самом деле, пусть $\xi$ --- гамильтоново векторное поле функции $F$ на $M$.
Все, что нам нужно проверить, --- это равенство нулю производной Ли $L_\xi \Omega$.
Это видно из следующих вычислений: 
\[L_\xi \Omega = i_\xi \d\Omega + \d (i_\xi \Omega) = -\d\d F = 0.\]
Обозначим через $\Ham (M, \Omega)$ множество всех гамильтоновых диффеоморфизмов.

Путь диффеоморфизмов в $\Ham (M, \Omega)$ называется гамильтоновым путём.
Естественно назвать поток со значениями в $\Ham (M, \Omega)$ гамильтоновым потоком.
Однако, немного подумав, мы понимаем, что попали в беду.
Действительно, гамильтоновы потоки уже были определены выше по-другому.
Заранее совсем не ясно, почему векторное поле, соответствующее гамильтонову пути, является гамильтоновым векторным полем!
К счастью, это правда.
Этот чрезвычайно важный факт был установлен Баньягой \cite{B1}.
Вот его точная формулировка.

\begin{thm}{Предложение}\label{1.4.B}
Для каждого гамильтонова пути $\{f_t\}$, $t \z\in I$, существует (зависящий от времени) нормализованный гамильтониан $F\: M \times I \to \RR$ такой, что 
\[\tfrac \d{\d t} f_t x = \sgrad F_t (f_t x)\]
для всех $x \in M$ и $t \in I$.
\end{thm}

Функция $F$ называется нормализованным гамильтонианом
пути $\{f_t\}$.

Обсудим этот результат.
Сначала предположим, что многообразие $M$ удовлетворяет следующему
топологическому условию: его первая \change{когомология}{группа когомологий} де Рама с компактными носителями равна нулю, то есть $H_{\mathrm{comp}}^1 (M, \RR) = 0$.
Чтобы иметь перед собой пример, представьте себе двумерную сферу или линейное пространство.
В этом случае приведённое выше предложение доказывается очень легко.
Обозначим через $\xi_t$ векторное поле, порождённое $f_t$.
Поскольку гамильтоновы диффеоморфизмы сохраняют симплектическую форму $\Omega$, имеем $L_{\xi_t} \Omega = 0$.
Следовательно, $\d i_{\xi_t} \Omega = 0$, поэтому $i_{\xi_t} \Omega$ --- замкнутая форма.
Ввиду нашего топологического условия форма $i_{\xi_t} \Omega$ точна.
Следовательно, существует единственное гладкое семейство функций $F_t (x) \in \A$ такое, что $-dF_t = i_{\xi_t} \Omega$.
Отсюда следует, что $F (x, t)$ --- нормированный гамильтониан $\{f_t\}$.
Это завершает доказательство.
Однако, когда $H^1_{\mathrm{comp}} (M, \RR) \ne 0$, нельзя  заранее знать, точны ли замкнутые формы $i_{\xi_t} \Omega$.
В этом случае следует дополнительно использовать тот факт, что каждое отдельное $f_t$ гамильтоново.
Это требует некоторых новых идей, см. \cite{B1}, \cite{MS}.

\begin{thm}{Замечание}\label{1.4.C}
Пусть $\Symp (M, \Omega)$ --- группа всех диффеоморфизмов $f$ с компактным носителем на $M$, сохраняющих $\Omega$, то есть $f^\ast\Omega = \Omega$.
Такие диффеоморфизмы называются \emph{симплектоморфизмами}.
Обозначим через $\Symp_0 (M, \Omega)$ компоненту линейной связности единицы в $\Symp (M, \Omega)$.
По определению она содержит те симплектоморфизмы $f$, которые можно соединить путем симплектоморфизмов с тождественным отображением.
Рассуждая точно так же, как в доказательстве \ref{1.4.B} выше, мы видим, что такой путь является гамильтоновым, если $H^1_{\mathrm{comp}} (M, \RR) = 0$.
Следовательно, в этом случае
\[\Symp_0 (M, \Omega) = \Ham (M, \Omega).\]
Если $H^1_{\mathrm{comp}} (M, \RR) \ne 0$, то это равенство может не выполняться.
Например, рассмотрим двумерный тор $T^2 = \RR^2 (p, q) / \ZZ^2$, наделённый формой площади $\Omega = dp \wedge dq$.
Сдвиг $(p, q) \to (p + a, q)$, очевидно, лежит в $\Symp_0 (T^2, \Omega)$.
Однако для нецелого $a$ можно показать, что он \change{не является Гамильтоновым}{не гамильтонов}.
С другой стороны, разница между $\Ham$ и $\Symp_0$ не слишком велика, и ее можно описать довольно просто, см. \ref{14.1} ниже.
\end{thm}

Следующее предложение содержит замечательную элементарную формулу, которая будет многократно использоваться ниже.

\begin{thm}[(Гамильтониан произведения)]{Предложение}\label{1.4.D}
Рассмотрим два гамильтонова пути $\{f_t\}$ и $\{g_t\}$.
Пусть $F$ и $G$ --- их нормированные гамильтонианы.
Тогда путь произведения $h_t = f_t g_t$ является гамильтоновым путём, порождённым нормированным гамильтонианом 
\[H(x,t) = F(x,t) + G(f_t^{-1} x, t).\]
\end{thm}

Докажем эту формулу.
Нам дано, что 
\[\tfrac{\d}{\d t} \d f_t x = \sgrad F_t
\quad\text{и}\quad
\tfrac{\d}{\d t}g_t x = \sgrad G_t
\]
Таким образом, 
\[\tfrac{\d}{\d t} (f_t g_t) x = \sgrad F_t + f_{t\ast} \sgrad G_t.\]
Ввиду упражнения \ref{1.3.B} выше, второе слагаемое в правой части равно $\sgrad  (G \circ f_t^{-1})$.
Отсюда 
\[\tfrac{\d}{\d t} h_t x = \sgrad  (F_t + G \circ f_t^{-1}) = \sgrad H_t.\]
Формула доказана.

Теперь можно обосновать название этого раздела.

\begin{thm}{Предложение}
Множество гамильтоновых диффеоморфизмов является группой относительно композиции.
\end{thm}

Действительно, возьмём два гамильтоновых диффеоморфизма $f$ и $g$.
Ввиду \ref{1.4.A}, можно записать $f = f_1$ и $g = g_1$ для некоторых гамильтоновых потоков $\{f_t\}$, $\{g_t\}$, определённых для $t \in [0; 1]$.
Предложение \ref{1.4.D} означает, что путь $\{f_t g_t\}$ является гамильтоновым потоком.
Таким образом, его отображение $f g$ в единичное время является гамильтоновым диффеоморфизмом.
В частности, множество $\Ham (M, \Omega)$ замкнуто относительно композиции диффеоморфизмов.
Осталось проверить, что $f^{-1}$ --- гамильтонов диффеоморфизм.
Это следует из следующего упражнения.

\begin{thm*}{Упражнение} Покажите, что путь $\{f_t^{-1}\}$ является гамильтоновым потоком, порождённым гамильтонианом $-F (f_t x, t)$.
\emph{Подсказка:} продифференцируйте тождество $f_t \circ f_t^{-1} = \1$ по $t$ и рассуждайте, как в доказательстве \ref{1.4.D} выше.
\end{thm*}



\begin{thm}{Замечание}
В дифференциальной геометрии обычно имеют дело с группами преобразований, сохраняющих определённые структуры на многообразии (например, группу $\Symp (M, \Omega)$ всех симплектоморфизмов).
Группа гамильтоновых диффеоморфизмов не имеет такого понятного описания (гамильтоновы диффеоморфизмы не определяются как морфизмы в определённой естественной категории).
Это приводит к очень неожиданным сложностям.
Например, следующий вопрос (известный как гипотеза о потоке, см. главу~\ref{chap:14}) все ещё открыт для большинства симплектических многообразий $M$.
Предположим, что $M$ замкнуто.
Предположим, что некоторая последовательность гамильтоновых диффеоморфизмов $C^\infty$-сходится к симплектическому диффеоморфизму $f$.
Является ли $f$ гамильтоновым?
\end{thm}

Чрезвычайно полезно думать о $\Ham (M, \Omega)$ в терминах теории групп Ли.
Эта точка зрения является фундаментальной для развития геометрической интуиции, необходимой для наших целей.
Давайте разработаем этот язык.
Мы будем рассматривать $\Ham (M, \Omega)$ как подгруппу Ли группы всех диффеоморфизмов $M$.
Таким образом, алгебра Ли%
\footnote{Как векторное пространство алгебра Ли по определению является касательным пространством к
группе в единице.
Касательные пространства к группе во всех остальных точках 
отождествляется с алгеброй Ли с помощью правых сдвигов группы.}
группы $\Ham (M, \Omega)$ --- это просто алгебра всех векторных полей $\xi$ на $M$ вида 
\[\xi(x)=\left.\frac{\d}{\d t}\right|_{t=0}f_tx\]
где $\{f_t\}$ --- гладкий путь в $\Ham (M, \Omega)$ с $f_0 = \1$.
Каждое такое поле гамильтоново.
В самом деле, $\xi = \sgrad F_0 (x),$ где $F (x, t)$ --- (единственный!) нормализованный гамильтониан, порождающий путь, и $F_0 (x) = F (x, 0)$.
Отметим, что $F_0 \in \A$.
Наоборот, для любой функции $F \in \A$ векторное поле $\sgrad F$ по определению является производной в точке $t = 0$ соответствующего гамильтонова потока.
Мы заключаем, что \emph{алгебру Ли} группы $\Ham (M, \Omega)$ можно отождествить с $\A$.

\begin{thm}{Упражнение}\label{1.4.G}
Покажите, что на этом языке касательный вектор к гамильтонову пути $\{f_t\}$ в точке $t = s$ является функцией $F_s \in \A$.
\emph{Подсказка:} идентификация касательных пространств к группе осуществляется с помощью правого сдвига (см. сноску).
Таким образом, рассматриваемый касательный вектор отождествляется с
касательным вектором к пути \?{$\{f_tf_s^{-1}\}$}{Надо бы как-то
  указать, что параметр пути это $t$} в точке $t \z= s$.
\end{thm}

Следующее важное понятие --- присоединённое действие группы Ли на её алгебре Ли.
Напомним, что эта операция определяется следующим образом.
Выберем элемент $f$ группы $\Ham (M, \Omega)$ и элемент $G$ из алгебры Ли $\A$.
Пусть $\{g_t\}$, $g_0 = \1$ --- путь на группе, касающийся $G$.
В нашей ситуации условие касания означает, конечно, что нормализованный гамильтониан потока $\{g_t\}$ в момент времени $t = 0$ равен $G$.
По определению 
\[\Ad_f G = \left.\frac{\d}{\d t}\right|_{t=0} fg_tf^{-1}.\]
Продифференцировав, получаем, что векторное поле в правой части равно $f_\ast \sgrad G$, а это в точности $\sgrad  (G\circ f^{-1})$ ввиду упражнения \ref{1.3.B}.
Возвращаясь к нашему отождествлению, получаем, что 
\[\Ad_f G = G \circ f^{-1}.\]
Таким образом, \emph{присоединённое действие $\Ham (M, \Omega)$ на $\A$ --- это просто обычное действие диффеоморфизмов на функции}.

Наконец, давайте обсудим скобку Ли на $\A$.
Выберем два элемента $F,G \in \A$, и пусть $\{f_t\}$, $f_0 = \1$, гамильтонов путь, касающийся $F$ в~$0$.
Скобка Ли $\{F, G\}$ элементов $F$ и $G$ называется скобкой Пуассона и определяется следующим образом: 
\[\{F, G\}=\left.\frac{\d}{\d t}\right|_{t=0}Ad_{f_t}G\]
Вычисляя выражение в правой части, получаем, что 
\[\{F, G\} = -\d G (\sgrad F) = \Omega (\sgrad G, \sgrad F).\]
Отметим, что в терминах векторных полей скобка Ли с точностью до знака совпадает с обычным коммутатором.
Читателю предлагается проверить, что 
\[[\sgrad F, \sgrad G] = -\sgrad  \{F, G\},\]
где коммутатор $[X, Y]$ двух векторных полей определяется как $L_{[X, Y]} \z= L_X L_Y -L_Y L_X$.

\?{}{убрал «понятий, которые играют важную роль в этой книге»}
\begin{framed}
\parbf{Внимание:} Разные авторы могут использовать разные
знаки в следующих определениях:
гамильтоново векторное поле,
скобка Пуассона,
коммутатор векторных полей
и кривизна связности.
\end{framed}

\begin{thm}{Пример}\label{1.4.H}
Рассмотрим единичную сферу $S^2$ в евклидовом пространстве $\RR^3$.
Пусть $\Omega$ --- индуцированная форма площади на сфере.
Группа $\SO(3)$ действует на $S^2$ диффеоморфизмами, сохраняющими площадь.
Поскольку группа $SO(3)$ линейно связна, она содержится в $\Symp_0 (S^2)$.
Применяя \ref{1.4.C} получаем, что $\Symp_0 (S^2) \z= \Ham (S^2)$, а значит, $\SO (3)$ является подгруппой в $\Ham (S^2)$.
В частности, каждый элемент алгебры Ли $\so(3)$ может быть однозначно представлен как нормализованный гамильтониан на $S^2$.
Давайте подробно опишем это соответствие.
Отождествим $\so (3)$ с $\RR^3$ следующим образом.
Каждый вектор $a \in \RR^3$ рассматривается как кососимметричное преобразование $x \to [x, a]$ пространства, где скобки обозначают стандартное векторное произведение.
Отождествим касательную плоскость к $S^2$ в точке $x$ с ортогональным
дополнением к $x$ в объемлющем пространстве.
По тавтологическим причинам гамильтоново векторное поле $v$ потока $x \z\mapsto \exp (ta) x$ на сфере задается формулой $v (x) = [x, a]$.
Мы утверждаем, что \emph{соответствующий нормированный гамильтониан является
функцией высоты $F(x)=(a,x)$} \?{в направлении}{добавили, что $(a,x)$ --- скалярное произведение} вектора $a$, то есть скалярным произведением с $a$.

Прежде всего, отражение в ортогональном дополнении к $a$ переводит $F$ в $-F$, таким образом, $F$ имеет нулевое среднее.
Далее заметим, что $\Omega (\xi, \eta) = (\eta, [x, \xi])$ для $\xi, \eta \in T_x S^2$.
Обозначим через $a'$ ортогональную проекцию a на $T_x S^2$.
Таким образом, $\Omega (\xi, v(x)) = ([x, a'], [x, \xi])$ для любого $\xi \in T_x S^2$.
Поскольку векторное произведение с $x$ является ортогональным преобразованием $T_x S^2$, последнее выражение равно $(a', \xi)$,
а это в точности $\d F(\xi)$.
Утверждение следует.
\end{thm}

\section[\texorpdfstring{Алгебраические свойства $\Ham(M,\Omega)$}{Алгебраические свойства Ham(M,Ω)}]%
{Алгебраические свойства $\bm{\Ham(M, \Omega)}$}

Алгебраические свойства группы гамильтоновых диффеоморфизмов изучались А. Баньягой \cite{B1,B2}.
В частности, он доказал следующий поразительный результат.
Напомним, что группа $D$ называется простой, если каждая её нормальная подгруппа тривиальна, то есть она либо $\{\1\}$, либо вся $D$.

\begin{thm}{Теорема}\label{1.5.A}
Пусть $(M, \Omega)$ --- замкнутое симплектическое многообразие.
Тогда группа $\Ham (M, \Omega)$ проста.
\end{thm}

Существует версия этого утверждения и для открытого $M$.
Обратите внимание, что абелева группа проста тогда и только тогда, когда каждый элемент порождает всю группу (и поэтому это должна быть конечная циклическая группа, порядок которой является простым числом).
Таким образом, говоря интуитивно, общие простые группы далеки от абелевых.
Ниже приводится элементарное утверждение, поясняющее этот принцип для группы гамильтоновых диффеоморфизмов.
Нам он потребуется в следующей главе.

\begin{thm}{Предложение}\label{1.5.B}
Пусть $(M, \Omega)$ --- симплектическое многообразие, и $U \subset M$ --- непустое открытое подмножество.
Существуют такие $f, g \in \Ham (M, \Omega)$, что $\supp (f)$, $\supp (g) \subset U$ и $f g \ne gf$.
\end{thm}

Доказательство основано на следующем факте.

\begin{thm}{Предложение}\label{1.5.C}
Пусть $\{f_t\}$ и $\{g_t\}$ --- гамильтоновы потоки, порождённые независимыми от времени нормализованными гамильтонианами $F$ и $G$ соответственно.
Если $f_t g_t = g_t f_t$ для всех $t$, то $\{F, G\} = 0$.
\end{thm}

\parbf{Доказательство:}
Согласно \ref{1.4.D}, гамильтонианы соответствующие потокам $f_t g_t$ и $g_t f_t$ равны
\[F(x)+G(f_t^{-1} (x))
\quad\text{и}\quad
G (x) + F (g_t^{-1}(x)).
\]
Но поскольку оба определяют один и тот же поток, получаем, что 
\[F (x) + G (f_t^{-1} (x)) = G (x) + F (g_t^{-1} (x))\]
для всех $t$.
Продифференцировав по $t$, получаем
\[\d G(-\sgrad F)=\d F(-\sgrad G),\]
а значит $\{F, G\} = \{G, F\}$.
Из антикоммутативности скобки Ли, получаем $\{F, G\} = 0$.

\parbf{Доказательство \ref{1.5.B}:} 
Выберем точку $x\in U$ и касательные векторы $\xi, \eta \in T_x U$ такие, что $\Omega (\xi, \eta) \ne 0$.
Теперь выберем ростки функций $F$ и $G$ (см. Упражнение ниже), для которых $\sgrad F (x) = \xi$, $\sgrad G (x) = \eta$.
Расширим эти функции нулём за пределы $U$.
Если $M$ открыто, то задача решена.
Если $M$ замкнуто, добавим константу, чтобы гарантировать, что $F$ и $G$ имеют нулевое среднее.
Таким образом, функции $F$ и $G$ принадлежат $\A$.
Кроме того, они постоянны вне $U$, поэтому носители соответствующих гамильтоновых диффеоморфизмов $f_t$ и $g_t$ лежат в $U$.
Поскольку $\{F, G\} \ne 0$, мы видим, что для некоторого $t$ диффеоморфизмы $f_t$ и $g_t$ не коммутируют.

\begin{thm*}{Упражнение}
Используя локальные канонические координаты в $x$, докажите, что $F$ и $G$ в доказательстве следствия действительно существуют.
\end{thm*}

Завершим этот раздел \?{формулировкой}{добавил} следующего результата А. Баньяги \cite{B2}.

\begin{thm}{Теорема}\label{1.5.D}
Пусть $(M_1, \Omega_1)$ и $(M_2, \Omega_2)$ --- два замкнутых симплектических многообразия, группы гамильтоновых диффеоморфизмов которых изоморфны.
Тогда многообразия \emph{конформно симплектоморфны}: существуют диффеоморфизм $f\: M_1 \to  M_2$ и число $c \ne 0$ такие, что $f^\ast \Omega_2 = c\Omega_1$. 
\end{thm}

Другими словами, алгебраическая структура группы гамильтоновых диффеоморфизмов определяет симплектическое многообразие с точностью до множителя.

\?{\hspace{100mm}}{До сюда посмотрел.}


