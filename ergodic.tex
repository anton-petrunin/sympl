\chapter[Эргодическая теория]{Приложение к эргодической теории}\label{chap:11}

В настоящей главе мы обсудим асимптотический геометрический инвариант,
связанный с фундаментальной группой $\Ham(M, \Omega)$, и применим его
в классической эргодической теории (см. \cite{P9}). 

\section{Гамильтоновы петли как динамические объекты}\label{sec:11.1}

Пусть $(M,\Omega)$ — замкнутое симплектическое многообразие.
Для заданного иррационального числа $\alpha$ и гладкой петли $h\:S^1\z\to\Ham(M,\Omega)$ можно определить отображение \rindex{косое произведение}\emph{косого произведения} $T_{h,\alpha} \: M \times S^1 \z\to M\times S^1$ как $T_{h,\alpha}(y,t)=(h(t)y,t+\alpha)$.
Наша цель — связать геометрию и топологию гамильтоновых петель с динамикой соответствующих косых произведений.% 
\footnote{В этой главе петли свободные и мы не предполагаем, что $h(0) = \1$.}

Приведённое выше определение является частным случаем гораздо более
общего понятия косого произведения  \cite[с. 231]{CFS}, которое
интенсивно изучалось несколько десятилетий. 
Есть по крайней мере две важные причины интереса к этому понятию.
Во-первых, оно служит основой для математических моделей случайной
динамики (см. обзор \cite{Ki}). 
Во-вторых, даёт нетривиальные примеры систем с интересными
динамическими свойствами. 

Нас будет интересовать строгая эргодичность.
Напомним, что гомеоморфизм $T$ компактного топологического пространства $X$ \rindex{строгая эргодичность}\emph{строго эргодичен}, если он имеет ровно одну инвариантную борелевскую вероятностную меру, скажем, $m$, которая, кроме того, положительна на непустых открытых подмножествах. 
Строго эргодические гомеоморфизмы эргодичны и обладают рядом
дополнительных замечательных свойств. 
Упомянем одно свойство, которое сыграет решающую роль.
А именно, если $T$ строго эргодичен, то для произвольной непрерывной
функции $F$ на $X$ средние по времени
$\tfrac1N\sum_{i=0}^{N-1}F(T^ix)$ равномерно сходятся к среднему по
пространству $\int_XFdm$ и, в частности, сходятся для всех $x \in X$. 
Заметим, что в общем случае, для эргодических преобразований такая сходимость
имеет место только {}\emph{почти везде}. 
Контраст между «везде» и «почти везде» становится более чётким, когда
налицо наличие чисто топологических препятствий к строгой
эргодичности. 
Например, 2-сфера не допускает строго эргодических гомеоморфизмов.
Действительно, из теоремы Лефшеца следует, что каждый гомеоморфизм
$S^2$ имеет либо неподвижную точку, либо орбиту периода $2$, и мы
видим, что инвариантная мера, сосредоточенная на такой орбите,
противоречит определению строгой эргодичности. 
В главе~\ref{sec:11.2} мы опишем более хитрое препятствие к строгой
эргодичности, вытекающее из симплектической топологии. 

Мы говорим, что петля $h\:S^1\to\Ham(M,\Omega)$ \rindex{строго эргодичная петля}\emph{строго эргодична}, если косое произведение $T_{h,\alpha}$ строго эргодично
для некоторого $\alpha$.%
\footnote{Заметим, что каждый класс $T_{h,\alpha}$ сохраняет
  каноническую меру на $M \times S^1$, индуцированную симплектической
  формой.
Таким образом, в нашей постановке строгая эргодичность означает, что
эта мера является (с точностью до множителя) единственной инвариантной
мерой.} 
На этом языке наш центральный вопрос можно сформулировать следующим образом.

\begin{ex*}{Вопрос}
Какие гомотопические классы $S^1 \to \Ham(M, \Omega)$ могут быть
представлены строго эргодическими петлями? 
\end{ex*}

В следующем примере, можно полностью ответить на этот вопрос.
Пусть $M_\ast$ — раздутие комплексной проективной плоскости $\CP^2$
в одной точке. 
Выберем кэлерову симплектическую структуру $\Omega_\ast$ на $M_\ast$,
интеграл которой по прямой общего положения равен 1, а интеграл по
исключительному дивизору равен $\tfrac13$. 
Ввиду \ref{9.2.E}, можно также думать, что $M_\ast = \PP(T \oplus
C)$, где $T$ и $C$ — тавтологическое и тривиальное голоморфное
линейные расслоения над $\CP^1$ соответственно. 
На этом языке исключительный дивизор соответствует нашему старому
приятелю — $\Sigma$ из главы \ref{sec:10.3}, а общая прямая гомологична
сумме $\Sigma$ со слоем. 
Периоды симплектической формы выбираются таким образом, что класс её
когомологий был кратен первому классу Черна многообразия $M$
(см. определение \ref{11.3.A}). 
Легко видеть, что $(M_\ast, \Omega_\ast)$ допускает эффективное
гамильтоново действие унитарной группы $\U(2)$, то есть существует
мономорфизм $i\: \U(2) \to \Ham(M_\ast, \Omega_\ast)$. 
Фундаментальная группа $\U(2)$ равна $\ZZ$.
Недавно \rindex{Абреу}Абреу и Макдафф \cite{AM} доказали, что включение $\pi_1(U(2))
\to \pi_1 (\Ham(M_\ast, \Omega_\ast))$ является изоморфизмом и,
следовательно, $\pi_1(\Ham(M_\ast, \Omega_\ast)) = \ZZ$. 
Насколько мне известно, это простейший пример симплектического
многообразия с $\pi_1 (\Ham) = \ZZ$. 

\begin{thm}{Теорема}\label{11.1.A}
Тривиальный класс $0\in\pi_1(\Ham(M_\ast, \Omega_\ast))$ —
единственный класс, который может быть представлен строго эргодической
петлёй. 
\end{thm}

Доказательство этой теоремы состоит из двух частей.
Прежде всего необходимо установить существование стягиваемых строго
эргодических петель. 
Это можно сделать чисто эргодическими методами в достаточно общем случае.
Мы отсылаем читателя к \cite{P9} за подробностями.
Во-вторых, нужно доказать, что каждый класс $\gamma \ne 0$ не может
быть представлен строго эргодической петлёй. 
Оказывается, препятствие исходит из геометрии $\Ham(M, \Omega)$.
Мы обсудим это подробнее.

\section{Асимптотической спектр длин}\label{sec:11.2}
\rindex{асимптотической спектр длин}

Определим \rindex{асимптотическая норма}\emph{асимптотическую норму} элемента
$\gamma\in\pi_1(\Ham(M, \Omega))$ как
\[\nu_\infty(\gamma)=\lim_{k\to+\infty}\frac{\nu(k\gamma)}{k},\]
где $\nu$ — норма, введённая в \ref{sec:7.3}. 
Это понятие аналогично асимптотическому росту $\mu$ определённому в
\ref{sec:8.2}.
Предел существует, поскольку последовательность $\nu(k\gamma)$ субаддитивна.

\begin{thm}{Теорема}\label{11.2.A}
Пусть $\gamma\in\pi_1(\Ham(M,\Omega))$ — класс, представленный
гладкой строго эргодической петлёй.
Тогда асимптотическая норма $\nu_\infty (\gamma)$ обращается в нуль.
\end{thm}

\parit{Доказательство.}
Доказательство основано на процедуре асимптотического сокращения
кривой в духе \ref{sec:8.3}. 
Пусть $h\: S^1 \to \Ham(M,\Omega)$ — гладкая петля гамильтоновых
диффеоморфизмов, определяющая строго эргодическое косое произведение
$T(y, t) = (h(t)y, t+\alpha)$. 
Пусть $\gamma$ — соответствующий элемент из $\pi_1(\Ham(M, \Omega))$.
Обозначим через $H(x, t)$ нормированный гамильтониан, порождающий
петлю $h(t)^{-1}$. 
Пусть $h_k(t) = h(t + k\alpha)^{-1}$ и  
\[f_N(t) = h_0(t) \circ \dots \circ h_{N-1}(t).\]
Из \ref{1.4.D} следует, что петля $f_N$ порождается нормированным гамильтонианом 
\begin{align*}
F_N(y,t)
&=
H(y,t)
+ H(h_0(t)^{-1}y,t+\alpha)
+\dots
\\
&\dots
+
H(h_{N-2} (t)^{-1} \circ \dots\circ h_0(t)^{-1}y, t + (N - 1)\alpha).
\end{align*}
Это выражение можно переписать следующим образом:
\[F_N(y, t) = \sum_{k=0}^{N-1} H\circ T^k(y, t).\]
Так как гомеоморфизм $T$ строго эргодичен и функция $F_N$ имеет
нулевое среднее, мы заключаем, что
\[\frac1N\int_0^1\max_{y\in M}F_N(y, t) - \min F_N(y, t)\,dt \to 0\]
при $N \to \infty$.

Но выражение в левой части в точности равно $\frac1N\length\{f_N(t)\}$.
Обратите внимание, что петля $\{f_N(t)\}$ представляет элемент $-N\gamma$.
Поскольку $\nu(N\gamma) = \nu(-N\gamma)$, мы получаем, что
$\nu(N\gamma)/N$ стремится к нулю при $N \to \infty$, 
то есть асимптотическая норма $\gamma$ обращается в нуль.
\qeds

Я не знаю точного значения $\nu_\infty (\gamma)$ ни в одном примере,
где эта величина строго положительна (например, для раздутия $\CP^2$ в
\ref{sec:11.1}).
Трудность в том, что во всех известных примерах, где можно точно
вычислить хоферовскую норму $\nu(\gamma)$, существует замкнутая петля
$h(t)$, минимизирующая длину в своём гомотопическом классе (то есть
\emph{замкнутая кратчайшая}).
Однако оказывается, что всякая непостоянная замкнутая кратчайшая
перестаёт быть таковой после достаточного числа итераций.
Другими словами, петлю $h(Nt)$ можно сократить, если $N$ достаточно велико.
Доказательство этого утверждения основано на следующем обобщении
описанной выше процедуры сокращения.
Пусть $H(y, t)$ — нормированный гамильтониан петли $h(t)^{-1}$.
Не умоляя общности можно предположить, что $h(0) = \1$, и что $H(y,0)$
не обращается в нуль тождественно.
Обозначим через $\Gamma$ множество всех точек $M$, в которых функция
$|H(y, 0)|$ достигает максимального значения.
Поскольку $M \setminus \Gamma$ — непустое открытое подмножество, а
группа гамильтоновых диффеоморфизмов действует транзитивно на $M$,
можно выбрать последовательность
\[\1=\phi_0,\phi_1,\dots,\phi_{N-1}\in\Ham(M,\Omega)\]
такую, что 
\[\Gamma\cap\phi_1(\Gamma)\cap\dots\cap\phi_{N-1}(\Gamma)=\emptyset\]
Рассмотрим петлю $f_N(t) = h(t)^{-1} \circ \phi_1h(t)^{-1}\phi_1^{-1}
\circ \dots \circ \phi_{N-1}h(t)^{-1}\phi_{N-1}$.
Покажем, что она короче чем петля $h(Nt)$.
Действительно, заметим, что её гамильтониан $F_N$ в момент времени $t
= 0$ может быть записан следующим образом:
\[F_N(y,0) = \sum_{i=0}^{N-1} H(\phi_i^{-1}y, 0).\]
Пусть
\[a(t)
=
\max_{y\in M} F_N(y, t) - \min_{y\in M}  F_N(y, t)
\]
и
\[b(t)
=
N(\max_{y\in M} H(y, t) - \min_{y\in M}  H(y, t)).
\]
Выбор последовательности $\{\phi_i\}$ влечёт, что $a(0) < b(0)$.
Поскольку $a(t) \z\le b(t)$ для всех $t$, мы получаем, что
$\int_0^1a(t)\,dt < \int_0^1 b(t)\,dt$, и это доказывает утверждение.
Мы заключаем, что если \textit{ненулевой класс $\gamma\in
  \pi_1(\Ham(M, \Omega))$ представлен минимальной замкнутой
  геодезической, то $\nu_\infty (\gamma)$ строго меньше, чем
  $\nu(\gamma)$.}

Было бы интересно исследовать дальнейшие ограничения на гомотопические
классы гладких строго эргодических петель в группе гамильтоновых
диффеоморфизмов. 

\section{Алгебра в помощь}

Вернёмся к теореме \ref{11.1.A}.
В этом разделе мы наметим доказательство того, что нестягиваемые петли
в $\Ham(M_\ast, \Omega_\ast)$ не могут быть строго эргодичными.

Пусть $(M^{2n},\Omega)$ — замкнутое симплектическое многообразие.
Рассмотрим отображение
\[I\:\pi_1(\Ham(M,\Omega))\to\RR\]
определённое следующим образом.
Пусть $\gamma\in\pi_1(\Ham(M,\Omega))$,
рассмотрим её симплектическое расслоение $P(\gamma)$.
Обозначим через $u$ первый класс Черна вертикального касательного
расслоения над $P(\gamma)$.
Слой этого расслоения в точке $P(\gamma)$ есть (симплектическое)
векторное пространство, касательное к слою, проходящему через эту
точку. 
Как и раньше, $c$ обозначает класс спаривания.
Определим «характеристическое число» 
\[I(\gamma)= \int_{P(\gamma)} c^n\smallsmile u.\]
Легко видеть, что $I\: \pi_1(\Ham(M, \Omega))\to \RR$ — гомоморфизм
(\rindex{Лалонд}\cite{P6,LMP2}). 

\begin{ex}{Определение}\label{11.3.A} Симплектическое многообразие
  $(M,\Omega)$ называется \rindex{монотонное
    многообразие}\emph{монотонным}, если $[\Omega]$ является 
  положительным кратным $c_1 (\T M)$.
\end{ex}

\begin{thm}[(\cite{P6})]{Теорема}\label{11.3.B}
Пусть $(M, \Omega)$ — замкнутое монотонное симплектическое многообразие.
Тогда существует положительная константа $C>0$ такая, что
$\nu(\gamma)\ge C|I(\gamma)|$ для всех $\gamma\z\in\pi_1(\Ham(M,\Omega))$.
\end{thm}

Другими словами, гомоморфизм $I$ калибрует хоферовскую норму на
фундаментальной группе.
Доказательство этой теоремы основано на теории, описанной в двух
предыдущих главах, в сочетании с результатами из \cite{Se}.
Недавно \rindex{Зейдель}Зейдель получил обобщение этого неравенства на немонотонные
симплектические многообразия.
Так как $I$ — гомоморфизм, оценка в \ref{11.3.B} проходит для
асимптотической хоферовской нормы:
\[\nu_\infty(\gamma)\ge C|I(\gamma)|\]
В отличии от нормы $\nu$, гомоморфизм $I$ можно относительно легко вычислить — в этом его большое преимущество.
Например, можно показать, что $I(\gamma) \ne 0$, где $\gamma$ —
образующая $\pi_1(\Ham(M_\ast,\Omega_\ast)) = \ZZ$.
Таким образом, из \ref{11.3.B} следует, что асимптотическая норма
каждого нетривиального элемента $\pi_1(\Ham(M_\ast, \Omega_\ast))$
строго положительна.
Из \ref{11.2.A} следует, что такой элемент не может быть представлен
строго эргодической петлёй.
