\chapter*{Предисловие}

Группа гамильтоновых диффеоморфизмов $\Ham(M,\Omega)$ симплектического
многообразия $(M,\Omega)$ играет
основополагающую роль в геометрии и
механике.
Для геометра, по крайней мере при некоторых предположениях
о многообразии $M$, это связная компонента тождественного отображения
в группе всех симплектических диффеоморфизмов.
С точки зрения
механики, $\Ham(M,\Omega)$ является группой всех допустимых движений.
Какое минимальное количество энергии, необходимо для реализации
данного гамильтонова диффеоморфизма $f$?
Попытка формализовать этот естественный вопрос и ответить на него привёл \rindex{Хофер}Х. Хофера \cite{H1} (1990) к удивительному открытию.
Оказалось, что решение этой вариационной задачи можно интерпретировать как {}\emph{геометрическую величину}, а именно как расстояние между $f$ и тождественным
отображением.
Кроме того, это расстояние связано с \?{канонической биинвариантной метрикой}{Может сказать, что имеется ввиду хоферовская метрика?} на $\Ham(M,\Omega)$.
Начиная с работ Хофера, эта новая геометрия интенсивно изучалась в рамках современной
симплектической топологии.
В настоящей книге я опишу кое-что из полученых результатов.

Хоферовская геометрия позволяет изучать различные понятия и задачи из знакомой нам конечномерной геометрии в контексте группы гамильтоновых диффеоморфизмов.
При этом они сильно отличаются от обычного круга задач, рассматриваемых в симплектической топологии и, таким образом, значительно расширяют горизонты в симплектического мира.
Бесконечен ли диаметр $\Ham(M,\Omega)$?
Какие там кратчайшие?
Как найти спектр длин?
В общем случае, эти вопросы остаются открытыми.
Однако некоторые частные ответы существуют и будут нами рассмотрены.

Есть ещё одна, на мой взгляд, даже более важная причина, почему полезно иметь каноническую геометрию на группе гамильтониановых диффеоморфизмы.
Рассмотрим зависящее от времени векторное поле $\xi_t$, $t\in \RR$ на многообразии $M$.
Обыкновенное дифференциальное уравнение
\[\dot x=\xi(x,t)\]
на $M$ определяет поток $f_t\: M \to M$, который отображает точку $x(0)$ в $x(t)$ --- значение решения в момент времени $t$.
Траектории потока образуют сложную систему кривых на многообразии.
Обычно, чтобы разобраться в динамике, нужно следить за этими кривым на многообразии и изучить их поведение в разных регионах.
Сменив точку зрения мы видим, что наш поток становится простым
геометрическим объектом --- одной кривой $t \mapsto f_t$ на группе всех диффеоморфизмов многообразия.
Хочется надеяться, что геометрические свойства этой кривой отражают динамику и, в таком случае, сложную динамику можно изучать чисто геометрическими средствами.
Разумеется за это придётся платить.
Дело в том, что объемлющее пространство --- группа
диффеоморфизмов --- бесконечномерно.
Более того, встаёт другая серьёзная проблема:
в общем случае у нас нет канонических инструментов для выполнения геометрических измерений по этой группе.
Примечательно, что хоферовская метрика даёт такой инструмент для систем классической механики.
В главах 8 и 11 мы рассмотрим некоторые ситуации,
когда такой ход рассуждений оказывается полезным.

Как часто бывает с молодыми быстроразвивающимися областями математики, доказательства некоторых красивых утверждений хоферовской геометрии оказываются технически сложными.
Поэтому я выбирал самые простые нетривиальные случаи основных утверждений (на свой вкус, конечно), не пытаясь представить их в максимально возможной общности.
По той же причине опущены многие технические детали.
Хотя формально эта книга не требует специальных знаний в симплектической топологии (по крайней мере, необходимые определения и формулировки приведены), я рекомендую читателю  два замечательных вводных текста \cite{HZ} и \cite{MS}.
Оба содержат главы по геометрии группы гамильтониановых диффеоморфизмов.
Я постарался избежать повторов.
Книга содержит упражнения, которые предположительно помогут читателю вникнуть в предмет.

Эта книга возникла из двух источников.
Первый --- это лекции для аспирантов, а именно миникурсы в университетах Фрайбурга и Уорика, а также курс лекций в Швейцарской высшей технической школе Цюриха.
Вторым источником является обзорная статья \cite{P8}, которая содержит
вкратце материал приведённый ниже.

Теперь я коротко опишу содержание книги.
Дифффеоморфизм $f$ симплектического многообразия $(M,\Omega)$
называется гамильтоновым если его можно включить в гамильтонов поток
$f_t$ с компактным носителем удовлетворяющий $f_0=\1$ и $f_1 =f$.
Такой поток определяется гамильтонианом $F\: M \times [0;1] \to \RR$.
На языке классической механики, $F$ --- энергия механического движения, описывающая $f_t$.
Мы понимаем полную энергию потока как длину соответствующего пути диффеоморфизмов:
\[\length \{f_t\} =
\int_0^1\max_{x\in M}F(x,t)-\min_{x\in M}F(x,t)\, dt 
\]%добавил скобки
Определим функцию
\[\rho\: \Ham(M,\Omega) \times \Ham(M,\Omega) \to \RR\]
как
\[\rho (\phi, \psi) = \inf \length \{f_t\},\]
где точная нижняя грань берётся по всем гамильтоновым потокам $\{f_t\}$, которые
порождают гамильтонов диффеоморфизм $f = \phi\psi^{-1}$.
Легко видеть, что $\rho$ неотрицательна, симметрична, обращается в ноль на диагонали и удовлетворяет неравенству треугольника.
Более того, $\rho$ биинвариантно по отношению к групповой структуре на $\Ham(M,\Omega)$.
Другими словами, $\rho$ --- биинвариантная псевдометрика.
Глубокий факт состоит в том, что $\rho$ --- настоящая \?{метрика}{Было «distance function», но «метрика» больше подходит}, то есть $\rho (\phi, \psi)$ строго положительна при $\phi \ne \psi$.
Метрика $\rho$ называется хоферовской метрикой.

Группа $\Ham(M,\Omega)$ и хоферовская псевдометрика $\rho$ обсужадются в главах 1 и 2 соответственно.
В главе 3 доказывается, что $\rho$ является настоящей метрикой в случае, когда $M$ есть стандартное симплектическое линейное пространство $\RR^{2n}$.
Наш подход основан на теории Громова голоморфных дисков с лагранжевыми граничными условиями, см.
главу~4.

Затем мы переходим к изучению основных геометрических инвариантов $\Ham(M,\Omega)$.
Существует (пока ещё открытая!) гипотеза о том, что диаметр $\Ham(M,\Omega)$ бесконечен.
В главах 5--7 эта гипотеза доказывается для замкнутых поверхностей.

В главе 8 обсуждается рост однопараметрической подгруппы $\{f_t\}$ группы $\Ham(M,\Omega)$, которая описывает асимптотическое поведение функции $\rho (\1, f_t)$ при $t \to \infty$.
Мы описываем связь между ростом и динамикой $\{f_t\}$ в контексте
теориии инвариантных торов классической механики.

Во многих интересных ситуациях пространство $\Ham(M,\Omega)$ имеет сложную топологию и, в частности, нетривиальную фундаментальную группу.
Для $\gamma\in\pi_1(\Ham(M,\Omega))$, положим $\nu(\gamma) \z= \inf \length \{f_t\}$, где
точная нижняя грань берётся по всем петлям $\{f_t\}$ гамильтоновых
диффеоморфизмов (то есть, по периодическим гамильтоновым потокам),
которые представляют класс петель.
Множество
\[\set{\nu(\gamma)}{\gamma\in\pi_{1}\Ham(M,\Omega)}\]
называется спектром длин группы $\Ham(M,\Omega)$.
В главе 9 мы представляем подход к оценке спектра длин, основанный на
теории симплектических расслоений. 
Важным компонентом этого подхода является теория Громова
псевдоголоморфных кривых, обсуждаемая в главе~10. 
В главе 11 приводятся приложения наших результатов к спектру длин в
классической эргодической теории.

В главах 12 и 13 развиваются два разных подхода к теории геодезических на $\Ham(M,\Omega)$.
Один из них элементарный, а другой требует мощного инструмента --- гомологий Флоера.
Глава 13 предлагает читателю краткое введение в гомологии Флоера.

Наконец, в главе 14 мы имеем дело с негамильтоновыми симплектическими диффеоморфизмами, которые естественно появляются в хоферовской геометрии как изометрии $\Ham(M,\Omega)$.
Кроме того, мы формулируем и обсуждаем знаменитую гипотезу потока,
заключающююся в том, что группа $\Ham(M,\Omega)$ замкнута в группе всех симплектических диффеоморфизмов, наделённой $C^\infty$-топологией. 

\?{}{Нет уверенности, что русификация имён правильная}
\paragraph*{Благодарности.}
Я сердечно благодарен \rindex{Аквельд}\?{Мейке Аквельд}{Meike Akveld --- she+Swiss?} %
за её незаменимую помощь в напечатании предварительной версии рукописи,
подготовку рисунков и огромную редакционную работу.
Я очень благодарен \?{Паулю Бирану}{Paul Biran} и Карлу Фридриху Зибургу за их подробные комментарии к рукописи и за улучшение изложения.
Я признателен \?{Рами Айзенбуду}{Rami Aizenbud}, Диме Гуревичу, Мише Энтову, Осе Полтерович и Зеэву Руднику за указание на ряд неточностей в предварительной версии книги.
Книга была написана во время моего пребывания в Швейцарской высшей технической школе Цюриха в 1997--1998 учебном году и во время моих визитов в Институт высших научных исследований, Бюр-сюр-Иветт в 1998 и 1999 годах.
Я благодарю оба этих института за прекрасную исследовательскую атмосферу. 
