\chapter{Спектр длин}

В этой главе мы опишем метод вычисления спектра длин в хоферовской геометрии.
Наш подход основан на теории симплектических расслоений над двумерной сферой.

\section{Положительная и отрицательная части хоферовской нормы}

Пусть $(M, \Omega)$ --- замкнутое симплектическое многообразие.
Для $\gamma \in \pi_1 (\Ham(M, \Omega))$ положим 
\begin{align*}
\nu_+ (\gamma) &= \inf_F \int_0^1 \max_x F (x, t)dt = \inf_F \max_{x,t}F (x, t),
\\ 
\nu_- (\gamma) &= \inf_F \int_0^1 -\min_x F (x, t)dt = \inf_F \min_{x,t}F (x, t),
\end{align*}
где точная нижняя грань берётся по всем нормированным периодическим гамильтонианам $F \in \H$, порождающим петлю в классе~$\gamma$.
\?{Правильность нашего определения}{Лучше «равенство в определениях»} доказывается точно так же, как и лемма \ref{5.1.C} выше.

\begin{thm}{Упражнение}
Докажите, что $\nu_+ (\gamma) = \nu_- (-\gamma)$ и $\nu(\gamma) \z\ge \nu_- (\gamma) + \nu_+ (\gamma)$
(сравните с открытой задачей в \ref{sec:2.4} выше).
\end{thm}

Вычисление $\nu_+ (\gamma)$ оказывается нетривиальным даже в следующем простейшем случае.
Рассмотрим $(S^2, \Omega)$ нормализованное так, что $\int_{S^2} \Omega = 1$.
Пусть $\gamma$ --- класс одного оборота $\{f_t\}$, и пусть $F \in \H$ --- гамильтониан, порождающий $\{f_t\}$.
Как мы видели в \ref{6.3.C} выше,  $\max F = - \min F = \frac12$.
(Множитель $2\pi$ в \ref{6.3.C} исчез ввиду приведенной выше нормировки $\Omega$.)
Таким образом, $\nu_+ (\gamma) \le \frac12$.
Но ма самом деле имеет место равенство!

\begin{thm}{Теорема}\label{9.1.A}
$\nu_+ (\gamma) = \frac12$.
\end{thm}

Гамильтонова петля $\{f_t \}$, представляющая класс $\gamma \ne 0$, называется замкнутой кратчайшей, если $\length\{f_t \} = \nu(\gamma)$.
Обратите внимание на то, что замкнутая кратчайшая не может быть кратчайшей.

\begin{thm}[(\cite{LM2})]{Следствие} $\nu(\gamma) = 1$ и $\{f_t\}$ --- замкнутая кратчайшая.
\end{thm}

\parit{Доказательство.}
Прежде всего, $\length\{f_t\} =\frac12 - (-\frac12 ) = 1$ и, следовательно, $\nu(\gamma) \le 1$.
Поскольку $2\gamma = 0$, 
\[\nu_- (\gamma) = \nu_+ (-\gamma) = \nu_+ (\gamma) =\frac12.\]
Кроме того, $\nu(\gamma) \ge \nu_- (\gamma) + \nu_+ (\gamma) = 1$.
Отсюда $\nu(\gamma) = 1$ и $\{f_t\}$ --- замкнутая кратчайшая.

Теорема \ref{9.1.A} доказана в \ref{sec:9.4} ниже.
Обобщение на на $\CP^n$ при $n \ge 2$ дано в \cite{P3}.
