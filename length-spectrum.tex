\chapter{Спектр длин}\label{chap:9}

В этой главе мы опишем метод вычисления спектра длин в хоферовской геометрии.
Наш подход основан на теории симплектических расслоений над двумерной сферой.

\section{Положительная и отрицательная части хоферовской нормы}

Пусть $(M, \Omega)$ --- замкнутое симплектическое многообразие.
Для $\gamma \in \pi_1 (\Ham(M, \Omega))$ положим \index[symb]{$\nu_\pm$}
\begin{align*}
\nu_+ (\gamma) &= \inf_F \int_0^1 \max_x F (x, t)\,dt = \inf_F \max_{x,t}F (x, t),
\\ 
\nu_- (\gamma) &= \inf_F \int_0^1 -\min_x F (x, t)\,dt = \inf_F \min_{x,t}F (x, t),
\end{align*}
где точная нижняя грань берётся по всем нормализованным периодическим
гамильтонианам $F \in \H$, порождающим петлю в классе~$\gamma$. 
Равенства в определениях доказывается точно так же, как и лемма \ref{5.1.C} выше.

\begin{ex*}{Упражнение}
Докажите, что $\nu_+ (\gamma) = \nu_- (-\gamma)$ и $\nu(\gamma) \z\ge
\nu_- (\gamma) + \nu_+ (\gamma)$ 
(сравните с открытой задачей в \ref{sec:2.4} выше).
\end{ex*}

Вычисление $\nu_+ (\gamma)$ оказывается нетривиальным даже в следующем
простейшем случае. 
Рассмотрим $(S^2, \Omega)$ нормализованное так, что $\int_{S^2} \Omega = 1$.
Пусть $\gamma$ --- класс одного оборота $\{f_t\}$, и пусть $F \in \H$
--- гамильтониан, порождающий $\{f_t\}$. 
Как мы видели в \ref{6.3.C} выше,  $\max F = - \min F = \frac12$.
(Множитель $2\pi$ в \ref{6.3.C} исчез ввиду приведенной выше
нормировки $\Omega$.) 
Таким образом, $\nu_+ (\gamma) \le \frac12$.
Но на самом деле выполняется равенство!

\begin{thm}{Теорема}\label{9.1.A}
$\nu_+ (\gamma) = \frac12$.
\end{thm}

Гамильтонова петля $\{f_t \}$, представляющая класс $\gamma \ne 0$,
называется \rindex{замкнутая кратчайшая}\emph{замкнутой кратчайшей}, если $\length\{f_t \} =
\nu(\gamma)$.
Обратите внимание на то, что замкнутая кратчайшая не может быть
кратчайшей.

\begin{thm}[(\cite{LM2})]{Следствие} $\nu(\gamma) = 1$ и $\{f_t\}$ ---
  замкнутая кратчайшая.
\end{thm}

\parit{Доказательство.}
Прежде всего, $\length\{f_t\} =\frac12 - (-\frac12 ) = 1$ и,
следовательно, $\nu(\gamma) \le 1$. 
Поскольку $2\gamma = 0$, 
\[\nu_- (\gamma) = \nu_+ (-\gamma) = \nu_+ (\gamma) =\frac12.\]
Кроме того, $\nu(\gamma) \ge \nu_- (\gamma) + \nu_+ (\gamma) = 1$.
Отсюда $\nu(\gamma) = 1$ и $\{f_t\}$ --- замкнутая кратчайшая.

Теорема \ref{9.1.A} доказана в \ref{sec:9.4} ниже.
Обобщение на на $\CP^n$ при $n \ge 2$ дано в \cite{P3}.

\section[\texorpdfstring{Симплектические расслоения над $S^2$}{Симплектические расслоения над S²}]{Симплектические расслоения над $\bm{S^2}$}
\label{sec:9.2}

Пусть $(M, \Omega)$ --- замкнутое симплектическое многообразие.
Далее будем считать, что $H^1 (M, \RR) = 0$.%
\footnote{От этого предположения легко избавиться, см. теорию
  гамильтоновых симплектических расслоений в \cite{MS} и \cite{P4}.} 
Как следствие $\Ham(M, \Omega)$ совпадает со связной компонентой $\1$
в $\Symp(M, \Omega)$. 
Пусть $p \: P \to S^2$ --- гладкое расслоение со слоем $M$ со
следующей послойной симплектической структурой.
Для каждого $x \in S^2$ на $p^{-1} (x)$ задана симплектическая форма
$\Omega_x$ такая, что $\Omega_x$ гладко зависит от $x$ и $(p^{-1} (x),
\Omega_x)$  симплектоморфно $(M, \Omega)$.
Кроме того, мы всегда выбираем ориентацию на $S^2$ как часть данных
(поэтому $P$ также ориентировано).
Назовём $p\: P\to S^2$ \rindex{симплектическое расслоение}\emph{симплектическим расслоением} (подробнее
см. \cite{MS}). 

Каждая петля $\{f_t\}$ гамильтоновых диффеоморфизмов $M$ порождает
симплектическое расслоение.
Возьмём две копии единичного $2$-диска $D_+^2$, $D_-^2$ такие, что
$D_+^2$ имеет положительную ориентацию, а $D_-^2$ --- обратную.
Определим новое многообразие
\[P =  M  \times D_-^2 \cup_\psi M \times D_+^2,\]
где отображение склейки $\psi$ задаётся следующим образом
\[\psi \: M \times S^1 \to M \times S^1,\quad (z, t) \mapsto (f_t z, t).\]
Ясно, что $P$ имеет естественную структуру симплектического расслоения
над $S^2$, так как $f_t$ --- симплектоморфизмы (в добавок, $S^2$
получает ориентацию по построению).

Гомотопные петли приводят к изоморфным симплектическим расслоениям, то
есть существуют гладкие изоморфизмы, сохраняющие послойную
симплектическую структуру и ориентацию.
Кроме того, это построение можно обратить --- по заданному расслоению
$P \to S^2$ с выбранной тривиализацией над одной точкой можно
восстановить гомотопический класс $\gamma$.
Заметим также, что класс $\gamma = 0$ соответствует тривиальному
расслоению $S^2 \times (M, \Omega)$.
Оставляем доказательство этих утверждений читателю.
Будем писать $P = P(\gamma)$, где $\gamma$ --- гомотопический класс
$\{f_t\}$.

Давайте посмотрим на расслоение $P(\gamma)$ для одного оборота $S^2$.
Заметим, что база, и слой этого расслоения $S^2$.
Полезно будет отождествить $S^2$ с комплексной проективной прямой
$\CP^1$ и перейти к комплексной точке зрения.

Прежде чем продолжать, сделаем отступление о симплектической геометрии
комплексных проективных пространств. 
Пусть $E$ --- $2n$-мерное вещественное векторное пространство с
комплексной структурой $j$ 
(здесь $j$ --- линейное преобразование $E\to E$ такое, что $j^2 = -\1$),
скалярным произведением $g$ и симплектической формой~$\omega$, такой что
\[g(\xi, \eta ) = \omega(\xi, j\eta)\]
для всех $\xi, \eta \in E$ (ср. \ref{sec:4.1} выше).
Конечно, используя $j$, мы можем рассматривать $E$ как комплексное
векторное пространство. 
Такая пара $(\omega, g)$ называется эрмитовой структурой на
комплексном пространстве $(E, j)$.
Рассмотрим единичную сферу
\[S = \set{\xi \in E}{g(\xi, \xi) = 1}\]
и действие окружности на $S$, определённое формулой 
\[\xi \mapsto e^{2\pi jt} \xi,\quad t \in \RR/\ZZ.\]
Орбиты этого действия --- это в точности множества вида $S \cap l$,
где $l$ --- комплексная прямая в $E$. 
Таким образом, пространство орбит $S/S^1$ можно канонически
отождествить с \rindex{проективное пространство}\emph{комплексным проективным пространством} $\PP(E)$.
Действие сохраняет сужение $\omega$ на $T S$.
Обозначим через $\Omega$
\?{проекцию}{По-хорошему, надо бы ещё сказать, что касательные к
  орбитам действия лежат в ядре сужения $\omega$ на $\T S$. Иначе не
  понятно, что за проекция.} $\tfrac1\pi \omega$ на $\PP(E)$.
Эта форма замкнута.
Далее, воспользовавшись элементарной линейной алгеброй получаем, что
$\Omega$ невырождена, и, таким образом, мы получаем симплектическую
форму на $\PP(E)$. 

Форма $\Omega$ называется стандартной (или формой Фубини --- Штуди) на
$\PP(E)$, ассоциированной с эрмитовой структурой $(\omega, g)$ на
комплексном пространстве $(E, j)$.
Приведённое выше построение является частным случаем редукции \rindex{Марсден}Марсдена --- \rindex{Вайнштейн}Вайнштейна \cite{MS}, играющей ключевую роль в теории действий
групп на симплектических многообразиях.
Фактор $\tfrac1\pi$ выше выбран по следующей причине.

\begin{ex}{Упражнение}\label{9.2.A}
  Покажите, что интеграл от $\Omega$ по проективной прямой в $\PP(E)$
  равен $1$.
\end{ex}

Из теоремы \rindex{Мозер}Мозера \cite{MS} следует, что различные эрмитовы структуры
на $(E, j)$ порождают диффеоморфные симплектические формы на
$\PP(E)$. 

\begin{ex}{Упражнение}\label{9.2.B}
  Рассмотрим пространство $\CC^n$ со стандартной эрмитовой структурой
  (см. \ref{sec:4.1}).
  Покажите, что стандартная симплектическая форма на $\PP(\CC^n ) =
  \CP^{n-1}$ инвариантна относительно группы $\PU (n) = \U (n)/S^1$ и
  что эта группа действует на $\CP^{n-1}$ гамильтоновыми
  диффеоморфизмами.
  Покажите, что \?{}{$\1$;$-\1$ или $\1$,$-\1$ ?\\С: Запятая.}
  \[\PU (2) = \SU (2)/\{\1; -\1\}.\]
\end{ex}
 
\begin{ex}{Упражнение}\label{9.2.C}
Рассмотрим петлю проективных унитарных преобразований $\CP^1$,которая в однородных координатах $(z_1 \z: z_2)$ на $\CP^1$ задается формулой
\[(z_1 : z_2 ) \mapsto (e^{-2\pi it} z_1 : z_2 ),\quad t \in [0; 1].\]
Покажите, что эта петля представляет собой нетривиальный элемент $\pi_1 (\Ham(\CP^1 ))$.
\emph{Подсказка:} используйте канонический изоморфизм $\SU (2)/\{\1; -\1\}\to \SO(3)$ (см. \cite{DFN}).  
\end{ex}

У приведенного выше построения существует естественный
«параметрический» вариант, который даёт симплектические расслоения со
слоем $\CP^{n-1}$, со стандартной симплектической формой. 
Пусть $E \to S^2$ --- комплексное векторное расслоение ранга $n$, и
пусть $\PP(E)$ --- его проективизация. 
Каждая эрмитова структура на $E$ порождает послойную симплектическую
форму $\Omega_x$ на $\PP(E)$. 
Здесь $\Omega_x$ равна стандартной симплектической форме на слое
$\PP(E_x)$.
Различные эрмитовы структуры приводят к изоморфным симплектическим
расслоениям. 

Вернёмся теперь к симплектическому расслоению $P(\gamma)$, где
$\gamma$ --- нетривиальный элемент $\pi_1 (\Ham(S^2 ))$. 
Пусть $T \to \CP^1$ --- тавтологическое расслоение,
то есть его слой над комплексной прямой в $\CC^2$ есть сама прямая. 
Пусть $C = \CC \times \CP^1$ --- тривиальное расслоение.
 
\begin{ex}{Упражнение}\label{9.2.D}
Докажите, что симплектическое расслоение $P(\gamma)$ изоморфно $\PP(T
\oplus C)$. 
\emph{Подсказка:} представьте базу $\CP^1$ в виде объединения двух дисков
\?{}{до и после они $D^2_\pm$} 
\[D_- = \set{(x_0 : x_1 ) \in \CP^1}{ |x_0 /x_1 | \le 1}\]
и
\[D_+ = \set{(x_0 : x_1 ) \in \CP^1}{|x_1 /x_0 | \le 1}.\]
Рассмотрите координату $t$ на окружности $S^1 = \partial D_+ =
\partial D_-$ такую, что $x_1 /x_0 = e^{2\pi it}$. 
Расслоение $T \oplus C$ можно тривиализовать над $D_-$ и $D_+$ так,
что функция перехода $S^1 \times \CC^2 \to S^1 \times \CC^2$ имеет вид  
\[(t, z_1, z_2 ) \mapsto (t, e^{-2\pi it} z_1, z_2 )\]
(подробное описание тавтологического линейного расслоения дано в
\cite{GH}). 
Теперь результат следует из \ref{9.2.C} выше.
\end{ex}

\begin{ex}{Упражнение}\label{9.2.E}
Покажите, что $\PP(T \oplus C)$ биголоморфно эквивалентно комплексному
раздутию $\CP^2$ в одной точке. 
Расслоение получается собственным прообразом пучка прямых, проходящих
через точку раздутия. 
\end{ex}

\section{Симплектические связности}

Пусть $p\: P\to S^2$ --- симплектическое расслоение со слоем $(M,\Omega)$.
Связность $\sigma$ на $P$ (то есть поле двумерных подпространств,
трансверсальных слоям, см. рис. \ref{pic-9}) называется
симплектической, если её параллельный перенос сохраняет послойную
симплектическую структуру. 
Можно показать, что всякое симплектическое расслоение допускает
симплектическую связность, см. \cite{GLS,MS}. 
\rindex{симплектическая связность}
\rindex{кривизна симплектической связности}

{

\begin{wrapfigure}[10]{o}{40 mm}
\vskip-0mm
\centering
\includegraphics{mppics/pic-9}
\caption{}\label{pic-9}
\vskip0mm
\end{wrapfigure}

\begin{ex*}{Пример}
  Пусть $E \to S^2$ --- комплексное векторное расслоение с эрмитовой
  метрикой.  Каждая эрмитова связность на $E$ индуцирует
  симлектическую связность на проективизированном расслоении $\PP(E)
  \to S^2$.
\end{ex*}

Давайте вспомним определение кривизны связности.
Для данных $x\in S^2$ и $\xi$, $\eta \z\in \T_x S^2$, продолжим $\xi$ и $\eta$ до полей в окрестность $x$,
пусть $\tilde\xi$ и $\tilde\eta$ --- горизонтальные поднятия
полученных векторных полей. 
По определению, кривизна $\rho^\sigma$ в точке $x$ есть 2-форма,
принимающая значения в алгебре Ли векторных полей на слое
$p^{-1}(x)$. 
Она определяется как $\rho^\sigma (\xi, \eta) \z= ([\tilde\xi,
  \tilde\eta])^\vert$, где «$\vert$» обозначает проекцию $[\tilde\xi,
  \tilde\eta]$ на слой $p^{-1} (x)$. 
Таким образом, если $\sigma$ --- симплектическая связность, то
$\rho^\sigma (\xi, \eta)$ лежит в алгебре Ли группы
$\Symp(p^{-1}(x))$, и поскольку $H^1 (M, \RR) = 0$, кривизна
$\rho^\sigma (\xi, \eta)$ является гамильтоновым векторным полем на
слое. 
Отождествляя гамильтоновы векторные поля с нормализованными
гамильтонианами, можно рассматривать $\rho^\sigma (\xi, \eta)$ как
нормализованный гамильтониан 
\[p^{-1} (x) \to \RR.\]
Зафиксируем форму площади $\tau$ на $S^2$ такую, что $\int_{S^2} \tau
= 1$ (здесь нам потребовалась ориентация $S^2$). 
Поскольку каждая 2-форма на $S^2$ в любой точке
пропорциональна $\tau$, мы можем написать
\[\rho^\sigma = L^\sigma \tau,\]
где $L^\sigma$ --- функция на $P$.

}

Теория симплектических связностей была развита недавно в работах
Гиймена, Лермана и Штернберга \cite{GLS,MS}. 
Основными объектами этой теории являются \?{форма спаривания}{coupling
  form} симплектической связности и \?{класс спаривания}{coupling
  class} симплектического расслоения. 
Для точки $(x, z) \in P$ можно написать разложением по симплектической
связности $\sigma$: 
\[\T_{(x,z)} P = \T_z p^{-1} (x) \oplus \T_x S^2.\]
Определим \rindex{форма спаривания}\emph{форму спаривания} $\delta^\sigma$ связности $\sigma$
как 2-форму на $P$, заданную формулой  
\[\delta^\sigma (v \oplus \xi, w \oplus \eta) = \Omega_x (v, w) -
\rho^\sigma (\xi, \eta)(z).\] 
Здесь $z \in p^{-1} (x)$, $v$, $w \in \T_z p^{-1} (x)$ и $\xi$, $\eta
\in \T_x S^2$, а $\rho^\sigma (\xi, \eta)$ рассматривается как функция
на слое $p^{-1} (x)$. 

Оказывается форма спаривания замкнута.
Обозначим через $c$ её класс когомологий в $H^2 (P, \RR)$.
Очевидно, что сужение $c$ на любой слой $p^{-1}(x)$ совпадает с
\?{классом}{Вроде прямо с самой формой совпадает(?)} $[\Omega_x]$
симплектической формы.  
Кроме того, можно доказать, что $c^{n+1} = 0$, где $2n = \dim M$.
Следующий результат показывает, что класс $c$ однозначно определяется
этими двумя свойствами. 

\begin{thm}[(См. \cite{GLS,MS})]{Теорема}\label{9.3.A}\rindex{Штернберг}
  Класс $c$ --- это единственный класс когомологий в $H^2 (P, \RR)$,
  такой что $c|_{\fiber} = [\Omega_x]$ и $c^{n+1} \z= 0$.
\end{thm}

В частности, $c$ --- инвариант симплектического расслоения $P$, не
зависящий от выбора связности $\sigma$. 
Мы называем $c$ \rindex{класс спаривания}\emph{классом спаривания} $P$.
Подробные доказательства всех этих результатов даны в \rindex{Гиймен}\cite{GLS} и \cite{MS}.

Следующее построение играет важную роль в нашем подходе к спектру длин.

\parbf{Построение слабого спаривания.}
(\cite{GLS,MS})\rindex{Лерман}
Для достаточно малого $\epsilon > 0$ существует гладкое семейство
замкнутых 2-форм $\omega_t$ на $P$ с $t \in [0;\epsilon)$ такое, что 
\begin{itemize}
\item $\omega_0 = p^\ast \tau$
\item $[\omega_t] = tc + p^\ast [\tau]$
\item $\omega_t|_{\fiber} = t\Omega_x$
\item $\omega_t$ симплектическая при всех $t > 0$.
\end{itemize}

Положим $\epsilon(P) = \sup \{\epsilon\}$, где точная верхняя грань
берётся по всем таким деформациям. 
Эта величина измеряет, насколько сильным может быть слабое спаривание.
Заметим, что $\epsilon(P) = +\infty$ для тривиального расслоения $P =
M \times S^2$.
Таким образом, в некотором смысле $\epsilon(P)$ измеряет
нетривиальность расслоения.

Есть ещё один способ измерения нетривиальности расслоения, который
работает и для расслоений с другими структурными группами.
Он был предложен \rindex{Громов}Громовым \cite{G2} для унитарных векторных
расслоений, здесь мы дадим симплектическую версию.
Идея состоит в том, чтобы измерить минимально возможную норму кривизны
симплектической связности на $P$.
Введём следующее понятие.

\begin{ex*}{Определение}
\[\chi_+ (P) = \sup_\sigma \frac1{\max_P L^\sigma}\]  --- (положительная часть) \rindex{$K$-площадь}\emph{симплектической $K$-площади} $P$.
Здесь точная верхняя грань берётся по всем симплектическим связностям на $P$, а $L^\sigma$ определяется равенством $\rho^\sigma = L^\sigma \tau$.
\end{ex*}

\begin{ex*}{Упражнение}
Докажите, что обе введенные выше величины, $\epsilon(P)$ и $\chi_+
(P)$, не зависят от выбора формы площади $\tau$ на $S^2$ с $\int_{S^2}
\tau = 1$. 
\emph{Подсказка:} сначала применим теорему Мозера \cite{MS}, о том, что для
любых двух таких форм, скажем, $\tau_1$ и $\tau_2$, существует
диффеоморфизм $a\:S^2\to S^2$, изотопный единице и такой, что $a^\ast
\tau_2 = \tau_1$.
Затем поднимем $a$ до послойно симплектического диффеоморфизма $A$
расслоения $P$, это означает, что $p(A(z)) = a(p(z))$ для всех $z \in
P$ и сужение $A_x$ диффеоморфизма $A$ на любой слой $p^{-1} (x)$
удовлетворяет $(A_x)^\ast \Omega_{a(x)} = \Omega_x$.
Такой подъём можно построить с помощью любой симплектической связности
на $P$. 
Обратите внимание, что $A$ переводит любую деформацию слабого
спаривания для $\tau_2$, в деформацию слабого спаривания для
$\tau_1$.
Это доказывает, что $\epsilon(P)$ не зависит от выбора формы площади.
Далее, $A$ действует на пространстве симплектических связностей на $P$
по правилу $\sigma \mapsto A_\ast \sigma$.
Покажите, что 
\[\rho^{A_\ast \sigma} (a_\ast \xi, a_\ast \eta)(Az) = \rho^\sigma (\xi, \eta)(z)\]
для каждой точки $z \in P$ и любой пары векторов $\xi$, $\eta \in
\T_{p(z)} S^2$.
Тот факт, что $\chi_+(P)$ не зависит от выбора формы площади, является
простым следствием из этой формулы.
\end{ex*}

\begin{thm}[(\cite{P4})]{Теорема}\label{9.3.B}
  Пусть $P = P(\gamma)$.
  Тогда $\epsilon(P) = \chi_+ (P) \z= \frac1{\nu_+(\gamma)}.$
\end{thm}

Мы докажем более слабое утверждение, а именно, что $\epsilon(P) \z\ge
\chi_+ (P)$, но этого будет достаточно для доказательства теоремы
\ref{9.1.A}.

\parit{Доказательство неравенства $\epsilon(P) \ge \chi_+ (P)$.}
Пусть $\sigma$ --- симплектическая связность.
Рассмотрим 
\[\omega_t = p^\ast \tau + t\delta^\sigma,\]
где $\delta^\sigma$ --- форма спаривания.
В точке $(x, z) \in P$ имеем 
\[\omega_{t,(x,z)} = t\Omega_x \oplus -tL^\sigma (x, z)\tau + \tau
= t\Omega_x \oplus (1 - tL^\sigma (x, z))\tau.\]
Ясно, что $\omega_t$ удовлетворяет первым трём свойствам в построении
слабого спаривания.
Форма $\omega_t$ является симплектической пока $1 \z- tL^\sigma (x, z)
> 0$ или, что то же самое, пока $L^\sigma (x, z) < \frac1t$ при всех
$x$,~$z$.
Это условие означает, что 
$\max_P L^\sigma (x, z) < \frac1t$
или 
\[\frac1{\max_P L^\sigma (x, z)} > t.\]
Выберем теперь произвольное $\kappa > 0$ и симплектическую связность
$\sigma$ такие, что  
\[\frac{1}{\max_P L^\sigma(x, z)} > \chi_+ (P) - \kappa.\]
Таким образом, существует деформация спаривания $\omega_t$ для $t \in
[0;\chi_+ (P) \z- \kappa)$.
То есть для всех $\kappa > 0$, $\epsilon(P) \ge \chi_+ (P) - \kappa$,
и мы заключаем, что
\[\epsilon(P) \ge \chi_+ (P).\]
\qeds

Заметим, что мы доказали существование деформации слабого спаривания.

\parit{Доказательство неравенства $\chi_+ (P) \ge \frac{1}{\nu_+ (\gamma)}$.}
Доказательство основано на следующем упражнении.

\begin{ex*}{Упражнение}
  Пусть $p\: P \to S^2$ --- симплектическое расслоение.
  Пусть $\omega$ --- замкнутая 2-форма на $P$ такая, что
  $\omega|_{\fiber} = \Omega_x$.
  Положим 
  \[\sigma(x,z) = \set{\xi \in \T_{(x,z)} P }{ i_\xi \omega =
    0\ \text{на}\  \T_z p^{-1} (x)}.\] 
  Покажите, что $\sigma$ определяет симплектическую
  связность на $P$. 
\end{ex*}

Пусть $\{f_t \}$, $t \in [0;1]$, --- произвольная петля гамильтоновых
диффеоморфизмов, порождённая нормализованным гамильтонианом $F \z\in
\H$.
Зафиксируем полярные координаты $u \in (0, 1]$ (радиус) и $t \z\in S^1
= \RR/\ZZ$ (нормированный угол) на $D^2$. 
Возьмём монотонную функцию срезки $\phi(u)$ такую, что $\phi(u) = 0$
вблизи $u = 0$ и $\phi(u) = 1$ вблизи $u = 1$. 
Положим 
\[P = M \times D_-^2 \cup_\psi M \times D_+^2,\]
где $\psi(z, t) = (f_t z, t)$.
Определим замкнутую 2-форму $\omega$ на $P$ равенством 
\[\omega=
\begin{cases}
\quad\Omega&\text{на}\ M\times D^2_+,
\\
\quad\Omega+d(\phi(u)H_t(z))\wedge dt&\text{на}\ M\times D^2_-,
\end{cases}
\]
где $H_t (z) = F (f_t z , t)$.


\begin{ex*}{Упражнение}
  Докажите, что $\omega$ хорошо определена, то есть, покажите, что
  $\psi^\ast \Omega = \Omega + dH_t \wedge dt$ (это можно проделать
  прямым вычислением).
\end{ex*}

Вычислим кривизну $\rho^\sigma$ симплектической связности 2,
ассоциированной с $\omega$. 
Заметим, что $\rho^\sigma$ обращается в нуль на $D_+^2$ (поскольку
$\Omega$ индуцирует плоскую связность на $D_+^2$), поэтому остается
вычислить $\rho^\sigma$ на $D_-^2$. 
Заметим, что $\rho^\sigma = 0$ вблизи $0$ на $D_-^2$, и это хорошо, поскольку мы избегаем сингулярности в нуле.
Чтобы вычислить кривизну, мы должны горизонтально поднять $\tfrac{\partial}{\partial u}$ и $\tfrac{\partial}{\partial t}$ в точках $(x, z) \in M \times D_-^2$.

Пусть горизонтальный подъём $\tfrac{\partial}{\partial u}$ имеет вид 
\[\widetilde{\frac{\partial}{\partial u}}  =\frac{\partial}{\partial u} + v
\quad\text{для некоторого}\quad
v \in \T_z p^{-1} (x).\]
По определению связности, $\omega(\widetilde{\frac{\partial}{\partial u}},w)=0$ для всех $w\in \T_zM$, значит 
\[0=\omega\left(\widetilde{\frac{\partial}{\partial u}}, w\right) = \omega(v, w) = \Omega(v, w)\]
для всех $w \in \T_z p^{-1} (x)$.
Из невырожденности $\Omega$ следует, что $v = 0$ и, значит, 
\[\widetilde{\frac{\partial}{\partial u}}=\frac{\partial}{\partial u}\]
Полагая, что
\[\widetilde{\frac{\partial}{\partial t}}=\frac{\partial}{\partial t}+v
\quad\text{для другого}\quad
v \in \T_z p^{-1} (x),\]
как и раньше получаем, что для всех $w \in \T_z p^{-1} (x)$
\[0=\omega\left(\widetilde{\frac{\partial}{\partial t}}, w\right) = \omega(\frac{\partial}{\partial t}+v, w) = \Omega(v, w)- d(\phi(u)H_t)(w).
\]
Итак, 
$i_v \Omega = d(\phi(u)H_t)$ и, следовательно, 
$v \z= - \sgrad \phi(u)H_t \z= -\phi(u) \sgrad H_t$,
и можно сделать вывод, что 
\[\widetilde{\frac{\partial}{\partial t}}=\frac{\partial}{\partial t}-\phi(u) \sgrad H_t.\]
Вычислив кривизну, получим 
\[\rho^\sigma
\left(\frac{\partial}{\partial t},\frac{\partial}{\partial u}\right)
=
\left[\frac{\partial}{\partial t}-\phi(u)\sgrad H_t,\frac{\partial}{\partial u}\right]^\vert
=
\phi'(u)\sgrad H_t.\]
(На самом деле коммутатор уже вертикальное векторное поле.)
Переходя от векторных полей к гамильтонианам в определении кривизны, мы видим, что
\[\rho^\sigma\left(\frac{\partial}{\partial t},\frac{\partial}{\partial u}\right)
=\phi'(u) H_t.
\]

Зафиксируем $\kappa > 0$ и выберем форму площади на $D_-^2$ вида $(1
\z- \kappa)dt \wedge du$ (напомним, что $D_-^2$ имеет обратную
ориентацию).

Расширим её до формы площади на $D_+^2$ такой, что $\Area (D_+^2) = \kappa$.
Полученную форму обозначим $\tau$.
Мы видим, что $\rho^\sigma = L^\sigma (u, t, z)\tau$, где 
\[
L^\sigma(u,t,z)=
\begin{cases}
0&\text{on\ } M\times D^2_+
\\
\frac{\phi'(u)H_t(z)}{1-\kappa}&\text{on\ } M\times D^2_-
\end{cases}
\]
Наконец, выберем $\phi$ так, что $\phi' (u) \le 1 + \kappa$, и
$\{f_t\}$ так, что $\max_z F_t \z= \max_z H_t \le \nu_+ (\gamma) +
\kappa$. 
Получаем
\[\max_P L^\sigma \le  \frac{1+\kappa}{1-\kappa} (\nu_+ (\gamma) + \kappa),\]
значит 
\begin{align*}
\chi_+ (P) &= \sup_\sigma \frac1{\max_PL^\sigma}\ge
\\
&\ge\frac{1-\kappa}{(1 + \kappa)(\nu_+ (\gamma) + \kappa)}. 
\end{align*}
Поскольку $\kappa$ произвольно, мы доказали, что 
\[\chi_+ (P) \ge \frac{1}{\nu_+ (\gamma)}.\]
\qeds

\section{Приложение к спектру длин}\label{sec:9.4}

Последний шаг в доказательстве теоремы \ref{9.1.A} состоит в следующей
оценке на $\epsilon(P)$, которая будет обсуждаться в следующей главе.


\begin{thm}{Теорема}\label{9.4.A}
Пусть $P = P(\gamma)$, где $\gamma$ --- поворот $S^2$ на 1 оборот.
Тогда
\[\epsilon(P) \le 2.\]
\end{thm}

\parit{Доказательство \ref{9.1.A}.}
Мы знаем, что $\nu_+ (\gamma) \le \tfrac12$.
С другой стороны, из \ref{9.3.B} и \ref{9.4.A} получаем, что
\[2 \ge \epsilon(P) \ge \chi_+ (P) \ge\frac1{\nu_+(\gamma)},\]
поэтому $\nu_+ (\gamma) = \tfrac12$ и $\epsilon(P) = \chi_+ (P) = 2$.
\qeds

\begin{ex}{Упражнение}
Рассмотрим голоморфные линейные расслоения $C$ и $T$ над $\CP^1$ ---
соответственно тривиальное и тавтологическое (см. \ref{sec:9.2}). 
Пусть $\nabla_C$ --- естественная плоская связность на~$C$.
Существует единственная связность, скажем, $\nabla_T$, на $T$, которая
сохраняет каноническую эрмитову структуру  на $T$ полученную из
$\CC^2$, и такую, что её $(0,1)$-часть совпадает с
$\bar\partial$-оператором (см. \cite{GH}). 
Рассмотрим симплектическое расслоение $P = \PP(T \oplus C)$ и
обозначим через $\sigma$ связность на $P$, полученную из $\nabla T
\oplus \nabla C$. 
Очевидно, что $\sigma$ симплектична (на самом деле её параллельный
перенос сохраняет как симплектическую, так и комплексную структуру на
слоях). 
Пусть $\tau$ есть форма площади Фубини --- Штуди на $\CP^1$,
определенная в \ref{sec:9.2} выше. 
Докажите, что 
\[\frac1{\max_P L^\sigma}=2,\]
где $\rho^\sigma = L^\sigma \tau$.
В частности, $\sigma$ является связностью с минимально возможной
кривизной, где кривизна «меряется» с  помощью $\tau$.  
\end{ex}
