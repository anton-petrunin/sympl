\chapter{Спектр длин}

В этой главе мы опишем метод вычисления спектра длин в хоферовской геометрии.
Наш подход основан на теории симплектических расслоений над двумерной сферой.

\section{Положительная и отрицательная части хоферовской нормы}

Пусть $(M, \Omega)$ --- замкнутое симплектическое многообразие.
Для $\gamma \in \pi_1 (\Ham(M, \Omega))$ положим 
\begin{align*}
\nu_+ (\gamma) &= \inf_F \int_0^1 \max_x F (x, t)dt = \inf_F \max_{x,t}F (x, t),
\\ 
\nu_- (\gamma) &= \inf_F \int_0^1 -\min_x F (x, t)dt = \inf_F \min_{x,t}F (x, t),
\end{align*}
где точная нижняя грань берётся по всем нормированным периодическим гамильтонианам $F \in \H$, порождающим петлю в классе~$\gamma$.
\?{Правильность нашего определения}{Лучше «равенство в определениях»} доказывается точно так же, как и лемма \ref{5.1.C} выше.

\begin{thm}{Упражнение}
Докажите, что $\nu_+ (\gamma) = \nu_- (-\gamma)$ и $\nu(\gamma) \z\ge \nu_- (\gamma) + \nu_+ (\gamma)$
(сравните с открытой задачей в \ref{sec:2.4} выше).
\end{thm}

Вычисление $\nu_+ (\gamma)$ оказывается нетривиальным даже в следующем простейшем случае.
Рассмотрим $(S^2, \Omega)$ нормализованное так, что $\int_{S^2} \Omega = 1$.
Пусть $\gamma$ --- класс одного оборота $\{f_t\}$, и пусть $F \in \H$ --- гамильтониан, порождающий $\{f_t\}$.
Как мы видели в \ref{6.3.C} выше,  $\max F = - \min F = \frac12$.
(Множитель $2\pi$ в \ref{6.3.C} исчез ввиду приведенной выше нормировки $\Omega$.)
Таким образом, $\nu_+ (\gamma) \le \frac12$.
Но на самом деле выполняется равенство!

\begin{thm}{Теорема}\label{9.1.A}
$\nu_+ (\gamma) = \frac12$.
\end{thm}

Гамильтонова петля $\{f_t \}$, представляющая класс $\gamma \ne 0$, называется замкнутой кратчайшей, если $\length\{f_t \} = \nu(\gamma)$.
Обратите внимание на то, что замкнутая кратчайшая не может быть кратчайшей.

\begin{thm}[(\cite{LM2})]{Следствие} $\nu(\gamma) = 1$ и $\{f_t\}$ --- замкнутая кратчайшая.
\end{thm}

\parit{Доказательство.}
Прежде всего, $\length\{f_t\} =\frac12 - (-\frac12 ) = 1$ и, следовательно, $\nu(\gamma) \le 1$.
Поскольку $2\gamma = 0$, 
\[\nu_- (\gamma) = \nu_+ (-\gamma) = \nu_+ (\gamma) =\frac12.\]
Кроме того, $\nu(\gamma) \ge \nu_- (\gamma) + \nu_+ (\gamma) = 1$.
Отсюда $\nu(\gamma) = 1$ и $\{f_t\}$ --- замкнутая кратчайшая.

Теорема \ref{9.1.A} доказана в \ref{sec:9.4} ниже.
Обобщение на на $\CP^n$ при $n \ge 2$ дано в \cite{P3}.


\section[\texorpdfstring{Симплектические расслоения над $S^2$}{Симплектические расслоения над S²}]{Симплектические расслоения над $\bm{S^2}$}

Пусть $(M, \Omega)$ --- замкнутое симплектическое многообразие.
Далее будем считать, что $H^1 (M, \RR) = 0$.%
\footnote{От этого предположения нетрудно избавиться, см. теорию гамильтоновых симплектических расслоений в \cite{MS} и \cite{P4}.}
Как следствие $\Ham(M, \Omega)$ совпадает со связной компонентой $\1$ в $\Symp(M, \Omega)$.
Пусть $p \: P \to S^2$ --- гладкое расслоение со слоем $M$, снабжённое следующей послойной симплектической структурой.
Для каждого $x \in S^2$ на $p^{-1} (x)$ задана симплектическая форма $\Omega_x$ такая, что $\Omega_x$ гладко зависит от $x$ и $(p^{-1} (x), \Omega_x)$  симплектоморфно $(M, \Omega)$.
Кроме того, мы всегда выбираем ориентацию на $S^2$ как часть данных (поэтому $P$ также ориентировано).
Назовём $p\: P\to S^2$ симплектическим расслоением (подробнее см. \cite{MS}).

Каждая петля $\{f_t\}$ гамильтоновых диффеоморфизмов $M$ порождает симплектическое расслоение.
Возьмём две копии единичного $2$-диска $D_+^2$, $D_-^2$ такие, что $D_+^2$ имеет положительную ориентацию, а $D_-^2$ --- обратную.
Определим новое многообразие
\[P =  M  \times D_-^2 \cup_\psi M \times D_+^2,\]
где $\psi$ --- отображение склейки 
\[\psi \: M \times S^1 \to M \times S^1,\quad (z, t) \mapsto (f_t z, t).\]
Ясно, что $P$ имеет естественную структуру симплектического расслоения над $S^2$, так как $f_t$ --- симплектоморфизмы (и, кроме того, $S^2$ получает ориентацию согласно построению).

Гомотопические петли приводят к изоморфным симплектическим расслоениям, то есть существуют гладкие изоморфизмы, сохраняющие послойную симплектическую структуру и ориентацию.
Кроме того, это построение можно обратить --- по заданному расслоению $P \to S^2$ с выбранной тривиализацией над одной точкой можно восстановить гомотопический класс $\gamma$.
Заметим также, что класс $\gamma = 0$ соответствует тривиальному расслоению $S^2 \times (M, \Omega)$.
Оставляем доказательство этих утверждений читателю.
Будем писать $P = \PP(\gamma)$, где $\gamma$ --- гомотопический класс $\{f_t\}$.

Давайте посмотрим на расслоение $\PP(\gamma)$ для одного оборота $S^2$.
Заметим, что база, и слой этого расслоения $S^2$.
Полезно будет отождествить $S^2$ с комплексной проективной линией $\CP^1$ и перейти к комплексной постановке.

Прежде чем продолжать, сделаем отступление о симплектической геометрии комплексных проективных пространств.
Пусть $E$ --- $2n$-мерное вещественное векторное пространство с комплексной структурой $j$
(здесь $j$ --- линейное преобразование $E\to E$ такое, что $j^2 = -\1$),
скалярным произведением $g$ и симплектической формой~$\omega$, такой что
\[g(\xi, \eta ) = \omega(\xi, j\eta)\]
для всех $\xi, \eta \in E$ (ср. \ref{sec:4.1} выше).
Конечно, используя $j$, мы можем рассматривать $E$ как комплексное векторное пространство.
Такая пара $(\omega, g)$ называется эрмитовой структурой на комплексном пространстве $(E, j)$.
Рассмотрим единичную сферу
\[S = \set{\xi \in E}{g(\xi, \xi) = 1}\]
и действие окружности на $S$, определённое формулой 
\[\xi \to e^{2\pi jt} \xi,\quad t \in \RR/\ZZ.\]
Орбиты этого действия --- это в точности множества вида $S \cap l$, где $l$ --- комплексная прямая в $E$.
Таким образом, пространство орбит $S/S^1$ можно канонически отождествить с комплексным проективным пространством $\PP(E)$.
Действие сохраняет сужение $\omega$ на $T S$.
Обозначим через $\Omega$ проекцию $\tfrac1\pi \omega$ на $\PP(E)$.
Эта форма замкнута.
Далее, воспользовавшись элементарной линейной алгеброй получаем, что $\Omega$ невырождена, и, таким образом, мы получаем симплектическую форму на $\PP(E)$.

Форма $\Omega$ называется стандартной (или формой Фубини --- Штуди) на $\PP(E)$, ассоциированной с эрмитовой структурой $(\omega, g)$ на комплексном пространстве $(E, j)$.
Приведённая выше конструкция является частным случаем редукции Марсдена --- Вайнштейна \cite{MS}, играющей ключевую роль в теории действий групп на симплектических многообразиях.
Фактор $\tfrac1\pi$ выше выбран по следующей причине.

\begin{thm}{Упражнение}\label{9.2.A}
Покажите, что интеграл от $\Omega$ по проективной прямой в $\PP(E)$ равен $1$.
\end{thm}

Из теоремы Мозера \cite{MS} следует, что различные эрмитовы структуры на $(E, j)$ порождают диффеоморфные симплектические формы на $\PP(E)$.

\begin{thm}{Упражнение}\label{9.2.B}
Рассмотрим пространство $\CC^n$ со стандартной эрмитовой структурой (см. \ref{sec:4.1}).
Покажите, что стандартная симплектическая форма на $\PP(\CC^n ) = \CP^{n-1}$ инвариантна относительно группы $\PU (n) = \U (n)/S^1$ и что эта группа действует на $\CP^{n-1}$ гамильтоновыми диффеоморфизмами.
Покажите, что \?{}{; или , ?}
\[\PU (2) = \SU (2)/\{\1; -\1\}.\]

\end{thm}

\begin{thm}{Упражнение}\label{9.2.C}
Рассмотрим петлю проективных унитарных преобразований $\CP^1$, которая в однородных координатах $(z_1 \z: z_2)$ на $\CP^1$ задается формулой 
\[(z_1 : z_2 ) \to (e^{-2\pi it} z_1 : z_2 ),\quad t \in [0; 1].\]
Покажите, что эта петля представляет собой нетривиальный элемент $\pi_1 (\Ham(\CP^1 ))$.
Подсказка: используйте канонический изоморфизм $\SU (2)/\{\1; -\1\} \to \SO(3)$ (см. \cite{DFN}).
\end{thm}

У приведенного выше построения существует естественный «параметрический» вариант, который даёт симплектические расслоения со слоем $\CP^{n-1}$, со стандартной симплектической формой.
Пусть $E \to S^2$ --- комплексное векторное расслоение ранга $n$, и пусть $\PP(E)$ --- его проективизация.
Каждая эрмитова структура на $E$ порождает послойную симплектическую форму $\Omega_x$ на $\PP(E)$.
Здесь $\Omega_x$ равна стандартной симплектической форме на слое $\PP(E_x)$.
Различные эрмитовы структуры приводят к изоморфным симплектическим расслоениям.

Вернёмся теперь к симплектическому расслоению $\PP(\gamma)$, где $\gamma$ --- нетривиальный элемент $\pi_1 (\Ham(S^2 ))$.
Пусть $T \to \CP^1$ --- тавтологическое линейное расслоение,
то есть его слой над комплексной прямой в $\CC^2$ есть сама прямая.
Пусть $C = \CC \times \CP^1$ --- тривиальное расслоение.

\begin{thm}{Упражнение}\label{9.2.D}
Докажите, что симплектическое расслоение $\PP(\gamma)$ изоморфно $\PP(T \oplus C)$.
Подсказка: представьте базу $\CP^1$ в виде объединения двух дисков \?{}{раньше они были $D^2_\pm$}
\[D_- = \set{(x_0 : x_1 ) \in \CP^1}{ |x_0 /x_1 | \le 1}\]
и
\[D_+ = \set{(x_0 : x_1 ) \in \CP^1}{|x_1 /x_0 | \le 1}.\]
Рассмотрите координату $t$ на окружности $S^1 = \partial D_+ = \partial D_-$ такую, что $x_1 /x_0 = e^{2\pi it}$.
Расслоение $T \oplus C$ можно тривиализовать над $D_-$ и $D_+$ так, что функция перехода $S^1 \times \CC^2 \to S^1 \times \CC^2$ имеет вид 
\[(t, z_1, z_2 ) \to (t, e^{-2\pi it} z_1, z_2 )\]
(подробное описание тавтологического линейного расслоения дано в \cite{GH}).
Теперь результат следует из \ref{9.2.C} выше.
\end{thm}

\begin{thm}{Упражнение}\label{9.2.E}
Покажите, что $\PP(T \oplus C)$ биголоморфно эквивалентно комплексному раздутию $\CP^2$ в одной точке.
Расслоение получается \?{подходящим}{proper} преобразованием пучка прямых, проходящих через точку раздутия.
\end{thm}
