\chapter{Рост и динамика}

В этой главе мы обсуждаем асимптотическое геометрическое поведение однопараметрических подгрупп в группе гамильтоновых диффеоморфизмов.
Мы описываем связь между геометрией и инвариантными торами в классической механике.

\section{Инвариантные торы в классической механике}

Инвариантные лагранжевые торы гамильтоновых динамических систем играют важную роль в классической механике.
Начнём с препятствия к существованию инвариантных торов, которое исходит из геометрии группы гамильтоновых диффеоморфизмов (см. \ref{8.1.C} ниже).
Рассмотрим $n$-мерный тор $\TT^n$ с евклидовой метрикой $ds^2 \z= \sum^n_{j = 1} dq_j^2$.
Евклидов геодезический поток описывается гамильтоновой системой на кокасательном расслоении $T^\ast \TT^n$ со стандартной симплектической формой $\Omega = dp \wedge dq$.
Функция Гамильтона задается формулой $F (p, q) = \tfrac12 | p |^2$.
Решая гамильтонову систему 
\[
\begin{cases}
\dot p &= 0,\\
\dot q &= p,
\end{cases}
\]
находим гамильтонов поток $f_t (p, q) = (p, q + pt)$.
Опишем динамику этого потока.
Каждый тор $\{p = a\}$ инвариантен относительно $\{f_t\}$, и, более того, ограничение потока на каждый такой тор представляет собой просто (квази)-периодическое движение $q \to q + at$.
Все эти торы гомологичны нулевому сечению кокасательного расслоения.
Геометрически они соответствуют семействам параллельных евклидовых прямых на $\TT^n$.
Приведенный выше поток принадлежит к важному классу не зависящих от времени гамильтоновых систем, называемых интегрируемыми системами (см. \cite{Ar}).
Интегрируемые системы можно охарактеризовать тем, что их энергетические уровни расслоены (с точностью до нуля) инвариантными торами средней размерности, на которых движение (квази-)периодическое.
Традиционно они считаются простейшей динамикой в классической механике.
Возникает естественный вопрос: что происходит с инвариантными торами при возмущении системы.
Теория Колмогорова--Арнольда--Мозера (или, КАМ-теория, см. \cite{Ar}) говорит нам, что при малых возмущений выживает б\'{о}льшая часть инвариантных торов.
Однако необходимо потребовать, чтобы вектор вращения был «достаточно иррациональным».
В частности, это означает, что если в некоторых угловых координатах $\theta$ на инвариантном торе динамика задается формулой $\dot\theta = a$, где $a = (a_1 ,\dots, a_n) \in \RR^n$, то $a_1 ,\dots, a_n$ должны быть линейно независимыми над $\QQ$.
При б\'{о}льших возмущениях «топологически существенные» торы могут исчезнуть.
Например, можно продеформировать евклидову метрику на $\TT^n$ до римановой метрики, геодезический поток которой не имеет инвариантных торов, несущих квазипериодическое движение и гомологичных нулевому сечению (см. \cite{AL}).
Заметим, что в нашем начальном примере инвариантные торы $\{p = a\}$ лагранжевы.
Это общий феномен, который мы сейчас объясним.

\begin{thm}{Упражнение}\label{8.1.А}
Пусть $F\: M \to \RR$ --- не зависящий от времени гамильтониан на симплектическом многообразии $M$.
Рассмотрим замкнутое подмногообразие $L \subset \{H = c\}$.
Покажите, что если $L$ лагранжево, то $L$ инвариантно относительно гамильтонова потока $F$.
Подсказка: используйте линейную алгебру, чтобы доказать, что $\sgrad F$ касается $L$.
\end{thm}

В некоторых интересных случаях приведенное выше утверждение можно обратить.

\begin{thm}[(\cite{He})]{Предложение}\label{8.1.B}
Пусть $F\: T^\ast \TT^n \to \RR$ --- гамильтониан, не зависящий от времени.
Рассмотрим инвариантный тор $L \subset \{H = c\}$, несущий
квазипериодическое движение $\dot\theta = a$, где координаты $a$, как и раньше, линейно независимы над $\QQ$.
Тогда $L$ лагранжев.
\end{thm}

\parbf{Доказательство:}
Возьмём точку $x \in L$ и предположим, что $\Omega|_{T_x L}$ имеет вид $\sum b_{ij} d\theta_i \wedge d\theta_j$.
Поскольку динамика --- это просто сдвиг, $\theta (t) = \theta (0) + at$, мы получаем, что $\Omega|_{T_y L} = \sum b_{ij} d\theta_i \wedge d\theta_j$ для каждой точки $y$, лежащей на траектории $x$.
Заметим, что каждая траектория плотна на торе, поэтому заключаем, что $\Omega = \sum b_{ij} d\theta_i \wedge d\theta_j$ всюду на~$L$.
Но $\Omega|_{TL}$ --- точная 2-форма.
Следовательно, $b_{ij} = 0$ при всех $i$, $j$, и значит $\Omega|_{T L} = 0$. 
Мы показали, что $L$ лагранжево.
\qeds

Пусть $F$ --- не зависящий от времени гамильтониан с компактным носителем в $(T^\ast \TT^n, \Omega)$.
Определим число $E (F) = \sup | E |$, где точная верхняя грань берётся по всем $E$ таким, что уровень энергии $\{F \z= E\}$ содержит лагранжев тор, гомологичный нулевому сечению.
Как найти нетривиальную оценку сверху на $E (F)$?
Такого сорта вопросы изучаются в рамках обратной КАМ-теории.%
\footnote{Цель КАМ-теории --- доказать существование инвариантных лагранжевых торов, в то время как обратная КАМ-теория изучает препятствия к их существованию, см. например \cite{Mac} и тамошние ссылки.}
Обозначим через $\{f_t\}$ гамильтонов поток, порожденный $F$.

\begin{thm}[(ср. \cite{BP2}, \cite{P8})]{Теорема}\label{8.1.C}
$\rho (\1, f_t) \ge E (F) t$ при всех $t \in \RR$.
\end{thm}

\parbf{Доказательство:}
Отметим, что достаточно доказать неравенство при $t = 1$ (провести репараметризацию по времени).
Пусть $L \z\subset \{H = E\}$ --- лагранжев тор, гомологичный нулевому сечению.
Тогда $L$ обладает устойчивым свойством лагранжева пересечения (см. \ref{6.2.D}).
Поскольку $(T^\ast \TT^n, \Omega)$ --- многообразие Лиувилля, спектр длин $\Ham (M, \Omega)$ равен $\{0\}$ (см. \ref{7.3.B}).
Следовательно, все условия теоремы \ref{7.4.A} выпонены.
Из этой теоремы следует, что $\rho (\1, f_1) \ge E$.
Взяв верхнюю грань по всем таким $E$, получаем требуемую оценку.
\qeds

Геометрическое содержание этой оценки следует понимать в более общем контексте роста однопараметрических подгрупп гамильтоновых диффеоморов.
