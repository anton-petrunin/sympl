\chapter{Рост и динамика}

В этой главе обсуждается асимптотическое геометрическое поведение
однопараметрических подгрупп в группе гамильтоновых диффеоморфизмов и
описывается связь между геометрией и инвариантными торами в
классической механике. 

\section[Инвариантные торы]{Инвариантные торы в классической механике}

Инвариантные лагранжевы торы гамильтоновых динамических систем играют
важную роль в классической механике. 
Начнём с препятствия к существованию инвариантных торов, которое
происходит из геометрии группы гамильтоновых диффеоморфизмов
(см. \ref{8.1.C}). 

Рассмотрим $n$-мерный тор $\TT^n$ с евклидовой метрикой $ds^2 \z=
\sum^n_{j = 1} dq_j^2$. 
Евклидов геодезический поток описывается гамильтоновой системой на
кокасательном расслоении $\T^\ast \TT^n$ со стандартной симплектической
формой $\Omega = dp \wedge dq$. 
Функция Гамильтона задается формулой $F (p, q) = \tfrac12 | p |^2$.
Решая гамильтонову систему 
\[
\begin{cases}
\quad\dot p &= 0,\\
\quad\dot q &= p,
\end{cases}
\]
находим гамильтонов поток $f_t (p, q) = (p, q + pt)$.
Опишем динамику этого потока.
Каждый тор $\{p = a\}$ инвариантен относительно $\{f_t\}$. 
Более того, ограничение потока на каждый такой тор представляет собой
обычное (квази)-периодическое движение $q \mapsto q + at$. 
Все эти торы гомологичны нулевому сечению кокасательного расслоения.
Геометрически они соответствуют семействам параллельных евклидовых
прямых на $\TT^n$. 

Приведенный выше поток принадлежит важному классу независящих от
времени гамильтоновых систем, называемых \rindex{интегрируемая система}\emph{интегрируемыми} системами
(см. \cite{Ar}). 
Интегрируемые системы можно охарактеризовать тем, что их
энергетические уровни расслоены (в дополнении множеств меры ноль)
инвариантными торами средней размерности с (квази-)периодической
динамикой. 
Традиционно они считаются простейшей динамикой в классической механике.
Возникает естественный вопрос: что происходит с инвариантными торами
при возмущении системы. 

Теория \rindex{Колмогоров}\rindex{Арнольд}\rindex{Мозер}Колмогорова — Арнольда — Мозера (или \rindex{КАМ-теория}КАМ-теория, см. \cite{Ar})
говорит нам, что при малых возмущениях выживает б\'{о}льшая часть
инвариантных торов. 
Однако необходимо потребовать, чтобы вектор вращения был «достаточно
иррациональным». 
В частности, это означает, что если в некоторых угловых координатах
$\theta$ на инвариантном торе динамика задается формулой $\dot\theta =
a$, где $a = (a_1 ,\dots, a_n) \in \RR^n$, то $a_1 ,\dots, a_n$ линейно независимы над $\QQ$. 
При б\'{о}льших возмущениях «топологически существенные» торы могут исчезнуть.
Можно, например, продеформировать евклидову метрику на $\TT^n$ до римановой метрики, геодезический поток которой не имеет инвариантных торов, несущих квазипериодическое движение и гомологичных нулевому сечению (см. \cite{AL}). 

Заметим, что в нашем начальном примере инвариантные торы $\{p = a\}$
лагранжевы. 
Это общее явление, которое сейчас прояснится.

\begin{ex}{Упражнение}\label{8.1.A}
Пусть $F\: M \to \RR$ — независящий от времени гамильтониан на симплектическом многообразии~$M$.
Рассмотрим замкнутое подмногообразие $L \subset \{H = c\}$.
Покажите, что если $L$ лагранжево, то $L$ инвариантно относительно гамильтонова потока~$F$.
\emph{Подсказка:} используйте линейную алгебру, чтобы доказать, что $\sgrad F$ касается~$L$.
\end{ex}

В некоторых интересных случаях приведенное выше утверждение можно обратить.

\begin{thm}[(\cite{He})]{Предложение}\label{8.1.B}\rindex{Эрман}
Пусть $F\: \T^\ast \TT^n \to \RR$ — гамильтониан, независящий от времени.
Рассмотрим инвариантный тор $L \subset \{H = c\}$, несущий квазипериодическое движение $\dot\theta = a$, где координаты вектора $a$, как и раньше, линейно независимы над $\QQ$.
Тогда $L$ лагранжев.
\end{thm}

\parit{Доказательство.}
Возьмём точку $x \in L$.
Предположим, что $\Omega|_{\T_x L}$ имеет вид $\sum b_{ij} d\theta_i
\wedge d\theta_j$. 
Поскольку динамика — это просто сдвиг, имеем $\theta (t) = \theta
(0) + at$, и, значит, $\Omega|_{\T_y L} = \sum b_{ij} d\theta_i \wedge d\theta_j$ для каждой точки $y$, лежащей на траектории $x$. 
Заметим, что каждая траектория плотна на торе.
Поэтому $\Omega = \sum b_{ij} d\theta_i \wedge d\theta_j$ всюду на~$L$.
Но $\Omega|_{\T L}$ — точная 2-форма.
Следовательно, $b_{ij} = 0$ при всех $i$, $j$, и, значит, $\Omega|_{\T L} = 0$. 
Мы показали, что $L$ лагранжево.
\qeds

Пусть $F$ — независящий от времени гамильтониан на $(\T^\ast \TT^n, \Omega)$ с компактным
носителем. 
Определим число \index[symb]{$E(F)$}$E(F)=\sup|E|$, где точная верхняя грань берётся
по всем таким $E$, что уровень энергии $\{F \z= E\}$ \textit{содержит
лагранжев тор, гомологичный нулевому сечению}. 
Как найти нетривиальную оценку сверху на $E (F)$?
Такого сорта вопросы изучаются в рамках обратной КАМ-теории.%
\footnote{Цель КАМ-теории — доказать существование инвариантных
  лагранжевых торов, в то время как обратная КАМ-теория изучает
  препятствия к их существованию, см., например \cite{Mac} и
  приведённые там ссылки.} 
Обозначим через $\{f_t\}$ гамильтонов поток, порожденный $F$.

\begin{thm}[(ср. \cite{BP2,P8})
]{Теорема}\label{8.1.C}
  $\rho (\1, f_t) \ge E (F) t$ при всех $t \z\in \RR$.
\end{thm}

\parit{Доказательство.}
Отметим, что достаточно доказать неравенство при $t = 1$ (время можно
репараметризовать).  
Пусть $L \z\subset \{H = E\}$ — лагранжев тор, гомологичный нулевому сечению.
Тогда $L$ обладает свойством устойчивого лагранжева пересечения
(см. пример \ref{6.2.D}). 
Поскольку $(\T^\ast \TT^n, \Omega)$ — многообразие Лиувилля, спектр
длин $\Ham (M, \Omega)$ равен $\{0\}$ (см. \ref{7.3.B}). 
Следовательно, все условия теоремы \ref{7.4.A} выпонены. 
Из этой теоремы следует, что $\rho (\1, f_1) \ge E$. 
Взяв верхнюю грань по всем таким $E$, получаем нужную оценку.
\qeds

Геометрическое содержание этой оценки следует разуметь в более общем контексте роста однопараметрических подгрупп гамильтоновых диффеоморфизмов. 


\section{Рост однопараметрических подгрупп}\label{sec:8.2}

Пусть $(M, \Omega)$ — симплектическое многообразие и $\{f_t\}$ —
однопараметрическая подгруппа в $\Ham (M, \Omega)$, порожденная
нормализованным гамильтонианом $F \in \A$. 
Одна из центральных задач хоферовской геометрии — изучить
взаимосвязь между функцией $\rho (\1, f_t)$ и динамикой потока
$\{f_t\}$.
Например, теорема \ref{8.1.C} утверждает, что инвариантные торы
автономного гамильтонова потока на $\T^\ast \TT^n$,  гомологичные
нулевому сечению и с квазипериодической динамикой, вносят вклад в
линейный рост функции $\rho (\1, f_t)$. 
Существует еще одна, чисто геометрическая причина интереса к этой функции.
Она исходит из теории геодезических хоферовской метрики (см. главу~\ref{chap:12}). 
Гамильтонов путь $\{f_t\}$ называется
\rindex{кратчайшая}\emph{кратчайшей}, если каждый из его отрезков
минимизирует длину между своими концами.
Предположительно (см. \ref{12.6.A}) все {}\emph{достаточно короткие}
отрезки любой однопараметрической подгруппы являются кратчайшими
(иными словами, каждая однопараметрическая подгруппа локально
минимизирует длину). 
Однако, как мы увидим в \ref{8.2.H}, {}\emph{длинные отрезки} могут перестать
быть кратчайшими. 
Нарушение минимальности — любопытное явление, которое всё ещё не изучено.
В настоящее время к нему известно несколько подходов.
Один из них основан на теории сопряженных точек в хоферовской
геометрии, он обсуждается в главе \ref{chap:12}. 
Здесь мы обсудим подход, основанный на понятии \rindex{асимптотический рост}\emph{асимптотического роста}
однопараметрической подгруппы (см. \rindex{Бялый}\cite{BP2}). 
Асимптотический рост определяется следующим образом: \index[symb]{$\mu(F)$}
\[\mu(F)=\lim_{t \to + \infty}\frac{\rho (\1, f_t)}{ t \| F \|}.\]

\begin{ex*}{Упражнение}
Покажите, что указанный выше предел существует.
\emph{Подсказка:} используйте субаддитивность функции $\rho (\1, f_t)$, то есть $\rho (\1, f_{t + s}) \le \rho (\1, f_t) + \rho (\1, f_s)$. 
\end{ex*}

Ясно, что $\mu (F)$ лежит в интервале $[0; 1]$.
Если $\mu (F) <1$, то путь $\{f_t\}$ не является кратчайшей.

Рассмотрим несколько примеров поведения функции $\rho (\1, f_t)$.
\rindex{Хофер}Х. Хофер \cite{H2} доказал, что каждая однопараметрическая подгруппа в
$\Ham (\RR^{2n})$ локально кратчайшая. 
С другой стороны, \rindex{Сикорав}Ж.-К. Сикорав \cite{S2} обнаружил поразительное явление --- каждая такая однопараметрическая подгруппа лежит на конечном расстоянии от тождественного отображения (в частности $\mu=0$).
Таким образом, вся подгруппа не может быть кратчайшей. 

\begin{thm}{Теорема}\label{8.2.A}
Пусть $\{f_t\}$ — однопараметрическая подгруппа в $\Ham (\RR^{2n})$,
порожденная гамильтоновой функцией $F$ с компактным носителем. 
Предположим, что носитель $F$ содержится в евклидовом шаре радиуса
$r$. 
Тогда функция $\rho (\1, f_t)$ ограничена: $\rho (\1, f_t) \z\le 16\pi
r^2$. 
\end{thm}

Мы отсылаем читателя к \cite[с. 177]{HZ} за полным доказательством (см. также обсуждение в предложении \ref{12.6.E}). 

Вернёмся теперь к однопараметрическим подгруппам в $\Ham (\T^\ast \TT^n)$.
Прежде всего доказано, что все они локально кратчайшие \cite{LM2}.
Иными словами, $\rho (\1, f_t) = t \| F \|$ при условии, что $t$
достаточно мало. 
Конечно же, это означает, что в общем случае оценка в \ref{8.1.C} не является
точной при малых $t$. 
Действительно, в общем случае $E (F)$ строго меньше, чем $\| F \|$.
Тем не менее в случае $n = 1$ и $F \ge 0$ оценка \ref{8.1.C}
асимптотически точна! 
Заметим, что, поскольку каждая замкнутая кривая на цилиндре $\T^\ast
\TT^1$ является лагранжевой, величина $E (F)$ равна верхней грани тех
вещественных чисел $E$, для которых уровень $\{F = E\}$ содержит
нестягиваемую вложенную окружность. 

\begin{thm}[(\cite{PS})]{Теорема}\label{8.2.B}\rindex{Зибург}
  Пусть F — неотрицательный гамильтониан с компактным носителем на
  цилиндре $\T^\ast \TT^1$ с $\| F \| = 1$. 
  Тогда обратный КАМ-параметр $E (F)$ совпадает с асимптотическим
  ростом $\mu (F)$: $E (F) = \mu (F)$. 
\end{thm}

\parit{Доказательство.}
Нам достаточно доказать, что $\mu (F) \le E (F)$.
Тогда, примениив \ref{8.1.C}, мы получим желаемый результат.

Если $E (F) \z= \max F = 1$, то \ref{8.1.C} влечёт $\mu (F) = E (F)$.
Предположим теперь, что $E (F) <1$.
Идея состоит в разложении потока $\{f_t\}$ на два
коммутирующих потока с простой асимптотикой. 
Выберем $\epsilon> 0$ достаточно малым и рассмотрим гладкую
неубывающую функцию $u: [0; + \infty) \to [0; + \infty)$ со следующими
    свойствами:  
\begin{itemize}
\item $u (s) = s$ при $s \le E (F) + \epsilon$;
\item $u (s) = E (F) + 2\epsilon$ при $s \ge E (F) + 3\epsilon$;
\item $u (s) \le s$ при всех $s$.
\end{itemize}
Рассмотрим новые гамильтонианы $G = u \circ F$ и $H = F - G$ и
обозначим через $\{g_t\}$ и $\{h_t\}$ соответствующие гамильтоновы
потоки. 
Эти потоки коммутируют и $f_t = g_t h_t$. 
Таким образом, 
\begin{equation}\rho (\1, f_t) \le \rho (\1, g_t) + \rho (\1, h_t).
\label{eq:8.2.C}
\end{equation}
Обратите внимание, что $\| G \| \le E (F) + 2\epsilon$.
Следовательно, 
\begin{equation}
 \rho (\1, g_t) \le t (E (F) + 2\epsilon).
\label{eq:8.2.D}
\end{equation}
Далее, носитель $H$ содержится в подмножестве $D_\epsilon = \{F \ge E
(F) + \epsilon\}$. 
Для достаточно малого $\epsilon$ общего положения множество
$D_\epsilon$ является областью, граница которой состоит из {}\emph{стягиваемых}
замкнутых кривых (тут используется определение величины $E(F)$). 
Предположим теперь, что носитель гамильтониана $F$ содержится в кольце 
\[A = \set{(p, q) \in \T^\ast\TT^1}{| q | \le a / 2}\]
для некоторого $a> 0$. 
Отметим, что $\partial D_\epsilon \subset A$.
Следовательно, множество $D_\epsilon$ содержится в некотором множестве
$D' \subset A$, которое является конечным объединением попарно
непересекающихся замкнутых дисков с общей площадью не более $a$. 
Поскольку цилиндр имеет бесконечную площадь, из теоремы
Дакорогны — \rindex{Мозер}Мозера \cite[1.6]{HZ} легко вытекает существование симплектического вложения $i\: \RR^2 \to\T^\ast \TT^1$ и конечного объединения $D'' \subset \RR^2$ евклидовых дисков таких, что $i$
отображает $D''$ диффеоморфно на $D'$. 
Ясно, что $i$ индуцирует естественный гомоморфизм 
\[i_\ast: \Ham (\RR^2) \to \Ham (\T^\ast \TT^1).\]
Важно отметить, что $i_\ast$ не увеличивает хоферовские расстояния.
Наш поток $h_t$ лежит в образе $i_\ast$, то есть $h_t = i_\ast (e_{t})$,
где $e_{t}$ — однопараметрическая подгруппа в $\Ham (\RR^2)$,
гамильтониан которой имеет носитель в $D''$. 
Таким образом, теорема \ref{8.2.A} влечёт, что 
\[
\rho (\1, h_t) \le 16a.
\]
Комбинируя это неравенство с (\ref{eq:8.2.D}) и (\ref{eq:8.2.C}) получаем, что
\begin{equation}
  \rho (\1, f_t) \le t (E (F) + 2\epsilon) + 16a 
  \label{eq:8.2.E} 
\end{equation}
при всех $t> 0$.
Поделив на $t$ и перейдя к пределу при $t \to +\infty$, получаем, что
$\mu (F) \le E (F) + 2\epsilon$. 
Теорема следует поскольку $\epsilon$ произвольно мало.
\qeds

\begin{ex}{Замечание}\label{8.2.F}
То же доказательство показывает, что если $E (F) \z= 0$, то функция
$\rho (\1, f_t)$ ограничена. 
Действительно, поскольку (\ref{eq:8.2.E}) выполняется при всех
$\epsilon> 0$, получаем, что $\rho (\1, f_t) \z\le 16a$. 
Принимая во внимание $E (F) \le \mu (F)$, получаем следующее
утверждение типа «жёсткости»: \textit{если $\mu (F) = 0$, то функция $\rho
(\1, f_t)$ ограничена} (дальнейшее обсуждение в \ref{sec:8.4}). 
\end{ex}

Теорема \ref{8.2.B} и замечание \ref{8.2.F} верны для всех открытых
поверхностей бесконечной площади (см. \cite{PS}). 
Более того, можно изменить определение величины $E (F)$ и распространить эти
утверждения на произвольные (не обязательно неотрицательные)
гамильтонианы $F$. 
Пока что нет обобщений \ref{8.2.B} на высшие размерности.
Однако следующее рассуждение показывает, что оценка \ref{8.1.C} бывает точной,
по крайней мере в следующем очень частном случае. 

Пусть $F\: \T^\ast \TT^n \to \RR$ — гамильтониан с компактным носителем, независящий от времени, который удовлетворяет следующим условиям: 
\begin{enumerate}[(i)]
\item $F \ge 0 $
\item $\max F = 1$ 
\item множество максимума $\Sigma =\{F = 1\}$ является гладким сечением кокасательного расслоения. 
\end{enumerate}

Оказывается, геометрия соответствующего потока $\{f_t\}$ сильно
зависит от того, является ли $\Sigma$ лагранжевым или нет! 

Предположим, что $\Sigma$ — лагранжево подмногообразие.
Тогда по определению $E (F) = 1$.
Следовательно, теорема \ref{8.1.C} влечёт, что $\{f_t\}$ —
кратчайшая и, в частности, $E (F) = \mu (F)$. 

Предположим теперь, что $\Sigma$ нелагранжево, то есть $\Omega$ не
обращается в нуль хотя бы на одном касательном пространстве к
$\Sigma$. 
Тогда неизвестно, равны ли $\mu (F)$ и $E (F)$ между собой.
Однако мы утверждаем, что оценка из \ref{8.1.C} по крайней мере
нетривиальна, а именно  
\begin{equation}
E (F) \le \mu (F) <1.\label{8.2.G}
\end{equation}

Чтобы объяснить это неравенство, нам понадобится следующий довольно общий результат.
В некоторых интересных случаях он позволяет доказать, что $\mu(F)$ строго меньше $1$. 

\begin{thm}{Теорема}\label{8.2.H}
Пусть $F$ — независящий от времени нормализованный гамильтониан на
симплектическом многообразии $ (M, \Omega)$. 
Пусть $\Sigma_+$  и $\Sigma_-$ 
  множества минимума и максимума $F$ соответственно.
Предположим, что существует такой диффеоморфизм $\phi \in \Ham (M, \Omega)$, что
либо $\phi (\Sigma_+) \cap \Sigma_+ = \emptyset$, либо $\phi
(\Sigma_-) \cap \Sigma_- = \emptyset$. 
Тогда $\mu (F) <1$ и, в частности, гамильтонов поток
гамильтониана $F$ не образует
кратчайшую.  
\end{thm}

Доказательство теоремы приведено в разделе \ref{sec:8.3}.

Приведём набросок доказательства \ref{8.2.G}.
Поскольку $\Sigma$ нелагранжево, можно показать, что существует такой гамильтонов диффеоморфизм $\phi$, что
$\phi(\Sigma) \cap \Sigma = \emptyset$. 
Таким образом, комбинируя теоремы \ref{8.1.C} и \ref{8.2.H} получаем,
что $E (F) \le \mu (F) <1$. 
Доказательство существования гамильтонова диффеоморфизма $\phi$,
смещающего нелагранжево подмногообразие $\Sigma$, сложное. 
Оно основано на $h$-принципе \rindex{Громов}Громова для отношений в частных производных.
На самом деле можно доказать большее, а именно что энергия смещения
$\Sigma$ обращается в нуль. 
Мы отсылаем читателя к \rindex{Лауденбах}\rindex{Сикорав}\cite{P2,LS} за доказательствами этого
результата и его обобщений. 

Отметим также, что \rindex{Зибург}К. Ф. Зибург \cite{Si2} обобщил неравенство \ref{8.1.C}
на {}\emph{неавтономные} гамильтоновы потоки. 
Обратите внимание, что в случае, зависящем от времени, уже
нетривиально определить обратный КАМ-параметр из-за отсутствия
сохранения энергии. 
Определение Зибурга основано на идее минимального действия,
разработанной \rindex{Мазер}Дж. Мазером. 

\section[Вырямление кривых]{Вырямление кривых}\label{sec:8.3}

Пусть $(M, \Omega)$ — симплектическое многообразие.
Для любой функции $F \in \A$, $F \not\equiv 0$, положим 
\[\delta(F)=\inf_\phi \frac{\|F+ F \circ \phi\|}{2\|F\|},\]
где точная нижняя грань берётся по всем $\phi \in \Ham (M, \Omega)$.

\begin{thm}[(\cite{BP2})]{Теорема}\label{8.3.A}
\[\mu (F) \le \delta (F).\]
\end{thm}

Теорема \ref{8.2.H} является непосредственным следствием теоремы выше. 
В самом деле, если $F$ удовлетворяет предположениям \ref{8.2.H}, то
$\delta (F) <1$. 

\parit{Доказательство \ref{8.3.A}.}
Не умаляя общности, можно считать, что $\|F \| = 1$.
Выберем $\phi \in \Ham (M, \Omega)$ и $T> 0$.
Запишем 
\[f_{2T}= (f_T \circ \phi \circ f_T \circ \phi^{-1}) \circ (\phi \circ
f_T^{-1} \circ \phi^{-1} \circ f_T) = A \circ B.\] 
Заметим, что $B$ — коммутатор, поэтому $\rho (\1, B) \le 2\rho (\1, \phi)$.
Диффеоморфизм $A$ порождается путем $g_t = f_t \phi f_t \phi ^{-1}$
при $t \in [0;T]$, и его гамильтониан равен  
\[G (x, t) = F (x) + F (\phi^{-1} f_t^{-1} x).\]
Таким образом, имеем 
\begin{align*}
\|G_t \| &= \|F + F \circ \phi ^{-1} \circ f_t ^{-1} \| =
\\
&=\|F \circ f_t + F \circ \phi ^{-1} \| =
\\
&=\|F + F \circ \phi ^{-1} \|,\end{align*}
поскольку $F \circ f_t = F$ (по закону сохранения энергии).
Итак, 
\[\rho (\1, f_{2T}) \le T \| F + F \circ \phi ^{-1} \| + 2\rho (\1, \phi)\]
и, следовательно, 
\[\frac{\rho (\1, f_{2T})}{2T}
\le\frac12 \|F + F \circ \phi ^{-1} \| + \frac{\rho (\1, \phi)}{T}\]
при всех $\phi \in \Ham (M , \Omega)$.
Переходя к пределу при $T \to \infty$, получаем $\mu (F) \le \delta (F)$.
\qeds

Отсылаем читателя к \cite{LM2} и \cite{P9}, где описаны различные
процедуры выпрямления кривых в хоферовской геометрии. 
Мы вернемся к этому вопросу в главе \ref{chap:11}.
Величина $\delta (F)$ допускает следующее естественное обобщение.
Положим 
\[
\delta_N(F)
=
\inf_{\phi_1 ,\dots, \phi_{N-1}}
\frac1N
\frac{\|\sum_{j=0}^{N-1} F \circ \phi_j\|}{\|F\|}
\] 
где $\phi_0 = \1$, а нижняя грань берётся по всем последовательностям $\{\phi_j\}$, $j = 1,\dots, N-1 $ гамильтоновых диффеоморфизмов. 
На этом языке $\delta = \delta_2$. 
Можно показать \cite{P9}, что на замкнутом симплектическом многообразии $\delta_N (F) \to 0$ при $N \to + \infty$ для любой функции $F \in \A$.
Заметим, что при $N>2$ неравенство $\mu(F)\le\delta_N(F)$, напоминающее теорему~\ref{8.3.A},
становится вообще говоря неверным.
Мне неизвестна скорость убывания последовательности $\delta_N (F)$. 

\section{А что если рост нулевой?}\label{sec:8.4}

Как мы видели в \ref{8.2.F}, на цилиндре 
(и, в более общем смысле, на открытых поверхностях бесконечной площади)
каждая однопараметрическая подгруппа с нулевым асимптотическим ростом является ограниченной. 
А что происходит на других симплектических многообразиях?

\begin{ex}{Задача}\label{8.4.A}
Существует ли симплектическое многообразие $(M, \Omega)$ и
однопараметрическая подгруппа $\{f_t\}$ в $\Ham (M, \Omega)$ такие,
что функция $\rho (\1, f_t)$ имеет промежуточный рост
(например, растет как $\sqrt{t}$)? 
\end{ex}

Эта задача открыта даже для такого простого симплектического
многообразия, как двумерный тор $\TT^2$. 
В оправдание невежества, заметим, что, как показывает следующий
результат, гамильтониан $F \z\in \A (\TT^2)$ с $\mu (F) = 0$, не может находится в общем положении.%
\footnote{
%% On any closed symplectic manifold, μ(F ) > 0 for a C ∞ -generic
%% function F , see [13, Section 6.3.1].
Для любого замкнутого симплектического многообразия и
$C^{\infty}$-общей функции $F$ на нём выполняется $\mu(F)>0$,
см.~\cite[Section 6.3.1]{PR14}.\dpp}

\begin{thm}{Теорема}\label{8.4.B}
Если 0 — регулярное значение $F \in \A (\TT^2)$, то $\mu (F)> 0$.
\end{thm}

\parit{Доказательство.}
Множество $D = \{F = 0\}$ состоит из конечного числа попарно
непересекающихся вложенных окружностей. 
Значит, существует нестягиваемая простая замкнутая кривая $L \z\subset
\TT^2$ такая, что $L\cap D = \emptyset$. 
Таким образом, $|F (x)| \z> C$ при фиксированном $C> 0$ и всех $x \in L$.
Поскольку $\pi_1 (\Ham (\TT^2)) = 0$ (см. пример \ref{7.2.B}), из
\ref{7.4.A} следует, что $\rho (\1, f_t) \ge Ct$ при всех $t$ и, в
частности, $\mu (F)> 0$. 
\qeds
