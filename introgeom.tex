 \chapter{Знакомство с геометрией}

В этой главе мы обсуждаем биинвариантные финслеровы метрики на группе гамильтоновых диффеоморфизмов и определяем геометрию Хофера.

\section{Вариационная задача}\label{2.1}

Какая минимальная энергия необходима для получения данного гамильтонова диффеоморфизма $\phi$? 
Этот естественный вопрос можно формализовать следующим образом.
Рассмотрим всевозможные гамильтоновы потоки $\{f_t\}$, $t \in [0; 1]$ такие, что $f_0 = \1$ и $f_1 = \phi$.
Для каждого потока возьмём, соответсвующий его нормализованный гамильтониан $F_t(x)$ и «измерим его величину».
Затем минимизируем результат измерения по всем таким потокам.
Остаётся понять, что значит «измерить его величину».
Напомним, что для любого $t$ функция $F_t$ является элементом алгебры Ли $\A$.
Возьмём любую безкоординатную норму $\|\ \|$ на $\A$,
т. е. потребуем, чтобы 
\begin{equation}
 \|H \circ \psi^{-1}\|
= \| H \|
\quad\text{для всех}\quad
H \in \A,\  \psi \in \Ham (M, \Omega).
\label{2.1.A}
\end{equation}
Теперь определим величину как  $\int_0^1\| F_t \| dt.$
Собирав все ингредиенты этой процедуры, мы получаем следующую вариационную задачу 
\begin{equation}
\inf\int_0^1 \| F_t \| dt, 
\label{2.1.B}
\end{equation}
где $\phi$ фиксировано, а точная нижняя грань берётся по всем потокам $\{f_t\}$, как сказано выше.

\section[Биинвариантные геометрии на $\Ham(M, \Omega)$]{Биинвариантные геометрии на $\bm{\Ham(M, \Omega)}$}
\label{2.2}

Приведённая вариационная задача может быть переформулирована в чисто геометрических терминах.
Для этого надо вспомнить понятие \emph{финслервой структуры} на многообразии.
Мы говорим, что многообразие $Z$ наделено финслерной структурой, если его касательные пространства $T_z Z$ оснащены нормой, которая гладко зависит от точек $z \in Z$.
Конечно же, римановы структуры образуют частный случай этого понятия.
Однако в общем случае, нормы могут и не задаваться скалярными произведениями.
Финслерву структуру можно использовать для определения длины кривой точно так, как риманову структуру 
\[\\length \{z(t)\}_{t\in [a; b]} =\int_a^b \norm(\dot z (t)) dt.\]
Кроме того, можно ввести расстояние между двумя точками $z$ и $z'$ в $Z$ как инфимум длин всех кривых, соединяющих $z$ с $z'$.

Вернёмся к ситуации, описанной в предыдущем разделе.
Поскольку все касательные пространства группы $\Ham(M, \Omega)$ можно отождествить с $\A$ (см. раздел \ref{1.4} выше), каждый выбор нормы $\|\ \|$ на $\A$ определяет финслерову структуру на группе.
Таким образом, можно определить длину гамильтонова пути и расстояние между двумя гамильтоновыми диффеоморфизмами.
В частности, длина гамильтонова пути $\{f_t\}$, $t \in [a; b]$ с нормализованным гамильтонианом $F$ определяется следующим образом 
\[\\length\{f_t\}=\int_a^b \| F_t \| dt.\]
А расстояние между двумя гамильтоновыми диффеоморфизмами $\phi$ и $\psi$ определяется как
\[\rho (\phi, \psi) = \inf \\length \{f_t\},\] 
где инфимум берётся по всем гамильтоновым путям $\{f_t\}$, $t \in [a; b]$ с $f_a = \phi$ и $f_b = \psi$.
Конечно, длина пути не зависит от параметризации, таким образом, в приведённом определении расстояния можно считать, что $a = 0$ и $b = 1$.
На этом языке, решение вариационной задачи
\ref{2.1.B} есть ничто иное, как расстояние $\rho (\1, \phi)$!

Следующие свойства $\rho$ легко проверить и даются в качестве упражнения:
\begin{itemize}
\item $\rho (\phi, \psi) = \rho (\psi, \phi)$;
\item неравенство треугольника: $\rho (\phi, \psi) + \rho (\psi, \theta) \ge \rho (\phi, \theta)$;
\item $\rho (\phi, \psi) \ge 0$.
\end{itemize}

Напомним теперь об условии \ref{2.1.A}, которое было накложено на норму $\|\ \|$ в разделе \ref{2.1}.
На геометрическом языке это означает, что норма инвариантна относительно присоединённого действия группы на её алгебре Ли (см. \ref{1.4}).
В дальнейшем мы будем иметь дело только с такими нормами.

\begin{thm*}{Упражнение}
Покажите что из \ref{2.1.A} следует, что функция $\rho$ --- биинвариантна%
\footnote{без \ref{2.1.A} мы получаем только правoинвариантность $\rho$, то есть $\rho (\phi, \psi) \z= \rho (\phi\theta, \psi\theta)$.
Такие метрики играют важную роль в гидродинамике, см. \cite{AK}.}%
:
\[\rho (\phi, \psi) = \rho (\phi \theta, \psi \theta) = \rho (\theta\phi, \theta\psi)\]
для всех $\phi, \psi, \theta \in \Ham (M, \Omega)$.
\end{thm*}

Было бы честней назвать функцию $\rho$ \emph{псевдометрикой}.
Действительно, как мы видели выше, она удовлетворяет всем аксиомам метрики, за исключением, возможно, невырожденности, то есть 
\begin{equation}
\rho (\phi, \psi)> 0
\quad\text{при}\quad
\phi \ne \psi.
\label{eq:2.2.A}
\end{equation}

Даже в конечномерном случае, невырожденность проверить непросто.
В этом случае, доказательство использует локальнокомпактность многообразия.
В нашей ситуации группа $\Ham (M, \Omega)$ бесконечномерна и не имеет никаких свойств компактности.
Таким образом, априори у \ref{eq:2.2.A} нет причин быть верным.
Как мы скоро увидим, свойство невырожденности $\rho$ очень чувствительно к выбору нормы~$\|\ \|$.

\section[Выбор нормы: $L_p$ или $L_\infty$]{Выбор нормы: \bm{$L_p$} или \bm{$L_\infty$}}

Среди норм на $\A$, удовлетворяющих предположению инвариантности \ref{2.1.A} существует очень естественный класс, который включает нормы $L_p$, $p = 1, 2, 3,\dots$
\[\|H\|_p=\left(\int_M|H|^p\Vol\right)^{\frac1p},\]
и $L_\infty$-норму 
\[\|H\|_\infty = \max H - \min H.\]
Обозначим через $\rho_p$ и $\rho_\infty$ соответствующие псевдометрики.

\begin{thm}{Теорема}\label{2.3.A}
Псевдометрика $\rho_p$ вырождена для всех конечных $p = 1, 2,\dots$
Более того, если многообразие замкнуто, то все такие $\rho_p$ тождественно обращаются в нуль.
\end{thm}

Этот результат получен в \cite{EP}.
Доказательство представлено ниже в этой главе (см. также книги \cite{HZ}, \cite{MS}, \cite{AK}).
Следующая теорема показывает разительный контраст между случаями $L_p$ и $L_\infty$.

\begin{thm}{Теорема}\label{2.3.B}
Псевдометрика $\rho_\infty$ невырождена.
\end{thm}

Эта теорема%
\footnote{Историческое отступление, приведенное ниже, отражает мое личное мнение.
Допускаю, что другие участники этих событий видят дело иначе.}
была сформулирована и доказана Хофером в \cite{H1} для случая $M = \RR^{2n}$ с использованием бесконечномерных вариационных методов.
В \cite{V1} Витербо вывел его для случая $M = \RR^{2n}$ используя свою теории производящих функций.
Для обоих, Хофера и Витербо, толчком послужил вопрос, заданный Элиашбергом в частной беседе.
В \cite{P1} это утверждение было распространено на широкий класс симплектических многообразий с «хорошим» поведением на бесконечности и, в частности, на все замкнутые симплектические многообразия, у которых класс когомологий симплектической формы является рациональным.
Подход \cite{P1} основан на теории псевдоголоморфных кривых Громова.
Наконец \cite{LM1} Лалонде и Макдафф доказали \ref{2.3.B} в полной общности, используя теорию Громова.
К настоящему времени найдены другие доказательства различных частных случаев этой теоремы, см., например, \cite{Ch}, \cite{O3}, \cite{Sch3}.
Изначальное доказательство Хофера подробно изложено ​​в книге \cite{HZ}.
Доказательство Лалонда и Макдафф даются в книге \cite{MS} и в обзоре \cite{L}.
Ниже мы приводим другое доказательство в случае $M = \RR^{2n}$, которое следует из \cite{P1}.
Все известные доказательства основаны на «жёстких» методах.%
\footnote{Более того, на мой вкус, все доказательства совсем не прозрачны.
Рассуждение главы 3 ниже, не является исключением.
Я твердо верю, что в будущем будет дано \emph{правильное объяснение} этому фундаментальному результату.}


\section{Понятие энергии смещения}

Какие инвариантные нормы $\|\ \|$ на $\A$ (то есть, удовлетворяющие \ref{2.1.A}) приводят к невырожденным метрикам $\rho$?
Ниже мы описываем очень полезную переформулировку этого вопроса, которая в конечном счёте позволит доказать теорему \ref{2.3.A}.
Она основана на прекрасной идее \?{энергии смещения}{выделить?}, введенной Хофером \cite{H1}.
Пусть $\rho$ --- биинвариантное псевдометрика на $\Ham (M, \Omega)$, и пусть \?{$A$}{$A$ похоже на $\A$} --- ограниченное подмножество $M$.

\begin{thm*}{Определение}
Энергия смещения $A$ определяется как
\[\e(A) = \inf \set{\rho (\1, f)}{f \in \Ham (M, \Omega), f (A) \cap A = \emptyset}.\]
\end{thm*}

Множество таких $f$ может быть пустым.
Мы будем следовать соглашению, по которому нижняя грань пустого множества равна $+\infty$.
Если $\e (A) \ne 0$, то мы говорим, что $A$ имеет положительную энергию смещения.

Отметим пару очевидных, но важных свойств $\e(A)$.
Прежде всего, $\e$ является монотонной функцией подмножеств: если $A \subset B$, то $\e(A) \le \e(B)$.
Во-вторых, $\e$ --- инвариантна, то есть, $\e (A) = \e (f (A))$ для любого гамильтонова диффеоморфизма $f$ на $M$.
Мы оставляем доказательства читателю.

\parbf{Пример.}
Рассмотрим $(\RR^2, \omega)$ и возьмем открытый квадрат $A$, рёбра которого имеют длину $u$ и параллельны осям координат.
Оценим энергию смещения $A$ относительно расстояния $\rho_\infty$.
Рассмотрим гамильтониан $H (p, q) = up$.
Соответствующая гамильтонова система имеет вид 
\[
\begin{cases}
\dot q &= u
\\
\dot p &= 0
\end{cases}
\]
Следовательно, за единичное время, $(p, q)$ переходит в $(p, q + u)$.
Обратите внимание, что движение квадрата происходит в прямоугольнике $K = \Closure (A \cup h (A))$.
Рассмотрим \?{срезку}{cut-off} $F$ гамильтониана $H$ вне малой окрестности $K$.%
\footnote{Пусть $Y$ --- замкнутое подмножество многообразия $Z$, и пусть $H$ --- гладкая функция, определенная в окрестности $V$ многообразия $Y$.
Под обрезкой $F$ области $H$ вне малой окрестности $Y$ мы понимаем следующее.
Выберем окрестность $W\supset Y$, замыкание которой содержится в $V$.
Возьмем гладкую функцию $a\colon Z \to [0; 1]$, который равен $1$ на $W$ и обращается в нуль вне $V$.
Определим $F$ как $aH$ на $V$ и продолжим нулём вне $V$.}
Обратите внимание, что $F$ нормализован (в отличие от $H$).
Поскольку $F = H$ на $K$, гамильтонов поток $f$, порождённый $F$, по-прежнему смещает квадрат $A$.
Мы всегда можем выполнить обрезку так, что $L_\infty$-норма $F$ была произвольно близка к перепаду $H$ на $K$.
Перепад определёется как 
\[\max_K H - \min_K H = u^2 - 0 = u^2.\]
Таким образом, получаем, что 
\[\e (A) \le u^2 = \Area (A).\]
Обратите внимание, что квадрат симплектоморфен диску той же площади в $\RR^2$ (это неверно в более высоких размерностях!).
Таким образом, мы доказали, что $\e (B^2 (r)) \le \pi r^2$.
Глубокий результат Хофера \cite{H1} утверждает, что на самом деле существует равенство во всех размерностях, то есть, $\e (B^{2n} (r)) = \pi r^2$.
Обобщение на произвольные симплектические многообразия дано в \cite{LM1}.
Мы докажем нижнюю оценку на $\e(B^{2n} (r))$ в следующей главе.
В общем случае, если энергия смещения всех непустых открытых подмножеств относительно некоторой биинвариантной псевдометрики $\rho$ положительна, то $\rho$ невырождена.
И в самом деле, любой $f \in \Ham (M, \Omega)$ такой, что $f \ne \1$, должен сместить маленький шар $A \subset M$.
Таким образом, получаем, что $\rho (\1, f) \ge \e (A)> 0$.
На самом деле верно и обратное.

\begin{thm}[(\cite{EP})]{Теорема}\label{2.4.A}
Если $\rho$ невырождено, то $\e (A)> 0$ для любого непустого открытого подмножества $A$.
\end{thm}

В доказательстве нам понадобится следующая лемма.

\begin{thm}{Лемма}\label{2.4.B}
Пусть $A \subset M$ --- непустое открытое подмножество.
Для всех $\phi, \psi \in \Ham (M, \Omega)$ с $\supp (\phi) \subset A$ и $\supp (\psi) \subset A$ имеем $\e (A) \ge \tfrac14 \rho (\1, [\phi, \psi])$.
\end{thm}

Здесь $[\phi, \psi]$ обозначает коммутатор $\psi^{-1} \phi^{-1} \psi\phi$.
Теорема сразу следует из леммы по модулю предложения \ref{1.5.B} предыдущей главы.

\parit{Доказательство \ref{2.4.A}:} 
Согласно \ref{1.5.B} существуют $\phi$, $\psi$ с носителем в $A$ такие, что $[\phi, \psi] \ne \1$.
Поскольку $\rho$ невырождено, отсюда следует, что $\e (A) \ge \tfrac14 \rho (\1, [\phi, \psi])> 0$.
\qeds

\parit{Доказательство \ref{2.4.B}:}
Предположим, что существует $h \in \Ham (M, \Omega)$ такой, что $h (A) \cap A = \emptyset$ (если такого $h$ нету, то дело сделано --- в этом случае $\e (A) = + \infty$).
Пусть $\theta = \phi h^{-1} \phi^{-1} h = [h, \phi^{-1}]$.
Если $x \in A$, то $h (x) \notin A$.
Поскольку $\phi = \1$ вне $A$, получаем $\phi^{-1} h (x) = h (x)$.
Итак, мы видим, что $h^{-1} \phi^{-1} h (x) = x$ и, следовательно, $\theta|_A = \phi|_A$.
Далее, $\supp (\psi) \subset A$ и, следовательно, $\phi^{-1} \psi\phi = \theta^{-1} \psi\theta$, откуда следует, что $[\phi, \psi] = [\theta, \psi]$.
Отметим, что 
\begin{align*}
\rho (\1, [\theta, \psi]) &= \rho (\1, \psi^{-1} \theta^{-1} \psi\theta) =
\\
&=\rho (\theta^{-1}, \psi^{-1} \theta^{-1} \psi) \le
\\
&\le \rho (\1, \theta^{-1}) + \rho (\1, \psi^{-1} \theta^{-1} \psi) =
\\
&= 2\rho (\1, \theta).
\end{align*}
Здесь мы использовали биинвариантность $\rho$ и неравенство треугольника.
Аналогично 
\[\rho (\1, \theta) = \rho (\1, [h, \phi^{-1}]) \le 2\rho (\1, h).\]
Из двух этих неравенств, получаем 
\[\rho (\1, [\phi, \psi]) = \rho (\1, [\theta, \psi]) \le 2\rho (\1, \theta) \le 4\rho (\1, h).\]
Поскольку это верно для всех $h \in \Ham (M, \Omega)$ с $h (A) \cap A =\emptyset$, то взяв нижнюю грань, получаем $4\e (A) \ge \rho (\1, [\phi, \psi])$.
\qeds

Напомним, что теорема \ref{2.3.A} утверждает, что $L_p$-норма задаёт вырожденную псевдометрику при $p <\infty$;
более того, для замкнутых многообразий она вовсе ноль.
Сейчас мы докажем это утверждение и увидим, почему рассуждение не работает в $L_\infty$-случае.

\parbf{Доказательство \ref{2.3.A}:} Покажем, что энергия смещения маленького шарика обнуляется.
Тогда вырождение $\rho_p$ следует из \ref{2.4.A}.
Пусть $U$ --- открытое подмножество в $M$, наделенное каноническими координатами $(x, y)$.
В этих координатах симплектическая форма $\Omega$ равна $\sum dx_i\wedge dy_i$.
Без ограничения общности можно предположить, что $U$ содержит шар $(x^2_j + y_j^2) <10$.
Пусть $A \subset U$ --- шар с тем же центром радиуса $0{,}1$.
Рассмотрим (частично определенный) поток $h_t$, $t \in [0; 1]$ на $U$, который представляет собой простой сдвиг на $t$ по координате $y_1$.
Такой сдвиг порождается (ненормализованым!) гамильтонианом $H (x, y) = x_1$ на $U$.
Ясно, что $h_1 (A) \cap A = \emptyset$.
Пусть $S_t$ есть сфера $h_t (\partial A)$.
Рассмотрим новый (зависящий от времени) нормализованный гамильтониан $G_t = F_t + c_t$, где $F_t$ обрезка $H$ вне небольшой окрестности $S_t$, а $c_t$ является (зависящей от времени) постоянной.
Конечно же, $c_t = 0$, если многообразие $M$ открыто, и $c_t = -\int_M F_t \Vol$, если $M$ замкнуто.
Поскольку для любого $t$ функция $G_t$ совпадает с $H$ вблизи $S_t$ с точностью до аддитивной константы, заключаем, что $\sgrad G_t = \sgrad H$ вблизи $S_t$.
Следовательно, поток $\{g_t\}$ гамильтониана $G$ удовлетворяет условию $g_t (\partial A) = h_t (\partial A)$ и, значит, $g_1 (\partial A) \cap \partial A = \emptyset$.
Но отсюда, очевидно, следует, что $g_1 (A) \cap A = \emptyset$.
Заметим теперь, что с помощью отсечения вне очень малых окрестностей $S_t$ мы можем добиться, чтобы $L_p$-норма каждой функции $G_t$ была произвольно мала (в отличие от $L_\infty$-нормы!).
Следовательно, $L_p$-энергия смещения $A$ обращается в нуль.
Это завершает доказательство вырождения $\rho_p$.

Обратимся теперь ко второму утверждению теоремы, где мы предполагаем, что многообразие $M$ замкнуто.
Положим
\[G = \set{g \in \Ham (M, \Omega)}{\rho (\1, g) = 0}.\]
Возьмем $f, g \in G$.
Тогда, конечно, $g^{-1} \in G$.
Далее из неравенства треугольника следует, что $\rho (\1, f g) = \rho (f^{-1}, g) \le \rho (\1, f) + \rho (\1, g) = 0$, так что $G$ является подгруппой $\Ham (M, \Omega)$.
По биинвариантности мы знаем, что если $f \in G$ и $h \in \Ham (M, \Omega)$, то $hf h^{-1} \in G$ и, значит, $G$ --- нормальная подгруппа.
По теореме Баньяги \ref{1.5.A}, группа $\Ham (M, \Omega)$ проста.
Следовательно, либо $G = {\1}$, либо $G = \Ham (M, \Omega)$.
Мы уже доказали, что $\rho_p$ вырождено, поэтому $G \ne {\1}$.
Таким образом, $G$ совпадает со всей группой $\Ham (M, \Omega)$.
Мы заключаем, что $\rho_p$ тождественно обращается в нуль.

\begin{thm*}{Упражнение}
Докажите, что энергия смещения $S^{2n - 2} \subset \RR^{2n-1} \subset \RR^{2n}$ относительно $\rho_\infty$ равна нулю.
С другой стороны, как будет видно в \ref{2.4.B} следующей главы, существуют подмногообразия половинной размерности в $\RR^{2n}$, которые имеют положительную энергию смещения (сравни \ref{1.1.C} выше).
\end{thm*}

\begin{thm*}{Открытая задача}
Какие инвариантные нормы на $A$ порождают невырожденные функции расстояния $\rho$?
Верно ли, что такие нормы всегда ограничены снизу константой $\|\ \|_\infty$?
Сложность задачи в том, что нет классификации $\Ham (M, \Omega)$-инвариантных норм.
Возможный подход состоит в исследовании срезок.
Если срезка способна произвольно уменьшить норму, то приведенное выше рассуждение показывает, что $\rho$ вырождено.
\end{thm*}

\begin{thm*}[\cite{EP}]{Открытая задача} 
Вполне естественно рассматривать отдельно положительную и отрицательную части метрики~$\rho_\infty$.
Более явно, положим
\[\rho_+ (\1, f) = \inf\int_0^1 \max F_t dt\]
и
\[\rho_- (\1, f) = \inf \int_0^1-\min F_t dt.\]
Тогда очевидно, что
\[\rho (\1, f) \ge \rho_+ (\1, f) + \rho_- (\1, f).\]
Однако во всех известных мне примерах получается равенство!
Было бы интересно доказать общее утверждение или найти контрпример.
Заметим, что из \cite{V1} следует, что на $\Ham (\RR^{2n})$ сумма $\rho_+ + \rho_-$  определяет биинвариантную метрику.
Насколько мне известно, для общих симплектических многообразий никаких аналогов этого утверждения не доказано.
\end{thm*}

\parbf{Соглашение.}
Если в дальнейшем не указано иное, мы используем обозначение $\|\ \|$ для $L_\infty$-нормы на $\A(M)$.
Через $\rho$ обозначим метрику $\rho_\infty$ и назовем ее метрикой Хофера.
Величина $\rho (\1, f)$ называется хоферовской нормой $f$.
Мы пишем $\length \{f_t\}$ для длины гамильтонова пути $\{f_t\}$ относительно $L_\infty$-нормы (см. \ref{2.2} выше).
