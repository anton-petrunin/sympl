 \chapter{Знакомство с геометрией}

В этой главе мы обсуждаем биинвариантные финслеровы метрики на группе гамильтоновых диффеоморфизмов и определяем геометрию Хофера.

\section{Вариационная задача}

Какая минимальная энергия необходима для получения данного гамильтонова диффеоморфизма $\phi$? 
Этот естественный вопрос можно формализовать следующим образом.
Рассмотрим всевозможные гамильтоновы потоки $\{f_t\}$, $t \in [0; 1]$ такие, что $f_0 = \1$ и $f_1 = \phi$.
Для каждого потока возьмём, соответсвующий ему нормированый гамильтониан $F_t(x)$ и «измерим его величину».
Затем минимизируем результат измерения по всем таким потокам.
Остаётся понять, что значит «измерить его величину».
Напомним, что для любого $t$ функция $F_t$ является элементом алгебры Ли $\A$.
Возьмём любую безкоординатную норму $\|\ \|$ на $\A$,
т. е. потребуем, чтобы 
\begin{equation}
 \|H \circ \psi^{-1}\|
= \| H \|
\quad\text{для всех}\quad
H \in \A, \psi \in \Ham (M, \Omega).
\label{2.1.A}
\end{equation}
Теперь определим величину как  $\int_0^1\| F_t \| dt.$
Собирая вместе все ингредиенты этой процедуры, мы получаем вариационную задачу 
\begin{equation}
\inf\int_0^1 \| F_t \| dt, 
\label{2.1.B}
\end{equation}
где $\phi$ фиксировано, а точная нижняя грань берётся по всем потокам $\{f_t\}$, как указано выше.
