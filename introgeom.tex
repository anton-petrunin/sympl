 \chapter{Знакомство с геометрией}

В этой главе мы обсуждаем биинвариантные финслеровы метрики на группе гамильтоновых диффеоморфизмов и определяем геометрию Хофера.

\section{Вариационная задача}\label{2.1}

Какая минимальная энергия необходима для получения данного гамильтонова диффеоморфизма $\phi$? 
Этот естественный вопрос можно формализовать следующим образом.
Рассмотрим всевозможные гамильтоновы потоки $\{f_t\}$, $t \in [0; 1]$ такие, что $f_0 = \1$ и $f_1 = \phi$.
Для каждого потока возьмём, соответсвующий ему нормированый гамильтониан $F_t(x)$ и «измерим его величину».
Затем минимизируем результат измерения по всем таким потокам.
Остаётся понять, что значит «измерить его величину».
Напомним, что для любого $t$ функция $F_t$ является элементом алгебры Ли $\A$.
Возьмём любую безкоординатную норму $\|\ \|$ на $\A$,
т. е. потребуем, чтобы 
\begin{equation}
 \|H \circ \psi^{-1}\|
= \| H \|
\quad\text{для всех}\quad
H \in \A,\  \psi \in \Ham (M, \Omega).
\label{2.1.A}
\end{equation}
Теперь определим величину как  $\int_0^1\| F_t \| dt.$
Собирав все ингредиенты этой процедуры, мы получаем следующую вариационную задачу 
\begin{equation}
\inf\int_0^1 \| F_t \| dt, 
\label{2.1.B}
\end{equation}
где $\phi$ фиксировано, а точная нижняя грань берётся по всем потокам $\{f_t\}$, как сказано выше.

\section[Биинвариантные геометрии на $\Ham(M, \Omega)$]{Биинвариантные геометрии на $\bm{\Ham(M, \Omega)}$}

Приведённая вариационная задача может быть переформулирована в чисто геометрических терминах.
Для этого надо вспомнить понятие \emph{финслервой структуры} на многообразии.
Мы говорим, что многообразие $Z$ наделено финслерной структурой, если его касательные пространства $T_z Z$ оснащены нормой, которая гладко зависит от точек $z \in Z$.
Конечно же, римановы структуры образуют частный случай этого понятия.
Однако в общем случае, нормы могут и не задаваться скалярными произведениями.
Финслерву структуру можно использовать для определения длины кривой точно так, как риманову структуру 
\[\length \{z(t)\}_{t\in [a; b]} =\int_a^b \norm(\dot z (t)) dt.\]
Кроме того, можно ввести расстояние между двумя точками $z$ и $z'$ в $Z$ как инфимум длин всех кривых, соединяющих $z$ с $z'$.

Вернёмся к ситуации, описанной в предыдущем разделе.
Поскольку все касательные пространства группы $\Ham(M, \Omega)$ можно отождествить с $\A$ (см. раздел \ref{1.4} выше), каждый выбор нормы $\|\ \|$ на $\A$ определяет финслерову структуру на группе.
Таким образом, можно определить длину гамильтонова пути и расстояние между двумя гамильтоновыми диффеоморфизмами.
В частности, длина гамильтонова пути $\{f_t\}$, $t \in [a; b]$ с нормализованным гамильтонианом $F$ определяется следующим образом 
\[\length\{f_t\}=\int_a^b \| F_t \| dt.\]
А расстояние между двумя гамильтоновыми диффеоморфизмами $\phi$ и $\psi$ определяется как
\[\rho (\phi, \psi) = \inf \length \{f_t\},\] 
где инфимум берётся по всем гамильтоновым путям $\{f_t\}$, $t \in [a; b]$ с $f_a = \phi$ и $f_b = \psi$.
Конечно, длина пути не зависит от параметризации, таким образом, в приведённом определении расстояния можно считать, что $a = 0$ и $b = 1$.
На этом языке, решение вариационной задачи
\ref{2.1.B} есть ничто иное, как расстояние $\rho (\1, \phi)$!

Следующие свойства $\rho$ легко проверить и даются в качестве упражнения:
\begin{itemize}
\item $\rho (\phi, \psi) = \rho (\psi, \phi)$;
\item неравенство треугольника: $\rho (\phi, \psi) + \rho (\psi, \theta) \ge \rho (\phi, \theta)$;
\item $\rho (\phi, \psi) \ge 0$.
\end{itemize}

Напомним теперь об условии \ref{2.1.A}, которое было накложено на норму $\|\ \|$ в разделе \ref{2.1}.
На геометрическом языке это означает, что норма инвариантна относительно присоединённого действия группы на её алгебре Ли (см. \ref{1.4}).
В дальнейшем мы будем иметь дело только с такими нормами.

\begin{thm*}{Упражнение}
Покажите что из \ref{2.1.A} следует, что функция $\rho$ --- биинвариантна%
\footnote{без \ref{2.1.A} мы получаем только правoинвариантность $\rho$, то есть $\rho (\phi, \psi) \z= \rho (\phi\theta, \psi\theta)$.
Такие метрики играют важную роль в гидродинамике, см. \cite{AK}.}%
:
\[\rho (\phi, \psi) = \rho (\phi \theta, \psi \theta) = \rho (\theta\phi, \theta\psi)\]
для всех $\phi, \psi, \theta \in \Ham (M, \Omega)$.
\end{thm*}

Было бы честней назвать функцию $\rho$ \emph{псевдометрикой}.
Действительно, как мы видели выше, она удовлетворяет всем аксиомам метрики, за исключением, возможно, невырожденности, то есть 
\begin{equation}
\rho (\phi, \psi)> 0
\quad\text{при}\quad
\phi \ne \psi.
\label{eq:2.2.A}
\end{equation}

Даже в конечномерном случае, невырожденность проверить непросто.
В этом случае, доказательство использует локальнокомпактность многообразия.
В нашей ситуации группа $\Ham (M, \Omega)$ бесконечномерна и не имеет никаких свойств компактности.
Таким образом, априори у \ref{eq:2.2.A} нет причин быть верным.
Как мы скоро увидим, свойство невырожденности $\rho$ очень чувствительно к выбору нормы~$\|\ \|$.

\section[Выбор нормы: $L_p$ или $L_\infty$]{Выбор нормы: \bm{$L_p$} или \bm{$L_\infty$}}

Среди норм на $\A$, удовлетворяющих предположению инвариантности \ref{2.1.A} существует очень естественный класс, который включает нормы $L_p$, $p = 1, 2, 3,\dots$
\[\|H\|_p=\left(\int_M|H|^p\Vol\right)^{\frac1p},\]
и $L_\infty$-норму 
\[\|H\|_\infty = \max H - \min H.\]
Обозначим через $\rho_p$ и $\rho_\infty$ соответствующие псевдометрики.

\begin{thm}{Теорема}\label{2.3.A}
Псевдометрика $\rho_p$ вырождена для всех конечных $p = 1, 2,\dots$
Более того, если многообразие замкнуто, то все такие $\rho_p$ тождественно обращаются в нуль.
\end{thm}

Этот результат получен в \cite{EP}.
Доказательство представлено ниже в этой главе (см. также книги \cite{HZ}, \cite{MS}, \cite{AK}).
Следующая теорема показывает разительный контраст между случаями $L_p$ и $L_\infty$.

\begin{thm}{Теорема}\label{2.3.B}
Псевдометрика $\rho_\infty$ невырождена.
\end{thm}

Эта теорема%
\footnote{Историческое отступление, приведенное ниже, отражает мое личное мнение.
Допускаю, что другие участники этих событий видят дело иначе.}
была сформулирована и доказана Хофером в \cite{H1} для случая $M = \RR^{2n}$ с использованием бесконечномерных вариационных методов.
В \cite{V1} Витербо вывел его для случая $M = \RR^{2n}$ используя свою теории производящих функций.
Для обоих, Хофера и Витербо, толчком послужил вопрос, заданный Элиашбергом в частной беседе.
В \cite{P1} это утверждение было распространено на широкий класс симплектических многообразий с «хорошим» поведением на бесконечности и, в частности, на все замкнутые симплектические многообразия, у которых класс когомологий симплектической формы является рациональным.
Подход \cite{P1} основан на теории псевдоголоморфных кривых Громова.
Наконец \cite{LM1} Лалонде и Макдафф доказали \ref{2.3.B} в полной общности с используя теорию Громова.
К настоящему времени найдены другие доказательства различных частных случаев этой теоремы, см., например, \cite{Ch}, \cite{O3}, \cite{Sch3}.
Первоначальное доказательство Хофера подробно представлена в книге
\cite{HZ}.
Доказательство Лалонда и Макдафф изложены в книге \cite{MS} и в обзоре \cite{L}.
Ниже мы приводим другое доказательство для случая $M = \RR^{2n}$, которое следует из \cite{P1}.
Все известные доказательства основаны на «жёстких» методах.%
\footnote{Более того, на мой вкус, все доказательства далеко не прозрачны.
Аргумент, представленный в главе 3 ниже, не является исключением.
Я твердо верю, что в будущем будет найдено \emph{правильное объяснение} этому фундаментальному результату.}
