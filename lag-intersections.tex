\chapter{Лагранжевы пересечения}

Теория лагранжевых пересечений изучает одно из самых удивительных явлений симплектической топологии.
В этой главе мы рассмотрим некоторые результаты этой теории, которые в сочетании с идеей линеаризации, изложенной выше, дают довольно мощный инструмент для исследования геометрии группы гамильтоновых диффеоморфизмов.

\section{Точные лагранжевые изотопии}
Пусть $(V^{2n}, \omega)$ --- симплектическое многообразие, а $N^n$ --- замкнутое многообразие.
Пусть 
\[\Phi\: N \times [0, 1] \to V\]
--- гладкое семейство лагранжевых вложений, то есть $\Phi$ --- лагранжева изотопия.
Заметим, что $\Phi^\ast \omega$ должно иметь вид $\alpha_s \wedge ds$, где $\{\alpha_s\}$ --- семейство 1-форм на $N$ (поскольку $\Phi^\ast \omega$ обращается в нуль на слоях $N \times \{\point\}$).
Кроме того, заметим, что $d\Phi^\ast \omega = d\alpha_s \wedge ds = 0$, откуда следует, что $\alpha_s$ замкнута при всех $s$.

\begin{thm*}{Определение}
Лагранжева изотопия $\Phi$ точна, если $\alpha_s$ точна при всех $s$.
\end{thm*}

\begin{thm}{Упражнение}\label{6.1.A}
Покажите, что лагранжева изотопия точна тогда и только тогда, когда она может быть расширена до объемлющей гамильтоновой изотопии $V$.
Подсказка: Напишите $\alpha_s \z= dH_s$ на $N$ и продолжите нормализованную гамильтонову функцию $H_s \circ \Phi^{-1}_s$ на $V$.
\end{thm}

\begin{thm*}{Пример}
Пусть $V$ --- поверхность и $N = S^1$.
Тогда $\Phi$ является точным, тогда и только тогда ориентированная площадь между $\Phi (N \times {0})$ и $\Phi (N \times {s})$ равна нулю при всех $s$.
См. Рисунок 6 для случая $V = S^2$ и рисунок 7 для случая $V = T^\ast S^1 = \RR \times S^1$.
Отметим, что в случае цилиндра можно найти симплектическую изотопию, описывающую правую картину.
\end{thm*}

РИСУНОК 6

РИСУНОК 7

Следующий результат играет важную роль в нашем исследовании хоферовской геометрии.
Предположим, что $\{h_t\}$ --- петля гамильтоновых диффеоморфизмов на многообразии $(M, \Omega)$, порождённая $H \in \H$.
Пусть $L \subset M$ --- замкнутое лагранжево подмногообразие.
Рассмотрим лагранжеву надстройку (см. \ref{3.1.E} выше)
\[L \times S^1 \to (M \times T^\ast S^1, \Omega + dr \wedge dt),\]
\[(x, t) \mapsto (h_t x, -H (h_t x, t), t).\]
Наша цель --- исследовать поведение этого лагранжевого вложения при однопараметрической деформации.
Пусть $\{h_{t, s}\}$, $s \in [0; 1]$ --- гладкое семейство гамильтоновых петель.
Обозначим через $\Phi\: L \z\times S^1 \z\times [0; 1] \z\to M \z\times T^\ast S^1$ соответствующее семейство лагранжевых надстроек.

\begin{thm}{Теорема}\label{6.1.B}
Лагранжева изотопия $\Phi$ точна.
\end{thm}

Другими словами, гомотопия гамильтоновых петель порождает точную лагранжевую изотопию лагранжевых надстроек.
Доказательство теоремы основано на следующем вспомогательном результате.
Обозначим через $H (x, t, s)$ нормализованный гамильтониан петли $\{h_{t, s}\}$.

\begin{thm}{Предложение}\label{6.1.C}
Для любых $x \in M$ и $s \in [0; 1]$
\[\int_0^1 \frac{\partial H}{\partial s} (h_{t, s} x, t, s) dt = 0.\]
\end{thm}

\parbf{Доказательство предложения:}
Доказательство основано на следующей формуле, которая справедлива для произвольного двупараметрического семейства диффеоморфизмов на многообразии.
Проверка формулы предоставляется читателю (см. также \cite{B1}).
Рассмотрим векторные поля $X_{t, s}$ и $Y_{t, s}$ на $M$ такие, что 
\[\frac{d}{dt} h_{t, s} x = X_{t, s} (h_{t, s} x)
\quad\text{и}\quad
\frac{d}{ds} h_{t, s} x = Y_{t, s} (h_{t, s} x)
\]
Тогда 
\[\frac{\partial}{\partial s}  X_{t, s} = \frac{\partial}{\partial t}Y_{t, s} + [X_{t, s}, Y_{t, s}].\]
Обратите внимание, что для любых $t$ и $s$ поля $X_{t, s}$ и $Y_{t, s}$ являются гамильтоновыми векторными полями.
Конечно, $X_{t, s} = \sgrad H_{t, s}$, где $H$ определено выше.
Напишем $Y_{t, s} = \sgrad F_{t, s}$.
Напомним, что 
\[[\sgrad H, \sgrad F] = -\sgrad  \{H, F\}.\]
Таким образом, получаем, что 
\[\frac{\partial H_{t, s}}{\partial s}
= \frac{\partial F_{t, s}}{\partial t} - \{H_{t, s}, F_{t, s}\}
= \frac{\partial F_{t, s}}{\partial t}+dF_{t,s}(\sgrad H_{t,s}).
\]
Но последнее выражение, вычисленное в точке $h_{t, s} x$, равно
\[\frac{d}{dt} F (h_{t, s} x, t),\]
и мы заключаем, что 
\[\frac{\partial H_{t, s}}{\partial s} (h_{t, s} x)\]
--- полная производная периодической функции.
Поэтому её интеграл по периоду обнуляется.
Это завершает доказательство.
\qeds

\parbf{Доказательство \ref{6.1.B}:}
Запишем $\Phi^\ast (\Omega + dr \wedge dt)$ как $\alpha_s \wedge ds$.
Требуется проверить, что $\alpha_s$ точна.
Форму $\alpha_s$ можно вычислить явно.

\begin{thm*}[(сравни с \ref{3.1.E})]{Упражнение}
Докажите, что равенство
\[\alpha_s (\xi) = \Omega (h_{t, s\ast} \xi, \frac{\partial h_{t, s}}{\partial s}x)\] 
выполняется при всех
$x \in L$, $\xi \in T_x L$ и 
\[\alpha_s (\tfrac{\partial}{\partial t}) = \frac{\partial H}{\partial s}(h_{t, s}x, t, s).\]
\end{thm*}

Заметим, что первая группа гомологий $H_1 (L \times S^1, \ZZ)$ порождается расщеплёнными циклами вида $C = \beta \times \{0\}$ и $D = \{y\} \times S^1$.
Здесь $\beta$ --- цикл на $L$, а $y$ --- точка $L$.
Чтобы доказать точность формы $\alpha_s$, достаточно проверить, что её интегралы по всем 1-циклам обнуляются.
Для циклов вида $C$ это следует из того, что $h_{0, s} \equiv \1$ при всех $s$.
Таким образом, из приведенного выше упражнения следует, что $\alpha_s$ обращается в нуль на всех векторах, касающихся $L \times \{0\}$.
Далее, 
\[\int_D \alpha_s
= \int_0^1 \frac{\partial H}{\partial s} (h_{t, s} y, t, s) dt. 
\]
Это выражение обращается в нуль ввиду предложения \ref{6.1.C} выше, что завершает доказательство.
\qeds
