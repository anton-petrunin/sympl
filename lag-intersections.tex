\chapter{Линеаризация хоферовской геометрии}

В этой главе мы будем думать про $\rho (\1, \phi)$ как про расстояние между точкой и подмножеством в линейном нормированном пространстве.
Позже это позволит нам получить оценки снизу на $\rho (\1, \phi)$ в некоторых интересных случаях.

\section{Пространство периодических гамильтонианов}

Обозначим через $\F$ пространство всех гладких нормированных гамильтонинанов $F\: M \times \RR \to \RR$, 1-периодических по времени: $F (x, t + 1) = F (x, t)$ для всех $x \in M$.
Мы часто будем рассматривать такие $F$ как функции на $M \times S^1$, где $S^1 = \RR / \ZZ$.
Для гамильтониана $F \in \F$ обозначим через $\phi_F$ отображение $f_1$ соответствующего гамильтонова потока $\{f_t\}$.
Заметим, что каждый гамильтонов диффеоморфизм $\phi$ может быть представлен таким образом.
В самом деле, пусть $\{g_t\}$ --- любой поток с $g_1 = \phi$.
Выберем функцию $a\: [0; 1] \to [0; 1]$ такую, что $a \equiv 0$ в окрестности нуля и $a \equiv 1$ в окрестности единицы.
Рассмотрим новый поток $f_t = g_{a(t)}$, 
продолжим его на всё $\RR$ по формуле $f_{t+1} = f_t f_1$.
Ясно, что поток получится гладким.
Утверждение следует из следующего упражнения.

\begin{thm}{Упражнение}\label{5.1.A}
Докажите, что \?{гамильтонов}{добавил} поток $\{f_t\}$, $t \in \RR$, порождается гамильтонианом из $\F$ тогда и только тогда, когда $f_{t+1} = f_t f_1$ для всех $t$.
\end{thm}

Рассмотрим подмножество $\H \subset \F$, определенное как 
\[\H = \set{H \in \F}{\phi_H = \1}.\]
Другими словами, гамильтонианы из $\H$ порождают петли гамильтоновых диффеоморфизмов (или гамильтоновы петли).
Определим норму на $\F$ 
\[\VERT F \VERT = \max \| F_t \| = \max (\max F (x, t) - \min F (x, t)).\]
\?{}{Может добавить: Из леммы \ref{5.1.C} ниже и невырожденности хоферовской метрики следует, что $\VERT\ \VERT$ является нормой на линейном пространстве $\F$.}
Теперь мы можем сформулировать основную теорему этой главы.

\begin{thm}{Теорема}\label{5.1.B}
\[\rho (\1, \phi_F) = \inf_{H\in\H} \VERT F - H \VERT\]
для любого $F \in \F$ .
\end{thm}

\begin{figure}[ht!]
\vskip0mm
\centering
\includegraphics{mppics/pic-5}
\caption{}\label{pic-5}
\vskip0mm
\end{figure}

Обратите внимание, что правая часть --- это просто расстояние от $F$ до $\H$ в смысле нашей нормы (см. рисунок 5).
Таким образом, множество $\H$ многое помнит о хоферовской геометрии.
В следующих главах мы установим некоторые интересные свойства $\H$ и внимательно посмотрим на гамильтоновы петли.

Теорема~\ref{5.1.B} --- простое следствие следующего факта.

\begin{thm}{Лемма}\label{5.1.C}
\[\rho (\1, \phi) = \inf \VERT F \VERT\]
для любого $\phi \in \Ham (M)$, здесь нижняя грань берётся по всем гамильтонианам из $\F$, порождающим $\phi$.
\end{thm}

Используя терминологию некоторых моих работ, это значит, что «грубая» хоферовская норма совпадает с обычной.

\parbf{Доказательство \ref{5.1.C}:}
Для $\phi \in \Ham (M, \Omega)$ положим $r (\1, \phi) \z= \inf \VERT F \VERT$ где $F$ пробегает все гамильтонианы $F \in \F$, порождающие~$\phi$.
Ясно, что $r (\1, \phi) \ge \rho (\1, \phi)$.
Остаётся доказать обратное неравенство.
Зафиксируем положительное число \?{$\epsilon$}{У Лёни epsilon и varepsilon в ходу --- оставляем только varepsilon.}.
Выберем путь $\{f_t\}$, $t \in [0; 1]$ гамильтоновых диффеоморфизмов таких, что $f_0 = \1$, $f_1 = \phi$ и  $\int_0^1 m (t) dt \le \rho (\1, \phi) + \epsilon$, где $m (t) = \| F_t \|$.
Не умоляя общности, можно считать, что $F \in \F$ и $m (t)> 0$ для всех $t$.
В самом деле, чтобы гарантировать периодичность, можно провести репараметризацию времени, как это делается в начале раздела.
Обоснование второго предположения даётся в следующем разделе.
Обозначим через $\C$ пространство всех $C^1$-гладких диффеоморфизмов $S^1$, сохраняющих ориентацию и фиксирующих $0$.
Отметим, что для $a \in \C$ путь $f_a = \{f_{a(t)}\}$ порождается нормированным гамильтонианом $F^a (x, t) = a' (t) F (x, a(t))$, где $a'$ обозначает производную по $t$ (см. \ref{1.4.A} выше).
Пусть $a(t)$ есть обратная функция к 
\[b(t)
=
\frac{\int_0^t m(s)ds}{\int_0^1 m(s)ds}.\]
Обратите внимание, что 
\[\VERT F \VERT = \max a' (t) m (a (t)) = \max (m (t) / b'(t)) = \int_0^1m (t) dt.\]
Мы заключаем, что $\VERT F^a \VERT \le \rho (\1, \phi) + \epsilon$.
Приближая $a$ в топологии $C^1$ гладким диффеоморфизмом из $\C$, мы видим, что можно найти гладкий нормализованный гамильтониан, скажем $\tilde F$, который порождает $\phi$ и такой, что $\VERT \tilde F \VERT \le \rho (\1, \phi) + 2\epsilon$.
Поскольку это можно сделать для произвольного $\epsilon$, заключаем, что $r (\1, \phi) \le \rho (\1, \phi)$, что завершает доказательство.
\qeds

\parbf{Доказательство \ref{5.1.B}:}
Будем обозначать через $\{f_t\}$ гамильтонов поток, порожденный $F$.
Пусть $\{g_t\}$ --- любой другой гамильтонов поток, порожденный $G \in \F$ с $g_1 = \phi_F$.
Разложим $g_t$ как $h_t \circ f_t$.
Как следует из \ref{5.1.A}, $\{h_t\}$ является петлей гамильтоновых диффеоморфизмов, то есть его нормированный гамильтониан $H$ принадлежит $\F$ и $h_0 = h_1 = \1$.
Наоборот, для каждой петли $\{h_t\}$ поток $\{h_t \circ f_t\}$ порождается гамильтонианом из $\F$,%
\footnote{Вообще говоря случае поток $\{f_t \circ h_t\}$ (порядок важен) \emph{не} порождается периодическим гамильтонианом!}
и в единичное временя он равен~$\phi_F$.
Далее, 
\[G (x, t) = H (x, t) + F (h^{-1}_t x, t).\]
Пусть $H' (x, t) = -H (h_t x, t)$.
Обратите внимание, что $H'$ порождает петлю $\{h^{-1}_t\}$ и, следовательно, $H'\in\H$.
С другой стороны, из приведённого выше выражения для $G$ следует, что $\VERT G \VERT = \VERT F - H' \VERT$.
Ввиду вышесказанного, каждому $G$ соответствует единственный $H'$ и наоборот.
Таким образом, требуемое утверждение немедленно следует из леммы \ref{5.1.C}.

\section{Регуляризация}\label{5.2}

В этом разделе мы обоснуем предположение $m(t)>0$, в доказательстве леммы \ref{5.1.C}.
Поток $\{f_t\}$ называется регулярным, если для любого $t$ нормализованный гамильтониан $F_t$ не обращается в нуль тождественно.
Другими словами, если для любого $t$ касательный вектор к пути $\{f_t\}$ не обращается в ноль.

\begin{thm}{Предложение}\label{5.2.А}
Пусть $\{f_t\}$ --- поток, порожденный гамильтонианом из $\F$.
Тогда существует произвольно малая (в $C^\infty$-смысле) петля $\{h_t\}$ такая, что поток $\{h^{-1}_t f_t\}$ регулярен.
\end{thm}

Доказательство разбито на несколько шагов.

1) Сначала поймём что собственно надо доказывать.
Предположим, что $\{h_t\}$ --- петля, порожденная гамильтонианом $H \in \H$.
Тогда гамильтониан потока $\{h^{-1}_t f_t\}$ задаётся выражением $-H (h_t x, t) \z+ F (h_t x, t)$.
Мы должны доказать, что для любого $t$ это выражение не обращается в нуль тождественно.
Но это равносильно утверждению, что\?{}{это уравнение должно быть 5.2.B, НО нет уравнения 5.2.А!} 
\begin{equation}
F (x, t) - H (x, t) \not\equiv 0\label{eq:5.2.B}
\end{equation}
при всех $t$.
Итак, нам надо построить сколь угодно малый гамильтониан $H \in \H$, удовлетворяющий \ref{eq:5.2.B}

2) Введём ещё одно полезное понятие:
$k$-мерной вариации постонной петли --- это гладкое семейство петель $\{h_t (\epsilon)\}$, где $\epsilon$ принадлежит окрестности $0$ в $\RR^k$ и $h_t (0) = \1$ для всех $t$.
Если $M$ открыто, то мы дополнительно требуем, чтобы носители всех $h_t (\epsilon)$ лежали в некотором компактном подмножестве $M$.

Вот удобный способ создания вариаций.
Начнём с однопараметрического случая.
Возьмем гамильтониан $G \in \F$ такой, что 
\begin{equation}
\int_0^1 G (x, t) dt = 0 
\label{eq:5.2.C}
\end{equation}
для любого $x \in M$.
Затем определим $h_t (\epsilon) \in \Ham (M, \Omega)$ как гамильтонов поток в момент $\epsilon$, порожденный не зависящим от времени гамильтонианом $\int_0^t G(x,s) ds$.

\begin{thm}{Упражнение}\label{5.2.D}
Пусть $H (x, t, \epsilon)$ --- нормализованный гамильтониан петли $\{h_t (\epsilon)\}$.
Докажите, что 
\[\frac{\partial}{\partial \epsilon}|_{\epsilon=0} H (x, t, \epsilon) = G (x, t).\]
\end{thm}

Для построения $k$-мерной вариации, естественно взять композицию одномерных: 
\[
h_t (\epsilon_1 ,\dots, \epsilon_k)
=
h_t^{(1)} (\epsilon_1) \circ\dots
\circ h_t^{(k)} (\epsilon_k).
\]

Каждый $h^{(j)}$ строится с помощью функции $G^{(j)}$, как указано выше.
Упражнение~\ref{5.2.D} говорит, что частная производная гамильтониана $H (x, t, \epsilon)$ по $\epsilon_j$ при $\epsilon = 0$ равна $G (j)$.

3) Выберем точку $y\in M$ и рассмотрим $2n$-мерное линейное пространство $E = T_y^\ast M$.
Выберем $2n$ гладких замкнутых кривых $\alpha_1 (t),\z\dots, \alpha_{2n} (t)$ (где $t \in S^1$), удовлетворяющие следующим условиям:
\begin{itemize}
\item $\int_0^1 \alpha_j (t) dt = 0$ для всех $j = 1,\dots, 2n$; 
\item векторы $\alpha_1 (t),\dots, \alpha_{2n} (t)$ линейно независимы при любом $t$.
\end{itemize}
Вот конструкция такой системы кривых.
Выбираем базис $u_1, v_1,\z\dots, u_n, v_n$ в $E$ и возьмём кривые вида $u_j\cos 2πt + v_j\sin 2πt$ и $-u_j\sin 2πt \z+ v_j\cos 2πt$.

4) Теперь выберем функции $G_1(x,t),\dots, G_{2n}(x,t)$ в $\F$, удовлетворяет условию \ref{eq:5.2.C} выше и такие, что $d_y G_t^{(j)} = \alpha_j (t)$.
Рассмотрим соответствующую $2n$-мерную вариацию $\{h_t (\epsilon)\}$ постоянной петли, как на шаге 2.
Рассмотрим отображение $\Phi\: S^1 \times \RR^{2n} (\epsilon_1 ,\dots, \epsilon_{2n}) \to E$, определенное как 
\[(t, \epsilon) \to d_y (F_t - H_t (\epsilon)).\]
Заметим, что $\Phi$ --- субмерсия в некоторой окрестности $U$ окружности $\{\epsilon = 0\}$.
Действительно, наша конструкция вместе с обсуждением на шаге 2 влечёт, что 
\[\frac{\partial}{\partial\epsilon}|_{\epsilon = 0} \Phi (t, \epsilon) = \alpha_j (t),\]
а эти векторы порождают всё $E$.
Обозначим через $\Psi$ сужение $\Phi$ на $S^1 \times U$.
Поскольку $\Psi$ является субмерсией, множество $\Psi^{-1} (0)$ является одномерным подмногообразием в $S^1 \times U$, поэтому его проекция на $U$ нигде не плотна.
Таким образом, существуют произвольные малые значения параметра $\epsilon$ такие, что $d_y (F_t - H_t (\epsilon)) \ne 0$ при всех~$t$.
Следовательно, для каждого $t$ выполняется условие \ref{eq:5.2.B} выше.
Конец доказательства.
\qeds

\section{Пути в данном гомотопическом классе}

Для нас гомотопия будет гладким однопараметрическим семейством путей.
Мы будем рассматривать гомотопии незамкнутых путей с фиксированными конечными точками, а также гомотопии петель, с базовой точкой $\1$, если не указано иное.
Когда многообразие $M$ открыто, мы, как обычно предполагаем, что носители всех диффеоморфизмов, входящих в двупараметрические семейства, содержатся в компактном подмножестве объемлющего $M$.

Возьмём гамильтониан $F \in \F$ и обозначим через $\{f_t\}$ соответствующий гамильтонов поток.
Рассмотрим величину \[\l(F) = \inf \length \{g_t\},\] где нижняя грань берётся по всем гамильтоновым путям $\{g_t\}$, $t \in [0; 1]$ с $g_0 = \1$, $g_1 = \phi_F$, которые гомотопны $\{f_t\}$ с фиксированными концами.
Сделаем набросок полезной интерпретации этой величины.
Рассмотрим универсальное покрытие \?{$Z$}{далее используется $\widetilde\Ham(M, \Omega)$} пространства $(\Ham (M, \Omega), \1)$.
Оно определяется стандартным образом с небольшим исключением, что мы рассматриваем только гладкие пути и гладкие гомотопии.
Финслерова структура на $\Ham (M, \Omega)$ канонически поднимается в универсальное накрытие.
Отсюда возникает понятие длины гладкой кривой на $Z$ а значит и метрика $\tilde\rho$ на $Z$.

Обозначим через $\tilde\1$ канонический подъём $\1$ в $Z$, а через $\tilde\phi_F$ --- поднятие $\phi_F$, определяемое путём $\{f_t\}$, $t \in [0; 1]$.
На этом языке $l (F) = \tilde\rho (\tilde\1, \tilde\phi_F)$, поэтому эта величина отвечает за геометрию универсального покрытия.
Обозначим через $\H_c$ множество всех гамильтонианов из $\H$, порождающих стягиваемые петли.
Другими словами, $\H_c$ --- компонента линейной связности $0$ в $\H$.

\begin{thm}{Теорема}\label{5.3.A}
Для любого $F \in \F$ 
\[l (F) = \inf_{H\in\H_c} \VERT F - H \VERT.\]
\end{thm}

Доказательство то же, что в \ref{5.1.B} выше.
В ходе доказательства следует учитывать следующие простые дополнительные наблюдения:
\begin{itemize}
\item Временная репараметризация, а также процедура регуляризации \ref{5.2} не меняют гомотопический класс пути с фиксированными концами.
Таким образом минимизировать $\VERT G \VERT$ надо по всем $G \in \F$ с $\phi_G = \phi_F$ и таких, что гамильтонов поток $\{g_t\}$ гомотопен $\{f_t\}$ (см. \ref{5.1.C} выше).
\item Если $g_t = h_t \circ f_t$, где $\{f_t\}$ и $\{g_t\}$ гомотопны с фиксированными концами, то петля $h_t$ стягиваема (сравни с концом доказательства \ref{5.1.B}).
\end{itemize}
Предоставляем читателю завершить доказательство.
