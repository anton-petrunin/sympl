\chapter{Лагранжевы пересечения}\label{chap:6}

Теория лагранжевых пересечений изучает одно из самых удивительных явлений симплектической топологии.
В этой главе мы рассмотрим некоторые результаты этой теории, которые в сочетании с идеей линеаризации, изложенной выше, дают довольно мощный инструмент для исследования геометрии группы гамильтоновых диффеоморфизмов.

\section{Точные лагранжевы изотопии}
Пусть $(V^{2n}, \omega)$ — симплектическое многообразие, а $N^n$ —
замкнутое многообразие. 
Рассмотрим лагранжеву изотопию
\[\Phi\: N \times [0;1] \to V,\]
то есть $\Phi$ — гладкое семейство лагранжевых вложений.
Заметим, что $\Phi^\ast \omega$ должно иметь вид $\alpha_s \wedge \d
s$, где $\{\alpha_s\}$ — семейство 1-форм на $N$ (поскольку
$\Phi^\ast \omega$ обнуляетрся на слоях $N \times \{\point\}$). 
Кроме того, заметим, что $\d \Phi^\ast \omega = \d \alpha_s \wedge \d
s = 0$, откуда следует, что $\alpha_s$ замкнута при всех $s$. 

\begin{ex*}{Определение}
Лагранжева изотопия $\Phi$ \rindex{точная изотопия}\emph{точна}, если $\alpha_s$ точна при всех $s$.
\end{ex*}

\begin{ex}{Упражнение}\label{6.1.A}
Покажите, что лагранжева изотопия точна тогда и только тогда, когда
она может быть расширена до объемлющей гамильтоновой изотопии $V$. 
\emph{Подсказка:} Заметим, что $\alpha_s \z= \d  H_s$ на $N$ и продолжим $H_s
\circ \Phi^{-1}_s$ до нормализованного гамильтониана на $V$. 
\end{ex}

\begin{ex*}{Пример}
Пусть $V$ — поверхность и $N = S^1$.
Лагранжева изотопия $\Phi$ точна, тогда и только тогда когда
ориентированная площадь между $\Phi (N \times {0})$ и $\Phi (N \times
{s})$ равна нулю при всех $s$. 
Случай $V = S^2$ показан на рис.~\ref{pic-6}, а  случай $V = \T^\ast S^1 \z= \RR \times S^1$ на рис.~\ref{pic-7}.
Заметим, что в случае цилиндра можно найти симплектическую изотопию, описывающую правую картину.
\end{ex*}


\begin{figure}[ht!]
\begin{minipage}{.48\textwidth}
\centering
\includegraphics{mppics/pic-6}
\end{minipage}\hfill
\begin{minipage}{.48\textwidth}
\centering
\includegraphics{mppics/pic-7}
\end{minipage}

\medskip

\begin{minipage}{.48\textwidth}
\centering
\caption{}\label{pic-6}
\end{minipage}\hfill
\begin{minipage}{.48\textwidth}
\centering
\caption{}\label{pic-7}
\end{minipage}
\vskip-4mm
\end{figure}

Следующий результат играет важную роль в дальнейшем исследовании хоферовской геометрии.
Предположим, что $\{h_t\}$ — петля гамильтоновых диффеоморфизмов, порождённая гамильтонианом $H \in \H$ на $(M, \Omega)$.
Пусть $L \subset M$ — замкнутое лагранжево подмногообразие.
Рассмотрим лагранжеву надстройку (см. \ref{3.1.E})
\[L \times S^1 \to (M \times \T^\ast S^1, \Omega + \d r \wedge \d t),\]
\[(x, t) \mapsto (h_t x, -H (h_t x, t), t).\]
Наша цель — исследовать поведение этого лагранжева вложения при однопараметрической деформации.
Пусть $\{h_{t, s}\}$, $s \in [0; 1]$ — гладкое семейство гамильтоновых петель.
Обозначим через $\Phi\: L \z\times S^1 \z\times [0; 1] \z\to M \z\times \T^\ast S^1$ соответствующее семейство лагранжевых надстроек.

\begin{thm}{Теорема}\label{6.1.B}
Лагранжева изотопия $\Phi$ точна.
\end{thm}

Другими словами, гомотопия гамильтоновых петель порождает точную
лагранжевую изотопию лагранжевых надстроек. 
Доказательство теоремы основано на следующем вспомогательном результате.
В его формулировке $H (x, t, s)$ обозначает нормализованный
гамильтониан петли $\{h_{t, s}\}$.
(Здесь $t$ и $s$ выполняют разную роль, $t$ — временн\'{а}я переменная, а $s$ параметр гомотопии.)

\begin{thm}{Предложение}\label{6.1.C}
Равенство 
\[\int_0^1 \frac{\partial H}{\partial s} (h_{t, s} x, t, s)\,\d t = 0\]
выполняется при любых $x \in M$ и $s \in [0; 1]$.
\end{thm}

\parit{Доказательство предложения.}
Доказательство основано на следующей формуле, которая выполняется для произвольного двупараметрического семейства диффеоморфизмов на многообразии.
Проверка формулы предоставляется читателю (см. также \cite{B1}).
Рассмотрим векторные поля $X_{t, s}$ и $Y_{t, s}$ на $M$ определяемые как 
\[\frac{\d }{\d t} h_{t, s} x = X_{t, s} (h_{t, s} x)
\quad\text{и}\quad
\frac{\d }{\d s} h_{t, s} x = Y_{t, s} (h_{t, s} x).
\]
Тогда 
\[\frac{\partial}{\partial s}  X_{t, s} = \frac{\partial}{\partial t}Y_{t, s} + [X_{t, s}, Y_{t, s}].\]
Обратите внимание, что $X_{t, s}$ и $Y_{t, s}$ являются гамильтоновыми векторными полями при любых $t$ и $s$.
Конечно же, $X_{t, s} = \sgrad H_{t, s}$ для $H$ определённого выше.
Напишем $Y_{t, s} = \sgrad F_{t, s}$.
Напомним, что 
\[[\sgrad H, \sgrad F] = -\sgrad  \{H, F\}.\]
Таким образом, получаем, что 
\[\frac{\partial H_{t, s}}{\partial s}
= \frac{\partial F_{t, s}}{\partial t} - \{H_{t, s}, F_{t, s}\}
= \frac{\partial F_{t, s}}{\partial t}+\d F_{t,s}(\sgrad H_{t,s}).
\]
Но последнее выражение, вычисленное в точке $h_{t, s} x$, равно
\[\frac{\d }{\d t} F (h_{t, s} x, t),\]
и мы заключаем, что 
\[\frac{\partial H_{t, s}}{\partial s} (h_{t, s} x)\]
равна полной производной периодической функции.
В частности её интеграл по периоду равен нулю, и предложение следует.
\qeds

\parit{Доказательство \ref{6.1.B}.}
Запишем $\Phi^\ast (\Omega + \d r \wedge \d t)$ как $\alpha_s \wedge \d s$.
Требуется проверить, что $\alpha_s$ точна.
Форму $\alpha_s$ можно вычислить явно.

\begin{ex*}[(ср. с \ref{3.1.E})]{Упражнение}
Докажите, что равенство
\[\alpha_s (\xi) = \Omega (h_{t, s\ast} \xi, \frac{\partial h_{t, s}}{\partial s}x)\] 
выполняется при всех
$x \in L$, $\xi \in \T_x L$ и 
\[\alpha_s (\tfrac{\partial}{\partial t}) = \frac{\partial H}{\partial s}(h_{t, s}x, t, s).\]
\end{ex*}

Заметим, что первая группа гомологий $H_1 (L \times S^1;\ZZ)$ порождается расщеплёнными циклами вида $C = \beta \times \{0\}$ и $D = \{y\} \times S^1$.
Здесь $\beta$ — цикл на $L$, а $y$ — точка $L$.
Чтобы доказать точность формы $\alpha_s$, достаточно проверить, что её интегралы по всем 1-циклам обнуляются.
Для циклов вида $C$ это следует из того, что $h_{0, s} \equiv \1$ при всех~$s$.
Таким образом, из приведённого выше упражнения следует, что~$\alpha_s$ обращается в нуль на всех векторах, касающихся $L \times \{0\}$.
Далее, 
\[\int_D \alpha_s
= \int_0^1 \frac{\partial H}{\partial s} (h_{t, s} y, t, s)\,\d t. 
\]
По предложению \ref{6.1.C}, это выражение равно нулю, что завершает доказательство.
\qeds

\section{Лагранжевы пересечения}

Мы говорим, что лагранжево подмногообразие $N \subset V$ обладает \rindex{свойство лагранжева пересечения}\emph{свойством лагранжева пересечения}, если $N$ пересекает свой образ при любой точной лагранжевой изотопии.
Согласно упражнению~\ref{6.1.A}, это можно переформулировать так: $N \cap \phi (N) \ne \emptyset$ при всех $\phi \in \Ham (V, \omega)$, или, другими словами, энергия смещения $N$ бесконечна: $e (N) = + \infty$.

\subsection*{Примеры} 

\begin{ex}[Задача об инфинитезимальном лагранжевом пересечении.]{}\\
Пусть $F$ — автономный гамильтониан на $V$, а $\xi = \sgrad F$ — его гамильтоново
векторное поле. 
Тогда $\xi$ касается $N$ в критических точках $F|_N$ и только в них
(упражнение). 
Поскольку $N$ замкнуто, $F|_N$ должно иметь критические точки, и,
следовательно, нельзя сдвинуть с себя $N$ бесконечно малой
гамильтоновой изотопией. 
\end{ex}

\begin{ex}[Теорема Громова]{}\label{6.2.B}\rindex{Громов}
Если $\pi_2 (V, N) = 0$ и $V$ имеет «хорошее» поведение на
бесконечности (скажем, $V$ является произведением замкнутого
многообразия и кокасательного расслоения), Громов \cite{G1} (см. также
статью \rindex{Флоер}Флоера \cite{F}) показал, что $N$ обладает свойством лагранжева
пересечения. 
В частности, это относится к окружности $\{r = 0\}$ в $\T^\ast S^1$
(конечно же, это можно доказать элементарным подсчётом площадей,
см. рис.~7 и обсуждение выше). 
В более общем смысле это справедливо для любой нестягиваемой кривой на
ориентированной поверхности. 
Набросок доказательства теоремы Громова представлен в разделе \ref{3.2.G}.
Полное доказательство дано в \cite[Chap. X]{AL}. 

При $\pi_2 (V, N) \ne 0$, свойство лагранжева пересечения может нарушаться.
Возьмём, например, крошечную окружность $N$ на $V \z= S^2$ и сместим его, см. рис.~8.
Однако свойство лагранжева пересечения, очевидно, выполняется для экватора (используйте то, что экватор делит сферу на равновеликие диски).
\end{ex}




\begin{ex}{Определение}\label{6.2.C}
Пусть $L$ — замкнутое лагранжево подмногообразие симплектического многообразия $(M, \Omega)$.
Мы говорим, что $L$ обладает свойством \rindex{устойчивое лагранжево пересечение}\emph{устойчивого лагранжева пересечения}, если $L \times \{r = 0\}$ обладает свойством лагранжева пересечения в $(M \z\times \T^\ast S^1, \Omega + \d r \wedge \d t)$.
\end{ex}


\begin{wrapfigure}{o}{25 mm}
\vskip-3mm
\centering
\includegraphics{mppics/pic-8}
\caption{}\label{pic-8}
\vskip0mm
\end{wrapfigure}


Пара важных примеров будет приведена в следующих главах.

\begin{ex}[Торы в $\T^\ast \TT^n$.]{}\label{6.2.D}
Рассмотрим лагранжев тор в кокасательном расслоении $\T^\ast \TT^n$, наделённый стандартной симплектической структурой (см. \ref{3.1.C}).
Предположим, что он гомологичен нулевому сечению.
Легко проверить, что топологическое предположение теоремы \ref{6.2.B} выполняется.
Следовательно, такие торы обладают свойством устойчивого лагранжева пересечения.
\end{ex}




\begin{ex}[Экватор на $S^2$.]{}\label{6.2.E}
Свойство устойчивого лагранжева пересечения также имеет место для экваторов в $S^2$.
Это следует из сложной теоремы О \cite{O1,O2}, основанной на хитроумной версии гомологий \rindex{Флоер}Флоера.
\end{ex}


Мне не известен пример замкнутого связного лагранжева подмногообразия, обладающего свойством лагранжева пересечения, но не обладающее его устойчивой версией.


\section{Приложение к гамильтоновым петлям}

Пусть $(M, \Omega)$ — симплектическое многообразие.
Предположим, что $L \subset M$ — замкнутое лагранжево подмногообразие, обладающее свойством устойчивого лагранжева пересечения.  
Пусть $\{g_t\}$ — петля гамильтоновых диффеоморфизмов, порождённая
гамильтонианом $G \in \H$. 
Предположим дополнительно, что
\begin{itemize}
\item $g_t (L) = L$ при всех $t \in S^1$; 
\item $G (x, t) = 0$ при всех $x \in L$, $t \in S^1$.
\end{itemize}
Очевидным примером такой петли является постоянная петля $g_t \z\equiv \1$.
Менее простой пример будет приведён в \ref{6.3.C}.

\begin{thm}{Теорема}\label{6.3.A}
Пусть $\{h_t\}$ — любая другая петля гамильтоновых диффеоморфизмов, гомотопная описанной выше петле $\{g_t\}$.
Пусть $H \in \H$ — её гамильтониан.
Тогда существуют такие $x \in L$ и $t \in S^1$, что $H (x, t) = 0$.
\end{thm}

Как мы увидим в следующей главе, этот результат приводит к нетривиальным нижним оценкам на хоферовские расстояния.

\parit{Доказательство.} 
Дважды применим к $L$ конструкцию лагранжевой надстройки, сначала для $\{g_t\}$, и затем для $\{h_t\}$.
Обозначим через $NG$ и $NH$ соответствующие лагранжевы подмногообразия в $M \times \T^\ast S^1$.
По формуле для лагранжевой надстройки мы знаем, что $NG = L \times \{r = 0\}$.
Теорема \ref{6.1.B} говорит, что существует точная лагранжева
изотопия из $NH$ в $NG$. 
Таким образом, из свойства устойчивого лагранжева пересечения следует,
что $NH \cap NG \ne \emptyset$. 
Пусть $(x, 0, t)$, $x \in L$ — точка пересечения.
Поскольку она лежит в $NH$, имеем $x = h_t y$ и $0 = -H (h_t y, t, 0)$
при некотором $y \in L$. 
Мы заключаем, что $H (x, t) = 0$.
\qeds

Напомним, что $\H_c$ обозначает пространство всех 1-периодических
гамильтонианов, порождающих стягиваемые петли гамильтоновых
диффеоморфизмов. 
Как непосредственное следствие приведённой выше теоремы мы получаем
следующий результат. 

\begin{thm}{Следствие}\label{6.3.B}
Пусть $L \subset M$ — замкнутое лагранжево подмногообразие,
обладающее свойством устойчивого лагранжева пересечения. 
Тогда для любого $H \in \H_c$ существуют $x \in L$ и $t \in S^1$ такие, что $H (x, t) = 0$.
\end{thm}


Например, это верно, когда $M = S^2$, а $L$ — экватор в $S^2$.
Заметим, что утверждение следствия в общем случае становится неверным,
если мы не предполагаем свойства устойчивого лагранжева пересечения.

\begin{ex}{Пример}\label{6.3.C}
Рассмотрим евклидово пространство $\RR^3 (x_1, x_2, x_3)$.
Пусть $M = S^2$ — единичная сфера в пространстве, наделённая
индуцированной формой площади. 
Полный оборот вокруг оси $x_3$ — это гамильтонова петля, порождённая
нормализованной функцией Гамильтона $F_1 (x) = 2\pi x_3$. 
(В этом можно убедиться, используя вычисления в \ref{1.4.H}.)
Таким образом, $F_k (x) = 2\pi k x_3$ порождает петлю из $k$ оборотов.
Поскольку гамильтониан $F_k$ тождественно обращается в нуль на экваторе $L = \{x_3
= 0\}$, он должен иметь нуль на каждой простой
замкнутой кривой на $S^2$, разделяющей сферу на равновеликие части. 
Обратите внимание на то, что, когда $k$ чётно, $k$ оборотов сферы
$S^2$ представляет собой стягиваемую петлю в $\SO (3)$ и,
следовательно, в $\Ham (S^2)$. 
С другой стороны, $F_k \equiv 2\pi k\epsilon$ на окружности
$C_\epsilon = \{x_3 = \epsilon\}$. 
Отсюда делаем вывод, что явление, описанное в \ref{6.3.B} очень жёсткое.
Оно полностью исчезает, если рассматривать окружности, которые делят
сферу на произвольно близкие, но неравные по площади части. 
В самом деле, для любого положительного $\epsilon$ можно выбрать $k$ так,
чтобы $F_k$, сколь угодно велико на $C_\epsilon$!
\?{}{Вроде здесь Лёня хочет что-то добавить, но я не совсем понял что, он говорит You can displace
such $C_\epsilon$
"very far away".
This construction had a recent impact - to discuss.
}
Решающим моментом, конечно же, является то, что окружность
$C_\epsilon$ не обладает свойством лагранжева пересечения. 
Её можно сместить с себя элементом $\SO (3)$ (ср. с рис.~\ref{pic-8}). 
\end{ex}
