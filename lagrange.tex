\chapter{Лагранжевы подмногообразия}\label{chap:3}

Цель этой и следующей глав --- доказать невырожденность хоферовской метрики на $\RR^{2n}$.
Мы будем использовать подход из \cite{P1}.
Для этого мы введём понятие лагранжевых подмногообразий в симплектических многообразиях.
Лагранжевы подмногообразия играют основную роль в симплектической топологии, а также в её приложениях к механике и вариационному исчислению.
Они будут многократно появляться в этой книге далее.

\section{Определения и примеры}\label{3.1}

\begin{ex*}{Определение}
Пусть $(M^{2n}, \Omega)$ --- симплектическое многообразие.
Подмногообразие $L \subset M$ называется \rindex{лагранжево многообразие}\emph{лагранжевым}, если $\dim L \z= \tfrac12 \dim M = n$ и $\Omega|_{TL} \equiv 0$.
Вложение (или погружение) $f\colon L^n \to M^{2n}$ называется лагранжевым, если $f^\ast \Omega \equiv 0$.
\end{ex*}

Перечислим некоторые важные примеры лагранжевых подмногообразий.

\begin{ex}[Кривые на поверхностях.]{}\label{3.1.A} 
Пусть $(M^2, \Omega)$ ориентированная поверхность с формой площади.
Тогда каждая кривая лагранжева (поскольку касательное пространство к кривой одномерно и $\Omega$ обнуляется на паре пропорциональных векторов).
\end{ex}


\begin{ex}[Расщеплённый тор.]{}\label{3.1.B} (ср. с \ref{1.1.C} выше).
Тор $S^1 \times\dots\times S^1
\subset 
\RR^2\times\dots\times \RR^2 
= 
\RR^{2n}$ лагранжев (2-форма $\Omega$ на $\RR^{2n}$ расщепляется).
\end{ex}


\begin{ex}[Графики замкнутых 1-форм в кокасательных расслоениях.]{}\label{3.1.C}\\ 
Этот пример играет важную роль в классической механике.
Пусть $N^n$ --- произвольное многообразие.
Рассмотрим кокасательное расслоение $M = \T^\ast N$ с естественной
проекцией $\pi\colon \T^\ast N \to N$, $(p, q) \mapsto q$.\?{}{мы исправили $\to$ на $\mapsto$ где было нужно.}
Определим 1-форму $\lambda$ на $M$ --- так называемую \rindex{форма Лиувилля}\emph{форму Лиувилля}.
Для $q \in N$, $(p, q) \in \T^\ast N$ и $\xi \in \T_{(p, q)} \T^\ast N$ положим $\lambda (\xi) \z= \langle p, \pi_\ast \xi \rangle$, где $\langle\ ,\ \rangle$ --- естественное спаривание $\T^\ast N$ и $TN$.
Мы утверждаем, что $\Omega = d\lambda$ --- симплектическая форма на $\T^\ast M$.
Используя локальные координаты $(p_1,\dots, p_n, q_1,\dots, q_n)$ на $\T^\ast N$, положим $\xi = (\dot p_1 ,\dots, \dot p_n, \dot q_1,\dots,\dot q_n)$;
тогда $\pi_\ast \xi = (\dot q_1,\dots, \dot q_n)$.
В этих обозначениях, $\langle p, \pi_\ast \xi\rangle=\sum p_i \dot q_i$, откуда следует, что $\lambda (\xi) =\sum p_i dq_i$.
Следовательно, $\Omega = d\lambda =\sum dp_i \z\wedge dq_i$ --- узнём стандартную симплектическую форму на $\RR^{2n}$ и утверждение следует.
\end{ex}


\begin{ex*}{Упражнение}\label{1-form-lagrange}
Покажите, что график 1-формы $\alpha$ на $N$ является лагранжевым подмногообразием в $\T^\ast N$ тогда и только тогда, когда форма $\alpha$ замкнута.
\end{ex*}


\begin{ex}[Симплектоморфизмы как лагранжевые подмногообразия.]{}\label{3.1.D}\\ 
Пусть $f\colon (M, \Omega) \to (M, \Omega)$ --- диффеоморфизм.
Рассмотрим новое симплектическое многообразие $(M \times M, \Omega \oplus -\Omega)$.
Оставим как упражнение читателю показать, что график$(f) \subset (M \times M, \Omega \oplus -\Omega)$ является лагранжевым тогда и только тогда, когда $f$ --- симплектоморфизм.
\?{\textit{Подсказка:}}{добавлена подсказка} Докажите, что прообраз формы Лиувилля на $\T^\ast M$ при естественной параметризации графика любой 1-формы $\alpha$ на $M$ это собственно $\alpha$.
\end{ex}

\begin{ex}[Лагранжева надстройка]{}\label{3.1.E}
\rindex{надстройка}
Пусть $L\subset(M,\Omega)$ --- лагранжево подмногообразие.
Рассмотрим петлю гамильтоновых диффеоморфизмов $\{h_t\}$, $t\in S^1$, $h_0=h_1=\1$, порождённую 1-периодическим гамильтонианом $H(x,t)$.
\end{ex}

\begin{thm*}{Предложение}
Пусть $M \times \T^\ast S^1$ --- симплектическое многообразие с
симплектической формой $\sigma = \Omega + dr \wedge dt$, где $M$ то
же, что выше, а $(r, t)$ --- координаты на $\T^\ast S^1 = \RR \times S^1$.
Тогда
\[\phi\colon L \times S^1 \to M \times \T^\ast S^1,
\quad
(x, t) \mapsto (h_t (x), -H (h_t (x), t), t)\]
--- лагранжево вложение.
\end{thm*}

\parit{Доказательство.}
Достаточно доказать, что $\phi^\ast \sigma$ обращается в нуль на парах $(\xi, \xi')$ и на парах  $(\xi, \tfrac{\partial}{\partial t})$ при $\xi, \xi' \in T L$ и $\tfrac{\partial}{\partial t} \in T S^1$.
Вычислим 
\begin{align*}
\phi_\ast \xi
&= h_{t\ast} \xi - \langle dH_t, h_{t\ast} \xi\rangle
\frac{\partial}{\partial r},
\\
\phi_\ast \xi'
&= h_{t\ast} \xi' - \langle dH_t, h_{t\ast} \xi'\rangle
\frac{\partial}{\partial r}.
\end{align*}
Поскольку $L$ лагранжево, $\phi^\ast \sigma (\xi, \xi') = \Omega (h_{t\ast} \xi, h_{t\ast} \xi') = \Omega (\xi, \xi') = 0$.
Более того,
\begin{align*}
\phi_\ast\frac{\partial}{\partial t}
&= \sgrad H_t - 
\left( \langle dH_t, \sgrad H_t \rangle +\frac{\partial H}{\partial t}\right) \frac{\partial}{\partial r} +\frac{\partial}{\partial t}=
\\
&=\sgrad H_t - \frac{\partial H}{\partial t}\frac{\partial}{\partial r} +\frac{\partial}{\partial t},
\end{align*}
так \?{что}{может убрать вторую строчку ниже?}
\begin{align*}
\phi^\ast \Omega (\xi, \frac{\partial}{\partial t})
&= \Omega (h_{t\ast} \xi, \sgrad H_t) + dr \wedge dt (-\langle dH_t, h_{t\ast} \xi\rangle \frac{\partial}{\partial r},\frac{\partial}{\partial t}) =
\\
&=\Omega (h_{t\ast} \xi, \sgrad H_t) - \langle dH_t, h_{t\ast} \xi\rangle =
\\
&=dH_t (h_{t\ast} \xi) - \langle dH_t, h_{t\ast} \xi\rangle = 0.
\end{align*}
\qeds

\section[Класс Лиувилля]{Класс Лиувилля лагранжевых\\подмногообразий в $\bm{\RR^{2n}}$}

Пусть $L \subset (\RR^{2n}, dp \wedge dq)$ --- лагранжево подмногообразие.
Рассмотрим сужение $\lambda|_{TL}$ формы Лиувилля 
\[\lambda = p_1 dq_1+\dots+ p_n dq_n\]
на $L$.
Ясно, что $d (\lambda|_{TL}) = \Omega|_{TL} = 0$.
Класс когомологий \index[symb]{$\lambda_L$}$\lambda_L\in H^1(L,\RR)$ этой замкнутой 1-формы называется \rindex{класс Лиувилля}\emph{классом Лиувилля} лагранжева подмногообразия $L$.
Аналогично, для лагранжева вложения или погружения $\phi\colon L \to \RR^{2n}$ класс Лиувилля определяется как $[\phi^\ast \lambda]$.
Класс Лиувилля лагранжева подмногообразия можно интерпретировать геометрически следующим образом.
Для 1-цикла $a\in H^1 (L)$ выберем 2-цепь $\Sigma$ в $\RR^{2n}$ такую, что $\partial\Sigma = a$.
Тогда
\[(\lambda_L, a) = \int_a\lambda_L = \int_\Sigma\Omega.\]
Это число не зависит от выбора $\Sigma$.
Из-за этой формулы значение $(\lambda_L, a)$ иногда называют \rindex{симплектическая площадь}\emph{симплектической площадью} класса $a$.
Обобщение этой конструкции на произвольное лагранжево многообразие $L$
\?{точного}{добавлено} симплектического многообразия $M$, даёт естественный гомоморфизм $H_2 (M, L; \ZZ) \to \RR$.
Важным свойством класса $\lambda_L$ является то, что он инвариантен относительно симплектоморфизмов $\RR^{2n}$,
то есть,
$f^\ast \lambda_{f (L)} = \lambda_L$.

\begin{thm}[(\cite{G1})]{Теорема}\label{3.2.A}\rindex{Громов}
Пусть $L \subset \RR^{2n}$ --- замкнутое лагранжево подмногообразие.
Тогда $\lambda_L \ne 0$.
\end{thm}

Подчеркнём, что $L$ вложено.
Для лагранжевых погружений это утверждение, вообще говоря, неверно.
В случае $n = 1$ это можно увидеть следующим образом.
Ясно, что любая замкнутая вложенная кривая ограничивает область положительной площади, но, например, погруженная восьмёрка может ограничивать нулевую площадь.
Это явление отражает «жёсткость лагранжевых вложений».

\begin{ex*}{Определение}
Замкнутое лагранжево подмногообразие $L \subset (\RR^{2n}, \omega)$
называется \rindex{рациональное подмногообразие}\emph{рациональным}, если $\lambda_L (H^1 (L; \ZZ))$ ---
дискретная подгруппа в $\RR$.
В таком случае, обозначим её положительную образующую через \index[symb]{$\gamma(L)$}$\gamma(L)$.
\end{ex*}

\begin{ex*}{Пример}
Расщеплённый тор $L = S^1 (r) \times\dots\times S^1 (r) \subset
\RR^{2n}$ рационален. 
Действительно, поскольку каждая окружность $S^1 (r)$ имеет
симплектическую площадь $\pi r^2$, получаем $\gamma (L) = \pi r^2$. 
Однако тор $S^1(1)\times S^1(\sqrt[3]{2}) \subset \RR^4$ не является
рациональным.
Симплектические площади двух окружностей равны $\pi$  и
$\sqrt[3]{4}\pi$ соответственно, и они порождают плотную подгруппу в
$\RR$.
\end{ex*}

\begin{thm}[(\cite{S1})]{Теорема}\label{3.2.B}
  Пусть $L \subset B^2 (r) \times \RR^{2n - 2}$ --- замкнутое
  рациональное лагранжево подмногообразие. 
  Тогда $\gamma (L) \le \pi r^2$.
\end{thm}

\begin{wrapfigure}[7]{o}{25 mm}
\vskip-3mm
\centering
\includegraphics{mppics/pic-1}
\caption{}\label{pic-1}
\vskip0mm
\end{wrapfigure}

Предположение, что $L$ вложено, необходимо.
На рис. \ref{pic-1} показано лагранжево погружение произвольной
симплектической площади.

Наш следующий результат даёт нижнюю оценку на \rindex{энергия смещения!лагранжева подмногообразия}\emph{энергию смещения} $\e (L)$ рационального лагранжева подмногообразия $L$ относительно хоферовской метрики.

\begin{thm}{Теорема}\label{3.2.C}
  Пусть $L \subset \RR^{2n}$ --- замкнутое рациональное лагранжево
  подмногообразие.
  Тогда $\e (L) \ge \tfrac12 \gamma (L)$.
\end{thm}

Теоремы \ref{3.2.A} и \ref{3.2.B} будут доказаны в следующей главе.
Теорема \ref{3.2.C} следует из \ref{3.2.B} (см. раздел \ref{3.3} ниже).
Выведем некоторые следствия из этих результатов.

\begin{ex}[Невырожденность хоферовской метрики.]{}\label{3.2.D}
Из теоремы \ref{3.2.C} следует, что хоферовская метрика на $\Ham (\RR^{2n}, \omega)$ невырождена.
Действительно, каждый шар 
$B^{2n}(r) = \{p_1^2 +\z\dots\z+ p_n^2 \z+ q_1^2+\z\dots\z+ q_n^2 \le r^2\}$
содержит рациональный расщеплённый тор 
\[
S^1(\tfrac r{\sqrt{n}}) \times\dots\times S^1(\tfrac r{\sqrt{n}})
=
\{p_1^{2}+q_1^2=\dots=p_n^{2}+q_n^2=\tfrac{r^2}{n}\}.
\]
Таким образом, $\e (B^{2n} (r)) \ge \tfrac{\pi r^2}{2n}> 0$, и, как было объяснено в \ref{sec:2.4} выше, получаем желаемую невырожденность $\rho$.
Эта оценка не точна.
\rindex{Хофер}Хофер доказал в \cite{H1}, что $\e (B^{2n} (r)) = \pi r^2$.
\end{ex}

\begin{ex}[Свойство несжимаемости.]{}\label{3.2.E}\rindex{несжимаемость}
Отметим, что $\gamma (L)$ --- симплектический инвариант, то есть, для симплектоморфизма $f\colon \RR^{2n} \to \RR^{2n}$ имеем $\gamma (f (L)) = \gamma (L)$.
Таким образом, из теоремы \ref{3.2.B} следует теорема о несжимаемости \ref{1.1.C}.
Напомним, что она утверждает, что расщеплённый тор с большим $\gamma (L) = \pi R^2$ не может быть перемещён гамильтоновым диффеоморфизмом в $B^2 (r) \times \RR^{2n - 2}$ при $r<R$.
\end{ex}

\begin{ex}[Цилиндрическая симплектическая ёмкость.]{}\label{3.2.F}
Пусть $A \subset \RR^{2n}$ --- ограниченное подмножество и положим \index[symb]{$c(A)$}
\[c(A) = \inf \set{\pi r^2}{ \exists\  g\colon \RR^{2n} \to \RR^{2n}},\]
где $g$ --- симплектоморфизм такой, что $g (A) \subset B^2 (r) \times \RR^{2n - 2}$.
Эта функция, определённая на подмножествах $\RR^{2n}$, называется \rindex{цилиндрическая ёмкость}\emph{цилиндрической симплектической ёмкостью}.
На этом языке теорема \ref{3.2.B} читается так:
для замкнутого рационального лагранжева подмногообразия $c(L) \ge \gamma (L)$.
Эта ёмкость является симплектическим инвариантом и удовлетворяет следующему свойству монотонности:
$c (A) \z\le c (B)$ если $A \subset B$ (сравните с аналогичным свойством
монотонности энергии смещения, раздел \ref{sec:2.4} \?{выше}{может убрать эти выше/ниже?}).
\end{ex}



\begin{ex}[Некоторые обобщения.]{}\label{3.2.G}
Пусть $(M, \Omega)$ --- симплектическое многообразие.
Если $M$ открыто, то мы полагаем, что оно имеет «хорошее» поведение на
бесконечности (этот класс включает, например, любое кокасательное
расслоение, наделённое стандартной симплектической структурой, а также
произведение кокасательного расслоения с любым замкнутым
симплектическим многообразием). 
Возьмём лагранжево подмногообразие $L \subset M$ и рассмотрим
гомоморфизм $\lambda_L\colon \pi_2 (M, L) \to \RR$, переводящий любой
диск $\Sigma$ в $M$, граница которого лежит на $L$, в его
симплектическую площадь $\int_\Sigma \Omega$. 
Точно так же, как в случае $M = \RR^{2n}$, мы говорим, что $L$
рационально, если образ $\lambda_L$ дискретен; для рационального $L$
определим $\gamma (L)$ как положительную образующую образа
$\lambda_L$. 
Если $\lambda_L = 0$, то положим $\gamma (L) = + \infty$.
Наше доказательство теоремы \ref{3.2.C} без существенных изменений
распространяется на эту более общую постановку (см. \cite{P1}). 
А именно, мы получаем, что $\e (L) \ge \tfrac12 \gamma (L)$.
В качестве следствия%
\footnote{Этот момент был упущен в \cite[с. 359]{P1}.}
можно получить следующее важное утверждение, доказанное Громовым в \cite{G1}:
$\e (L) = + \infty$ при $\lambda_L = 0$.\?{}{Может сказать, что $L$ вложенное?}
Это можно интерпретировать как свойство лагранжева пересечения: если $\lambda_L = 0$, то  для любого гамильтонова диффеоморфизма $\phi$ образ $\phi (L)$ пересекает $L$.
Применение этого результата к хоферовской геометрии будет обсуждаться в главе \ref{chap:6} ниже.

Отметим также, что эти оценки были значительно улучшены в \rindex{Чеканов}\cite{Ch} при помощи гомологий Флоера (см. также \cite{O3}).
В частности, было показано, что каждое (не обязательно рациональное) замкнутое лагранжево подмногообразие $L \subset M$ имеет положительную энергию смещения.
\end{ex}

\begin{ex}[Изопериметрическое неравенство.]{}\rindex{изопериметрическое неравенство}
Мы завершаем этот раздел формулировкой следующего замечательного результата, принадлежащего Витербо \cite{V2}.
Пусть $L \subset \RR^{2n}$ --- замкнутое лагранжево подмногообразие.
Обозначим $n$-мерный евклидов объем $L$ как $V$.
Тогда 
\[\e (L) \le 2^{n(n-1)/2} n^n V.\]
Точную константу в этом неравенстве ещё предстоит найти. 
\end{ex}

\section{Оценка энергии смещения}\label{3.3}

В этом разделе мы выводим теорему \ref{3.2.C} из теоремы \ref{3.2.B} используя элементарную геометрию.

\parbf{Шаг 1.}
Пусть $L$ --- замкнутое рациональное лаграново подмногообразие и пусть\?{}{интервалы то $[a;b]$, то $[a,b]$} $h_t$, $t \in [0, 1]$ --- путь гамильтоновых диффеоморфизмов такой, что $h_0 = \1$ и $h_1 (L) \cap L = \emptyset$.
Выберем $\epsilon> 0$.
Не умаляя общности можно считать, что $h_t = \1$ при $t \in [0, \epsilon]$ и $h_t = h_1$ при $t \in [1 - \epsilon, 1]$.
Этого можно добиться подходящей репараметризацией потока, сохраняющей его длину (используйте упражнение \ref{1.4.A} выше).
Пусть $H (x, t)$ --- соответствующий гамильтониан.
Положим
\[l
=
\length\{h_t\} 
=
\int_0^1 \max_x H_t - \min_x H_t\,dt.\]
Нам нужно доказать, что $l \ge \tfrac12 \gamma (L)$.
Генеральный план состоит в том, чтобы закодировать движение $L$ в потоке как замкнутое лагранжево подмногообразие в $\RR^{2n+2}$, а затем применить \ref{3.2.B}.
Мы воспользуемся лагранжевой надстройкой описанной \?{в разделе
  \ref{3.1.E}}{добавил}.
Нам придётся построить петлю гамильтоновых диффеоморфизмов.

\parbf{Шаг 2.}
Рассмотрим следующую петлю гамильтоновых диффеоморфизмов при $t \in [0, 2]$
\begin{align*}
g_t
&=
\begin{cases}
h_t&\text{при}\ t\in [0,1] 
\\
h_{2-t}&\text{при}\ t\in [1,2]
\end{cases}
\intertext{с гамильтонианом}
G (x, t)&=
\begin{cases}
H(x,t)&\text{при}\ t\in [0,1]
\\
-H(x,2-t)&\text{при}\ t\in [1,2]
\end{cases}
\end{align*}

\begin{ex*}{Упражнение}
Покажите, что $\int_0^2G (g_t (x), t)\,dt = 0$ при всех $x$.
\end{ex*}

\begin{figure}[ht!]
\vskip-0mm
\centering
\includegraphics{mppics/pic-2}
\caption{}\label{pic-2}
\vskip0mm
\end{figure}

Взяв её лагранжеву надстройку (см. \ref{3.1.E} выше),
получаем новое лагранжево подмногообразие $L' \subset \RR^{2n} \z\times T^{\ast} S^1$ как образ $L \times S^1$ при отображении 
\[(x, t) \mapsto \Big(g_t (x), -G \big(g_t (x), t\big), t\Big).\]
Напомним, что здесь $S^1 = \RR / 2\ZZ$.
Определим две функции 
\begin{align*}a_+ (t) &= - \min_x G (x, t) + \epsilon
&&\text{и}&
a_- (t) &= - \max_x G (x, t) - \epsilon.
\end{align*}
Ясно, что $L' \subset \RR^{2n} \times C \subset \RR^{2n} \times \T^\ast S^1$, где $C$ обозначает кольцо 
$\{a_- (t) <r <a_+ (t)\}$
(см. рисунок 2).

\parbf{Шаг 3.}
Теперь мы хотим перейти от $\RR^{2n} \times T^{\ast} S^1$ к $\RR^{2n} \times \RR^2$.
Рассмотрим специальное симплектическое погружение $\theta\: C \to \RR^2$
(этот трюк известен как громовская восьмерка, см. \cite{G1}, \cite{AL}).

\begin{ex*}{Упражнение}
(См. Рисунок 3).
Покажите, что существует симплектическое погружение $\theta: C \to \RR^2 (p, q)$ со следующими свойствами: 
\begin{itemize}
\item $\theta$ переводит нулевое сечение $\{r = 0\}$ в восьмерку с равновеликими ушами, и, таким образом, замкнутая форма $\theta^\ast pdq - rdt$ точна (она замкнута, поскольку $\theta$ --- симплектоморфизм).
\item $\theta$ --- это вложение за пределами пары тонких горловин и оно склеивает эти горловины вместе.
\item площадь внутренних ушей произвольно мала, скажем, по $\epsilon$ у каждого. 
\end{itemize}
\end{ex*}

Заметим, что \?{}{$\area=\Area$}
\begin{align*}
\area (C) 
&= \int_0^2(a_+ (t) - a_- (t))\,dt = 
\\
&=2 \int_0^1(\max_x H_t - \min H_t) \,dt + 4\epsilon =
\\
&=2l + 4\epsilon.
\end{align*}
Таким образом, образ $\theta (C)$ может быть заключён в диск $B$
площадью $2l + 10\epsilon$ (мы добавили $10\epsilon$, \?{чтобы}{to make up for the extra bits} учесть
  площади внутренних ушей и для зазора).

\begin{figure}[ht!]
\vskip-0mm
\centering
\includegraphics{mppics/pic-3}
\caption{}\label{pic-3}
\vskip0mm
\end{figure}

\parbf{Шаг 4.}
Теперь рассмотрим симплектическое погружение \?{}{$\id=\1$?}
\[\theta' = \id \times \theta: \RR^{2n} \times C \to \RR^{2n} \times \RR^2.\]
Очевидно, что $\theta' (L')$ --- погруженное лагранжево подмногообразие лежащее в $\RR^{2n} \times B$. 
Покажем, что $L'' = \theta' (L')$ вложено.
Единственное место, где могут возникнуть двойные точки, --- это тонкие горловины.
Но $g_t (L) = L$ при $t \in [-\epsilon, \epsilon]$, 
и $g_t (L) = h_1 (L)$ при $t \in [1 - \epsilon, 1 + \epsilon]$, а по предположению $h_1 (L) \cap L = \emptyset$, поэтому двойных точек нет и $\theta'$ --- вложение.

\parbf{Шаг 5.}
Нам осталось показать, что $L''$ рационально.
После этого мы сможем применить \ref{3.2.B}.
Докажем, что $\gamma (L) = \gamma (L'')$.
Пусть $\phi$ --- композиция лагранжевой надстройки и $\theta'$,
то есть отображение 
\[\phi\colon 
L \times S^1
\to
\RR^{2n} \times \T^\ast S^1
\to
\RR^{2n} (p_1,\dots, p_n, q_1,\dots, q_n) \times \RR^2 (p, q)\]
отправляющее $(x, t)$ в $(g_t (x), \theta (-G (g_t (x), t), t))$.
Тогда $L''$ --- образ $L \z\times S^1$ при $\phi$.
Группа $H_1 (L'')$ порождается циклами вида $\phi (b)$, где либо $b \subset L \times {0}$, либо $b = {x_0} \times S^1$ при $x_0 \in L$.
В первом случае $\phi (b) = b \times \{\theta (0, 0)\}$, поэтому симплектические площади $b$ и $\phi (b)$ совпадают.
Осталось рассмотреть второй случай.
Обозначим через $\alpha$ орбиту $\{g_t x_0\}$, $t \in [0; 2]$.
Тогда\?{}{+скобки под второй интеграл.}
\begin{align*}
\int_b\phi^\ast (p_1 dq_1 +\dots
+ p_n dq_n + pdq)
&= \int_\alpha (p_1 dq_1 +\dots
+ p_n dq_n) + \int_{\Gamma}\theta^\ast pdq =
\\
&= 0 + \int_{\Gamma} r\,dt =
\\
&= - \int_0^2G (g_t (x_0), t)\,dt 
= 0.
\end{align*}
где $\Gamma\subset \T^\ast S^{1}$ это график функции $t\mapsto -G_{t}(g_{t}x_{0})$, то есть проекция цикла $\phi(x_{0}\times S^{1})$ на $\T^\ast S^{1}$.\?{}{Добавлена $\Gamma$.}
Это завершает доказательство того, что $L''$ --- рациональное лагранжево подмногообразие с $\gamma (L) = \gamma (L'')$.
Напомним, что $L''$ содержится в $\RR^{2n} \times B$.
Принимая во внимание \ref{3.2.B}, получим что
\[\gamma (L'') \le \area (B) = 2l + 10\epsilon\]
при всех $\epsilon> 0$.
Следовательно, $\e(L) \ge \tfrac12 \gamma (L)$.
\qeds

