\chapter{Лагранжевы подмногообразия}

Цель этой следующей главы --- доказать невырожденность метрики Хофера на $\RR^{2n}$.
Мы используем подход из \cite{P1}.
Для этого нам потребуются лагранжевы подмногообразия симплектических многообразий.
Лагранжевы подмногообразия играют фундаментальную роль в симплектической топологии, а также в её приложениях к механике и вариационному исчислению.
Они будут многократно появляться в этой книге.

\section{Определения и примеры}\label{3.1}

\begin{thm*}{Определение}
Пусть $(M^{2n}, \Omega)$ --- симплектическое многообразие и $L \subset M$ --- подмногообразие.
Мы называем $L$ лагранжевым, если $\dim L = \tfrac12 \dim M = n$ и $\Omega|_{TL} \equiv 0$.
Вложение (или погружение) $f\colon L^n \to M^{2n}$ называется лагранжевым, если $f^\ast \Omega \equiv 0$.
\end{thm*}

Перечислим некоторые важные примеры лагранжевых подмногообразий.

\begin{thm}[Кривые на поверхностях]{}\label{3.1.A} 
\end{thm}


Пусть $(M^2, \Omega)$ ориентированная поверхность с формой площади.
Тогда каждая кривая является лагранжевой (поскольку касательное пространство к кривой одномерно и $\Omega$ даёт нуль, на паре пропорциональных векторов).

\begin{thm}[Расщепленный тор]{}\label{3.1.B} (сравни с \ref{1.1.C} выше).
\end{thm}

Тор $S^1 \times\dots\times S^1
\subset 
\RR^2\times\dots\times \RR^2 
= 
\RR^{2n}$ лагранжев (2-форма $\Omega$ на $\RR^{2n}$ расщепляется).


\begin{thm}[Графики замкнутых 1-форм в кокасательных расслоениях]{}\label{3.1.C}
\end{thm}

Этот пример играет важную роль в классической механике.
Пусть $N^n$ --- произвольное многообразие.
Рассмотрим кокасательное расслоение $M = T^\ast N$ с естественной проекцией $\pi\colon T^\ast N \to N$, $(p, q) \mapsto q$.\?{}{выше вместо $\mapsto$ писалось $\to$. Исправить?}
Определим следующую 1-форму $\lambda$ на $M$, которая называется формой Лиувилля.
Для $q \in N$, $(p, q) \in T^\ast N$ и $\xi \in T_{(p, q)} T^\ast N$ положим $\lambda (\xi) = \langle p, \pi_\ast \xi \rangle$, где $\langle\ ,\ \rangle$ --- естественное спаривание между $T^\ast N$ и $TN$.
Мы утверждаем, что $\Omega = d\lambda$ --- симплектическая форма на $T^\ast M$.
Давайте использовать локальные координаты $(p_1,\dots, p_n, q_1,\dots, q_n)$ на $T^\ast N$; пусть $\xi = (\dot p_1 ,\dots, \dot p_n, \dot q_1,\dots,\dot q_n)$, так, что $\pi_\ast \xi = (\dot q_1,\dots, \dot q_n)$.
В этих обозначениях спаривание выражается как $\langle p, \pi_\ast \xi\rangle =\sum p_i \dot q_i$, откуда следует, что $\lambda (\xi) =\sum p_i dq_i$.
Следовательно, $\Omega = d\lambda =\sum dp_i \z\wedge dq_i$, и мы узнаем стандартную симплектическую форму на $\RR^{2n}$.
Утверждение следует.

\begin{thm*}{Упражнение}
Пусть $\alpha$ --- 1-форма на $N$.
Покажите, что график $\alpha$ является лагранжевым подмногообразием в $T^\ast N$ тогда и только тогда, когда форма $\alpha$ замкнута.
\end{thm*}

\begin{thm}[Симплектоморфизмы как лагранжевые подмногообразия.]{}\label{3.1.D}
\end{thm}

Пусть $f\colon (M, \Omega) \to (M, \Omega)$ --- диффеоморфизм.
Рассмотрим новое симплектическое многообразие $(M \times M, \Omega \oplus -\Omega)$.
Оставляем как упражнение показать, что график$(f) \subset (M \times M, \Omega \oplus -\Omega)$ является лагранжевым тогда и только тогда, когда $f$ --- симплектоморфизм.


\begin{thm}[Лагранжева надстройка]{}\label{3.1.E}
\end{thm}

Пусть $L \subset (M, \Omega)$ --- лагранжево подмногообразие.
Рассмотрим петлю гамильтоновых диффеоморфизмов $\{h_t\}$, $t \in S^1$, $h_0 = h_1 = \1$, порожденную 1-периодическим гамильтонианом $H (x, t)$.


\begin{thm*}{Предложение}
Пусть $M \times T^\ast S^1$ --- симплектическое многообразие с симплектической формой $\sigma = \Omega + dr \wedge dt$, где $M$ то же, что выше, а $(r, t)$ --- координаты на $T^\ast S^1 = R \times S^1$.
Тогда
\[\phi: L \times S^1 \to M \times T^\ast S^1,
\quad
(x, t) \mapsto (h_t (x), -H (h_t (x), t), t)\]
--- лагранжево вложение.
\end{thm*}

\parit{Доказательство.}
Достаточно доказать, что $\phi^\ast \sigma$ обращается в нуль на парах $(\xi, \xi')$ и на парах  $(\xi, \tfrac{\partial}{\partial t})$ для $\xi, \xi' \in T L$ и $\tfrac{\partial}{\partial t} \in T S^1$.
Вычислим 
\begin{align*}
\phi_\ast \xi
&= h_{t\ast} \xi - \langle dH_t, h_{t\ast} \xi\rangle
\frac{\partial}{\partial r},
\\
\phi_\ast \xi'
&= h_{t\ast} \xi' - \langle dH_t, h_{t\ast} \xi'\rangle
\frac{\partial}{\partial r}.
\end{align*}
Поскольку $L$ лагранжево, $\phi^\ast \sigma (\xi, \xi') = \Omega (h_{t\ast} \xi, h_{t\ast} \xi') = \Omega (\xi, \xi') = 0$.
Более того,
\begin{align*}
\phi_\ast\frac{\partial}{\partial t}
&= \sgrad H_t - 
\left( \langle dH_t, \sgrad H_t \rangle +\frac{\partial H}{\partial t}\right) \frac{\partial}{\partial t} +\frac{\partial}{\partial t}=
\\
&=\sgrad H_t - \frac{\partial H}{\partial t}\frac{\partial}{\partial r} +\frac{\partial}{\partial t},
\end{align*}
так что получаем
\begin{align*}
\phi^\ast \Omega (\xi, \frac{\partial}{\partial t})
&= \Omega (h_{t\ast} \xi, \sgrad H_t) + dr \wedge dt (-\langle dH_t, h_{t\ast} \xi\rangle \frac{\partial}{\partial r},\frac{\partial}{\partial t}) =
\\
&=\Omega (h_{t\ast} \xi, \sgrad H_t) - \langle dH_t, h_{t\ast} \xi\rangle =
\\
&=dH_t (h_{t\ast} \xi) - \langle dH_t, h_{t\ast} \xi\rangle = 0.
\end{align*}
\qeds
