\chapter[Геодезические]{Элементы вариационной теории геодезических}\label{chap:12}

Мы уже обсудили ряд результатов о геодезических группы гамильтоновых диффеоморфизмов.
В этой главе мы посмотрим на геодезические с точки зрения вариационного исчисления.
Пусть задан гладкий путь гамильтоновых диффеоморфизмов --- можно ли сократить его малой вариацией с фиксированными концами?
Этот вопрос мотивируется классической теорией геодезических на римановых многообразиях.
Интересно, что в хоферовской геометрии, по крайней мере, при некоторых предположениях о невырожденности можно дать на него довольно точный ответ с ясным динамическим смыслом \cite{U}.

\section{Что такое геодезическая?}

Римонава интуиция подсказывает, что геодезические следует определять как критические точки функционала длины.
Попробуем формализовать это определение.

Пусть $\{f_t\}$, $t\in[a; b]$ --- гладкий путь в $\Ham(M,\Omega)$.
\rindex{вариация}\emph{Вариацией} $\{f_t\}$ называется гладкое семейство путей $\{f_{t,\epsilon}\}$ с $t \in [a; b]$ и $\epsilon \in (-\epsilon_0, \epsilon_0)$, удовлетворяющих условиям
\[f_{a,\epsilon} = f_a,\quad f_{b,\epsilon} = f_b\quad\text{и}\quad f_{t,0} = f_t\]
для всех $t$ и $\epsilon$.
Мы всегда предполагаем, что общий носитель $\bigcup_{t,\epsilon}\supp f_t$ компактен.
При такой вариации рассмотрим длину пути $\{f_{t,\epsilon}\}$ как функцию от $\epsilon$ \index[symb]{$\ell(\epsilon)$}
\begin{align*}
\ell(\epsilon)&=\int_a^b\|F(\cdot,t,\epsilon)\|\,dt=
\\
&=\int_a^b \max_x F(x,t,\epsilon)-\min_x F(x,t,\epsilon)\,dt,
\end{align*} 
где $F(x, t, \epsilon)$ --- гамильтониан, порождающий путь $\{f_{t,\epsilon}\}$ для данного~$\epsilon$.

\begin{ex}{Предварительное определение}\label{12.1.A}
Путь $\{f_t\}$ является геодезической, если
\begin{itemize}
\item имеет постоянную скорость, т. е. $\|F(\cdot, t)\|$ не зависит от $t$;
\item для любой гладкой вариации $\{f_t\}$ функция длины $\ell(s)$ имеет критическую точку при $\epsilon = 0$.
\end{itemize}
\end{ex}

Здесь мы сразу сталкиваемся с трудностью --- даже при гладких вариациях функция $\ell(\epsilon)$ не обязана быть гладкой!
Таким образом, необходимо уточнить понятие критической точки.
Наша первая задача --- выяснить структуру функций длины $\ell(\epsilon)$, связанных с вариациями данного пути.

\begin{ex}{Предложение}\label{12.1.B}
Каждая функция длины $\ell(\epsilon)$ выпукла в $0$ с точностью до второго порядка.
То есть, существуют выпуклая функция $u(\epsilon)$, а также числа $\delta > 0$ и $C > 0$ такие, что 
\[|\ell(\epsilon) - u(\epsilon)| \le C\epsilon^2\]
для всех $\epsilon \in (-\delta, \delta)$.
\end{ex}

Обратите внимание, что каким бы ни было определение критической точки, оно не должно зависеть от членов второго порядка.
Единственным естественным кандидатом на критическую точку выпуклой функции является её минимум.
Таким образом, мы приходим к следующему понятию.

\begin{ex}{Определение}\label{12.1.C}
Точка $\epsilon = 0$ является критической точкой функции длины $\ell(\epsilon)$ тогда и только тогда, когда она является точкой минимума выпуклой функции $u(\epsilon)$, удовлетворяющей неравенству \ref{12.1.B}
\end{ex}

\begin{ex}{Упражнение}\label{12.1.D}
\begin{itemize}

\item Приведённое выше определение правильное, то есть не зависит от выбора выпуклой функции $u$, удовлетворяющей условию \ref{12.1.B}.


\item Если функция $\ell(\epsilon)$ гладкая, то определение \ref{12.1.C} совпадает с обычным.


\item Если $\ell$ достигает своего локального минимума или максимума в
  нуле, то ноль является критической точкой $\ell$ в смысле
  \ref{12.1.C}. 


\item Выведите из \ref{12.1.B} то, что росток функции $1 - |\epsilon|$
  в точке $0$ не может возникнуть как функция длины любой гладкой
  вариации.
\end{itemize}
\end{ex}

Теперь мы готовы ответить на вопрос, поставленный в заголовке раздела: \rindex{геодезическая}\emph{путь $\{f_t\}$, $t\in[a; b]$ называется геодезической, если он удовлетворяет \ref{12.1.A} и \ref{12.1.C}.}

Как мы увидим в \ref{12.2.A}, ограничение геодезической, определённой на $[a; b]$ на любой подотрезок снова является геодезической.
Таким образом, понятие геодезических очевидным образом распространяется на пути, определённые на произвольных интервалах времени.
Пусть $I\in \RR$ --- интервал, и пусть $f\: I\to\Ham(M,\Omega)$ --- гладкий путь.
Если ограничение $f$ на любой отрезок $[a; b] \subset I$ --- геодезическая, то и сам путь $f$ называется геодезической.
Далее мы сосредоточимся на геодезических $\{f_t\}$, которые определены на единичном интервале $[0; 1]$.
Более того, поскольку хоферовская метрика биинвариантна, мы всегда можем сдвинуть геодезическую и считать, что $f_0 = \1$. 

Перейдём к доказательству \ref{12.1.B}.
Нам будет удобно использовать идею линеаризации, представленную в главе 5.
Рассмотрим пространство \index[symb]{$\F_0$}$\F_0$ всех нормированных гамильтонианов $M \times [0; 1] \to \RR$
с нормой \index[symb]{$\VERT F \VERT_0$}
\[\VERT F \VERT_0 = \int_0^1 \max_x F(x,t) - \min_x F(x,t)\, dt.\]
Обозначим через \index[symb]{$\H_0$}$\H_0 \subset \F_0$ подмножество, состоящее из всех гамильтонианов, порождающих петлю $\{h_t\}$ гамильтоновых диффеоморфизмов: $h_0 = h_1 = \1$.
В отличие от соглашений в главе 5,
ме \textbf{не} предполагаем, что гамильтонианы в $\F_0$ и $\H_0$ периодичны по времени.
Обозначим через \index[symb]{$\V$}$\V$ множество всех гладких семейств $H(x, t, \epsilon)$ функций из $\H_0$ таких, что $H(x, t, 0) \equiv 0$.

\begin{thm}{Предложение}\label{12.1.E}
Пусть $\{f_t\}$ --- гладкий путь гамильтоновых диффеоморфизмов с $t \in [0; 1]$ и $f_0 = \1$.
Множество функций длины $\ell(\epsilon)$, связанных с вариациями $\{f_t\}$, состоит из всех функций вида 
\[\VERT F - H(\epsilon)\VERT_0,\]
где $H\in\V$.
\end{thm}

Предложение \ref{12.1.B} является непосредственным следствием этой формулы.
Действительно, положим $u(\epsilon) = \VERT F - \epsilon H'(0)\VERT_0$.
Ясно, что $u$ выпукло и совпадает с $\ell$ с точностью до второго порядка.

\parit{Доказательство \ref{12.1.E}.}
Любая вариация пути $f_t$ может быть записана в виде $f_{t,\epsilon} = h_{t,\epsilon}^{-1}\circ f_t$, где $h_{1,\epsilon}$ --- гладкое семейство петель с $h_{1,0} = \1$.
Гамильтониан $H(x,t,\epsilon)$ петель $\{h_{t,\epsilon}\}$ принадлежит $\V$.
Далее, 
\[F(x, t, \epsilon) = -H(h_{1,\epsilon}x, t, \epsilon) + F(h_{1,\epsilon}x, t),\]
поэтому
\[\ell(\epsilon) = \VERT F - H(\epsilon)\VERT_0.\]
\qeds

Решите следующие упражнения, используя \ref{12.1.E}.

\begin{ex}{Упражнение}\label{12.1.F}
Предположим, что при всяком $t$ функция $F(x, t)$ имеет единственную точку максимума и минимума соответственно, причём эти точки невырождены в смысле теории Морса.
Тогда \textit{для любой} вариации $\{f_t\}$ функция длины $\ell(\epsilon)$ гладкая.
\emph{Подсказка:} используйте теорему о неявной функции.
\end{ex}

\begin{ex}{Упражнение}\label{12.1.G}
Предположим, что множество максимума $F$ имеет непустую внутренность.
Постройте вариацию у которой функция длины негладкая в нуле.
\end{ex}

\section{Описание геодезических}\label{sec:12.2}

Будем говорить, что гамильтониан $F\in \F_0$ имеет \rindex{фиксированные экстремумы}\emph{фиксированные экстремумы}, если существуют две точки $x_-, x_+\in M$ такие, что $F(x_-, t) \z= \min_x F(x, t)$ и $F(x_+, t) = \max_x F (x, t)$ для всех $t\in[0; 1]$, причём функция $F(x_+, t) - F(x_-, t)$ не зависит от $t$.
Важнее всего то, что экстремальные точки $x_-$ и $x_+$ функции $F(\cdot, t)$ не зависят от времени.

\begin{thm}{Теорема}\label{12.2.A}
Путь $\{f_t\}$ является геодезическим тогда и только тогда, когда соответствующий гамильтониан $F\in\F_0$ имеет фиксированные экстремумы.
\end{thm}

В частности, каждый автономный гамильтонов поток является геодезическим.
Гамильтонианы с фиксированными экстремумами были введены в \cite{BP1}, где они были названы \rindex{квазиавтономный поток}\emph{квазиавтономными}.
Теорема \ref{12.2.A} по существу доказана в \rindex{Лалонд}\cite{LM2} (хотя там геодезическая определяется несколько по-другому).

Для доказательства теоремы \ref{12.2.A} нам предстоит более детально исследовать структуру вариаций.
Обозначим через $\V_1$ касательное пространство к $\H_0$ в точке $0$: 
\[\V_1=\set{\frac{\partial H}{\partial\epsilon}}{H\in \V}\]

\begin{thm}{Предложение}\label{12.2.B}
Пространство $\V_1$ состоит из всех функций $G\in\F_0$, удовлетворяющих условию
\[\int_0^1G(x,t)\,dt=0\]
при всех $x\in M$.
\end{thm}

Доказательство предложения абсолютно аналогично \ref{5.2.D} и \ref{6.1.C} выше, где решён случай периодических во времени вариаций.

\parit{Доказательство теоремы \ref{12.2.A}.}
Пусть $F\in\F_0$ --- такая функция, что $\|F(\cdot, t)\|$ не зависит от $t$.
Для семейства $H(\epsilon)$ из $\V$ положим $G = H'(0)\in\V_1$.
Определим функции $\ell(\epsilon) = \VERT F - H(\epsilon)\VERT_0$ и $v(\epsilon) = \VERT F - \epsilon G\VERT_0$.
С учётом \ref{12.1.E} гамильтониан $F$ порождает геодезическую тогда и только тогда, когда $\ell(\epsilon)$ имеет критическую точку в смысле \ref{12.1.C} при $\epsilon= 0$ для каждого $H\in\V$.
Поскольку $\ell(\epsilon)$ и $v(\epsilon)$ совпадают с точностью до членов второго порядка, а $v(\epsilon)$ выпукла, это эквивалентно тому, что $v(\epsilon)$ имеет точку минимума при $\epsilon= 0$ для любого $G\in\V_1$.
Поэтому для доказательства теоремы \ref{12.2.A} достаточно проверить эквивалентность следующих условий:
\begin{enumerate}[(i)]
\item\label{12.2.i} $F$ имеет фиксированные экстремумы;
\item\label{12.2.ii} $\VERT F - \epsilon G\VERT_0 \ge \VERT F\VERT_0$ для всех $G\in\V_1$, $\epsilon\in\RR$.
\end{enumerate} 

Предположим, что выполняется (\ref{12.2.i}).
Положим 
$u=F-\epsilon G$ для $G\in \V_1$.
Тогда из \ref{12.2.B} следует, что 
\[\VERT u\VERT_0 \ge \int_0^1 u(x_+,t)-u(x_-,t)\,dt=\VERT F\VERT_0,\]
и, следовательно, мы получаем (\ref{12.2.ii}).

А теперь предположим, что выполняется (\ref{12.2.ii}).
Возьмём 
\[G(x, t) = F(x, t) - \int_0^1F(x,t)\,dt.\]
Неравенство (\ref{12.2.ii}) даёт 
\[
\max_x\!\int_0^1 F(x, t)\,dt- \min_x\!\int_0^1 F(x, t)\,dt
\ge
\int_0^1 \max_xF(x, t)\,dt-\! \int_0^1\min_x F(x, t)\,dt,
\]
что возможно, только если $F$ имеет фиксированные экстремумы.
\qeds


\section{Устойчивость и сопряжённые точки}

Геодезическая называется \rindex{устойчивая геодезическая}\emph{устойчивой}, если при каждой вариации функция длины $\ell(\epsilon)$ достигает своего минимального значения в нуле.
Другими словами, устойчивые геодезические нельзя укоротить небольшими вариациями с фиксированными концами.
Задача описания устойчивых геодезических в полной общности остаётся открытой.%
\footnote{Думаю, её можно решить существующими методами негладкого анализа.
Заметим также, что первое утверждение теоремы \ref{12.3.A} ниже говорит что для {}\emph{всех} геодезических выполняется следующее условие, которое достаточно для устойчивости, см. \cite{LM3}.
Доказательство использует теорию псевдоголоморфных кривых.}
Ниже мы приведём решение для определённого класса \rindex{невырожденная геодезическая}\emph{невырожденных} геодезических.
По определению геодезическая невырождена, если для каждого $t$ соответствующий гамильтониан имеет единственную точку максимума и минимума соответственно, причём эти точки невырождены в смысле теории Морса.
Например, каждый автономный путь с единственными невырожденными максимумом и минимумом является невырожденной геодезической.
Напомним, что для любой вариации невырожденной геодезической функция длины $\ell(\epsilon)$ гладкая (см. \ref{12.1.F}).
Теория, развитая в этом и двух следующих разделах, восходит к Устиловскому \cite{U} (см. также \rindex{Лалонд}\cite{LM2}).

Пусть $\{f_t\}$, $t \in [0; 1]$, $f_0 = \1$ --- невырожденная геодезическая, порождённая гамильтонианом $F\in \F_0$.
Обозначим через $x_-$ и $x_+$ соответственно независимые от времени точки минимума и максимума функции $F(\cdot, t)$.
Заметим, что $x_\pm$ в этом случае являются неподвижными точками $\{f_t\}$.
Рассмотрим линеаризованные потоки $f_{t*}$ на $\T_{x_+}M$ и $\T_{x_-}M$.
Мы говорим, что такой поток имеет нетривиальную $T$-периодическую орбиту с $T\ne0$, если $f_{T*}\xi=\xi$ для некоторого касательного вектора $\xi\ne0$.

\begin{thm}{Теорема}\label{12.3.A}

\begin{itemize}
\item Предположим, что линеаризованные потоки не имеют нетривиальных $T$-периодических орбит с $T\in(0,1]$.
Тогда $\{f_t\}$ стабилен.
\item Предположим, что $\{f_t\}$ стабилен.
Тогда линеаризованные потоки не имеют нетривиальных периодических орбит с $T\z\in(0,1)$.
\end{itemize}
\end{thm}

Этот результат можно интерпретировать как описание сопряжённых точек вдоль геодезических хоферовской метрики ---  сопряжённые точки соответствуют нетривиальным замкнутым орбитам линеаризованного потока в точке $x_\pm$.
До сопряжённой точки геодезическая устойчива, а после неё теряет устойчивость (то есть геодезическую можно укоротить малой вариацией).
Мы отсылаем читателя к \cite{U} за дополнительной информацией.
Доказательство \ref{12.3.A} представлено в \ref{sec:12.5} ниже.
Оно основано на формуле второй вариации, которую мы опишем в следующем разделе.

\begin{ex}{Упражнение}\label{12.3.B}
Выведите из \ref{12.3.A}, что \textit{достаточно короткий} отрезок невырожденной геодезической устойчив.
\end{ex}

\section{Формула второй вариации}
\rindex{формула второй вариации}

Пусть $\{f_t\}$ --- невырожденная геодезическая, порождённая гамильтонианом $F(x, t)$ с точками максимума/минимума $x_\pm$.
Обозначим через $C_\pm(t)$ оператор линеаризованного уравнения в точке $x_\pm$,
то есть 
\[\frac{d}{dt} f_{t*}(x_\pm)=C_\pm f_{t*}(x_\pm).\]
Рассмотрим пространства 
\[V_\pm = \set{\text{гладкие отображения}\ v\: [0, 1] \to \T_{x_\pm}M}{v(0) = v(1) = 0}.\]
Для данной вариации $\{f_{t,\epsilon}\}$ геодезической $\{f_t\}$ определим элементы $v_\pm\in V_\pm$ формулой 
\[v_\pm = \left.\frac d{d\epsilon}\right|_{\epsilon=0} f_{t,\epsilon} x_\pm.\]
Будет удобно рассматривать максимальную и минимальную части функции длины, связанные с вариацией, по отдельности.
Положим 
\[\ell_+(\epsilon) =\int_0^1\max_x F(x,t,\epsilon)\,dt\]
и 
\[\ell_(\epsilon)=\int_0^1\min_x F(x,t,\epsilon)\,dt.\]
Ясно, что $\ell(\epsilon) = \ell_+(\epsilon) - \ell_(\epsilon)$.

\begin{thm}[\cite{U}]{Теорема}\label{12.4.A}
\[\frac{\d^2\ell_\pm}{d\epsilon^2}(0)=Q_\pm(v_\pm)\]
и, следовательно,
\[\frac{\d^2\ell_\pm}{d\epsilon^2}(0)=Q_+(v_+)-Q_-(v_-),\]
где
\[Q_\pm(v)=-\int_0^1\left(\Omega(C^{-1}_\pm \dot v,\dot v)+\Omega(\dot v,\dot v)\right).\]

\end{thm}

\begin{ex*}[(изопериметрическое неравенство)]{Пример}
Рассмотрим стандартную симплектическую плоскость $\RR^2(p, q)$ с симплектической формой $\omega = dp \wedge dq$.
Пусть $v\: [0, 1] \to\RR^2$ --- гладкая кривая такая, что $v(0)=v(1)=0$.
Напомним следующие понятия евклидовой геометрии: 
\begin{align*}
\length(v)&=\int_0^1|\dot v|\,\d t,
\\
\energy(v)&=\int_0^1|\dot v|^2\,\d t
\\
\area(v)&=\int_0^1\langle \tfrac12(p\d q-q\d p),\dot v\rangle\,\d t=
\\
&=\frac12\int_0^1 p\dot q-q\dot p\,\d t=
\\
&=\frac12\int_0^1\omega(v,\dot v)\,\d t
\end{align*}
где $v(t) = (p(t), q(t))$.
Изопериметрическое неравенство гласит, что
\[4\pi\area(v)\le\length(v)^2\le \energy(v).\]
Вернёмся к нашей симплектической задаче, мы предполагаем, что гамильтониан вблизи $x_-$ задаётся формулой $F(x) = \pi \lambda(p^2 \z+ q^2)$ с $\lambda > 0$.
Таким образом, мы получаем гамильтонову систему 
\[
\begin{cases}
\dot p &= -2\pi\lambda q,
\\
\dot q &= 2\pi\lambda p.
\end{cases}
\]
Полагая $z = p + iq$, мы получаем линейное уравнение $\dot z = 2\pi \lambda iz$.
В этом случае $C_-(t) = 2\pi\lambda i$ и $C_-^{-1}(t) = -\frac i{2\pi\lambda}$.
Учитывая, что $\omega(\xi,i\xi) \z= |\xi|^2$, получаем
\begin{align*}
Q_-(v)&=-\int_0^1\left(\omega(-\tfrac i{2\pi\lambda}\dot v,\dot v)+\omega(\dot v, v)\right)\,dt=
\\
&=-\frac 1{2\pi\lambda}\int_0^1|\dot v|^2\,dt-\int_0^1\omega(\dot v,v)\,dt=
\\
&=-\frac 1{2\pi\lambda}\energy(v)+2\area(v).
\end{align*}
Уравнение $\dot z = 2\pi\lambda iz$ имеет решение $z(t) = e^{2\pi\lambda it}z(0)$ при $t \in [0, 1]$ и, следовательно, при $\lambda < 1$ оно не имеет орбит периода $1$.
Применив теоремы \ref{12.3.A} и \ref{12.4.A} получаем, что $Q_-(v)\le 0$ для всех плоских кривых $v$ с $v(0) = v(1) = 0$.
Поэтому 
\[4\pi \lambda \area(v) \le \energy (v)\]
при всех $\lambda < 1$.
При $\lambda\to1$, получаем изопериметрическое неравенство.
\end{ex*}

В доказательстве \ref{12.4.A} нам понадобится следующее вспомогательное утверждение.

\begin{thm}{Лемма}\label{12.4.B}
Пусть $\{f_{t,\epsilon}\}$ --- вариация пути $\{f_t\}$, а $F(x, t,\epsilon)$ --- её гамильтониан.
Тогда 
\[\int_0^1\frac{\partial F}{\partial \epsilon}(f_{t,\epsilon},t,\epsilon)\,\d t=0.\] 

\end{thm}

\parit{Доказательство.}
\begin{enumerate}[1)]
\item Приведённая выше формула верна для любой вариации постоянной петли, т. е. когда $f_t = \1$ при всех $t$.
Доказательство аналогично доказательству \ref{6.1.C} выше.
\item Рассмотрим теперь общий случай.
Запишим $f_{t,\epsilon} = f_t \circ h_{1,\epsilon}$, где $h_{1,\epsilon}$ --- вариация постоянной петли.
Обозначим через $H(x, t, \epsilon)$ гамильтониан петли $\{h_{1,\epsilon}\}$.
Тогда $F(x,t,\epsilon) = F(x,t) + H(f_t^{-1}x,t,\epsilon)$, и значит, 
\[\frac{\partial F}{\partial \epsilon}(x,t,\epsilon)=\frac{\partial H}{\partial \epsilon}(f_t^{-1}x,t,\epsilon).\]
Таким образом,
\[\frac{\partial F}{\partial \epsilon}(f_{t,\epsilon}x,t,\epsilon)=\frac{\partial H}{\partial \epsilon}(h_{t,\epsilon}x,t,\epsilon).\]
Требуемое утверждение следует теперь из шага 1.
\end{enumerate}


\parit{Доказательство \ref{12.4.A}.}
Вычислим вторую производную $\ell_+(\epsilon)$.
С $\ell_-(\epsilon)$ поступим аналогично.
Мы будем работать в стандартных симплектических координатах $x = (p, q)$ вблизи $x_+$ и $\xi\cdot\eta$ обозначает Евклидово \?{скалярное}{раньше было $(\xi,\eta)$} произведение.
Обозначим через $i$ комплексную структуру $(p, q) \mapsto (-q, p)$.
В этих обозначениях $\Omega(\xi, i\eta) = \xi\cdot\eta$ и гамильтониан запишется как
\[\frac{\d}{\d t}f_tx=i\frac{\partial F}{\partial x}(f_tx,t).\]
Значит
\[C_+(t)=i\frac{\partial^2F}{\partial x^2}(x_+,t).\]
Для упрощения формул введём следующие обозначения: 
\begin{align*}
a&=\frac{\partial^2 F}{\partial x\partial\epsilon}(x_+,t,0),
\\
b&=\frac{\partial x_+}{\partial\epsilon}(t,0),
\\
c&=\frac{\partial^2 F}{\partial\epsilon^2}(x_+,t,0),
\\
K&=\frac{\partial^2 F}{\partial x^2}(x_+,t)
\end{align*}

Теорема о неявной функции гарантирует, что $F(\cdot, t, \epsilon)$ имеет единственную точку максимума $x_+(t, \epsilon)$, которая гладко зависит от $t$ и $\epsilon$.
Поскольку 
\[\ell_+(\epsilon)=\int_0^1F(x_+(t,\epsilon),t,\epsilon)\,\d t\]
мы получаем, что
\[\frac{\d \ell_+}{\d \epsilon}
=
\int_0^1
\frac{\partial F}{\partial x}(x_+(t,\epsilon),t,\epsilon)\frac{\partial x_+}{\partial \epsilon}(t,\epsilon)
+
\frac{\partial F}{\partial x}(x_+(t,\epsilon),t,\epsilon)\,\d t.
\]
Поскольку $x_+(t,\epsilon)$ является критической точкой $F(\cdot,t,\epsilon)$, это выражение можно упростить
\[\frac{\d \ell_+}{\d \epsilon}
=
\int_0^1
\frac{\partial F}{\partial x}(x_+(t,\epsilon),t,\epsilon)\,\d t.
\]
Снова дифференцируя по $\epsilon$ и полагая $\epsilon = 0$, мы получаем 
\begin{equation}
\frac{\d^2 \ell_+}{\d\epsilon^2}=\int_0^1 (a\cdot b +c)\,\d t.
\label{eq:12.4.C}
\end{equation}
Дифференцируя уравнение 
$\frac{\partial F}{\partial x}(x_+ (t, \epsilon), t, \epsilon) = 0$ по $\epsilon$, получаем
\[Kb+a = 0.\]
Условие невырожденности гарантирует, что $K$ обратим и, таким образом, 
\begin{equation}
b=-K^{-1}a.
\label{eq:12.4.D}
\end{equation}

Согласно лемме~\ref{12.4.B}, 
\[\int_0^1\frac{\partial F}{\partial \epsilon}(f_{t,\epsilon},t,\epsilon)\,\d t=0.\]
Дифференцируя это равенство по $\epsilon$ при $\epsilon = 0$ и $x = x_+$, получаем 
\begin{equation}
\int_0^1(a\cdot v_++c)\,\d t = 0.
\label{eq:12.4.E}
\end{equation}

Рассмотрим уравнение Гамильтона
\[\frac{\d}{\d t}f_{t,\epsilon}x=i\frac{\partial F}{\partial x}(f_{t,\epsilon}x,t,\epsilon).\]
Продифференцировав его по $\epsilon$ при $\epsilon = 0$ и $x = x_+$,
получим
\[\dot v_+=iKv_++ia,\]
или, эквивалентно,
\begin{equation}
a=-i\dot v_+-Kv_+.
\label{eq:12.4.F}
\end{equation}

Применив \ref{eq:12.4.D} и \ref{eq:12.4.F} вместе, получаем
\[b=-K^{-1}(-i\dot v_+ - K v_+) = K^{-1}i\dot v_+ + v_+.\]
С учётом \ref{eq:12.4.C} получим
\begin{align*}
\frac{\d^2\ell_+}{\d\epsilon}(0)
&=
\int_0^1(a\cdot b+c)\,\d t=
\\
&=\int_0^1(a\cdot K^{-1}i\dot v_++a\cdot v_++c)\,\d t=
\\
&\!\!\!\stackrel{\text{\tiny \ref{eq:12.4.E}}}{=}\int_0^1 a\cdot K^{-1}i\dot v_+\,\d t
\\
&\!\!\!\stackrel{\text{\tiny \ref{eq:12.4.F}}}{=}-\int_0^1(i\dot v_++Kv_+)\dot K^{-1}i\dot v_+\,\d t=
\\
&\stackrel{{*}}{=}-\int_0^1(K^{-1}i\dot v_+\cdot i\dot v_+ +v_+\cdot i\dot v_+)\,\d t=
\\
&=-\int_0^1(-C_+^{-1}\dot v_+\cdot i\dot v_++v_+\cdot i\dot v_+)\,\d t=
\\
&=-\int_0^1(\Omega(C_+^{-1}\dot v_+,\dot v_+)+\Omega(\dot v_+,v_+))\,\d t.
\end{align*}
Равенство $({*})$ следует из симметрии $K$.
Кроме того, мы воспользовались тем, что $C_+^{-1} = -К^{-1} i$ и $\Omega(\xi, i\eta) = \xi\cdot \eta$
(или, эквивалентно, $-\Omega(\xi, \eta) = \xi\cdot i\eta$).

\section{Анализ формулы второй вариации}\label{sec:12.5}

\begin{thm}{Предложение}\label{12.5.A}
Для произвольных $a \in V_+$ и $b \in V_-$ существует вариация $\{f_{t,\epsilon}\}$ \?{пути}{?} $\{f_t\}$ такая, что $v_+ = a$ и $v_- = b$.
\end{thm}

\parit{Доказательство.}
Выберем гамильтониан $G\in\F_0$ такой, что
\[\int_0^1G(x, t)\,\d t = 0\]
для всех $x\in M$ (то есть $G\in\V_1$ в обозначениях \ref{sec:12.2}).
Мы уточним этот выбор позже.
Определим вариацию $\{h_{1,\epsilon}\}$ постоянной петли $h_{1,0} = \1$ следующим образом: $h_{1,\epsilon}$ --- поток гамильтониана $\int_0^tG(x,s)\,\d s$ с временной координатой $\epsilon$.
Рассмотрим вариацию $f_{t,\epsilon} = f_th_{1,\epsilon}$ \?{пути}{?} $\{f_t\}$.
Тогда
\begin{align*}
v_+(t) &= \left.\frac{\d}{\d \epsilon}\right|_{\epsilon=0} f_{t,\epsilon} x_+ =
\\
&=\left.\frac{\d}{\d \epsilon}\right|_{\epsilon=0}f_th_{t,\epsilon} x_+=
\\
&=
f_{t*} \left.\frac{\d}{\d \epsilon}\right|_{\epsilon=0} h_{1,\epsilon} x_+.
\end{align*}
По построению $\{h_{1,\epsilon}\}$,
\[\left.\frac{\d}{\d \epsilon}\right|_{\epsilon=0}h_{t,\epsilon}x_+
=
\int_0^t\sgrad G(x_+,s)\,\d s.\]
Пусть 
\[c(t) = \frac{\d}{\d t}f^{-1}_{t*} a(t).\]
Выберем теперь $G \in \V_1$ так чтоб 
\[G(x, t) = \Omega(x - x_+, c(t)).\]
в стандартных симплектических координатах вблизи $x_+$.
Поскольку $a(0) = a(1) = 0$, по определению $V_+$, это условие согласовано с тем, что $G\in\V_1$.
Явное выражение для $G$ даёт 
\[\sgrad G(x_+, t) = c(t) = \frac{\d}{\d t}f^{-1}_{t*} a(t).\]
Интегрируя это равенство и учитывая, что $a(0) = 0$, получаем
\[\int_0^t\sgrad G(x_+, s)\,\d s = f_{t*}^{-1}a(t).\]
Отсюда следует, что $v_+(t) = a(t)$.
Тот же аргумент работает для $x_-$.
Поскольку $x_+$ и $x_-$ различны, построенные нами вариации не мешают друг другу.
Это завершает доказательство.
\qeds

\parit{Доказательство теоремы \ref{12.3.A}.}
Учитя \ref{12.4.A} и \ref{12.5.A}, приходим к выводу, что \?{путь}{?} $\{f_t\}$ стабилен тогда и только тогда, когда оба функционала $Q_+$ и $-Q_-$ неотрицательны.
Классические методы вариационного исчисления дадут нам точные условия, когда такое происходит.
Мы сосредоточимся на $Q_+$, но те же рассуждения применимы и к $Q_-$.

Рассмотрим лагранжиан $\L(\dot v, v) = -\Omega(C_+^{-1}\dot v, \dot v) - \Omega(\dot v, v)$.
Требуется исследовать критические точки следующего функционала на~$V_+$:
\[\int_0^1 \L(\dot v(t), v(t))\,\d t.\]

Прежде всего мы утверждаем, что $\L$ удовлетворяет условию Лежандра, то есть, $\tfrac{\partial^2\L}{\partial\dot v^2}$ положительно определена.
Действительно, 
\[\L( \dot v, v) = -iC_+^{-1} \dot v\cdot\dot v - i\dot v\cdot v\]
и, следовательно, $\tfrac{\partial^2\L}{\partial\dot v^2}= - 2i\cdot C_+^{-1}$.

Так как 
\[C_+=i\frac{\partial^2F}{\partial x^2}(x_+)
\quad\text{получаем, что}\quad
-iC_+^{-1}=i\left(\frac{\partial^2F}{\partial x^2}(x_+)\right)^{-1}i\]
и поэтому 
\[
\left\langle i\left(\frac{\partial^2F}{\partial x^2}(x_+)\right)^{-1}i\xi,\xi\right\rangle
=
-\left\langle \left(\frac{\partial^2F}{\partial x^2}(x_+)\right)^{-1}i\xi,i\xi\right\rangle
\]
Точка $x_+$ является невырожденным максимумом $F$.
Таким образом, матрица $\frac{\partial^2F}{\partial x^2}(x_+)$
отрицательно определена, как и $\left(\frac{\partial^2F}{\partial x^2}(x_+)\right)^{-1}$.
Значит $\tfrac{\partial^2\L}{\partial\dot v^2}$ положительно определена и утверждение следует.

Так как $\L$ квадратичен по $v$, постоянный путь $v_0(t) \equiv 0$ экстремален.
Для квадратичных функционалов, удовлетворяющих условию Лежандра, классическое вариационное исчисление говорит нам следующее.
Если $v_0$ --- минимум, то при всех $T \in [0, 1)$ уравнение Эйлера --- Лагранжа (совпадающее с уравнением Якоби) не имеет нетривиальных решений с $v(0) = v(T) = 0$.
И наоборот, если таких решений нет при $T\in[0, 1]$, то $v_0$ минимально.
Остаётся связать уравнение Эйлера --- Лагранжа с периодическими орбитами линеаризованного потока $f_{t*}$ на $\T_{x_+}M$.
Уравнение Эйлера --- Лагранжа имеет вид
\begin{align*}
\frac{\d}{\d t}\frac{\partial \L}{\partial \dot v}
&=\frac{\partial \L}{\partial v}
\\
&\Updownarrow
\\
\frac{\d}{\d t}(-2iC_+^{-1}\dot v+iv)
&=-i\dot v
\\
&\Updownarrow
\\
\frac{\d}{\d t}(-2iC_+^{-1}\dot v)
&=
\frac{\d}{\d t}(-2iv)
\end{align*}
Если $v$ является решением, то $C_+^{-1}\dot v = v + \const$.
Пусть $w = v + \const$, тогда 
\[
\begin{cases}
C_+^{-1}\dot w=w,
\\
w(0)=w(T).
\end{cases}
\]
Но это как раз и означает, что $w$ является $T$-периодической орбитой линеаризованного потока, что доказывает теорему \ref{12.3.A}.
\qeds


\section{Кратчайшие}
Пусть $I \subset \RR$ --- интервал.
Рассмотрим геодезическую $\{f_t\}$, $t\in I$ с единичной скоростью, то есть
\[\|\tfrac{\d}{\d t}f_t\|=1\]
при всех $t\in I$.
Такая геодезическая называется \rindex{локально кратчайшая}\emph{локально кратчайшей}\?{}{замена термина --- строго кратчайшая = кратчайшая, а кратчайшая = слабая кратчайшая.} (см. \ref{sec:8.2} выше), если для каждого $t\in I$ существует окрестность $U$ точки $t$ в $I$ такая, что 
\[\rho(f_a,f_b)=|a-b|\]
при всех $a,b\in U$.

Каждая геодезическая на римановом многообразии локально кратчайшая.
По существу это следует из того, что экспоненциальное отображение связности по Леви-Чивиты является диффеоморфизмом в окрестности нуля.
Аналога такого факта в хоферовской геометрии нет.
Тем не менее различные примеры служат подверждением следующей гипотезы.

\begin{thm}{Гипотеза}\label{12.6.A}
Каждая геодезическая хоферовской метрики локально кратчайшая.
\end{thm}

Впервые это было доказано для стандартной симплектической структуры на $\RR^{2n}$ в \rindex{Бялый}\cite{BP1}.
Позже эта гипотеза была подтверждена для некоторых других симплектических многообразий \rindex{Лалонд}\cite{LM2}, включая кокасательные расслоения, замкнутые ориентированные поверхности и $\CP^2$.

Полезно ослабить понятие локально кратчайшей следующим образом.
Будем говорить, что геодезическая $\{f_t\}$, $t\in I$ \rindex{локально слабая кратчайшая}\emph{локально слабая кратчайшая}, если для каждого $t\in I$ существует окрестность $U$ точки $t$ в $I$ такая, что для любых $a, b\in U$ с $a < b$ геодезический отрезок $\{f_t\}$, $t\in [a, b]$ минимизирует длину в гомотопическом классе путей с фиксированными концами.

\begin{thm}{Гипотеза}\label{12.6.B}
Каждая геодезическая хоферовской метрики является локально слабой кратчайшая.
\end{thm}

В конечномерной римановой геометрии легко доказать, что любая локально слабая кратчайшая является локальную кратчайшей.
В хоферовской геометрии это верно при следующем дополнительном предположении.
Обозначим $S \subset [0; +\infty)$ спектр длин $\Ham(M, \Omega)$ (см. \ref{sec:7.3} выше).
Положим $a = \inf S\backslash\{0\}$.
Как обычно предполагается, что точная нижняя грань пустого множества равена $+\infty$.

\begin{thm}[(\cite{LM2})]{Предложение}\label{12.6.C}
Предположим, что $a>0$.
Тогда любая геодезическая является локально кратчайшей.
\end{thm}

\parit{Доказательство.}
Пусть $\{f_t\}$ --- локально слабая кратчайшая.
Не умоляя общности можно предположить, что её гамильтониан имеет единичную норму в каждый момент времени $t$.
Таким образом, существует $\epsilon \in (0; a/2)$ со следующим свойством.
Неравенство $\length\alpha \ge \epsilon$ выполняется для всякого пути $\alpha$ в $\Ham(M, \Omega)$, соединяющего $\1$ с $f_\epsilon$ и гомотопного $\{f_t\}$, $t\in[0; \epsilon]$.
Достаточно доказать, что $\rho(\1, f_\epsilon) = \epsilon$.
Рассуждая от противного, предположим, что $\length \beta < \epsilon$ для некоторого пути $\beta$, соединяющего $\1$ с $f_\epsilon$.
Обозначим через $\gamma$ путь $\{f_t\}$, $t\in[0; \epsilon]$.
Рассмотрим петлю, образованную $\gamma$ и $\beta$.
Его длина строго меньше $a$.
Следовательно, по определению $a$ эту петлю можно стянуть в сколь угодно короткую.
Но это означает, что путь $\gamma$ гомотопен пути $\gamma'$, длина которого сколь угодно близка к длине $\beta$.
Приходим к противоречию с неравенствами $\length \gamma' \ge\epsilon$ и $\epsilon > \length \beta$.
\qeds

В частности, если для многообразия $(M,\Omega)$ величина $a$ строго положительна, то \ref{12.6.B} влечёт \ref{12.6.A}.
Условие $a > 0$ проверить трудно.
Оно выполняется для многообразий Лиувилля (см. \ref{7.3.B} выше), в случае, когда $\pi_1(\Ham(M, \Omega)$) конечно (например, для поверхностей или $\CP^2$), и в некоторых других примерах в размерности 4.
Пока нет известных примеров многообразий с $a = 0$.

В дополнение, гипотеза \ref{12.6.B} подтверждена для следующего класса симплектических многообразий \rindex{Лалонд}\cite{LM2}.
Предположим, что $M$ полумонотонно (см. \ref{11.3.A}), то есть существует константа $k \ge 0$ такая, что $c_1(\lambda) = k([\Omega],\lambda)$ для любого сферического класса гомологий $A \in H_2 (M, \ZZ)$.
Далее, если $M$ открыто, предположим дополнительно, что $М$ «геометрически ограничено» на бесконечности.
Например, любое кокасательное расслоение $\T^\ast N$ со стандартной симплектической структурой, геометрически ограничено.
Точное определение дано в \cite{AL}.
Тогда выполняется \ref{12.6.B}.

Вернёмся к вопросу, который мы начали обсуждать в \ref{sec:8.2} выше.
Пусть дана геодезическая, что можно сказать о длине временного интервала, на котором она минимальна?
Существует прекрасный подход к этому вопросу, основанный на тщательном изучении замкнутых орбит соответствующего гамильтонова потока.
Он становится особенно прозрачен в случае автономных геодезических, то есть однопараметрических подгрупп группы $\Ham(M, \Omega)$.
Пусть $\{f_t\}$ --- однопараметрическая подгруппа $\Ham(M, \Omega)$.
Стандартная техника обычных дифференциальных уравнений даёт, что существует строго положительное число $\tau > 0$ со следующим свойством \cite[Sec. 5.7]{HZ}).
\rindex{замкнутая орбита}\emph{Каждая замкнутая орбита потока $\{f_t\}$ на временн\'{о}м интервале $[0; \tau]$ есть неподвижная точка потока.}

\begin{thm}{Гипотеза}\label{12.6.D}
Сужение любой однопараметрической подгруппы на $[0;\tau]$ является слабой кратчайшей.
\end{thm}

Это гипотеза была доказана \rindex{Хофер}Хофером для случая $\RR^{2n}$ \cite{H2}.
Его можно рассматривать как глобальную версию критерия сопряжённых точек \ref{12.3.A} выше.%
\footnote{На самом деле, открытие сопряжённых точек было мотивировано результатом Хофера.} 
Для обобщений на неавтономные потоки, а также на другие симплектические многообразия мы отсылаем к \rindex{Бялый}\cite{BP1, Si1,LM2,Sch3,MSl}.
Гипотеза \ref{12.6.D} даёт интересный способ доказательства существования нетривиальных замкнутых орбит автономных гамильтоновых потоков.
Действительно, предположим, что мы заранее знаем, что автономный поток $\{f_t\}$ \textbf{не} является минимальным на некотором временном интервале.
Это может следовать, например, из наличия сопряжённых точек или процедуры укорачивания, описанных в разделах \ref{sec:8.2} и \ref{sec:8.3} выше.
Тогда \ref{12.6.D} гарантирует существование нетривиальных замкнутых орбит на этом интервале.
Мы отсылаем читателя к \rindex{Лалонд}\cite{LM2} и \cite{P8} для дальнейшего обсуждения.

Изучение кратчайших привело к пониманию следующей удивительной особенности хоферовской геометрии.

\?{}{уплощениe?}
\begin{thm}[\cite{BP1}.]{$\bm{C^1}$-уплощение}\rindex{уплощение}
\label{12.6.E}
Существуют $C^1$-окрестность $\E$ единицы в группе $\Ham(\RR^{2n})$ и $C^2$-окрестность $\C$ нуля в её алгебре Ли $\A$ такие, что $(\E, \rho)$ изометрична $(C, \|\ \|)$.
\end{thm}

Некоторые обобщения даются в \cite{LM2}.
Поучительно сравнить явление $C^1$-уплощения с теоремой Сикорава \ref{8.2.A}, утверждающей, что каждая однопараметрическая подгруппа $\Ham(\RR^{2n})$ лежит на ограниченным расстоянием от единицы.
Это можно интерпретировать как положительный эффект типа кривизны, который интуитивно противоречит уплощению!
Пока неясно, как разрешить этот парадокс.
Мы отсылаем читателя к \cite{BP1}, \cite{HZ} за доказательством $C^1$-уплощения (см. также \cite{P8} для дальнейшего обсуждения).

Существует несколько различных подходов к представленным выше результатам о локальных кратчайших.
Все они достаточно сложные.
В следующей главе мы представляем одну из них, которую, наверное можно довести до доказательства гипотезы \ref{12.6.B} в полной общности.
Мы обсудим только следующий простейший случай.

\begin{thm}{Теорема}\label{12.6.F}
Пусть $(M, \Omega)$ --- замкнутое симплектическое многообразие с $\pi_2(M) = 0$.
Тогда подгруппа группы $\Ham(M, \Omega)$ с любым параметром, порождённая общим не зависящим от времени гамильтонианом на $M$ является локально слабой кратчайшей.
\end{thm}

Набросок доказательства приводится в разделе \ref{sec:13.4} ниже.
