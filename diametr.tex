\chapter{Диаметр}

В этой главе доказывается, что группа гамильтоновых диффеоморфизмов замкнутой ориентированной поверхности имеет бесконечный диаметр относительно хоферовской метрики.

\section{Начальная оценка}
Пусть $(M, \Omega)$ --- симплектическое многообразие, и $L \subset M$ --- замкнутое лагранжево подмногообразие со стабильным свойством  лагранжева пересечения.
Пусть $F \subset \F$ --- гамильтониан такой, что $| F (x, t) | \z\ge C$ для всех $x \in L$ и $t \in S^1$.
Здесь $C$ --- положительная постоянная.
Следующее предложение даёт оценку снизу на величину $l (F) \z= \tilde\rho (\tilde\1, \tilde\phi_F)$, введенной в \ref{5.3} выше.

\begin{thm}{Предложение}\label{7.1.A}
В данных предположениях $l (F) \ge C$.
\end{thm}

\parbf{Доказательство:}
Предложение непосредственно следует из \ref{5.3.A} и \ref{6.3.B} выше.
Действительно, \ref{6.3.B} утверждает, что каждая функция $H \in \H_c$ равна нулю в некоторой точке $(y, \tau)$, где $y \in L$ и $\tau \in S^1$.
Таким образом, $| F (y, \tau) - H (y, \tau) | \ge C$ и, следовательно, $\VERT F - H \VERT \z\ge C$.
Это верно для любого $H \in \H_c$.
Значит \ref{5.3.A} влечёт, что $l (F) \z\ge C$.
\qeds

Мы хотим распространить полученную оценку на $\rho (\1, \phi_F).$
Если группа $\Ham (M, \Omega)$ односвязна относительно $C^\infty$-топологии (сильной топологии Уитни), то все пути с общими концами гомотопны и, следовательно, $l (F) = \rho (\1, \phi_F)$.
Однако если фундаментальная группа $\pi_1 (\Ham (M, \Omega))$ нетривиальна, то, вообще говоря, нет никакой надежды на продолжение оценки \ref{7.1.A} без дополнительной информации.
Действительно, может случиться, что в ситуации описанной выше существует более короткий путь, соединяющий $\1$ с $\phi_F$, который, конечно, не гомотопен потоку $\{f_t\}$, $t \in [0; 1]$ гамильтониана $F$.
Тем не менее оказывается, что в некоторых интересных случаях эту трудность можно обойти.
Для этого, нужно более внимательно посмотреть на фундаментальную группу $\Ham (M, \Omega)$.

\section{Фундаментальная группа}
О $\pi_1 (\Ham (M, \Omega))$ известно немного.
Всё понятно для поверхностей (на основе классических методов) и для некоторых четырёхмерных многообразий \cite{G1}, \cite{A}, \cite{AM} (на основе теории псевдоголоморфных кривых Громова).
В высших размерностях доступны лишь некоторые частные результаты.
На самом деле, я не знаю ни одного симплектического многообразия $M$ размерности $\ge 6$, для которого можно было бы полностью описать фундаментальную группу $\Ham (M, \Omega)$.
Например, известно, что $\Ham (\RR^{2n})$ односвязно (и по сути стягиваемо) для $n = 1, 2$, а для $n = 3$ уже ничего не известно.
Нам потребуются следующие утверждения о $\pi_1 (\Ham (M, \Omega))$, где $M$ --- замкнутая ориентируемая поверхность.

\begin{thm}[Сфера (ср. \ref{1.4.H} и \ref{6.3.C} выше).]{}\label{7.2.A}
Включение $\SO (3) \to \Ham (S^2)$ индуцирует изоморфизм фундаментальных групп.
В частности, $\pi_1 (\Ham (S^2))=\ZZ_2$.
Нетривиальный элемент получается вращением сферы вокруг вертикальной оси на один оборот.
\end{thm}

\begin{thm}[Поверхности рода $\ge 1$.]{}
\label{7.2.B}
В этом случае можно показать, что группа гамильтоновых диффеоморфизмов односвязна.
\end{thm}

Эти факты хорошо известны специалистам, но, не похоже, что они публиковались.
Поэтому, я сделаю набросок доказателеьства и добавлю несколько ссылок;
надеюсь, что это позволит читателю восстановить полное доказательство.
Мы будем использовать $\Diff_0 (M)$ (соответственно $\Symp_0 (M)$) для обозначения компоненты единицы группы всех диффеоморфизмов (соответственно симплектоморфизмов) поверхности $M$.

\parbf{Набросок доказательства:}

1) Включение $\pi_1 (\Symp_0 (M)) \to \pi_1 (\Diff_0 (M))$ является изоморфизмом.
Чтобы убедиться в этом, рассмотрим пространство $X$ всех форм площади на $M$ с общей площадью равной $1$.
Зафиксируем форму площади $\Omega \in X$.
Рассмотрим отображение $\Diff_0 (M) \to X$, переводящее диффеоморфизм $f$ в $f^\ast \Omega$.
Можно воспользоваться трюком Мозера \cite[p. 94--95]{MS}, чтобы показать, что это отображение является расслоением Серра.
Заметим, что его слой есть не что иное, как $\Symp_0 (M, \Omega)$, а база $X$ стягиваема.
После этого желанный факт следует из точной гомотопической последовательности расслоения.

2) Топология $\Diff_0 (M)$ известна (см. \cite{EE}).
В частности, эта группа стягиваема для поверхностей рода $\ge 2$.
Кроме того, если $M \z= S^2$, то она содержит $\SO (3)$ как сильный деформационный ретракт.
В случае $M = \TT^2$, она содержит $\TT^2$ как сильный деформационный ретракт (здесь тор действует на себе сдвигами).

3) Включение $j\: \pi_1 (\Ham (M, \Omega)) \to \pi_1 (\Symp_0 (M, \Omega))$ инъективно (см. \cite[10.18 iii]{MS}).
На самом деле это верно для всех замкнутых симплектических многообразий.

4) Собрав все эти утверждения вместе, получаем \ref{7.2.B} для поверхностей рода $\ge 2$.
Учитывая, что $\Symp_0 (S^2) = \Ham (S^2)$ (см. \ref{1.4.C}) получаем \ref{7.2.A}.
Осталось разобраться со случаем тора $\TT^2$.

5) Выберем точку $y \in \TT^2$ и рассмотрим \?{отображение подстановки}{evaluation map} $\Diff_0 (\TT^2) \to \TT^2$, переводящее диффеоморфизм $f$ в точку $f(y)$.
Оно индуцирует отображение $e_D: \pi_1 (\Diff_0 (\TT^2)) \to \pi_1 (\TT^2)$.
Из шага 2 легко увидеть, что $e_D$ --- изоморфизм.
Рассмотрим теперь сужение отображения подстановки на $\Ham (\TT^2)$ и $\Symp_0 (\TT^2)$ и обозначим через $e_H$ и $e_S$ соответственно индуцированные ими гомоморфизмы фундаментальных групп.
Из шага 1 следует, что $e_S$ --- изоморфизм.
Используя шаг 3, получаем, что $e_H = e_S \circ j$, где $j$ инъективно.
Из теоремы Флоера следует, что $e_H$ обращается в ноль (см. \cite{LMP1}).
Таким образом, $\pi_1 (\Ham (\TT^2)) = 0$.
Доказательство завершено. 
\qeds

\begin{thm}[(\cite{P5})]{Теорема}\label{7.2.C}
Предположим, что 
\begin{itemize}
\item либо $M = S^2$ и $L \subset S^2$ --- экватор, 
\item либо $M$ --- замкнутая ориентируемая поверхность рода $\ge 1$ и $L$ --- несжимаемая замкнутая кривая.
\end{itemize}
Пусть $F \in \F$ --- такой гамильтониан, что $|F(x,t)| \ge C$ для всех $x \in L$ и $t \in S^1$, где $C$ --- положительная постоянная.
Тогда $\rho (\1, \phi_F) \z\ge C$.
\end{thm}

\parbf{Доказательство:}
Если $M$ имеет род $\ge 1$, то $l (F) = \rho (\1, \phi_F)$, поскольку группа $\Ham (M, \Omega)$ односвязна (здесь мы пользовались \ref{7.2.B}).
Таким образом, результат следует из \ref{7.1.A} выше.
Если $M = S^2$, то нетривиальный элемент $\pi_1 (\Ham (M, \Omega))$ представлен поворотом на один оборот (см. \ref{7.2.A}).
Его гамильтониан обращается в нуль на $L$ (см. \ref{6.3.C}).
Таким образом, теорема \ref{6.3.A} означает, что каждая функция из $\H$ должна обнуляется в некоторой точке $L \times S^1$.
Необходимое утверждение следует теперь из \ref{5.1.B} выше.
\qeds

\begin{thm}{Следствие}\label{7.2.D} Группа гамильтоновых диффеоморфизмов замкнутой поверхности имеет бесконечный диаметр относительно хоферовской метрики.
\end{thm}

В самом деле, пусть $M$ и $L$ те же, что в \ref{7.2.C} выше, и пусть $B \subset M$ --- открытый диск, не пересекающийся с $L$.
Возьмём не зависящий от времени гамильтониан $F \in \F$, тождественно равный $C$ вне $B$.
По теореме выше, $\rho (\1, \phi_F) \ge C$.
Выбирая $C$ произвольно большим, получаем, что диаметр бесконечен.
Отметим также, что в этом примере носитель $\phi_F$ содержится в $B$.
Таким образом, сокращая $B$ и одновременно увеличивая $C$, мы получаем последовательность гамильтоновых диффеоморфизмов, которая сходится к тождественному в $C^0$-смысле, но расходится в хоферовской метрике.

Для замкнутой поверхности $M$ рода $\ge 1$ существует по крайней мере два других доказательства того, что диаметр $\Ham (M, \Omega)$ бесконечен.
Одно из них выглядит следующим образом.

\begin{thm}[см. \cite{LM2}.]{Упражнение} \label{7.2.E}
Пусть $F$ --- нормализованный гамильтониан на $M$.
Предположим, что некоторое регулярное множество уровня $F$ содержит несжимаемую замкнутую кривую.
Рассмотрим поднятие $\tilde f_t$ соответствующего гамильтонова потока $f_t$ в универсальное покрытие $\tilde M$.
Докажите, что существует $\epsilon> 0$ и семейство дисков $D_t \subset \tilde M$ площади $\epsilon t$ такое, что $\tilde f_t D_t \cap D_t = \emptyset$ для всех достаточно больших $t$.
Выведите из теоремы \ref{3.2.C} выше, что $\rho (\1, f_t) \to + \infty$ при $t \to + \infty$.
\end{thm}

Другое доказательство (см. \cite{Sch3}) основано на гомологиях Флоера (см. Главу 13 о приложениях гомологий Флоера к хоферовской геометрии).
Наконец, тот факт, что $\diam \Ham (M, \Omega) = + \infty$, был доказан для некоторых многомерных многообразий (см. \cite{LM2}, \cite{Sch3}, \cite{P5}).
Общий случай однако остаётся открытым.
