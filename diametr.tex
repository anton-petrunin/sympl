\chapter{Диаметр}

В этой главе доказывается, что группа гамильтоновых диффеоморфизмов
замкнутой ориентированной поверхности имеет бесконечный диаметр
относительно хоферовской метрики. 

\section{Начальная оценка}
Пусть $(M, \Omega)$ — симплектическое многообразие, и $L \subset M$
— замкнутое лагранжево подмногообразие со свойством  устойчивого
лагранжева пересечения. 
Пусть $F \subset \F$ — гамильтониан такой, что $| F (x, t) | \z\ge C$
при всех $x \in L$ и $t \in S^1$. 
Здесь $C$ — положительная константа.
Следующее предложение даёт оценку снизу на величину $l (F) \z=
\tilde\rho (\tilde\1, \tilde\phi_F)$, введенной в \ref{5.3}. 

\begin{thm}{Предложение}\label{7.1.A}
В данных предположениях $l (F) \ge C$.
\end{thm}

\parit{Доказательство.}
Предложение непосредственно следует из \ref{5.3.A} и \ref{6.3.B}.
Действительно, \ref{6.3.B} утверждает, что каждая функция $H \z\in \H_c$ равна нулю в некоторой точке $(y, \tau)$, где $y \in L$ и $\tau \in S^1$.
Таким образом, $| F (y, \tau) - H (y, \tau) | \ge C$ и, следовательно, $\VERT F - H \VERT \z\ge C$.
Это верно для любого $H \in \H_c$.
Значит \ref{5.3.A} влечёт, что $l (F) \z\ge C$.
\qeds

Мы хотим распространить полученную оценку на $\rho (\1, \phi_F).$
Если группа $\Ham (M, \Omega)$ односвязна относительно
$C^\infty$-топологии (сильной топологии Уитни), то все пути с общими
концами гомотопны и, следовательно, $l (F) = \rho (\1, \phi_F)$. 
Однако если фундаментальная группа $\pi_1 (\Ham (M, \Omega))$
нетривиальна, то нет надежды на обобщение оценки \ref{7.1.A} без
дополнительной информации. 
Действительно, может случиться, что в ситуации описанной выше
существует более короткий путь, соединяющий $\1$ с $\phi_F$, который,
конечно, не гомотопен потоку $\{f_t\}$, $t \in [0; 1]$ гамильтониана
$F$. 
Тем не менее оказывается, что в некоторых интересных случаях эту трудность можно обойти.
Для этого, нужно более внимательно посмотреть на фундаментальную группу $\Ham (M, \Omega)$.

\section{Фундаментальная группа}

О группе $\pi_1 (\Ham (M, \Omega))$ известно немного.
Имеется полная картина для поверхностей (на основе классических
методов) и для некоторых четырёхмерных многообразий \rindex{Громов}\cite{G1},
\rindex{Абреу}\cite{A}, \cite{AM} (на основе теории псевдоголоморфных кривых
Громова). 
В высших размерностях доступны лишь некоторые частичные результаты.
На самом деле, я не знаю ни одного симплектического многообразия $M$
размерности $\ge 6$, для которого можно было бы полностью описать
фундаментальную группу $\Ham (M, \Omega)$. 
Например, известно, что $\Ham (\RR^{2n})$ \?{односвязно}{может ссылку
  добавить?} (на самом деле,
стягиваемо) при $n = 1, 2$, а при $n = 3$ уже ничего не известно. 
Нам потребуются следующие утверждения о $\pi_1 (\Ham (M, \Omega))$,
где $M$ — замкнутая ориентируемая поверхность. 

\begin{ex}[Сфера (ср. \ref{1.4.H} и \ref{6.3.C}).]{}\label{7.2.A}
Включение $\SO(3) \z\to \Ham(S^2)$ индуцирует изоморфизм фундаментальных групп.
В частности, $\pi_1 (\Ham (S^2))\z=\ZZ_2$.
Нетривиальный элемент получается вращением сферы вокруг вертикальной
оси на один оборот. 
\end{ex}

\begin{ex}[Поверхности рода $\ge 1$.]{}
\label{7.2.B}
В этом случае можно показать, что группа гамильтоновых диффеоморфизмов
односвязна. 
\end{ex}


Эти утверждения хорошо известны специалистам, но, на сколько мне известно,
они не публиковались.
Поэтому, я сделаю набросок доказателеьства и добавлю несколько ссылок,
надеясь, что это позволит читателю восстановить полное доказательство.
Мы будем использовать $\Diff_0 (M)$ (соответственно $\Symp_0 (M)$) для
обозначения компоненты единицы группы всех диффеоморфизмов
(соответственно симплектоморфизмов) поверхности $M$. 

\parbf{Набросок доказательства:}

1) \textit{Включение $\pi_1 (\Symp_0 (M)) \to \pi_1 (\Diff_0 (M))$ является
изоморфизмом.}
Чтобы убедиться в этом, рассмотрим пространство $X$ всех форм площади
на $M$ с общей площадью равной $1$. 
Зафиксируем форму площади $\Omega \in X$.
Рассмотрим отображение $\Diff_0 (M) \to X$, переводящее диффеоморфизм
$f$ в $f^\ast \Omega$. 
Можно воспользоваться трюком \rindex{Мозер}Мозера \cite[p. 94--95]{MS}, чтобы
показать, что это отображение является расслоением Серра. 
Заметим, что его слой есть не что иное, как $\Symp_0 (M, \Omega)$, а
база $X$ стягиваема. 
После этого желаемый факт следует из точной гомотопической
последовательности расслоения. 

2) \textit{Топология $\Diff_0 (M)$ известна} (см. \cite{EE}).
В частности, эта группа стягиваема для поверхностей рода $\ge 2$.
Кроме того, если $M \z= S^2$, то она содержит $\SO (3)$ как сильный
деформационный ретракт. 
В случае $M = \TT^2$, она содержит $\TT^2$ как сильный деформационный
ретракт (здесь тор действует на себе сдвигами). 

3) \textit{Включение $j\: \pi_1 (\Ham (M, \Omega)) \to \pi_1 (\Symp_0 (M,
\Omega))$ инъективно} (см. \cite[10.18 iii]{MS}). 
На самом деле это верно для всех замкнутых симплектических многообразий.

4) Собрав все эти утверждения вместе, получаем \ref{7.2.B} для
поверхностей рода $\ge 2$. 
Учитывая, что $\Symp_0 (S^2) = \Ham (S^2)$ (см. \ref{1.4.C}) получаем
\ref{7.2.A}. 
Осталось разобраться со случаем тора $\TT^2$.

5) Выберем точку $y \in \TT^2$ и рассмотрим отображение подстановки $\Diff_0 (\TT^2) \to \TT^2$,
переводящее диффеоморфизм $f$ в точку $f(y)$. 
Оно индуцирует отображение $e_D: \pi_1 (\Diff_0 (\TT^2)) \to \pi_1 (\TT^2)$.
Из шага 2 легко увидеть, что $e_D$ — изоморфизм.
Рассмотрим теперь сужение отображения подстановки на $\Ham (\TT^2)$ и
$\Symp_0 (\TT^2)$ и обозначим через $e_H$ и $e_S$ соответственно
индуцированные ими гомоморфизмы фундаментальных групп. 
Из шага 1 следует, что $e_S$ — изоморфизм.
Используя шаг 3, получаем, что $e_H = e_S \circ j$, где $j$ инъективно.
Из теоремы \rindex{Флоер}Флоера следует, что $e_H$ обращается в ноль (см. \cite{LMP1}).
Таким образом, $\pi_1 (\Ham (\TT^2)) = 0$. 
Доказательство завершено. 
\qeds

\begin{thm}[(\cite{P5})]{Теорема}\label{7.2.C}
Предположим, что 
\begin{itemize}
\item либо $M = S^2$ и $L \subset S^2$ — экватор, 
\item либо $M$ — замкнутая ориентируемая поверхность рода $\ge 1$ и
  $L$ — нестягиваемая простая замкнутая кривая. 
\end{itemize}
Пусть $F \in \F$ — такой гамильтониан, что $|F(x,t)| \ge C$ при всех $x \z\in L$ и $t \in S^1$, где $C$ — положительная постоянная.
Тогда $\rho (\1, \phi_F) \z\ge C$.
\end{thm}

\parit{Доказательство.}
Если $M$ имеет род $\ge 1$, то $l (F) = \rho (\1, \phi_F)$, поскольку группа $\Ham (M, \Omega)$ односвязна (здесь мы пользовались \ref{7.2.B}).
Таким образом, результат следует из \ref{7.1.A}.
Если $M = S^2$, то нетривиальный элемент $\pi_1 (\Ham (M, \Omega))$ представлен поворотом на один оборот (см. \ref{7.2.A}).
Его гамильтониан обращается в нуль на $L$ (см. \ref{6.3.C}).
Таким образом, теорема \ref{6.3.A} означает, что каждая функция из $\H$ должна обнуляется в некоторой точке $L \times S^1$.
Необходимое утверждение следует теперь из \ref{5.1.B}.
\qeds

\begin{thm}{Следствие}\label{7.2.D} Группа гамильтоновых диффеоморфизмов замкнутой поверхности имеет бесконечный диаметр относительно хоферовской метрики.
\end{thm}

В самом деле, пусть $M$ и $L$ те же, что в \ref{7.2.C}, и пусть $B \subset M$ — открытый диск, не пересекающийся с $L$.
Возьмём не зависящий от времени гамильтониан $F \in \F$, тождественно равный $C$ вне $B$.
По теореме выше, $\rho (\1, \phi_F) \ge C$.
Выбирая $C$ произвольно большим, получаем, что диаметр бесконечен.
Отметим также, что в этом примере носитель $\phi_F$ содержится в $B$.
Таким образом, сокращая $B$ и одновременно увеличивая $C$, мы получаем
последовательность гамильтоновых диффеоморфизмов, которая сходится к
тождественному в $C^0$-смысле, но расходится в хоферовской метрике. 

Для замкнутой поверхности $M$ рода $\ge 1$ существует по крайней мере
два других доказательства того, что диаметр $\Ham (M, \Omega)$
бесконечен. 
Одно из них выглядит следующим образом. 

\begin{ex}[(см. \cite{LM2}).]{Упражнение} \label{7.2.E}
Пусть $F$ — нормализованный гамильтониан на $M$.
Предположим, что некоторое регулярное множество уровня $F$ содержит
нестягиваемую замкнутую кривую.  
Рассмотрим поднятие $\tilde f_t$ соответствующего гамильтонова потока
$f_t$ в универсальное накрытие $\tilde M$.  
Докажите, что существует $\epsilon> 0$ и семейство дисков $D_t \subset
\tilde M$ площади $\epsilon t$ такое, что $\tilde f_t D_t \cap D_t =
\emptyset$ при всех достаточно больших $t$.  
Выведите из теоремы \?{\ref{3.2.C}}{Не могу разобраться, при чём тут
  3.2.С?}, что $\rho (\1, f_t) \to + \infty$ при $t \to + \infty$.  
\end{ex}

Другое доказательство (см. \rindex{Шварц}\cite{Sch3}) основано на гомологиях Флоера (см. Главу 13 о приложениях гомологий Флоера к хоферовской геометрии).
Наконец, то, что $\diam \Ham (M, \Omega) = + \infty$, было доказано для некоторых многомерных многообразий (см. \rindex{Лалонд}\cite{LM2,Sch3,P5}).
Общий случай остаётся открытым.

\section{Спектр длин}\label{sec:7.3}

Как было показано, наш метод даёт нижнюю оценку на $\rho (\1, \phi_F)$, при наличии точной информации о фундаментальной группе $\Ham (M, \Omega)$.
Напомним, что в размерностях $\ge 6$ такой информации нет.
Сейчас мы немного изменим ход рассуждений, расширив при этом класс многообразий, к которым применим метод.
Следующее понятие является одним из главных героев нашего повествования.

\begin{ex}{Определение}\label{7.3.A}
Определим {}\emph{норму} элемента $\gamma \in \pi_1 (\Ham (M, \Omega))$ как \index[symb]{$\nu(\gamma)$}
\[\nu(\gamma)=\inf\length\{h_t\},\]
где нижняя грань берётся по всем гамильтоновым петлям $\{h_t\}$, представляющим $\gamma$.
Множество 
\[\set{\nu (\gamma)}{\gamma \in \pi_1 (\Ham (M, \Omega))}\]
называется \rindex{спектр длин}\emph{спектром длин} группы $\Ham (M, \Omega)$.
\end{ex}

\begin{ex*}{Упражнение}

\begin{enumerate}[(i)]
 \item Покажите, что группа $\pi_1 (\Ham (M, \Omega))$ является абелевой. (Используйте то же рассуждение, что и для конечномерных групп Ли.)
Таким образом, мы можем обозначать групповую операцию через $+$, а нейтральный элемент через $0$.
 \item Докажите, что $\nu (\gamma) = \nu (-\gamma)$ и $\nu (\gamma + \gamma') \le \nu (\gamma) + \nu (\gamma')$.
\end{enumerate}

\end{ex*}

На данный момент нет общего утверждения о невырожденности $\nu$ (и
меня не удивит пример $\gamma \ne 0$ с $\nu (\gamma) = 0$, сравните с
примером \ref{7.3.B}). 
То есть, было бы честнее называть $\nu$ псевдонормой, но мы будем
называть её нормой. 
В главе \ref{chap:9} мы опишем метод, который даёт нижние оценки
на $\nu (\gamma)$ и, в частности, позволит вычислить спектр
длин~$S^2$. 

\begin{ex}[Многообразия Лиувилля.]{Пример}\label{7.3.B}\rindex{многообразие Лиувилля}
Мы говорим, что открытое симплектическое многообразие $(M, \Omega)$
обладает \rindex{свойство Лиувиля}\emph{свойством Лиувиля}, если
существует гладкое семейство диффеоморфизмов 
\[D_c: M \to M,\quad c \in (0; + \infty)\]
такое, что $D_1 = \1$ и $D_c^\ast \Omega = c\Omega$ при всех $c$.
Такие диффеоморфизмы $D_c$ конечно, не могут иметь компактных
носителей при $c \ne 1$. 
Важный класс примеров это кокасательные расслоения со стандартной
симплектической структурой (см. \ref{3.1.C}).
Диффеоморфизм $D_c$ в этом случае задаётся послойной гомотетией $(p,
q) \z\mapsto (cp, q)$. 
Мы утверждаем что спектр длин $\Ham (M, \Omega)$ равен $\{0\}$ при
условии, что $(M, \Omega)$ 
обладает свойством Лиувиля.
Доказательство этого утверждения основано на следующем простом факте.
\end{ex}

\begin{ex*}{Упражнение}
Пусть $\{f_t\}$ — гамильтонов поток, порожденный нормализованным гамильтонианом $F (x, t)$.
Тогда при любом $c> 0$ поток $\{D_c f_t D_c^{-1}\}$ снова является
гамильтоновым и его нормализованный гамильтониан равен $cF (D_c^{-1} x, t)$.
\end{ex*}

Пусть $\{h_t\}$ — петля гамильтоновых диффеоморфизмов.
Рассмотрим семейство гомотопных петель $\{D_c h_t D_c ^{-1}\}$.
Из упражнения выше следует, что длины петель стремятся к нулю при $c \to 0$, поэтому каждую петлю можно продеформировать в петлю произвольной малой длины.
Мы заключаем, что спектр длин равен $\{0\}$ (без каких-либо сведений о фундаментальной группе).
В следующей главе мы разберём этот пример в контексте классической механики.

\section{Уточнение оценки}

\begin{thm}{Теорема}\label{7.4.A}
Пусть $(M, \Omega)$ — симплектическое многообразие и $L \subset M$
— замкнутое лагранжево подмногообразие со свойством устойчивого
лагранжева пересечения. 
Предположим, что спектр длин группы $\Ham (M, \Omega)$ ограничен
сверху некоторым $K \ge 0$. 
Далее предположим, что гамильтониан $F \in \F$ такой, что что $| F (x,
t) | \z\ge C$ при всех $x \in L$ и $t \in S^1$. 
Тогда 
\[\rho (\1, \phi_F) \ge C - K.\]
\end{thm}

\parit{Доказательство.}
Выберем произвольное $\epsilon> 0$. 
Пусть $\{g_t\}$ — путь гамильтоновых диффеоморфизмов, соединяющий $\1$ с $\phi_F$.
Рассмотрим петлю $\{f_t \circ g_t ^{-1}\}$.
По нашему предположению эта петля гомотопна петле $\{h_t\}$, длина которой не превосходит $K + \epsilon$.
Путь $\{f_t\}$ гомотопен с фиксированными концами с композицией $\{h_t \circ g_t\}$.
Следовательно
\[l (F) \le \length \{h_t\} + \length \{g_t\}.\]
Поскольку $l (F) \ge C$, то в силу \ref{7.1.A}, получаем, что $\length \{g_t\} \ge C \z- K - \epsilon$.
Таким образом, $\rho (\1, \phi_F) \ge C - K$.
\qeds
