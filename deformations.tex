\chapter[Деформации симплектических форм]{Деформации симплектических форм и псевдоголоморфные кривые}

В этой главе мы получим верхнюю оценку на параметр спаривания \ref{9.4.A} и, таким образом,
завершим вычисление спектра длин $\Ham(S^2)$.
Эта оценка оказывается частным случаем более общей задачи о деформациях симплектических форм (см. \cite{P7}).
При решении задачи деформации, будет использована громовская теория псевдоголоморфных кривых.

\section{Задача деформации}

\begin{wrapfigure}[8]{r}{30 mm}
\vskip-6mm
\centering
\includegraphics{mppics/pic-10}
\caption{}\label{pic-10}
\vskip0mm
\end{wrapfigure}

Пусть $(P, \omega)$ --- замкнутое симплектическое многообразие и $l$ --- луч в $H^2 (P, \RR)$ с началом в $[\omega]$, см. рис. \ref{pic-10}.

\begin{ex*}{Задача}
Как далеко можно деформировать $\omega$ так, чтобы класс когомологий двигался вдоль $l$ и форма оставалась симплектической? 
\end{ex*}

Обратите внимание, что добавление малой замкнутой 2-формы к $\omega$ оставляет её симплектической --- вопрос в том, как далеко можно зайти.
В случае деформации спаривания сама \?{форма}{зачжем здесь квадратные скобки?} $[p^\ast \tau]$ вырождена, но форма $[p^\ast \tau ] + tc$ уже симплектична при малых $t$.
В этом случае нам удастся пройти по лучу в направлении класса спаривания $c$ ровно на $\epsilon(P)$.

Приведём пример препятствия к существованию бесконечной деформации.
Предположим, что $\dim P = 4$ и что $\Sigma \subset P$ --- вложенная 2-сфера такая, что $\omega|_{T \Sigma}$ --- форма площади (другими словами, $\Sigma$ --- симплектическое подмногообразие в $P$).
Через $(A, B)$ будет обозначаться индекс пересечения классов гомологий $A$ и $B$.

\begin{ex*}{Определение}
Пусть $\Sigma \subset P^4$ --- симплектическая вложенная сфера.
Если $([\Sigma], [\Sigma]) = -1$, то $\Sigma$ называется \emph{исключительной сферой}.
\end{ex*}

\begin{thm}[(\cite{McD1})]{Теорема}\label{10.1.A}
Пусть $\Sigma \subset (P^4, \omega)$ --- исключительная сфера.
Пусть $\omega_t$, $t \in [0, 1]$, --- \?{симплектическая деформация}{вместо ``thru symplectic forms''} $\omega$.
Тогда $([\omega_1 ], [\Sigma]) > 0$.
\end{thm}

Другими словами, гиперплоскость $(x, [\Sigma]) = 0$ в $H^2 (P, \RR)$ образует непроницаемую стенку для симплектических деформаций.

Доказательство основано на теории псевдоголоморфных кривых.
Ниже мы дадим набросок доказательства этого утверждения, а также приложение к доказательству \ref{9.4.A}.

\section[\texorpdfstring{И снова $\bar\partial$-уравнение}{И снова ∂-уравнение}]{И снова $\bm{\bar\partial}$-уравнение}

Почти комплексная структура $j$ на многообразии $P$ это поле эндоморфизмов $\T P \to \T P$ таких, что $j^2 = -\1$.
Важный класс примеров приходит из комплексной алгебраической геометрии.
Каждое комплексное многообразие (то есть многообразие с атласом у которого голоморфны отображения скеек) имеет каноническую почти комплексную структуру $\xi \to \sqrt{-1}\xi$.
Возникающие таким образом почти комплексные структуры называются \emph{интегрируемыми}.
Глубоким фактом является то, что интегрируемость эквивалентна обнулению некоторого тензора, связанного с почти комплексной структурой (\cite{NN}).
Как следствие этого результата получается, что всякая почти комплексная структура на (вещественной) поверхности \?{интегрируема}{Не нужно глубоких фактов, ведь почти комплексная структура = конформаня структура + ориентация}.
Более того, на 2-сфере все почти комплексные структуры диффеоморфны (это классический факт --- так называемая теорема об униформизации \cite{AS}).

\begin{ex*}{Определение}
Пусть $(P, \omega)$ --- симплектическое многообразие.
Почти комплексная структура $j$ называется \emph{совместимой} с $\omega$, если $g(\xi, \eta) \z= \omega(\xi, j\eta)$ определяет риманову метрику на $P$.
\end{ex*}


\begin{ex}{Упражнение}\label{10.2.A}
Пусть $(P, \omega)$ --- вещественная поверхность и пусть $j$ --- почти комплексная структура на $P$.
Тогда либо $j$, либо $-j$ совместима с $\omega$.
\end{ex}

Пусть $(P, \omega)$ --- симплектическое многообразие и $j$ --- совместимая интегрируемая почти комплексная структура на $P$.
Тройка $(P, \omega, j)$ называется \emph{кэлеровой} \?{структурой}{скорее многообразием}.
Например, $\CP^n$ со стандартной $\omega$ и $j$ является келеровым многообразием (см. \ref{sec:9.2} выше).
Таким образом, каждое комплексное подмногообразие в $\CP^n$ кэлерово относительно индуцированной структуры.
И наоборот, если $(P, \omega, j)$ --- замкнутое кэлерово многообразие такое, что класс когомологий $[\omega]$ целочисленен, то $(P, j)$ голоморфно вкладывается в $\CP^n$ для некоторого $n$ (это знаменитая теорема Кодаиры, см. \cite{GH}).

Важный шаг, сделанный Громовым \cite{G1}, сотоит в том что некоторые важные методы алгебраической геометрии можно обобщить на \emph{квазикелеровы многообразия} $(P, \omega, j)$.
Здесь $j$ --- совместимая почти комплексная структура, которая может быть не интегрируемой.
Примечательно то, что теория голоморфных кривых без существенных изменений распространяется на неинтегрируемый случай.
Важность такого обобщения обусловлена ​​следующей причиной ---
каждое симплектическое многообразие допускает совместимую почти комплексную структуру.
При этом существуют симплектические многообразия, которые не могут иметь кэлерову структуру \cite{MS}.

\begin{ex}[\cite{MS}]{Упражнение}\label{10.2.B}
Пусть $E$ --- чётномерное линейное пространство с невырожденной кососимметричной билинейной формой $\omega$.
Покажите, что пространство комплексных структур $j\: E \z\to E$, $j^2 = -\1$ стягиваемо.
\end{ex}

Таким образом совместимая почти комплексная структура на симплектическом многообразии --- это сечение расслоения, слои которого стягиваемы.
Следовательно, такие структуры существуют и, кроме того, образуют стягиваемое пространство.
По этой же причине различные задачи продолжения, связанные с почти комплексными структурами, допускают положительное решение.

\begin{ex}{Упражнение}\label{10.2.С}
Пусть $(P, \omega)$ --- четырёхмерное симплектическое многообразие, и пусть $\Sigma \subset P$ --- симплектическое подмногообразие.
Предположим, что $\Sigma$ снабжена почти комплексной структурой $j$, согласованной с $\omega|_{\T \Sigma}$.
Покажите, что $j$ продолжается до совместимой почти комплексной структуры на $P$.
\end{ex}


Теперь перейдем к теории псевдоголоморфных кривых на квазикэлеровых многообразиях.

\begin{ex*}{Определение}
Отображение $\phi\:(S^2, i) \to (P, j)$ является псевдоголоморфной (или $j$-голоморфной) кривой, если 
$\phi_\ast \circ i = j \circ \phi_\ast$.
\end{ex*}

\begin{ex}{Упражнение}\label{10.2.D}
Покажите, что на комплексном многообразии приведенное выше определение эквивалентно обычному уравнению Коши --- Римана.
\end{ex}

Положим 
\[\bar\partial\phi=\frac12(\phi_\ast+j\circ\phi_\ast\circ i)\]

\begin{ex}[(ср. \ref{4.1.A})]{Упражнение}\label{10.2.E}
Для заданных $(P, \omega, j, g)$ и $j$-голоморфной кривой $\phi \: (S^2, i) \to (P, j)$ покажите, что $\Area_g (\phi(S^2)) = \int_{S^2}\phi^\ast \omega$.
В частности, если $\phi$ непостоянна, то $\int_{S^2} \phi^\ast \omega > 0$.

Более того, если $\phi$ --- вложение, то $\phi(S^2)$ --- симплектическое подмногообразие в $P$.
Действительно, ограничение симплектической формы совпадает с формой римановой площади.
\end{ex}

Пусть $(P, \omega)$ --- симплектическое многообразие.
Выберем почти комплексную структуру $j$, совместимую с $\omega$.
Тогда касательное расслоение $\T P$ получает структуру комплексного векторного расслоения.
Поскольку пространство совместимых $j$ связно, соответствующие характеристические классы не зависят от выбора $j$.
Обозначим через $c_1$ первый класс Черна расслоения $\T P$ относительно любой согласованной почти комплексной структуры.

\begin{ex}[(формула присоединения)]{Упражнение}\label{10.2.F}
Пусть $(P^4, \omega)$ --- симплектическое многообразие с согласованной почти комплексной структурой $j$.
Пусть $\Sigma \subset P$ --- вложенная $j$-голоморфная сфера.
Покажите, что
\[1 +\frac12 (([\Sigma], [\Sigma]) - c_1 (\Sigma)) = 0.\]
Подсказка: используйте, что $([\Sigma], [\Sigma])$ --- самопересечение в комплексном нормальном расслоении $\nu_\Sigma$ и $\T_\Sigma P = \T\Sigma \oplus \nu_\Sigma$.
\end{ex}


\begin{thm*}{Следствие} Для симплектически вложенной сферы $\Sigma$, $c_1 = 1$ тогда и только тогда, когда $([\Sigma], [\Sigma]) = -1$.
\end{thm*}

\parit{Доказательство.}
Поскольку $\Sigma$ симплектически вложена, существует $\omega$-совместимая почти комплексная структура $j$ такая, что $\Sigma$ $j$-голоморфна (см. \ref{10.2.С}).
Теперь утверждение следует из формулы присоединения \ref{10.2.F}.
\qeds
