\chapter{Деформации симплектических форм и псевдоголоморфные кривые}

В этой главе мы завершим вычисление спектра длин $\Ham(S^2)$ доказав верхнюю оценку на параметр спаривания \ref{9.4.A}.
Эта оценка оказывается частным случаем более общей задачи о деформациях симплектических форм (см. \cite{P7}).
Мы будем использовать подход к этой задаче, основанный на громовской теории псевдоголоморфных кривых.

\section{Задача деформации}

Пусть $(P, \omega)$ --- замкнутое симплектическое многообразие и $l$ --- луч в $H^2 (P, \RR)$ с началом в $[\omega]$, см. рис. \ref{pic-10}.

\begin{thm*}{Задача}
Как далеко можно деформировать $\omega$ так, чтобы класс когомологий двигался вдоль $l$ и форма оставалась симплектической? 
\end{thm*}

Обратите внимание, что добавление малой замкнутой 2-формы к $\omega$ оставляет её симплектической, но как далеко можно зайти?
В случае деформации спаривания сама \?{форма}{зачжем здесь квадратные скобки?} $[p^\ast \tau]$ вырождена, но форма $[p^\ast \tau ] + tc$ уже симплектична при малых $t$.
В этом случае нам удастся пройти по лучу в направлении класса спаривания $c$ ровно на $\epsilon(P)$.

РИСУНОК 10


Приведём пример препятствия к существованию бесконечной деформации.
Предположим, что $\dim P = 4$ и что $\Sigma \subset P$ --- вложенная 2-сфера такая, что $\omega|_{T \Sigma}$ --- форма площади (другими словами, $\Sigma$ --- симплектическое подмногообразие в $P$).
Мы пишем $(A, B)$ для индекса пересечения двух классов гомологий $A$ и $B$.

\begin{thm*}{Определение}
Пусть $\Sigma \subset P^4$ --- симплектическая вложенная сфера.
Если $([\Sigma], [\Sigma]) = −1$, то $\Sigma$ называется \emph{исключительной сферой}.
\end{thm*}

\begin{thm}[(\cite{McD1})]{Теорема}\label{10.1.A}
Пусть $\Sigma \subset (P^4, \omega)$ --- исключительная сфера.
Пусть $\omega_t$, $t \in [0, 1]$, --- \?{симплектическая деформация}{вместо ``thru symplectic forms''} $\omega$.
Тогда $([\omega_1 ], [\Sigma]) > 0$.
\end{thm}

Другими словами, гиперплоскость $(x, [\Sigma]) = 0$ в $H^2 (P, \RR)$ --- это стенка, которую нельзя пересечь симплектической деформацией.

Доказательство основано на теории псевдоголоморфных кривых.
Ниже мы дадим набросок доказательства этого утверждения, а также приложение к доказательству \ref{9.4.A}
