\chapter[Деформации симплектических форм]{Деформации симплектических форм и псевдоголоморфные кривые}

В этой главе мы получим верхнюю оценку на параметр сцепления
\ref{9.4.A} и, таким образом, завершим вычисление спектра длин $\Ham(S^2)$.
Эта оценка оказывается частным случаем более общей задачи о
деформациях симплектических форм (см. \cite{P7}). 
В её решении, будет использована громовская теория псевдоголоморфных кривых.

\section{Задача деформации}

\begin{wrapfigure}[8]{r}{30 mm}
\vskip-6mm
\centering
\includegraphics{mppics/pic-10}
\caption{}\label{pic-10}
\vskip0mm
\end{wrapfigure}

Пусть $(P, \omega)$ — замкнутое симплектическое многообразие и $l$ —
луч в $H^2 (P;\RR)$ с началом в $[\omega]$, см. рис. \ref{pic-10}. 

\begin{ex*}{Задача}
Как далеко можно деформировать $\omega$ так, чтобы класс когомологий двигался вдоль $l$ и форма оставалась симплектической? 
\end{ex*}

Обратите внимание, что добавление малой замкнутой 2-формы к $\omega$ оставляет её симплектической — вопрос в том, как далеко можно зайти.
В деформации сцепления форма $\omega_0=p^\ast \tau$ вырождена, но форма $\omega_t$ в классе $[p^\ast \tau ] + tc$ становится симплектической при малых $t$.
В этом случае нам удастся пройти по лучу в направлении класса сцепления $c$ ровно на $\epsilon(P)$.

Приведём пример препятствия к существованию бесконечной деформации.
Предположим, что $\dim P = 4$ и что $\Sigma \subset P$ — вложенная 2-сфера такая, что $\omega|_{\T \Sigma}$ — форма площади (другими словами, $\Sigma$ — симплектическое подмногообразие в $P$).
Через $(A, B)$ будет обозначаться индекс пересечения классов гомологий $A$ и $B$.

\begin{ex*}{Определение}
Пусть $\Sigma \subset P^4$ — симплектическая вложенная сфера.
Если $([\Sigma], [\Sigma]) = -1$, то $\Sigma$ называется \rindex{исключительная сфера}\emph{исключительной сферой}.
\end{ex*}

\begin{thm}[(\cite{McD1})]{Теорема}\label{10.1.A}
Пусть $\Sigma \subset (P^4, \omega)$ — исключительная сфера.
Пусть $\omega_t$, $t \in [0;1]$, — симплектическая деформация $\omega$.
Тогда $([\omega_1 ], [\Sigma]) > 0$.
\end{thm}

Другими словами, гиперплоскость $(x, [\Sigma]) = 0$ в $H^2 (P;\RR)$ образует непроницаемую стенку для симплектических деформаций.

Доказательство основано на теории псевдоголоморфных кривых.
Ниже мы дадим набросок доказательства, а также
приложение к доказательству \ref{9.4.A}. 

\section[\texorpdfstring{И снова $\bar\partial$-уравнение}{И снова ∂-уравнение}]{И снова $\bm{\bar\partial}$-уравнение}

\rindex{почти комплексная структура}\emph{Почти комплексная структура} $j$ на многообразии $P$ это поле
эндоморфизмов $\T P \to \T P$ таких, что $j^2 = -\1$. 
Важный класс примеров приходит из комплексной алгебраической геометрии.
Каждое комплексное многообразие (то есть многообразие с атласом у
которого голоморфны отображения склеек) имеет каноническую почти
комплексную структуру $\xi \to \sqrt{-1}\xi$. 
Возникающие таким образом почти комплексные структуры называются
\emph{интегрируемыми}. 
Важно, что интегрируемость эквивалентна
обнулению некоторого тензора, связанного с почти комплексной
структурой (\cite{NN}). 
Как следствие получается, что всякая почти
комплексная структура на (вещественной) поверхности интегрируема.%
\footnote{Поскольку почти комплексная структура на поверхности это конформная структура плюс ориентация,
это утверждение также следует из теоремы об униформизации. — \textit{Прим. ред.}}
Более того, на 2-сфере любые две почти комплексные структуры диффеоморфны (это классическое утверждение — так называемая теорема об униформизации \cite{AS}).

\begin{ex*}{Определение}
Пусть $(P, \omega)$ — симплектическое многообразие.
Почти комплексная структура $j$ называется \rindex{совместимая
  комплексная структура}\emph{совместимой} с $\omega$, если $g(\xi,
\eta) \z= \omega(\xi, j\eta)$ определяет риманову метрику на $P$. 
\end{ex*}


\begin{ex}{Упражнение}\label{10.2.A}
Пусть $(P, \omega)$ — вещественная симплектическая
поверхность и пусть $j$ — почти комплексная структура на $P$. 
Тогда либо $j$, либо $-j$ совместима с $\omega$.
\end{ex}

Пусть $(P, \omega)$ — симплектическое многообразие и $j$ —
совместимая интегрируемая почти комплексная структура на $P$.
Тройка $(P, \omega, j)$ называется \rindex{кэлерово
  многообразие}\emph{кэлеровым многообразием}.  
Например, $\CP^n$ со стандартными $\omega$ и $j$ является кэлеровым
многообразием (см. \ref{sec:9.2}). 
Таким образом, каждое комплексное подмногообразие в $\CP^n$ кэлерово
относительно индуцированной структуры. 
И наоборот, если $(P, \omega, j)$ — замкнутое кэлерово многообразие
такое, что класс когомологий $[\omega]$ целочисленен, то $(P, j)$
голоморфно вкладывается в $\CP^n$ для некоторого $n$ (это знаменитая
теорема Кодаиры, см. \cite{GH}). 

Глубокое наблюдение, сделанное \rindex{Громов}Громовым \cite{G1}, состоит в том
что некоторые мощные методы алгебраической геометрии можно обобщить на
\rindex{квазикэлерово многообразие}\emph{квазикэлеровы многообразия}
$(P, \omega, j)$. 
Здесь $j$ — совместимая почти комплексная структура, возможно не интегрируемая. 
Примечательно, что теория голоморфных кривых без существенных
изменений распространяется на неинтегрируемый случай. 
Важность такого обобщения обусловлена тем, что
каждое симплектическое многообразие допускает совместимую почти
комплексную структуру.
При этом существуют симплектические многообразия, которые не допускают
кэлеровой структуры~\cite{MS}. 

\begin{ex}[\cite{MS}]{Упражнение}\label{10.2.B}
Пусть $E$ — чётномерное линейное пространство с невырожденной
кососимметричной билинейной формой~$\omega$. 
Покажите, что пространство комплексных структур $j\: E \z\to E$, $j^2
= -\1$ совместимых с $\omega$ стягиваемо. 
\end{ex}

Таким образом совместимая почти комплексная структура на
симплектическом многообразии — это сечение расслоения, слои которого
стягиваемы. 
Следовательно, такие структуры существуют и, кроме того, образуют
стягиваемое пространство. 
По этой же причине различные задачи продолжения, связанные с почти
комплексными структурами, допускают положительное решение. 

\begin{ex}{Упражнение}\label{10.2.C}
Пусть $(P, \omega)$ — четырёхмерное симплектическое многообразие, и
пусть $\Sigma \subset P$ — симплектическое подмногообразие. 
Предположим, что $\Sigma$ снабжена почти комплексной структурой $j$,
согласованной с $\omega|_{\T \Sigma}$. 
Покажите, что $j$ продолжается до совместимой почти комплексной
структуры на $P$. 
\end{ex}


Теперь перейдём к теории псевдоголоморфных кривых на квазикэлеровых многообразиях.

\begin{ex*}{Определение}
Отображение $\phi\:(S^2, i) \to (P, j)$ является псевдоголоморфной
(или $j$-голоморфной) кривой, если  
$\phi_\ast \circ i = j \circ \phi_\ast$.
\end{ex*}

\begin{ex}{Упражнение}\label{10.2.D}
Покажите, что на комплексном многообразии приведённое выше определение
эквивалентно обычному уравнению Коши — Римана.
\end{ex}

Определим 
\[\bar\partial\phi=\frac12(\phi_\ast+j\circ\phi_\ast\circ i)\]

\begin{ex}[(ср. \ref{4.1.A})]{Упражнение}\label{10.2.E}
Для заданных $(P, \omega, j, g)$ и $j$-го\-ло\-морф\-ной кривой $\phi
\: (S^2, i) \to (P, j)$ покажите, что $\area_g (\phi(S^2)) \z=
\int_{S^2}\phi^\ast \omega$. 
Более того, если $\phi$ непостоянна, то $\int_{S^2} \phi^\ast \omega
> 0$.  
\end{ex}

В дополнение сказанному, если $\phi$ — вложение, то $\phi(S^2)$ —
симплектическое подмногообразие в $P$. 
Действительно, ограничение симплектической формы совпадает с формой
римановой площади. 

Пусть $(P, \omega)$ — симплектическое многообразие.
Выберем почти комплексную структуру $j$, совместимую с $\omega$.
Тогда касательное расслоение $\T P$ получит структуру комплексного
векторного расслоения. 
Поскольку пространство совместимых $j$ связно, соответствующие
характеристические классы не зависят от выбора $j$. 
Обозначим через \index[symb]{$c_1$}$c_1$ первый класс Черна расслоения $\T P$
относительно любой согласованной почти комплексной структуры. 

\begin{ex}[(формула присоединения)]{Упражнение}\label{10.2.F}
Пусть $(P^4, \omega)$ — симплектическое многообразие с согласованной
почти комплексной структурой $j$. 
Пусть $\Sigma \subset P$ — вложенная $j$-голоморфная сфера.
Покажите, что
\[1 +\tfrac12 (([\Sigma], [\Sigma]) - c_1 (\Sigma)) = 0.\]
\emph{Подсказка:} используйте, что $([\Sigma], [\Sigma])$ —
самопересечение в комплексном нормальном расслоении $\nu_\Sigma$ и
$\T_\Sigma P = \T\Sigma \oplus \nu_\Sigma$. 
\end{ex}


\begin{thm*}{Следствие}
Пусть $\Sigma$ — симплектически вложенная сфера.
Тогда $c_1(\Sigma) = 1$ в том и только в том случае, когда $([\Sigma], [\Sigma]) = -1$. 
\end{thm*}

\parit{Доказательство.}
Поскольку $\Sigma$ симплектически вложена, существует
$\omega$-совместимая почти комплексная структура $j$ такая, что
$\Sigma$ $j$-голоморфна (см. \ref{10.2.C}). 
Теперь утверждение следует из формулы присоединения \ref{10.2.F}. 
\qeds


\section{Приложение к сцеплению}\label{sec:10.3}

В этом разделе мы выводим \ref{9.4.A} из \ref{10.1.A}.
Напомним, что мы изучаем деформацию сцепления расслоения $P(T \oplus C) \to \CP^1$, где $T$ и $C$ — соответственно, тавтологическое и тривиальное расслоения. 

\begin{ex*}{Упражнение}
Пусть $E$ — комплексное векторное пространство, и пусть $l \in P(E)$
— прямая в $E$. 
Покажите, что $\T_l P(E)$ канонически изоморфно $\Hom(l, E/l) = l^\ast \otimes E/l$.
\end{ex*}

Прежде всего, мы хотим вычислить кольцо (ко)гомологий $P \z= P(T \oplus C)$.
Обозначим через $[F]$ гомологический класс слоя, а через $\Sigma$
сечение, соответствующее подрасслоению $0\oplus C$ ранга $1$. 
Ясно, что $([F], [F]) = 0$ и $([F], [\Sigma]) = 1$.
Для вычисления $([\Sigma], [\Sigma])$, обратим внимание на то, что
нормальное расслоение $\nu_\Sigma$ — это просто ограничение на
$\Sigma$ касательного расслоения к слоям. 
Из приведённого выше упражнения следует, что 
\[\nu_\Sigma = \Hom(C, T \oplus C/C) = \Hom(C, T) = C^\ast \otimes T = T.\]
Таким образом, $c_1 (\nu_\Sigma) = -1$. 
Так как $([\Sigma], [\Sigma])$ это самопересечение в нормальном расслоении,
мы заключаем, что $([\Sigma], [\Sigma]) \z= -1$.

Пусть $\omega_t$ — деформация сцепления.
Из теоремы Мозера \cite{MS} легко следует, что для любого достотчно малого
$t$ форма $\omega_t$ симплектоморфна кэлеровой форме относительно
стандартной комплексной структуры на $P(T \oplus C)$. 
Не умоляя общности можно считать, что $\omega_t$ — кэлерова
форма при малых $t$. 
Поскольку $\Sigma$ — голоморфное сечение $P$, мы получаем, что
$\Sigma$ симплектична и вложена. 
Учитывая, что $([\Sigma], [\Sigma]) = -1$, получаем, что $\Sigma$ —
исключительная сфера. 
Таким образом, из теоремы \ref{10.1.A} следует, что $([\omega_t],
[\Sigma]) > 0$ для всех~$t$. 

Напомним, что $[\omega_t] = p^\ast [\tau] + tc$, где $[\tau]$ —
образующая $H^2 (S^2;\ZZ)$, соответствующая ориентации сферы, а $c$ — класс сцепления.  
Классы $p^\ast [\tau]$ и $c$ однозначно определяются следующими соотношениями: 
\begin{align*}
(p^\ast [\tau], [F]) &= 0,
&
(p^\ast [\tau], [\Sigma]) &= 1,
\\
(c, [F]) &= 1,
&
c^2 &= 0.
\end{align*}
Мы будем использовать двойственность Пуанкаре для отождествления гомологий и когомологий.

\begin{ex*}{Упражнение}
Докажите, что $p^\ast [\tau] = [F]$ и $c = [\Sigma] + \tfrac12 [F]$.
\end{ex*}

Таким образом, в силу теоремы \ref{10.1.A}
\[([\omega_t], [\Sigma]) = ([F] + t[\Sigma] + \tfrac t2[F], [\Sigma]) = 1 - t + \tfrac t2= 1 -\tfrac t2>0,\]
и, значит, $t < 2$.
Поскольку это верно для любой деформации сцепления, получаем $\epsilon(P)\le2$,
что завершает доказательство.
\qeds

\section{Псевдоголоморфные кривые}\label{sec:10.4}

Здесь мы кратко изложим громовскую теорию \rindex{псевдоголоморфная кривая}псевдоголоморфных кривых \cite{G1,AL}. 
Пусть $(P^{2n}, \omega)$ — симплектическое многообразие и пусть $A
\in H_2 (P; \ZZ)$ — примитивный класс;
то есть, $A$ нельзя представить в виде $kB$, где $k > 1$ — целое
число и $B \in H_2 (P; \ZZ)$. 
В частности, $A \ne 0$.
Пусть $\J$ — пространство всех $\omega$-совместимых почти
комплексных структур на $P$, а $\mathcal{N}$ — пространство всех
гладких отображений $f\: S^2 \to P$ таких, что $[f] = A$. 
Определим $\mathcal{X} \subset \mathcal{N} \times \J$ как 
\[\mathcal{X}
=
\set{(f,j)}{f \in \mathcal{N},\  j \in \J \ \text{и}\  \bar
  \partial_j f = 0}.\] 
Тогда $\mathcal{X}$ — гладкое подмногообразие в $\mathcal{N} \times
\J$, и проекция $\pi\:\mathcal{X}\z\to\J$~—
оператор Фредгольма, то есть 
$\Ker \pi_\ast$ и $\Coker \pi_\ast$ конечномерны.
Индекс $\pi$ удовлетворяет условию 
\[\Index \pi_\ast
\df
\dim (\Ker \pi_\ast) - \dim (\Coker \pi_\ast) 
= 
2(c_1 (A) + n).
\]
Фредгольмовость $\pi$ позволяет применить бесконечномерную версию теоремы Сарда \cite{Sm}.

\begin{figure}[ht!]
\vskip0mm
\centering
\includegraphics{mppics/pic-11}
\caption{}\label{pic-11}
\vskip0mm
\end{figure}

Пусть $j_0$ и $j_1$ — регулярные значения $\pi$ (то есть $\pi_\ast$
сюръективен для всех $x \in \pi^{-1} (j_k)$, $k = 0, 1$). 
Тогда $\pi^{-1} (j_0)$ и $\pi^{-1} (j_1)$ — гладкие подмногообразия.
Для пути $\gamma$ общего положения, соединяющего $j_0$ с $j_1$,
прообраз $\pi^{-1}(\gamma)$ является гладким подмногообразием
размерности $\Index (\pi) + 1$ и $\partial\pi^{-1} (\gamma) = \pi^{-1}
(j_0) \cup \pi^{-1} (j_1)$ (см. рис. \ref{pic-11}). 

Решающую роль играет свойства компактности $\pi^{-1} (\gamma)$.
Прежде всего заметим, что слой $\pi^{-1} (\gamma)$ сам по себе не
компактен, ведь на нём действует некомпактная группа $\PSL(2, \CC)$. 
Здесь $\PSL(2, \CC)$ — группа конформных преобразований $(S^2, i)$ и
она действует на $\pi^{-1} (j_0)$ поскольку, 
если $h\: S^2 \z\to S^2$ конформно и $(f, j_0) \in \pi^{-1} (j_0)$, то
и $(f \circ h, j_0) \in \pi^{-1} (j_0)$. 

\begin{ex*}{Упражнение}
Покажите, что это действие свободно.
(Воспользуйтесь тем, что класс $A$ примитивен.)
\end{ex*}

\begin{thm*}{Теорема Громова о компактности}\rindex{теорема о компактности}\rindex{Громов}
Либо пространство модулей $\pi^{-1} (\gamma)/\PSL(2, \CC)$ компактно, либо существует семейство $(f_k, j_k) \z\in \pi^{-1} (\gamma)$ такое, что $j_k \to j_\infty$ и $f_k$ «сходится» к составной $j_\infty$-голоморфной кривой в классе $A$.
\end{thm*}

Мы не определяем сходимость.
Для нас важно, что если пространство модулей $\pi^{-1} (\gamma)/\PSL(2, \CC)$ не компактно, то существует \rindex{составная кривая}\emph{составная} $j$-голоморфная кривая в классе $A$ для некоторого $j \in \J$.
Составная кривая определяется следующим образом.
Пусть $A \z= A_1 +\dots+ A_d$ при $d > 1$ — такое разложение $A$, что $A_k \ne 0$ для всех $k$, и пусть $\phi_k \: S^2 \to P$ — $j$-голоморфные кривые в классах $A_k$, $k = 1,\dots,d$.
Мы говорим, что эти данные определяют составную кривую в классе~$A$.
Объединение $\phi_k (A_k)$ называется его образом и обычно считается связным.

Рассмотрим теперь следующую ситуацию.
\begin{itemize}
\item Множество $\J'$ тех $j \in \J$, у которых есть составная кривая в классе $A$, имеет коразмерность не менее $2$ в $\J$.

\item $j_0 \in \J\setminus \J'$ является регулярным значением $\pi$ и $\pi^{-1} (j_0)/\PSL(2, \CC)$ (которое, таким образом, является компактным многообразием без края) не кобордантно нулю (то есть не ограничивает компактное многообразие).
 
\end{itemize}
Например, точка не кобордантна нулю, так как она не ограничивает компактное многообразие.

Тогда для любого регулярного значения $j_1 \in \J \setminus \J'$ множество $\pi^{-1} (j_1)/\PSL(2, \CC)$ не пусто.
В самом деле, если бы оно было пусто то, соединив $j_0$ с $j_1$ путём общего положения $\gamma$ в $\J \setminus\J'$, получим
\[\partial(\pi^{-1}(\gamma)/\PSL(2,\CC))
=
\pi^{-1}(j_0)/\PSL(2,\CC),
\]
а это противоречит тому, что $\pi^{-1} (j_0)/\PSL(2, \CC)$ не кобордантно нулю.
Рис. \ref{pic-12} иллюстрирует противоречие.

\begin{figure}[ht!]
\vskip0mm
\centering
\includegraphics{mppics/pic-12}
\caption{}\label{pic-12}
\vskip0mm
\end{figure}

Это явление соответствует \rindex{принцип продолжения}принципу продолжения, с которым мы уже встречались в
случае псевдоголоморфных дисков в разделе \ref{sec:4.2}: 
\textit{либо решения $\bar\partial$-уравнения сохраняются, либо происходит выдувание}.

\section{Сохранение исключительных сфер}

Приведём набросок доказательства теоремы \ref{10.1.A}.
Мы полагаем, что $(P^4, \omega)$ — симплектическое многообразие и $\Sigma \subset P$ — исключительная сфера с $A = [\Sigma]$.
Пусть $\omega_t$, $t \in [0;1]$, — деформация симплектической формы $\omega = \omega_0$.

\parbf{Шаг 1.}
Выберем $\omega$-совместимую почти комплексную структуру $j_0$ такую, что $\Sigma$ является $j_0$-голоморфной, и расширим $j_0$ до однопараметрического семейства $j_t$ такого, что $j_t$ является $\omega_t$-совместимой (ср. \ref{10.2.C}).

\parbf{Шаг 2.}
Теория, описанная в предыдущем разделе, без изменений работает для
множества всех $j$, совместимых с симплектическими структурами в
данном классе деформации. 
И значит
\[\Index\pi = 2(c_1 (A) + n) = 2(1 + 2) = 6.\]
Однако $\dim_\RR\PSL(2, \CC) = 6$, поэтому $\dim \pi^{-1}
(j_0)/\PSL(2, \CC) = 0$ при условии, что значение $j_0$ регулярно. 

\parbf{Шаг 3.}
В размерности 4 мы видим следующее.
Два различных ростка $j$-голоморфных кривых всегда пересекаются с
положительным индексом в общей точке. 
Это утверждение хорошо известно если $j$ интегрируема.
В неинтегрируемом случае оно просто следует из линейной алгебры при
условии, что пересечение трансверсально. 
Однако доказательство требует тонкого локального анализа
нетрансверсальных пересечений. 
Как следствие, в $A$ существует единственная $j_0$-голоморфная кривая
— если бы была другая, скажем, $\Sigma'$, то $([\Sigma], [\Sigma'])
= ([\Sigma], [\Sigma]) = -1$, что противоречит тому, что $([\Sigma],
[\Sigma']) \ge 0$. 
Мы заключаем, что $\pi^{-1} (j_0)/\PSL(2, \CC)$ состоит ровно из одной
точки. 

\parbf{Шаг 4.}
Выберем регулярные $j_0$ и $\{j_t\}$.
Мы утверждаем, что в общем случае составные кривые не появляются.
Действительно, положим $A = A_1 +\dots + A_d$, $d > 1$ и $c_1 (A) = 1
= c_1 (A_1) +\dots + c_1 (A_d)$. 
Таким образом, по крайней мере один из классов Черна в правой части неположителен.
Не умоляя общности можно предположить, что $c_1 (A_1) \le 0$ и что
$A_1$ представляется $j_s$-голоморфной кривой для некоторого $s \in
[0;1]$. 
Обозначим через $\pi_{A_1}$ проекцию для класса $A_1$ (см. начало \ref{sec:10.4}).
Мы видим, что $\pi_{A_1}^{-1} (\gamma)/\PSL(2, \CC)$ непусто.
С другой стороны, 
\[\dim \pi_{A_1}^{-1} (\gamma)/\PSL(2, \CC) = 2(c_1 (A_1) + 2) - 6 + 1 \le -1,\]
что невозможно.
Таким образом, в общем случае выдувания не происходит (это отражает то, что $\codim \J' \ge 2$). 
Следовательно, $A$ представляется $j_1$-голоморфной кривой и, следовательно $([\omega_1], [A]) > 0$ (см. \ref{10.2.E}).
